\documentclass{report}

%%%%%%%%%%%%%%%%%%%%%%%%%%%%%%%%%
% PACKAGE IMPORTS
%%%%%%%%%%%%%%%%%%%%%%%%%%%%%%%%%


\usepackage[tmargin=2cm,rmargin=1in,lmargin=1in,margin=0.85in,bmargin=2cm,footskip=.2in]{geometry}
\usepackage{amsmath,amsfonts,amsthm,amssymb,mathtools}
\usepackage[varbb]{newpxmath}
\usepackage{xfrac}
\usepackage[makeroom]{cancel}
\usepackage{mathtools}
\usepackage{bookmark}
\usepackage{enumitem}
\usepackage{hyperref,theoremref}
\usepackage{xparse}
\hypersetup{
	pdftitle={Hamboola my beloved},
	colorlinks=true, linkcolor=doc!90,
	bookmarksnumbered=true,
	bookmarksopen=true
}
\usepackage[most,many,breakable]{tcolorbox}
\usepackage{xcolor}
\usepackage{varwidth}
\usepackage{varwidth}
\usepackage{etoolbox}
\usepackage{bm}
%\usepackage{authblk}
\usepackage{pgfplots}
\usepackage{nameref}
\usepackage{multicol,array}
\usepackage{tikz-cd}
\usepackage[ruled,vlined,linesnumbered]{algorithm2e}
\usepackage{comment} % enables the use of multi-line comments (\ifx \fi) 
\usepackage{import}
\usepackage{xifthen}
\usepackage{pdfpages}
\usepackage{transparent}
\usepackage{minted}
\usepackage{fontspec}
\usepackage{tasks}
\usepackage{chemfig}
\usepackage[version=4]{mhchem}
\usepackage{suffix}
\usepackage{tabularx}
\usepackage{subcaption}
\usepackage{physics}


% \setmonofont{SpaceMono Nerd Font}
\setminted{fontsize=\footnotesize}

\newcommand\mycommfont[1]{\footnotesize\ttfamily\textcolor{blue}{#1}}
\SetCommentSty{mycommfont}
\newcommand{\incfig}[1]{%
	\def\svgwidth{\columnwidth}
	\import{./figures/}{#1.pdf_tex}
}

\usepackage{tikzsymbols}
% \renewcommand\qedsymbol{$\Laughey$}
\pgfplotsset{compat=1.18}

%\usepackage{import}
%\usepackage{xifthen}
%\usepackage{pdfpages}
%\usepackage{transparent}


%%%%%%%%%%%%%%%%%%%%%%%%%%%%%%
% SELF MADE COLORS
%%%%%%%%%%%%%%%%%%%%%%%%%%%%%%



\definecolor{myg}{RGB}{56, 140, 70}
\definecolor{myb}{RGB}{45, 111, 177}
\definecolor{myr}{RGB}{199, 68, 64}
\definecolor{mytheorembg}{HTML}{F2F2F9}
\definecolor{mytheoremfr}{HTML}{00007B}
\definecolor{mylenmabg}{HTML}{FFFAF8}
\definecolor{mylenmafr}{HTML}{983b0f}
\definecolor{mypropbg}{HTML}{f2fbfc}
\definecolor{mypropfr}{HTML}{191971}
\definecolor{myexamplebg}{HTML}{F2FBF8}
\definecolor{myexamplefr}{HTML}{88D6D1}
\definecolor{myexampleti}{HTML}{2A7F7F}
\definecolor{mydefinitbg}{HTML}{E5E5FF}
\definecolor{mydefinitfr}{HTML}{3F3FA3}
\definecolor{notesgreen}{RGB}{0,162,0}
\definecolor{myp}{RGB}{197, 92, 212}
\definecolor{mygr}{HTML}{2C3338}
\definecolor{myred}{RGB}{127,0,0}
\definecolor{myyellow}{RGB}{169,121,69}
\definecolor{myexercisebg}{HTML}{F2FBF8}
\definecolor{myexercisefg}{HTML}{88D6D1}
\definecolor{codebg}{HTML}{0D1117}

%%%%%%%%%%%%%%%%%%%%%%%%%%%%
% TCOLORBOX SETUPS
%%%%%%%%%%%%%%%%%%%%%%%%%%%%

\setlength{\parindent}{0cm}
%================================
% THEOREM BOX
%================================

\tcbuselibrary{theorems,skins,hooks}
\newtcbtheorem[number within=section]{Theorem}{Theorem}
{%
	enhanced,
	breakable,
	colback = mytheorembg,
	frame hidden,
	boxrule = 0sp,
	borderline west = {2pt}{0pt}{mytheoremfr},
	sharp corners,
	detach title,
	before upper = \tcbtitle\par\smallskip,
	coltitle = mytheoremfr,
	fonttitle = \bfseries\sffamily,
	description font = \mdseries,
	separator sign none,
	segmentation style={solid, mytheoremfr},
}
{th}

\tcbuselibrary{theorems,skins,hooks}
\newtcbtheorem[number within=chapter]{theorem}{Theorem}
{%
	enhanced,
	breakable,
	colback = mytheorembg,
	frame hidden,
	boxrule = 0sp,
	borderline west = {2pt}{0pt}{mytheoremfr},
	sharp corners,
	detach title,
	before upper = \tcbtitle\par\smallskip,
	coltitle = mytheoremfr,
	fonttitle = \bfseries\sffamily,
	description font = \mdseries,
	separator sign none,
	segmentation style={solid, mytheoremfr},
}
{th}


\tcbuselibrary{theorems,skins,hooks}
\newtcolorbox{Theoremcon}
{%
	enhanced
	,breakable
	,colback = mytheorembg
	,frame hidden
	,boxrule = 0sp
	,borderline west = {2pt}{0pt}{mytheoremfr}
	,sharp corners
	,description font = \mdseries
	,separator sign none
}

%================================
% Corollery
%================================
\tcbuselibrary{theorems,skins,hooks}
\newtcbtheorem[number within=section]{Corollary}{Corollary}
{%
	enhanced
	,breakable
	,colback = myp!10
	,frame hidden
	,boxrule = 0sp
	,borderline west = {2pt}{0pt}{myp!85!black}
	,sharp corners
	,detach title
	,before upper = \tcbtitle\par\smallskip
	,coltitle = myp!85!black
	,fonttitle = \bfseries\sffamily
	,description font = \mdseries
	,separator sign none
	,segmentation style={solid, myp!85!black}
}
{th}
\tcbuselibrary{theorems,skins,hooks}
\newtcbtheorem[number within=chapter]{corollary}{Corollary}
{%
	enhanced
	,breakable
	,colback = myp!10
	,frame hidden
	,boxrule = 0sp
	,borderline west = {2pt}{0pt}{myp!85!black}
	,sharp corners
	,detach title
	,before upper = \tcbtitle\par\smallskip
	,coltitle = myp!85!black
	,fonttitle = \bfseries\sffamily
	,description font = \mdseries
	,separator sign none
	,segmentation style={solid, myp!85!black}
}
{th}


%================================
% LENMA
%================================

\tcbuselibrary{theorems,skins,hooks}
\newtcbtheorem[number within=section]{Lenma}{Lenma}
{%
	enhanced,
	breakable,
	colback = mylenmabg,
	frame hidden,
	boxrule = 0sp,
	borderline west = {2pt}{0pt}{mylenmafr},
	sharp corners,
	detach title,
	before upper = \tcbtitle\par\smallskip,
	coltitle = mylenmafr,
	fonttitle = \bfseries\sffamily,
	description font = \mdseries,
	separator sign none,
	segmentation style={solid, mylenmafr},
}
{th}

\tcbuselibrary{theorems,skins,hooks}
\newtcbtheorem[number within=chapter]{lenma}{Lenma}
{%
	enhanced,
	breakable,
	colback = mylenmabg,
	frame hidden,
	boxrule = 0sp,
	borderline west = {2pt}{0pt}{mylenmafr},
	sharp corners,
	detach title,
	before upper = \tcbtitle\par\smallskip,
	coltitle = mylenmafr,
	fonttitle = \bfseries\sffamily,
	description font = \mdseries,
	separator sign none,
	segmentation style={solid, mylenmafr},
}
{th}


%================================
% PROPOSITION
%================================

\tcbuselibrary{theorems,skins,hooks}
\newtcbtheorem[number within=section]{Prop}{Proposition}
{%
	enhanced,
	breakable,
	colback = mypropbg,
	frame hidden,
	boxrule = 0sp,
	borderline west = {2pt}{0pt}{mypropfr},
	sharp corners,
	detach title,
	before upper = \tcbtitle\par\smallskip,
	coltitle = mypropfr,
	fonttitle = \bfseries\sffamily,
	description font = \mdseries,
	separator sign none,
	segmentation style={solid, mypropfr},
}
{th}

\tcbuselibrary{theorems,skins,hooks}
\newtcbtheorem[number within=chapter]{prop}{Proposition}
{%
	enhanced,
	breakable,
	colback = mypropbg,
	frame hidden,
	boxrule = 0sp,
	borderline west = {2pt}{0pt}{mypropfr},
	sharp corners,
	detach title,
	before upper = \tcbtitle\par\smallskip,
	coltitle = mypropfr,
	fonttitle = \bfseries\sffamily,
	description font = \mdseries,
	separator sign none,
	segmentation style={solid, mypropfr},
}
{th}


%================================
% CLAIM
%================================

\tcbuselibrary{theorems,skins,hooks}
\newtcbtheorem[number within=section]{claim}{Claim}
{%
	enhanced
	,breakable
	,colback = myg!10
	,frame hidden
	,boxrule = 0sp
	,borderline west = {2pt}{0pt}{myg}
	,sharp corners
	,detach title
	,before upper = \tcbtitle\par\smallskip
	,coltitle = myg!85!black
	,fonttitle = \bfseries\sffamily
	,description font = \mdseries
	,separator sign none
	,segmentation style={solid, myg!85!black}
}
{th}



%================================
% Exercise
%================================

\tcbuselibrary{theorems,skins,hooks}
\newtcbtheorem[number within=section]{Exercise}{Exercise}
{%
	enhanced,
	breakable,
	colback = myexercisebg,
	frame hidden,
	boxrule = 0sp,
	borderline west = {2pt}{0pt}{myexercisefg},
	sharp corners,
	detach title,
	before upper = \tcbtitle\par\smallskip,
	coltitle = myexercisefg,
	fonttitle = \bfseries\sffamily,
	description font = \mdseries,
	separator sign none,
	segmentation style={solid, myexercisefg},
}
{th}

\tcbuselibrary{theorems,skins,hooks}
\newtcbtheorem[number within=chapter]{exercise}{Exercise}
{%
	enhanced,
	breakable,
	colback = myexercisebg,
	frame hidden,
	boxrule = 0sp,
	borderline west = {2pt}{0pt}{myexercisefg},
	sharp corners,
	detach title,
	before upper = \tcbtitle\par\smallskip,
	coltitle = myexercisefg,
	fonttitle = \bfseries\sffamily,
	description font = \mdseries,
	separator sign none,
	segmentation style={solid, myexercisefg},
}
{th}

%================================
% EXAMPLE BOX
%================================

\newtcbtheorem[number within=section]{Example}{Example}
{%
	colback = myexamplebg
	,breakable
	,colframe = myexamplefr
	,coltitle = myexampleti
	,boxrule = 1pt
	,sharp corners
	,detach title
	,before upper=\tcbtitle\par\smallskip
	,fonttitle = \bfseries
	,description font = \mdseries
	,separator sign none
	,description delimiters parenthesis
}
{ex}

\newtcbtheorem[number within=chapter]{example}{Example}
{%
	colback = myexamplebg
	,breakable
	,colframe = myexamplefr
	,coltitle = myexampleti
	,boxrule = 1pt
	,sharp corners
	,detach title
	,before upper=\tcbtitle\par\smallskip
	,fonttitle = \bfseries
	,description font = \mdseries
	,separator sign none
	,description delimiters parenthesis
}
{ex}

%================================
% DEFINITION BOX
%================================

\newtcbtheorem[number within=section]{Definition}{Definition}{enhanced,
	before skip=2mm,after skip=2mm, colback=red!5,colframe=red!80!black,boxrule=0.5mm,
	attach boxed title to top left={xshift=1cm,yshift*=1mm-\tcboxedtitleheight}, varwidth boxed title*=-3cm,
	boxed title style={frame code={
					\path[fill=tcbcolback]
					([yshift=-1mm,xshift=-1mm]frame.north west)
					arc[start angle=0,end angle=180,radius=1mm]
					([yshift=-1mm,xshift=1mm]frame.north east)
					arc[start angle=180,end angle=0,radius=1mm];
					\path[left color=tcbcolback!60!black,right color=tcbcolback!60!black,
						middle color=tcbcolback!80!black]
					([xshift=-2mm]frame.north west) -- ([xshift=2mm]frame.north east)
					[rounded corners=1mm]-- ([xshift=1mm,yshift=-1mm]frame.north east)
					-- (frame.south east) -- (frame.south west)
					-- ([xshift=-1mm,yshift=-1mm]frame.north west)
					[sharp corners]-- cycle;
				},interior engine=empty,
		},
	fonttitle=\bfseries,
	title={#2},#1}{def}
\newtcbtheorem[number within=chapter]{definition}{Definition}{enhanced,
	before skip=2mm,after skip=2mm, colback=red!5,colframe=red!80!black,boxrule=0.5mm,
	attach boxed title to top left={xshift=1cm,yshift*=1mm-\tcboxedtitleheight}, varwidth boxed title*=-3cm,
	boxed title style={frame code={
					\path[fill=tcbcolback]
					([yshift=-1mm,xshift=-1mm]frame.north west)
					arc[start angle=0,end angle=180,radius=1mm]
					([yshift=-1mm,xshift=1mm]frame.north east)
					arc[start angle=180,end angle=0,radius=1mm];
					\path[left color=tcbcolback!60!black,right color=tcbcolback!60!black,
						middle color=tcbcolback!80!black]
					([xshift=-2mm]frame.north west) -- ([xshift=2mm]frame.north east)
					[rounded corners=1mm]-- ([xshift=1mm,yshift=-1mm]frame.north east)
					-- (frame.south east) -- (frame.south west)
					-- ([xshift=-1mm,yshift=-1mm]frame.north west)
					[sharp corners]-- cycle;
				},interior engine=empty,
		},
	fonttitle=\bfseries,
	title={#2},#1}{def}



%================================
% Solution BOX
%================================

\makeatletter
\newtcbtheorem{question}{Question}{enhanced,
	breakable,
	colback=white,
	colframe=myb!80!black,
	attach boxed title to top left={yshift*=-\tcboxedtitleheight},
	fonttitle=\bfseries,
	title={#2},
	boxed title size=title,
	boxed title style={%
			sharp corners,
			rounded corners=northwest,
			colback=tcbcolframe,
			boxrule=0pt,
		},
	underlay boxed title={%
			\path[fill=tcbcolframe] (title.south west)--(title.south east)
			to[out=0, in=180] ([xshift=5mm]title.east)--
			(title.center-|frame.east)
			[rounded corners=\kvtcb@arc] |-
			(frame.north) -| cycle;
		},
	#1
}{def}
\makeatother

%================================
% SOLUTION BOX
%================================

\makeatletter
\newtcolorbox{solution}{enhanced,
	breakable,
	colback=white,
	colframe=myg!80!black,
	attach boxed title to top left={yshift*=-\tcboxedtitleheight},
	title=Solution,
	boxed title size=title,
	boxed title style={%
			sharp corners,
			rounded corners=northwest,
			colback=tcbcolframe,
			boxrule=0pt,
		},
	underlay boxed title={%
			\path[fill=tcbcolframe] (title.south west)--(title.south east)
			to[out=0, in=180] ([xshift=5mm]title.east)--
			(title.center-|frame.east)
			[rounded corners=\kvtcb@arc] |-
			(frame.north) -| cycle;
		},
}
\makeatother

%================================
% Question BOX
%================================

\makeatletter
\newtcbtheorem{qstion}{Question}{enhanced,
	breakable,
	colback=white,
	colframe=mygr,
	attach boxed title to top left={yshift*=-\tcboxedtitleheight},
	fonttitle=\bfseries,
	title={#2},
	boxed title size=title,
	boxed title style={%
			sharp corners,
			rounded corners=northwest,
			colback=tcbcolframe,
			boxrule=0pt,
		},
	underlay boxed title={%
			\path[fill=tcbcolframe] (title.south west)--(title.south east)
			to[out=0, in=180] ([xshift=5mm]title.east)--
			(title.center-|frame.east)
			[rounded corners=\kvtcb@arc] |-
			(frame.north) -| cycle;
		},
	#1
}{def}
\makeatother

\newtcbtheorem[number within=chapter]{wconc}{Wrong Concept}{
	breakable,
	enhanced,
	colback=white,
	colframe=myr,
	arc=0pt,
	outer arc=0pt,
	fonttitle=\bfseries\sffamily\large,
	colbacktitle=myr,
	attach boxed title to top left={},
	boxed title style={
			enhanced,
			skin=enhancedfirst jigsaw,
			arc=3pt,
			bottom=0pt,
			interior style={fill=myr}
		},
	#1
}{def}



%================================
% NOTE BOX
%================================

\usetikzlibrary{hobby}
\usetikzlibrary{arrows,calc,shadows.blur}
\tcbuselibrary{skins}
\newtcolorbox{note}[1][]{%
	enhanced jigsaw,
	colback=gray!20!white,%
	colframe=gray!80!black,
	size=small,
	boxrule=1pt,
	title=\textbf{Note:-},
	halign title=flush center,
	coltitle=black,
	breakable,
	drop shadow=black!50!white,
	attach boxed title to top left={xshift=1cm,yshift=-\tcboxedtitleheight/2,yshifttext=-\tcboxedtitleheight/2},
	minipage boxed title=1.5cm,
	boxed title style={%
			colback=white,
			size=fbox,
			boxrule=1pt,
			boxsep=2pt,
			underlay={%
					\coordinate (dotA) at ($(interior.west) + (-0.5pt,0)$);
					\coordinate (dotB) at ($(interior.east) + (0.5pt,0)$);
					\begin{scope}
						\clip (interior.north west) rectangle ([xshift=3ex]interior.east);
						\filldraw [white, blur shadow={shadow opacity=60, shadow yshift=-.75ex}, rounded corners=2pt] (interior.north west) rectangle (interior.south east);
					\end{scope}
					\begin{scope}[gray!80!black]
						\fill (dotA) circle (2pt);
						\fill (dotB) circle (2pt);
					\end{scope}
				},
		},
	#1,
}

%%%%%%%%%%%%%%%%%%%%%%%%%%%%%%
% SELF MADE COMMANDS
%%%%%%%%%%%%%%%%%%%%%%%%%%%%%%


\newcommand{\thm}[2]{\begin{Theorem}{#1}{}#2\end{Theorem}}
\newcommand{\cor}[2]{\begin{Corollary}{#1}{}#2\end{Corollary}}
\newcommand{\mlenma}[2]{\begin{Lenma}{#1}{}#2\end{Lenma}}
\newcommand{\mprop}[2]{\begin{Prop}{#1}{}#2\end{Prop}}
\newcommand{\clm}[3]{\begin{claim}{#1}{#2}#3\end{claim}}
\newcommand{\wc}[2]{\begin{wconc}{#1}{}\setlength{\parindent}{1cm}#2\end{wconc}}
\newcommand{\thmcon}[1]{\begin{Theoremcon}{#1}\end{Theoremcon}}
\newcommand{\ex}[2]{\begin{Example}{#1}{}#2\end{Example}}
\newcommand{\dfn}[2]{\begin{Definition}[colbacktitle=red!75!black]{#1}{}#2\end{Definition}}
\newcommand{\dfnc}[2]{\begin{definition}[colbacktitle=red!75!black]{#1}{}#2\end{definition}}
\newcommand{\qs}[2]{\begin{question}{#1}{}#2\end{question}}
\newcommand{\pf}[2]{\begin{myproof}[#1]#2\end{myproof}}
\newcommand{\nt}[1]{\begin{note}#1\end{note}}

\newcommand*\circled[1]{\tikz[baseline=(char.base)]{
		\node[shape=circle,draw,inner sep=1pt] (char) {#1};}}
\newcommand\getcurrentref[1]{%
	\ifnumequal{\value{#1}}{0}
	{??}
	{\the\value{#1}}%
}
\newcommand{\getCurrentSectionNumber}{\getcurrentref{section}}
\newenvironment{myproof}[1][\proofname]{%
	\proof[\bfseries #1: ]%
}{\endproof}

\newcommand{\mclm}[2]{\begin{myclaim}[#1]#2\end{myclaim}}
\newenvironment{myclaim}[1][\claimname]{\proof[\bfseries #1: ]}{}

\newcounter{mylabelcounter}

\makeatletter
\newcommand{\setword}[2]{%
	\phantomsection
	#1\def\@currentlabel{\unexpanded{#1}}\label{#2}%
}
\makeatother




\tikzset{
	symbol/.style={
			draw=none,
			every to/.append style={
					edge node={node [sloped, allow upside down, auto=false]{$#1$}}}
		}
}


% deliminators
% \DeclarePairedDelimiter{\abs}{\lvert}{\rvert}
% \DeclarePairedDelimiter{\norm}{\lVert}{\rVert}

\DeclarePairedDelimiter{\ceil}{\lceil}{\rceil}
\DeclarePairedDelimiter{\floor}{\lfloor}{\rfloor}
\DeclarePairedDelimiter{\round}{\lfloor}{\rceil}

\newsavebox\diffdbox
\newcommand{\slantedromand}{{\mathpalette\makesl{d}}}
\newcommand{\makesl}[2]{%
	\begingroup
	\sbox{\diffdbox}{$\mathsurround=0pt#1\mathrm{#2}$}%
	\pdfsave
	\pdfsetmatrix{1 0 0.2 1}%
	\rlap{\usebox{\diffdbox}}%
	\pdfrestore
	\hskip\wd\diffdbox
	\endgroup
}
% \newcommand{\dd}[1][]{\ensuremath{\mathop{}\!\ifstrempty{#1}{%
% 			\slantedromand\@ifnextchar^{\hspace{0.2ex}}{\hspace{0.1ex}}}%
% 		{\slantedromand\hspace{0.2ex}^{#1}}}}
\ProvideDocumentCommand\dv{o m g}{%
	\ensuremath{%
		\IfValueTF{#3}{%
			\IfNoValueTF{#1}{%
				\frac{\dd #2}{\dd #3}%
			}{%
				\frac{\dd^{#1} #2}{\dd #3^{#1}}%
			}%
		}{%
			\IfNoValueTF{#1}{%
				\frac{\dd}{\dd #2}%
			}{%
				\frac{\dd^{#1}}{\dd #2^{#1}}%
			}%
		}%
	}%
}
\DeclareDocumentCommand\pdv{ s o m g g d() }
{ % Partial derivative
	% s: star for \flatfrac flat derivative
	% o: optional n for nth derivative
	% m: mandatory (x in df/dx)
	% g: optional (f in df/dx)
	% g: optional (y in d^2f/dxdy)
	% d: long-form d/dx(...)
	\IfBooleanTF{#1}
	{\let\fractype\flatfrac}
	{\let\fractype\frac}
	\IfNoValueTF{#4}
	{
		\IfNoValueTF{#6}
		{\fractype{\partial \IfNoValueTF{#2}{}{^{#2}}}{\partial #3\IfNoValueTF{#2}{}{^{#2}}}}
		{\fractype{\partial \IfNoValueTF{#2}{}{^{#2}}}{\partial #3\IfNoValueTF{#2}{}{^{#2}}} \argopen(#6\argclose)}
	}
	{
		\IfNoValueTF{#5}
		{\fractype{\partial \IfNoValueTF{#2}{}{^{#2}} #3}{\partial #4\IfNoValueTF{#2}{}{^{#2}}}}
		{\fractype{\partial^2 #3}{\partial #4 \partial #5}}
	}
}
% \providecommand*{\pdv}[3][]{\frac{\partial^{#1}#2}{\partial#3^{#1}}}
%  - others
\DeclareMathOperator{\Lap}{\mathcal{L}}
\DeclareMathOperator{\Var}{Var} % varience
\DeclareMathOperator{\Cov}{Cov} % covarience
\DeclareMathOperator{\E}{E} % expected

% Since the amsthm package isn't loaded

% I prefer the slanted \leq
\let\oldleq\leq % save them in case they're every wanted
\let\oldgeq\geq
\renewcommand{\leq}{\leqslant}
\renewcommand{\geq}{\geqslant}

% % redefine matrix env to allow for alignment, use r as default
% \renewcommand*\env@matrix[1][r]{\hskip -\arraycolsep
%     \let\@ifnextchar\new@ifnextchar
%     \array{*\c@MaxMatrixCols #1}}


%\usepackage{framed}
%\usepackage{titletoc}
%\usepackage{etoolbox}
%\usepackage{lmodern}


%\patchcmd{\tableofcontents}{\contentsname}{\sffamily\contentsname}{}{}

%\renewenvironment{leftbar}
%{\def\FrameCommand{\hspace{6em}%
%		{\color{myyellow}\vrule width 2pt depth 6pt}\hspace{1em}}%
%	\MakeFramed{\parshape 1 0cm \dimexpr\textwidth-6em\relax\FrameRestore}\vskip2pt%
%}
%{\endMakeFramed}

%\titlecontents{chapter}
%[0em]{\vspace*{2\baselineskip}}
%{\parbox{4.5em}{%
%		\hfill\Huge\sffamily\bfseries\color{myred}\thecontentspage}%
%	\vspace*{-2.3\baselineskip}\leftbar\textsc{\small\chaptername~\thecontentslabel}\\\sffamily}
%{}{\endleftbar}
%\titlecontents{section}
%[8.4em]
%{\sffamily\contentslabel{3em}}{}{}
%{\hspace{0.5em}\nobreak\itshape\color{myred}\contentspage}
%\titlecontents{subsection}
%[8.4em]
%{\sffamily\contentslabel{3em}}{}{}  
%{\hspace{0.5em}\nobreak\itshape\color{myred}\contentspage}



%%%%%%%%%%%%%%%%%%%%%%%%%%%%%%%%%%%%%%%%%%%
% TABLE OF CONTENTS
%%%%%%%%%%%%%%%%%%%%%%%%%%%%%%%%%%%%%%%%%%%

\usepackage{tikz}
\definecolor{doc}{RGB}{0,60,110}
\usepackage{titletoc}
\contentsmargin{0cm}
\titlecontents{chapter}[3.7pc]
{\addvspace{30pt}%
	\begin{tikzpicture}[remember picture, overlay]%
		\draw[fill=doc!60,draw=doc!60] (-7,-.1) rectangle (-0.9,.5);%
		\pgftext[left,x=-3.5cm,y=0.2cm]{\color{white}\Large\sc\bfseries Chapter\ \thecontentslabel};%
	\end{tikzpicture}\color{doc!60}\large\sc\bfseries}%
{}
{}
{\;\titlerule\;\large\sc\bfseries Page \thecontentspage
	\begin{tikzpicture}[remember picture, overlay]
		\draw[fill=doc!60,draw=doc!60] (2pt,0) rectangle (4,0.1pt);
	\end{tikzpicture}}%
\titlecontents{section}[3.7pc]
{\addvspace{2pt}}
{\contentslabel[\thecontentslabel]{2pc}}
{}
{\hfill\small \thecontentspage}
[]
\titlecontents*{subsection}[3.7pc]
{\addvspace{-1pt}\small}
{}
{}
{\ --- \small\thecontentspage}
[ \textbullet\ ][]

\makeatletter
\renewcommand{\tableofcontents}{%
	\chapter*{%
	  \vspace*{-20\p@}%
	  \begin{tikzpicture}[remember picture, overlay]%
		  \pgftext[right,x=15cm,y=0.2cm]{\color{doc!60}\Huge\sc\bfseries \contentsname};%
		  \draw[fill=doc!60,draw=doc!60] (13,-.75) rectangle (20,1);%
		  \clip (13,-.75) rectangle (20,1);
		  \pgftext[right,x=15cm,y=0.2cm]{\color{white}\Huge\sc\bfseries \contentsname};%
	  \end{tikzpicture}}%
	\@starttoc{toc}}
\makeatother

%From M275 "Topology" at SJSU
\newcommand{\id}{\mathrm{id}}
\newcommand{\taking}[1]{\xrightarrow{#1}}
\newcommand{\inv}{^{-1}}

%From M170 "Introduction to Graph Theory" at SJSU
\DeclareMathOperator{\diam}{diam}
\DeclareMathOperator{\ord}{ord}
\newcommand{\defeq}{\overset{\mathrm{def}}{=}}

%From the USAMO .tex files
\newcommand{\ts}{\textsuperscript}
\newcommand{\dg}{^\circ}
\newcommand{\ii}{\item}

% % From Math 55 and Math 145 at Harvard
% \newenvironment{subproof}[1][Proof]{%
% \begin{proof}[#1] \renewcommand{\qedsymbol}{$\blacksquare$}}%
% {\end{proof}}

\newcommand{\liff}{\leftrightarrow}
\newcommand{\lthen}{\rightarrow}
\newcommand{\opname}{\operatorname}
\newcommand{\surjto}{\twoheadrightarrow}
\newcommand{\injto}{\hookrightarrow}
\newcommand{\On}{\mathrm{On}} % ordinals
% \newcommand{\EE}{\mathbb{E}} % Expectance
\DeclareMathOperator{\img}{im} % Image
\DeclareMathOperator{\Img}{Im} % Image
\DeclareMathOperator{\coker}{coker} % Cokernel
\DeclareMathOperator{\Coker}{Coker} % Cokernel
\DeclareMathOperator{\Ker}{Ker} % Kernel
\DeclareMathOperator{\rank}{rank}
\DeclareMathOperator{\Spec}{Spec} % spectrum
\DeclareMathOperator{\Tr}{Tr} % trace
\DeclareMathOperator{\pr}{pr} % projection
\DeclareMathOperator{\ext}{ext} % extension
\DeclareMathOperator{\pred}{pred} % predecessor
\DeclareMathOperator{\dom}{dom} % domain
\DeclareMathOperator{\ran}{ran} % range
\DeclareMathOperator{\Hom}{Hom} % homomorphism
\DeclareMathOperator{\Mor}{Mor} % morphisms
\DeclareMathOperator{\End}{End} % endomorphism
% \DeclareMathOperator{\Pr}{Pr} % probability
% \DeclareMathOperator{\Var}{Var} % variance

\newcommand{\eps}{\epsilon}
\newcommand{\veps}{\varepsilon}
\newcommand{\ol}{\overline}
\newcommand{\ul}{\underline}
\newcommand{\wt}{\widetilde}
\newcommand{\wh}{\widehat}
\newcommand{\vocab}[1]{\textbf{\color{blue} #1}}
\providecommand{\half}{\frac{1}{2}}
\newcommand{\dang}{\measuredangle} %% Directed angle
\newcommand{\ray}[1]{\overrightarrow{#1}}
\newcommand{\seg}[1]{\overline{#1}}
\newcommand{\arc}[1]{\wideparen{#1}}
\DeclareMathOperator{\cis}{cis}
\DeclareMathOperator*{\lcm}{lcm}
\DeclareMathOperator*{\argmin}{arg min}
\DeclareMathOperator*{\argmax}{arg max}
\newcommand{\cycsum}{\sum_{\mathrm{cyc}}}
\newcommand{\symsum}{\sum_{\mathrm{sym}}}
\newcommand{\cycprod}{\prod_{\mathrm{cyc}}}
\newcommand{\symprod}{\prod_{\mathrm{sym}}}
\newcommand{\Qed}{\begin{flushright}\qed\end{flushright}}
\newcommand{\parinn}{\setlength{\parindent}{1cm}}
\newcommand{\parinf}{\setlength{\parindent}{0cm}}
% \newcommand{\norm}{\|\cdot\|}
\newcommand{\inorm}{\norm_{\infty}}
\newcommand{\opensets}{\{V_{\alpha}\}_{\alpha\in I}}
\newcommand{\oset}{V_{\alpha}}
\newcommand{\opset}[1]{V_{\alpha_{#1}}}
\newcommand{\lub}{\text{lub}}
\newcommand{\del}[2]{\frac{\partial #1}{\partial #2}}
\newcommand{\Del}[3]{\frac{\partial^{#1} #2}{\partial^{#1} #3}}
\newcommand{\deld}[2]{\dfrac{\partial #1}{\partial #2}}
\newcommand{\Deld}[3]{\dfrac{\partial^{#1} #2}{\partial^{#1} #3}}
\newcommand{\lm}{\lambda}
\newcommand{\uin}{\mathbin{\rotatebox[origin=c]{90}{$\in$}}}
\newcommand{\usubset}{\mathbin{\rotatebox[origin=c]{90}{$\subset$}}}
\newcommand{\lt}{\left}
\newcommand{\rt}{\right}
\newcommand{\bs}[1]{\boldsymbol{#1}}
\newcommand{\exs}{\exists}
\newcommand{\st}{\strut}
\newcommand{\dps}[1]{\displaystyle{#1}}
\newcommand{\va}[1]{\vec{\bm{\mathrm{#1}}}}
\WithSuffix\newcommand\va*[1]{\vec{\bm{#1}}}
\newcommand{\vb}[1]{\bm{\mathrm{#1}}}
\WithSuffix\newcommand\vb*[1]{\bm{#1}}
\newcommand{\vu}[1]{\hat{\bm{\mathrm{#1}}}}
\WithSuffix\newcommand\vu*[1]{\hat{\bm{#1}}}
\renewcommand{\dd}[1]{\mathrm{d}#1}
\renewcommand{\Re}{\mathrm{Re}}
\renewcommand{\Im}{\mathrm{Im}}
\DeclareMathOperator{\tr}{tr}
\newcommand{\csin}[1]{\mintinline{csharp}|#1|}
\renewcommand{\Pr}[1]{\mathrm{Pr}\lt( #1 \rt)}
\renewcommand{\Var}[1]{\mathrm{Var}\lt( #1 \rt)}
\newcommand{\cov}[1]{\mathrm{cov}\lt( #1 \rt)}

\newcommand{\sol}{\setlength{\parindent}{0cm}\textbf{\textit{Solution:}}\setlength{\parindent}{1cm} }
\newcommand{\solve}[1]{\setlength{\parindent}{0cm}\textbf{\textit{Solution: }}\setlength{\parindent}{1cm}#1 \Qed}

% Things Lie
\newcommand{\kb}{\mathfrak b}
\newcommand{\kg}{\mathfrak g}
\newcommand{\kh}{\mathfrak h}
\newcommand{\kn}{\mathfrak n}
\newcommand{\ku}{\mathfrak u}
\newcommand{\kz}{\mathfrak z}
\DeclareMathOperator{\Ext}{Ext} % Ext functor
\DeclareMathOperator{\Tor}{Tor} % Tor functor
\newcommand{\gl}{\opname{\mathfrak{gl}}} % frak gl group
\renewcommand{\sl}{\opname{\mathfrak{sl}}} % frak sl group chktex 6

% More script letters etc.
\newcommand{\SA}{\mathcal A}
\newcommand{\SB}{\mathcal B}
\newcommand{\SC}{\mathcal C}
\newcommand{\SF}{\mathcal F}
\newcommand{\SG}{\mathcal G}
\newcommand{\SH}{\mathcal H}
\newcommand{\OO}{\mathcal O}

\newcommand{\SCA}{\mathscr A}
\newcommand{\SCB}{\mathscr B}
\newcommand{\SCC}{\mathscr C}
\newcommand{\SCD}{\mathscr D}
\newcommand{\SCE}{\mathscr E}
\newcommand{\SCF}{\mathscr F}
\newcommand{\SCG}{\mathscr G}
\newcommand{\SCH}{\mathscr H}

% Mathfrak primes
\newcommand{\km}{\mathfrak m}
\newcommand{\kp}{\mathfrak p}
\newcommand{\kq}{\mathfrak q}

% number sets
\newcommand{\RR}[1][]{\ensuremath{\ifstrempty{#1}{\mathbb{R}}{\mathbb{R}^{#1}}}}
\newcommand{\NN}[1][]{\ensuremath{\ifstrempty{#1}{\mathbb{N}}{\mathbb{N}^{#1}}}}
\newcommand{\ZZ}[1][]{\ensuremath{\ifstrempty{#1}{\mathbb{Z}}{\mathbb{Z}^{#1}}}}
\newcommand{\QQ}[1][]{\ensuremath{\ifstrempty{#1}{\mathbb{Q}}{\mathbb{Q}^{#1}}}}
\newcommand{\CC}[1][]{\ensuremath{\ifstrempty{#1}{\mathbb{C}}{\mathbb{C}^{#1}}}}
\newcommand{\PP}[1][]{\ensuremath{\ifstrempty{#1}{\mathbb{P}}{\mathbb{P}^{#1}}}}
\newcommand{\HH}[1][]{\ensuremath{\ifstrempty{#1}{\mathbb{H}}{\mathbb{H}^{#1}}}}
\newcommand{\FF}[1][]{\ensuremath{\ifstrempty{#1}{\mathbb{F}}{\mathbb{F}^{#1}}}}
% expected value
\newcommand{\EE}{\ensuremath{\mathbb{E}}}
\newcommand{\charin}{\text{ char }}
\DeclareMathOperator{\sign}{sign}
\DeclareMathOperator{\Aut}{Aut}
\DeclareMathOperator{\Inn}{Inn}
\DeclareMathOperator{\Syl}{Syl}
\DeclareMathOperator{\Gal}{Gal}
\DeclareMathOperator{\GL}{GL} % General linear group
\DeclareMathOperator{\SL}{SL} % Special linear group

%---------------------------------------
% BlackBoard Math Fonts :-
%---------------------------------------

%Captital Letters
\newcommand{\bbA}{\mathbb{A}}	\newcommand{\bbB}{\mathbb{B}}
\newcommand{\bbC}{\mathbb{C}}	\newcommand{\bbD}{\mathbb{D}}
\newcommand{\bbE}{\mathbb{E}}	\newcommand{\bbF}{\mathbb{F}}
\newcommand{\bbG}{\mathbb{G}}	\newcommand{\bbH}{\mathbb{H}}
\newcommand{\bbI}{\mathbb{I}}	\newcommand{\bbJ}{\mathbb{J}}
\newcommand{\bbK}{\mathbb{K}}	\newcommand{\bbL}{\mathbb{L}}
\newcommand{\bbM}{\mathbb{M}}	\newcommand{\bbN}{\mathbb{N}}
\newcommand{\bbO}{\mathbb{O}}	\newcommand{\bbP}{\mathbb{P}}
\newcommand{\bbQ}{\mathbb{Q}}	\newcommand{\bbR}{\mathbb{R}}
\newcommand{\bbS}{\mathbb{S}}	\newcommand{\bbT}{\mathbb{T}}
\newcommand{\bbU}{\mathbb{U}}	\newcommand{\bbV}{\mathbb{V}}
\newcommand{\bbW}{\mathbb{W}}	\newcommand{\bbX}{\mathbb{X}}
\newcommand{\bbY}{\mathbb{Y}}	\newcommand{\bbZ}{\mathbb{Z}}

%---------------------------------------
% MathCal Fonts :-
%---------------------------------------

%Captital Letters
\newcommand{\mcA}{\mathcal{A}}	\newcommand{\mcB}{\mathcal{B}}
\newcommand{\mcC}{\mathcal{C}}	\newcommand{\mcD}{\mathcal{D}}
\newcommand{\mcE}{\mathcal{E}}	\newcommand{\mcF}{\mathcal{F}}
\newcommand{\mcG}{\mathcal{G}}	\newcommand{\mcH}{\mathcal{H}}
\newcommand{\mcI}{\mathcal{I}}	\newcommand{\mcJ}{\mathcal{J}}
\newcommand{\mcK}{\mathcal{K}}	\newcommand{\mcL}{\mathcal{L}}
\newcommand{\mcM}{\mathcal{M}}	\newcommand{\mcN}{\mathcal{N}}
\newcommand{\mcO}{\mathcal{O}}	\newcommand{\mcP}{\mathcal{P}}
\newcommand{\mcQ}{\mathcal{Q}}	\newcommand{\mcR}{\mathcal{R}}
\newcommand{\mcS}{\mathcal{S}}	\newcommand{\mcT}{\mathcal{T}}
\newcommand{\mcU}{\mathcal{U}}	\newcommand{\mcV}{\mathcal{V}}
\newcommand{\mcW}{\mathcal{W}}	\newcommand{\mcX}{\mathcal{X}}
\newcommand{\mcY}{\mathcal{Y}}	\newcommand{\mcZ}{\mathcal{Z}}


%---------------------------------------
% Bold Math Fonts :-
%---------------------------------------

%Captital Letters
\newcommand{\bmA}{\boldsymbol{A}}	\newcommand{\bmB}{\boldsymbol{B}}
\newcommand{\bmC}{\boldsymbol{C}}	\newcommand{\bmD}{\boldsymbol{D}}
\newcommand{\bmE}{\boldsymbol{E}}	\newcommand{\bmF}{\boldsymbol{F}}
\newcommand{\bmG}{\boldsymbol{G}}	\newcommand{\bmH}{\boldsymbol{H}}
\newcommand{\bmI}{\boldsymbol{I}}	\newcommand{\bmJ}{\boldsymbol{J}}
\newcommand{\bmK}{\boldsymbol{K}}	\newcommand{\bmL}{\boldsymbol{L}}
\newcommand{\bmM}{\boldsymbol{M}}	\newcommand{\bmN}{\boldsymbol{N}}
\newcommand{\bmO}{\boldsymbol{O}}	\newcommand{\bmP}{\boldsymbol{P}}
\newcommand{\bmQ}{\boldsymbol{Q}}	\newcommand{\bmR}{\boldsymbol{R}}
\newcommand{\bmS}{\boldsymbol{S}}	\newcommand{\bmT}{\boldsymbol{T}}
\newcommand{\bmU}{\boldsymbol{U}}	\newcommand{\bmV}{\boldsymbol{V}}
\newcommand{\bmW}{\boldsymbol{W}}	\newcommand{\bmX}{\boldsymbol{X}}
\newcommand{\bmY}{\boldsymbol{Y}}	\newcommand{\bmZ}{\boldsymbol{Z}}
%Small Letters
\newcommand{\bma}{\boldsymbol{a}}	\newcommand{\bmb}{\boldsymbol{b}}
\newcommand{\bmc}{\boldsymbol{c}}	\newcommand{\bmd}{\boldsymbol{d}}
\newcommand{\bme}{\boldsymbol{e}}	\newcommand{\bmf}{\boldsymbol{f}}
\newcommand{\bmg}{\boldsymbol{g}}	\newcommand{\bmh}{\boldsymbol{h}}
\newcommand{\bmi}{\boldsymbol{i}}	\newcommand{\bmj}{\boldsymbol{j}}
\newcommand{\bmk}{\boldsymbol{k}}	\newcommand{\bml}{\boldsymbol{l}}
\newcommand{\bmm}{\boldsymbol{m}}	\newcommand{\bmn}{\boldsymbol{n}}
\newcommand{\bmo}{\boldsymbol{o}}	\newcommand{\bmp}{\boldsymbol{p}}
\newcommand{\bmq}{\boldsymbol{q}}	\newcommand{\bmr}{\boldsymbol{r}}
\newcommand{\bms}{\boldsymbol{s}}	\newcommand{\bmt}{\boldsymbol{t}}
\newcommand{\bmu}{\boldsymbol{u}}	\newcommand{\bmv}{\boldsymbol{v}}
\newcommand{\bmw}{\boldsymbol{w}}	\newcommand{\bmx}{\boldsymbol{x}}
\newcommand{\bmy}{\boldsymbol{y}}	\newcommand{\bmz}{\boldsymbol{z}}

%---------------------------------------
% Scr Math Fonts :-
%---------------------------------------

\newcommand{\sA}{{\mathscr{A}}}   \newcommand{\sB}{{\mathscr{B}}}
\newcommand{\sC}{{\mathscr{C}}}   \newcommand{\sD}{{\mathscr{D}}}
\newcommand{\sE}{{\mathscr{E}}}   \newcommand{\sF}{{\mathscr{F}}}
\newcommand{\sG}{{\mathscr{G}}}   \newcommand{\sH}{{\mathscr{H}}}
\newcommand{\sI}{{\mathscr{I}}}   \newcommand{\sJ}{{\mathscr{J}}}
\newcommand{\sK}{{\mathscr{K}}}   \newcommand{\sL}{{\mathscr{L}}}
\newcommand{\sM}{{\mathscr{M}}}   \newcommand{\sN}{{\mathscr{N}}}
\newcommand{\sO}{{\mathscr{O}}}   \newcommand{\sP}{{\mathscr{P}}}
\newcommand{\sQ}{{\mathscr{Q}}}   \newcommand{\sR}{{\mathscr{R}}}
\newcommand{\sS}{{\mathscr{S}}}   \newcommand{\sT}{{\mathscr{T}}}
\newcommand{\sU}{{\mathscr{U}}}   \newcommand{\sV}{{\mathscr{V}}}
\newcommand{\sW}{{\mathscr{W}}}   \newcommand{\sX}{{\mathscr{X}}}
\newcommand{\sY}{{\mathscr{Y}}}   \newcommand{\sZ}{{\mathscr{Z}}}


%---------------------------------------
% Math Fraktur Font
%---------------------------------------

%Captital Letters
\newcommand{\mfA}{\mathfrak{A}}	\newcommand{\mfB}{\mathfrak{B}}
\newcommand{\mfC}{\mathfrak{C}}	\newcommand{\mfD}{\mathfrak{D}}
\newcommand{\mfE}{\mathfrak{E}}	\newcommand{\mfF}{\mathfrak{F}}
\newcommand{\mfG}{\mathfrak{G}}	\newcommand{\mfH}{\mathfrak{H}}
\newcommand{\mfI}{\mathfrak{I}}	\newcommand{\mfJ}{\mathfrak{J}}
\newcommand{\mfK}{\mathfrak{K}}	\newcommand{\mfL}{\mathfrak{L}}
\newcommand{\mfM}{\mathfrak{M}}	\newcommand{\mfN}{\mathfrak{N}}
\newcommand{\mfO}{\mathfrak{O}}	\newcommand{\mfP}{\mathfrak{P}}
\newcommand{\mfQ}{\mathfrak{Q}}	\newcommand{\mfR}{\mathfrak{R}}
\newcommand{\mfS}{\mathfrak{S}}	\newcommand{\mfT}{\mathfrak{T}}
\newcommand{\mfU}{\mathfrak{U}}	\newcommand{\mfV}{\mathfrak{V}}
\newcommand{\mfW}{\mathfrak{W}}	\newcommand{\mfX}{\mathfrak{X}}
\newcommand{\mfY}{\mathfrak{Y}}	\newcommand{\mfZ}{\mathfrak{Z}}
%Small Letters
\newcommand{\mfa}{\mathfrak{a}}	\newcommand{\mfb}{\mathfrak{b}}
\newcommand{\mfc}{\mathfrak{c}}	\newcommand{\mfd}{\mathfrak{d}}
\newcommand{\mfe}{\mathfrak{e}}	\newcommand{\mff}{\mathfrak{f}}
\newcommand{\mfg}{\mathfrak{g}}	\newcommand{\mfh}{\mathfrak{h}}
\newcommand{\mfi}{\mathfrak{i}}	\newcommand{\mfj}{\mathfrak{j}}
\newcommand{\mfk}{\mathfrak{k}}	\newcommand{\mfl}{\mathfrak{l}}
\newcommand{\mfm}{\mathfrak{m}}	\newcommand{\mfn}{\mathfrak{n}}
\newcommand{\mfo}{\mathfrak{o}}	\newcommand{\mfp}{\mathfrak{p}}
\newcommand{\mfq}{\mathfrak{q}}	\newcommand{\mfr}{\mathfrak{r}}
\newcommand{\mfs}{\mathfrak{s}}	\newcommand{\mft}{\mathfrak{t}}
\newcommand{\mfu}{\mathfrak{u}}	\newcommand{\mfv}{\mathfrak{v}}
\newcommand{\mfw}{\mathfrak{w}}	\newcommand{\mfx}{\mathfrak{x}}
\newcommand{\mfy}{\mathfrak{y}}	\newcommand{\mfz}{\mathfrak{z}}


\newcolumntype{Y}{>{\centering\arraybackslash}X}

\title{\Huge{Magnetism}\\Semester 5}
\author{Ahmad Abu Zainab}
\date{}

\newcommand{\midlabelline}[3]{
   \node (midlabel) at ($ (#1)!.5!(#2) $) {#3};
   \draw[{latex}-,thick] (#1) --  (midlabel);
   \draw[-{latex},thick] (midlabel) -- (#2);
}

\begin{document}

\maketitle
\newpage% or \cleardoublepage
% \pdfbookmark[<level>]{<title>}{<dest>}
\pdfbookmark[section]{\contentsname}{toc}
\tableofcontents
\pagebreak

\chapter{Electromagnetic Waves}

Electromagnetic waves are waves that are created by oscillating electric and magnetic fields. The electric and magnetic fields are perpendicular to each other and to the direction of propagation of the wave.
They travel at a speed of $c=299792458$ \unit{m/s} in a vacuum. The frequency of the wave is given by $f=\frac{c}{\lambda}$ where $\lambda$ is the wavelength of the wave.

The wavelength is the distance between two consecutive peaks of the wave.

\begin{figure}[ht]
	\centering
	\begin{tikzpicture}[domain=-4:0]
		\draw[->] (-4.2,0) -- (0.7,0);

		\draw[smooth,thick, samples=1000, densely dashed]   plot (\x,{sin(4*\x r)});
		\draw[smooth,thick, samples=1000, domain=-3.6:0.4]   plot (\x,{sin(4*\x r - 90)});
		\draw[<->] (-3.6,-1.2) -- (-3.1,-1.2) ;
		\draw (-3.35,-1.2) node[below] {$\lambda$};
	\end{tikzpicture}
\end{figure}

The wave moves a distance $x$ in a time $t$ with a speed
\[
	v = \frac{x}{t} = \frac{\omega}{k} = \frac{\lambda}{T} = \lambda f
	.\]

\[
	\lambda = \frac{2\pi}{f}
	.\]

The equation of the wave is given by

\[
	y = A\sin \lt( \omega t - kx + \phi \rt)
	.\]

The intensity of the wave is given by

\[
	I = \frac{P}{4 \pi r^2}
	.\]

\subsection{Poynting Vector}

The Poynting vector is a vector that represents the energy flux of the wave. It is given by

\[
	\va{S} = \frac{1}{\mu_0}\va{E} \times \va{B}
	.\]

However, since $\va{B}$ are perpendicular to $\va{E}$ then

\[
	S = \frac{1}{\mu_0}EB
	.\]

and since

\[
	c = \frac{E}{B}
	.\]

then

\[
	S = \frac{c}{\mu_0}B^2 = \frac{1}{c\mu_0}E^2
	.\]

\subsection{Maxwell's Equations}

\begin{align*}
	\pdv{\va{E}}{x} & = - \pdv{\va{B}}{t}                     \\
	\pdv{\va{B}}{x} & = - \varepsilon_0 \mu_0 \pdv{\va{E}}{t}
\end{align*}

\chapter{Optics}

The equation of an EM wave is given by

\[
	\va{E} = E_0 \sin \lt( \omega t - kx \rt) \vu{\jmath}
	.\]

Noting that electric field typically have all directions, whenever an non-polarised electric field hits a polariser, only the electric field components that are along the line of polarization are allowed to pass while all others are absorbed. If the electric field direction is perpendicular to the line of polarization then it is entirely absorbed while the electric field vectors that have an angle $\theta <\pi /2$ then only the projection of that vector on the polarization line will not be absorbed.

Given a polariser with angle $ \alpha $ then the electric field $\va{E } $ hitting that polariser has an equation

\[
	\va{E} = \underbrace{E \sin \alpha\vu{u}_1}_{\text{transmitted}} + \underbrace{E\cos \alpha\vu{u}_2}_{\text{absorbed}}
	.\]

\subsection{Reflection}

The angle of incidence is equal to the angle of reflection.

\[
	\theta_i = \theta_r
	.\]

\begin{figure}[H]
	\centering
	\begin{tikzpicture}
		\draw[->,thin] (-4,0) -- (4,0);
		\draw[->,thin] (0,0) -- (0,4);
		\draw[-{Latex},thick] (0,0) -- (2,2) node[above] {$\va{E}_i$};
		\draw[-{Latex},thick] (-2,2) node[above] {$\va{E}_r$} -- (0,0);
		\draw (0,0.5) arc (90:135:0.5) node[midway,above] {$\theta_i$};
		\draw (0,0.5) arc (90:45:0.5) node[midway,above] {$\theta_r$};
	\end{tikzpicture}
\end{figure}


\subsection{Refraction}
Snell's law
\[
	n_1\sin \theta_1 = n_2\sin \theta_2
	.\]

\begin{figure}[H]
	\centering
	\begin{tikzpicture}
		\draw[thin] (-3,0) -- (3,0);
		\draw[thin, dashed] (0,-2.7) -- (0,2.7);
		\draw[-{Latex},thick] (-2,2) -- (0,0);
		\draw[-{Latex},thick] (0,0) -- (1,-3);
		% Draw the angle 
		\draw (0,0.6) arc (90:135:0.6) node[midway,above] {$\theta_1$};
		\draw (0,-0.6) arc (-90:-73:0.6) node[midway,below] {$\theta_2$};
		\draw (3,1) node[above] {$n_1$};
		\draw (3,-1) node[above] {$n_2$};
	\end{tikzpicture}
\end{figure}

This implies that if $n_1 < n_2$ then $\theta_1 > \theta_2$.


\dfn{Critical Angle}{
	The critical angle the maximum angle of refraction where the refracted ray goes back in the incident medium

	\[
		\theta_c = \sin^{-1} \lt( \frac{n_2}{n_1} \rt)
		.\]
	\begin{figure}[H]
		\centering
		\begin{tikzpicture}
			\begin{scope}[xshift=-3cm, scale=0.75]
				\draw[thin] (-3,0) -- (3,0);
				\draw[thin, dashed] (0,-2.7) -- (0,2.7);

				\draw[-{Latex},thick] (-1.5,-1.5) -- (0,0);
				\draw[-{Latex},thick] (0,0) -- (2,0);

				\draw (2,1) node[above] {$\theta_1=\theta_c$};
				\draw (2,-1) node[above] {$\theta_2=\ang{90}$};

				\draw (0,-0.5) arc (-90:-135:0.5) node[midway,below] {$\theta_1$};
				\draw (0,0.5) arc (90:0:0.5) node[midway,above] {$\theta_2$};
			\end{scope}
			\begin{scope}[xshift=3cm, scale=0.75]
				\draw[thin] (-3,0) -- (3,0);
				\draw[thin, dashed] (0,-2.7) -- (0,2.7);

				\draw[-{Latex},thick] (-2.5,-1) -- (0,0);
				\draw[-{Latex},thick] (0,0) -- (2.5,-1);

				\draw (-2,1) node[above] {$\theta_1>\theta_c$};

				\draw (0,-0.5) arc (-90:-158.2:0.5) node[midway,below] {$\theta_1$};
				\draw (0,-0.5) arc (-90:-21.8:0.5) node[midway,below] {$\theta_2$};
			\end{scope}

		\end{tikzpicture}
	\end{figure}
}

\[
	v = \frac{c}{n}
	.\]

\begin{align*}
	\text{Speed of light in a medium} & = \frac{1}{\sqrt{\mu \varepsilon}}     \\
	\text{Speed of light in a vacuum} & = \frac{1}{\sqrt{\mu_0 \varepsilon_0}} \\
	\text{Refractive index}           & = \sqrt{\mu_r \varepsilon_r}
\end{align*}


\[
	\frac{n_1}{n_2} =\frac{v_2}{v_1}
	.\]

\section{Young's Double Slit Experiment}

\[
	d\sin \theta = m\lambda
	.\]

\[
	\Delta x = \frac{m\lambda L}{d}
	.\]

Bright fringes are where the path difference is an integer multiple of the wavelength.
\[
	x = \frac{m\lambda L}{d}
	.\]

Dark fringes are where the path difference is an odd multiple of half the wavelength.

\[
	x = \frac{\lt( 2m+1 \rt)\lambda L}{2d}
	.\]

Where $m$ is an integer, $d$ is the distance between the slits, $\lambda$ is the wavelength of the light, $L$ is the distance between the slits and the screen, and $x$ is the distance between the central bright fringe and the $m$th bright fringe.

\section{Diffraction Grating}

\[
	\sin \theta_1 = \frac{\lambda}{a}
	.\]

\[
	\sin \theta_n = n\frac{\lambda}{d}
	.\]

\chapter{Transmission Lines}

\section{The Phasor}

The phasor is a complex number that represents the amplitude and phase of a sinusoidal function. It is given by 2 forms

\[
	\text{Rectangular form} = A\cos \lt( \omega t + \varphi \rt) + jA\sin \lt( \omega t + \varphi \rt)
	.\]

\[
	\text{Polar form} = Ae^{j\lt( \omega t + \varphi \rt)} = A \phase{\omega t + \varphi}
	.\]

\section{The Role of Wavelength}

In a transmission line, the wavelength of the signal can affect the impedance of the line. The voltage at the source ($V_{AA'}$) is given by

\[
	V_{AA'} = V_0 \cos \omega t
	.\]

The voltage at the load ($V_{BB'}$) (assuming no losses across the transmission line) is given by

\[
	V_{BB'} = V_0 \cos \lt( \omega t - \varphi_0 \rt)
	.\]

\[
	\varphi_0 = \frac{\omega l}{c}
	.\]

\section{TL lumped element model}

We can model a transmission line using a series of resistors, inductors, and capacitors. The model is given by 4 parameters

\begin{align*}
	R' & = \text{Resistance per unit length } [\unit{\ohm\per\meter}]      \\
	L' & = \text{Inductance per unit length } [\unit{\henry\per\meter}]    \\
	C' & = \text{Capacitance per unit length } [\unit{\farad\per\meter}]   \\
	G' & = \text{Conductance per unit length } [\unit{\siemens\per\meter}]
\end{align*}

\begin{figure}[H]
	\centering
	\begin{circuitikz}[american]
		\draw (-1,0) node[above] {$N$} to[R=$R' \Delta z$,i>_=$i(z{,}t)$, o-] (2,0) to[L=$L' \Delta z$] (4,0) node[above] {$N+1$} to[short,i=$i(z+\Delta z {,} t)$, -o] (8,0);
		\draw (4,0) to[short, *-*] (4,-1);
		\draw (4,-1) to[short] (3.5,-1) to[C,l_=$C' \Delta z$] (3.5,-3) to[short, -*] (4,-3);
		\draw (4,-1) to[short] (4.5,-1) to[R=$G' \Delta z$] (4.5,-3) to[short, -*] (4,-3);
		\draw (4,-3) to[short, *-*] (4,-4);
		\draw (-1,-4) to[short, o-o] (8,-4);

		% \draw[<->] (-1,-0.5) -- node {$v(z{,}t)$} -- (-1,-3.5);
		\midlabelline{-1,-0.5}{-1,-3.5}{$v(z{,}t)$}
		\midlabelline{8,-0.5}{8,-3.5}{$v(z + \Delta z{,}t)$}
		\node[left] at (-1,-0.5) {$+$};
		\node[left] at (-1,-3.5) {$-$};

		\node[right] at (8,-0.5) {$+$};
		\node[right] at (8,-3.5) {$-$};

		\midlabelline{-1,-4.25}{8,-4.25}{$\Delta z$}
	\end{circuitikz}
\end{figure}

{
\renewcommand{\arraystretch}{2.5}
\everymath{\displaystyle}
\begin{table}[H]
	\centering
	\begin{tabularx}{\linewidth}{|c|YYYc|}
		\hline
		\textbf{Parameter} & \textbf{Coaxial}                                       & \textbf{Two-Wire}                                             & \textbf{Parallel-Plate}   & \textbf{Unit}               \\
		\hline
		$R'$               & $\frac{R_s}{2\pi} \lt( \frac{1}{a} + \frac{1}{b} \rt)$ & $\frac{2R_s}{\pi d}$                                          & $\frac{2R_s}{w}$          & $\unit{\ohm\per\meter}$     \\
		$L'$               & $\frac{\mu}{2\pi}\ln b/a $                             & $\frac{\mu}{\pi}\ln\lt[ (D/d)+\sqrt{(D/d)^2-1} \rt]$          & $\frac{\mu h}{w}$         & $\unit{\henry\per\meter}$   \\
		$G'$               & $\frac{2\pi \sigma}{\ln b/a}$                          & $\frac{\pi \sigma}{\ln\lt[ (D/d)+\sqrt{(D/d)^2-1} \rt]}$      & $\frac{\sigma w}{h}$      & $\unit{\siemens\per\meter}$ \\
		$C'$               & $\frac{2\pi \varepsilon}{\ln b/a}$                     & $\frac{\pi \varepsilon}{\ln\lt[ (D/d)+\sqrt{(D/d)^2-1} \rt]}$ & $\frac{\varepsilon w}{h}$ & $\unit{\farad\per\meter}$   \\
		\hline
	\end{tabularx}
\end{table}
}
Where $\mu$, $\varepsilon$, and $\sigma$ are the permeability, permittivity, and conductivity of the medium respectively. $R_s=\sqrt{\pi f \mu_c/\sigma_c}$ is the resistance of the conductor, $a$ is the radius of the inner conductor, $b$ is the radius of the outer conductor, $D$ is the distance between the two wires, $d$ is the diameter of the wire, $w$ is the width of the plate, and $h$ is the height of the plate.\\

\section{The Telegrapher's Equations}

The wave equation is given by

\[
	\dv[2]{\tilde{V}(z)}{z} - \lt( R' + j\omega L' \rt) \lt( G' + j\omega C' \rt) \tilde{V}(z) = 0
	.\]

Thus, we define the complex propagation constant $\gamma$ as
\[
	\gamma = \alpha + j\beta
	.\]

Where $\alpha$ is the attenuation constant and $\beta$ is the phase constant.

\begin{align*}
	\alpha & = \Re \lt( \gamma \rt)\quad [\unit{\neper\per\meter}]  \\
	\beta  & = \Im \lt( \gamma \rt)\quad [\unit{\radian\per\meter}]
\end{align*}

In uncoupled form, the telegrapher's equations are given by

\begin{align*}
	\dv[2]{\tilde{V}(z)}{z} - \gamma^2 \tilde{V}(z) & = 0 \\
	\dv[2]{\tilde{I}(z)}{z} - \gamma^2 \tilde{I}(z) & = 0
\end{align*}

The general solution to the wave equation is given by

\begin{align*}
	\tilde{V}(z) & = {V_0}^+ e^{-\gamma z} + {V_0}^- e^{\gamma z} \\
	\tilde{I}(z) & = {I_0}^+ e^{-\gamma z} + {I_0}^- e^{\gamma z}
\end{align*}

Where ${V_0}^+$ and ${I_0}^+$ are the incident voltage and current respectively and ${V_0}^-$ and ${I_0}^-$ are the reflected voltage and current respectively.\\

The characteristic impedance of the transmission line is given by

\[
	Z_0 = \frac{R' + j\omega L'}{\gamma} = \sqrt{\frac{R' + j\omega L'}{G' + j\omega C'}} \quad [\unit{\ohm}]
	.\]

\section{Lossless Transmission Line}

In a lossless transmission line, the resistance and conductance per unit length are zero. I.E. $R'\ll \omega L'$ and $G'\ll \omega C'$. Thus, the characteristic impedance is given by

\begin{align*}
	\alpha  & = 0                                                                                                               \\
	\beta   & = \omega \sqrt{L'C'} = \omega \sqrt{\mu\varepsilon}                                                               \\
	Z_0     & = \sqrt{\frac{L'}{C'}}                                                                                            \\
	\lambda & = \frac{2\pi}{\beta} = \frac{2\pi}{\omega \sqrt{L'C'}}                                                            \\
	u_p     & = \frac{\omega}{\beta} = \frac{1}{\sqrt{L'C'}} = \frac{1}{\sqrt{\mu\varepsilon}} \quad [\unit{\meter\per\second}]
\end{align*}

\section{Voltage Reflection Coefficient}

The voltage reflection coefficient is given by

\begin{align*}
	\Gamma & = \frac{{V_0}^-}{{V_0}^+}     \\
	       & = \frac{Z_L - Z_0}{Z_L + Z_0} \\
	       & = \frac{z_L -1}{z_L + 1}
\end{align*}

Where $z_L = Z_L/Z_0$ is the load impedance in terms of the characteristic impedance.\\

We represent the voltage and current on the transmission line as

\[
	{V_0}^- = \Gamma {V_0}^+
	.\]

\begin{align*}
	\tilde{V}(z) & = {V_0}^+ \lt( e^{-j\beta z} + \Gamma e^{j\beta z} \rt)             \\
	\tilde{I}(z) & = \frac{{V_0}^+}{Z_0} \lt( e^{-j\beta z} - \Gamma e^{j\beta z} \rt)
\end{align*}

And their magnitudes are given by

\begin{align*}
	\abs{\tilde{V}(d)} & = \abs{{V_0}^+} \lt[ 1 + \abs{\Gamma}^2 + 2 \abs{\Gamma} \cos(2\beta d - \theta_r) \rt]^{1/2}             \\
	\abs{\tilde{I}(d)} & = \frac{\abs{{V_0}^+}}{Z_0} \lt[ 1 + \abs{\Gamma}^2 - 2 \abs{\Gamma} \cos(2\beta d - \theta_r) \rt]^{1/2}
\end{align*}

where $d=-z$

\section{Special Line Conditions}

\begin{enumerate}
	\ii Matched Line: $Z_L = Z_0 \Rightarrow \Gamma = 0$ $V_\text{ref} = 0$
	\ii Open Circuit: $Z_L = \infty \Rightarrow \Gamma = 1$ $V_\text{ref} = V_\text{inc}$
	\ii Short Circuit: $Z_L = 0 \Rightarrow \Gamma = -1$ $V_\text{ref} = -V_\text{inc}$
\end{enumerate}

{
\renewcommand{\arraystretch}{2.5}
\everymath{\displaystyle}
\begin{table}[H]
	\centering
	\begin{tabularx}{0.8\linewidth}{|c|YY|}
		\hline
		\textbf{Load}              & $\bm{|\Gamma|}$                                       & $\bm{\theta_r}$                                                          \\
		\hline
		$Z_L = (r + jx)Z_0$        & $\lt[ \frac{(r-1)^2 + x^2}{(r+1)^2 + x^2} \rt]^{1/2}$ & $\tan^{-1} \lt( \frac{x}{r-1} \rt) - \tan^{-1} \lt( \frac{x}{r+1} \rt) $ \\
		$Z_L = Z_0$                & 0                                                     & Irrelevant                                                               \\
		(short)                    & 1                                                     & $\pm\pi$ (phase opposite)                                                \\
		(open)                     & 1                                                     & 0                                                                        \\
		$jX = j\omega L$           & 1                                                     & $\pm\pi - 2\tan^{-1}x$                                                   \\
		$jX = -\frac{j}{\omega C}$ & 1                                                     & $\pm 2\tan^{-1}x$                                                        \\
		\hline
	\end{tabularx}
\end{table}
}

\section{Standing Waves}

Standing waves are waves that do not propagate. They are formed when two waves of the same frequency and amplitude travel in opposite directions.

\begin{figure}[H]
	\centering
	\begin{tikzpicture}[>=latex]
		\begin{scope}[yshift=0cm]
			\begin{axis}
				[
					axis x line*=bottom,
					axis y line*=right,
					y axis line style={-stealth},
					x axis line style={stealth-},
					height=4cm,
					width=10cm,
					clip=false,
					ymin=0,
					ymax=1.6,
					ytick={0,0.2,...,1.6},
					yticklabels={$0$,$0.2$,$0.4$,$0.6$,$0.8$,$1.0$,$1.2$,$1.4$,$1.6\unit{\volt}$},
					ylabel={$\abs{\tilde{V}(d)}$},
					xmin=-4.1,
					xmax=0,
					xlabel={$d$},
					xtick={-4,...,0},
					xticklabels={$\lambda$,$\frac{3}{4}\lambda$,$\frac{1}{2}\lambda$,$\frac{1}{4}\lambda$,$0$},
				]
				\addplot[blue, thick, samples=1000, domain=-4:0] {0.35*cos(deg(pi*x + 4*pi/25))+1.05};
				\addplot[red, dashed, samples=1000, domain=-4:0] {1.4} node[above, pos=0] {$\abs{\tilde{V}}_\text{max}$};
				\addplot[red, dashed, samples=1000, domain=-4:0] {0.7} node[below, pos=0] {$\abs{\tilde{V}}_\text{min}$};
				\draw[dashed,green] (axis cs:-0.16,1.4) -- (axis cs:-0.16,0) node[anchor=north east] {$d_\text{max}$};
				\draw[dashed,green] (axis cs:-1.16,0.7) -- (axis cs:-1.16,0) node[anchor=north east] {$d_\text{min}$};
			\end{axis}
		\end{scope}

		\begin{scope}[yshift=-4.5cm]
			\begin{axis}
				[
					axis x line*=bottom,
					axis y line*=right,
					y axis line style={-stealth},
					x axis line style={stealth-},
					height=4cm,
					width=10cm,
					clip=false,
					ymin=0,
					ymax=1.6,
					ytick={0,0.2,...,1.6},
					yticklabels={0,5,...,35,$40\unit{\milli\ampere}$},
					ylabel={$\abs{\tilde{I}(d)}$},
					xmin=-4.1,
					xmax=0,
					xlabel={$d$},
					xtick={-4,...,0},
					xticklabels={$\lambda$,$\frac{3}{4}\lambda$,$\frac{1}{2}\lambda$,$\frac{1}{4}\lambda$,$0$},
				]
				\addplot[blue, thick, samples=1000, domain=-4:0] {-0.35*cos(deg(pi*x + 4*pi/25))+1.05};
				\addplot[red, dashed, samples=1000, domain=-4:0] {1.4} node[above, pos=0] {$\abs{\tilde{I}}_\text{max}$};
				\addplot[red, dashed, samples=1000, domain=-4:0] {0.7} node[below, pos=0] {$\abs{\tilde{I}}_\text{min}$};
				\draw[dashed,green] (axis cs:-1.16,1.4) -- (axis cs:-1.16,0) node[anchor=north east] {$d_\text{max}$};
				\draw[dashed,green] (axis cs:-0.16,0.7) -- (axis cs:-0.16,0) node[anchor=north east] {$d_\text{min}$};
			\end{axis}
		\end{scope}

	\end{tikzpicture}
\end{figure}

The figure above, shows the voltage and current on a transmission line. The voltage and current are in phase opposition. This general graph applies for all values of $\Gamma$ with the exception of 3 lines conditions. \\

\begin{figure}[H]
	\centering
	\begin{tikzpicture}
		\begin{scope}[yshift=0cm]
			\begin{axis}
				[
					axis x line*=bottom,
					axis y line*=right,
					y axis line style={-stealth},
					x axis line style={stealth-},
					height=4cm,
					width=10cm,
					clip=false,
					ymin=0,
					ymax=1.6,
					ytick={0,0.7},
					yticklabels={$0$, $\abs{{V_0}^+}$},
					ylabel={$\abs{\tilde{V}(d)}$},
					xmin=-4.1,
					xmax=0,
					xlabel={$d$},
					xtick={-4,...,0},
					xticklabels={$\lambda$,$\frac{3}{4}\lambda$,$\frac{1}{2}\lambda$,$\frac{1}{4}\lambda$,$0$},
				]
				\addplot[blue, thick, samples=1000, domain=-4:0] {0.7} node[above, red, pos=0.1] {Matched Line $Z_L = Z_0$};
			\end{axis}
		\end{scope}

		\begin{scope}[yshift=-4cm]
			\begin{axis}
				[
					axis x line*=bottom,
					axis y line*=right,
					y axis line style={-stealth},
					x axis line style={stealth-},
					height=4cm,
					width=10cm,
					clip=false,
					ymin=0,
					ymax=1.6,
					ytick={0,1.4},
					yticklabels={$0$, $2\abs{{V_0}^+}$},
					ylabel={$\abs{\tilde{V}(d)}$},
					xmin=-4.1,
					xmax=0,
					xlabel={$d$},
					xtick={-4,...,0},
					xticklabels={$\lambda$,$\frac{3}{4}\lambda$,$\frac{1}{2}\lambda$,$\frac{1}{4}\lambda$,$0$},
				]
				\addplot[blue, thick, samples=1000, domain=-4:0] {1.4 * sqrt(cos(deg(pi*x + pi)) + 1)/sqrt(2)} node[above, red, pos=0.1] {Short-circuited line $Z_L = 0$};
			\end{axis}
		\end{scope}

		\begin{scope}[yshift=-8cm]
			\begin{axis}
				[
					axis x line*=bottom,
					axis y line*=right,
					y axis line style={-stealth},
					x axis line style={stealth-},
					height=4cm,
					width=10cm,
					clip=false,
					ymin=0,
					ymax=1.6,
					ytick={0,1.4},
					yticklabels={$0$, $2\abs{{V_0}^+}$},
					ylabel={$\abs{\tilde{V}(d)}$},
					xmin=-4.1,
					xmax=0,
					xlabel={$d$},
					xtick={-4,...,0},
					xticklabels={$\lambda$,$\frac{3}{4}\lambda$,$\frac{1}{2}\lambda$,$\frac{1}{4}\lambda$,$0$},
				]
				\addplot[blue, thick, samples=1000, domain=-4:0] {1.4 * sqrt(cos(deg(pi*x)) + 1)/sqrt(2)} node[above, red, pos=0.1] {Open-circuited line $Z_L = \infty$};
			\end{axis}
		\end{scope}

	\end{tikzpicture}
\end{figure}

We want to find the minimum and maximum values of the voltage magnitude.

\begin{align*}
	d_\text{max} & = \frac{\theta_r +2n\pi}{2\beta} = \frac{\theta_r\lambda}{4\pi} + \frac{n\lambda}{2} \\
	             & \begin{cases}
		               n=1,2,3,\ldots & \text{if } \theta_r < 0    \\
		               n=0,1,2,\ldots & \text{if } \theta_r \geq 0
	               \end{cases}
\end{align*}

\[
	d_\text{min} = \begin{cases}
		d_\text{max} + \frac{\lambda}{4} & \text{if } d_\text{max} < \lambda/4    \\
		d_\text{max} - \frac{\lambda}{4} & \text{if } d_\text{max} \geq \lambda/4
	\end{cases}
\]

The ratio of $\abs{\tilde{V}}_\text{max}$ to $\abs{\tilde{V}}_\text{min}$ is called the voltage standing wave ratio $S$ (VSWR or SWR) and is given by

\[
	S = \frac{\abs{\tilde{V}}_\text{max}}{\abs{\tilde{V}}_\text{min}} = \frac{1 + \abs{\Gamma}}{1 - \abs{\Gamma}}
	.\]

And it represents the mismatch between the load and the transmission line. For a matched load line $\Gamma=0$ and $S=1$. For an open circuit or short circuit line $\Gamma=1$ and $S=\infty$.\\

\nt{
	The average power on a transmission line is given by
	\[
		P_\text{avg} = \frac{\abs{{V_0}^+}^2}{2Z_0} \lt( 1 - \abs{\Gamma}^2 \rt)
		.\]
}

\end{document}

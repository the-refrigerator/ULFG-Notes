\part{Polar Curves}

A polar curve defined to be
\[
	\va{F}(\theta) = \mqty(\rho(\theta)\cos(\theta) \\ \rho(\theta)\sin(\theta))
	.\]

It can be defined by
\[
	\rho(\theta)
	.\]

\section{Periodicity}
\[
	\begin{array}{l|l}
		\rho(\theta+T) = \rho(\theta)         & \rho\text{ is \(T\)-periodic}                                                         \\
		\rho(-\theta) = \rho(\theta)          & (Ox)\text{ is an axis of symmetry}                                                    \\
		\rho(-\theta) = -\rho(\theta)         & (Oy)\text{ is an axis of symmetry}                                                    \\
		\rho(\pi-\theta) = \rho(\theta)       & (Oy)\text{ is an axis of symmetry}                                                    \\
		\rho(\pi-\theta) = -\rho(\theta)      & (Ox)\text{ is an axis of symmetry}                                                    \\
		\rho(\pi+\theta) = \rho(\theta)       & O\text{ is center of symmetry}                                                        \\
		\rho(\pi+\theta) = -\rho(\theta)      & \rho\text{ is \(\pi\) periodic}                                                       \\
		\rho(2\pi+\theta) = \rho(\theta)      & \rho\text{ is \(2\pi\) periodic}                                                      \\
		\rho(\theta_0-\theta) = \rho(\theta)  & \text{then } \theta = \frac{\theta_0}{2} \text{ is an axis of symmetry}               \\
		\rho(\theta_0-\theta) = -\rho(\theta) & \text{then } \theta = \frac{\theta_0}{2}+\frac{\pi}{2} \text{ is an axis of symmetry}
	\end{array}
\]
\section{Study of Tangent Points}

We define the angle $\nu$ at a point of $\theta_0$
\[
	\tan(\nu) = \lim_{\theta \to \theta_0}\frac{\rho(\theta)}{\rho'(\theta)}
	.\]

The slope of the tangent at a point $\theta_0$ to be
\[
	\tan(\varphi) = \tan(\theta_0 + \nu)
	.\]

\section{Infinite Branches}
\begin{itemize}
	\item if $\rho(\theta)\sin(\theta-\theta_0)\underset{\theta\to\theta_0}{\longrightarrow}A$ then the line $y=A$ is an oblique asymptote relative to the orthonormal system $(O,\vu{u},\vu{v})$ where $(\vu{i},\vu{u}) = \theta_0$.
	\item if $\rho(\theta)\sin(\theta-\theta_0)\underset{\theta\to\theta_0}{\longrightarrow}\pm\infty$ then the curve admits a parabolic branch of direction $\theta=\theta_0$
\end{itemize}

Cartesian equation of an asymptote in the usual system
\[
	-\sin(\theta_0)x + \cos(\theta_0)y = A
	.\]
The equation in polar form
\[
	\rho = \frac{A}{\sin(\theta - \theta_0)}
	.\]

\subsection{When $\theta\to\pm\infty$}
\begin{itemize}
	\item if $\rho(\theta)\to 0$ then the curve admits $O$ as a point asymptote (limit point).
	\item if $\rho(\theta)\to \pm\infty$ then the curve admits a spiral asymptote.
	\item if $\rho(\theta)\to R$ then the curve admits a circle asymptote of radius $R$.
\end{itemize}

\begin{remark}
	The arc length of a polar curve
	\[
		L = \int_{\theta_1}^{\theta_2}\sqrt{\left(\dv{x}{\theta}\right)^2 + \left(\dv{y}{\theta}\right)^2 }\dd{\theta} = \int_{\theta_1}^{\theta_2}\sqrt{\rho^2 + \left(\dv{\rho}{\theta}\right)^2} \dd{\theta}
		.\]
	The area under a polar curve is
	\[
		A = \int_{\theta_1}^{\theta_2} \frac{1}{2}\rho^2 \dd{\theta}
		.\]
\end{remark}

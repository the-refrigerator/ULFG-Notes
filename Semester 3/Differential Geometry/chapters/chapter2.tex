\part{Parametric Curves}

A vector function/parametric curve is a function of the form

\begin{align*}
	\va{F}: \mathbb{R} & \longrightarrow \mathbb{R}^2        \\
	t                  & \longmapsto \va{F}(t) = (x(t),y(t))
	.\end{align*}

With a domain of definition $\mathbb{D}_{\va{F}} = \mathbb{D}_x \cap \mathbb{D}_y$

\begin{remark}
	The length of a curve when $t\in[a,b]$ is
	\[
		\int^b_a\sqrt{\left( \dv{y}{t}  \right)^2 +\left( \dv{x}{t}  \right)^2 } \dd {t}
		.\]

\end{remark}

\section{Symmetry}

Consider the domain of definition to be $\mathbb{R}$.

If a function is is even($f(-x)=f(x)$) or odd($f(-x)=-f(x)$) the domain of study $ \mathbb{D}_S$ is only $[0,+\infty[$, and it is symmetric with respect to some axis.(refer to the table)\\

If a curve $x(t+T)=x(t)$ and $y(t+T)=y(t)$ then the curve is  $T$-periodic. Then the domain if study $\mathbb{D}_S = [0,T]\cap\mathbb{D}_{\va{F}}$ or $=\left[-\frac{T}{2},\frac{T}{2}\right]\cap\mathbb{D}_{\va{F}}$.\\
\begin{margintable}
	\begin{tabular}{|l||l|l|}
		\hline
		\backslashbox{$y$}{$x$} & Even     & Odd        \\
		\hline\hline
		Even                    & None     & $y$-axis   \\
		\hline
		Odd                     & $x$-axis & Center $O$ \\
		\hline
	\end{tabular}
	\caption{Axis of symmetry of $\va{F}(t)$ depending on the nature of $x$ and $y$.}
\end{margintable}

\begin{remark}
	The tangent line of a curve at $t=t_0$ is
	\[
		-y'(t_0)(x-x(t_0)) + x'(t_0)(y-y(t_0))=0
		.\]
	and the normal is
	\[
		x'(t_0)(x-x(t_0))+y'(t_0)(y-y(t_0))=0
		.\]
\end{remark}

\section{Infinite Branches}

\begin{itemize}
	\item If $\lim_{t \to t_0} x(t)= \pm \infty \qand \lim_{t \to t_0}y(t)=y_0$ then the line $y=y_0$ is a horizontal asymptote.
	\item If $\lim_{t \to t_0} x(t)= x_0 \qand \lim_{t \to t_0}y(t)=\pm \infty$ then the line $x=x_0$ is a vertical asymptote.
	\item If $\lim_{t \to t_0} x(t)= \pm \infty \qand \lim_{t \to t_0}y(t)=\pm \infty$ then we study $\frac{y(t)}{x(t)}$
	      \begin{itemize}
		      \item If $\lim_{t \to t_0}  \frac{y(t)}{x(t)}=\pm \infty$ then the curve admits a parabolic directed by $(Oy)$.
		      \item If $\lim_{t \to t_0}  \frac{y(t)}{x(t)}=0$ then the curve admits a parabolic directed by $(Ox)$.
		      \item If $\lim_{t \to t_0}  \frac{y(t)}{x(t)}=a\in \mathbb{R}^*$ then we study $y(t)-ax(t)$
		            \begin{itemize}
			            \item If $\lim_{t \to t_0} y(t)-ax(t)=b\in \mathbb{R} $ then the curve admits an oblique asymptote $y=ax+b$
			            \item If $\lim_{t \to t_0} y(t)-ax(t)=\pm \infty$ then the curve admits an asymptotic direction $y=ax$
		            \end{itemize}
	      \end{itemize}
\end{itemize}

\section{Particular Points}

A point is said to be stationary if $\va{F}'(t) = 0$, regular if $\va{F}'(t)=0$, and biregular if $\det(\va{F}'(t),\va{F}''(t))\neq 0$.\\

The first non zero vector in the set $\{\va{F}'(t),\va{F}''(t),\va{F}'''(t),\ldots,\va{F}^{(k)}(t)\} $ is $\va{F}^{(p)}$ is used to define the tangent vector to the curve
\[
	\va{T}(t) = \frac{\va{F}^{(p)}(t)}{\left\| \va{F}^{(p)}(t) \right\| }
	.\]

\[
	(T):y=\frac{y^{(p)}(t)}{x^{(p)}(t)}(x-x(t)) + y(t)
	.\]
\begin{remark}
	\begin{itemize}
		\item $\va{F}'(t_0)=0\implies t=t_0$ is a stationary point (reflection point of 1/2 kind).
		\item $\va{F}'(t_0) \neq  0\implies t=t_0$ is an inflection point or normal shape point.
		\item $\det(\va{F}'(t_0),\va{F}''(t_0))=0\implies t=t_0$ is a reflection or inflection point (not biregular).
	\end{itemize}
\end{remark}
\newpage
\begin{table}[h]
	\centering
	\begin{tabular}{|c|c|c|}
		\hline
		\backslashbox{$q$}{$p$} & Even                       & Odd \\
		\hline\hline
		Even                    &
		\begin{tikzpicture}[>=latex]
			\begin{axis}[
					axis x line=center,
					axis y line=center,
					xtick={-3,-2,...,3},
					ytick={-3,-2,...,3},
					xlabel={$x$},
					ylabel={$y$},
					xlabel style={below right},
					ylabel style={above left},
					xmin=-3.5,
					xmax=3.5,
					ymin=-3.5,
					ymax=3.5]
				\addplot [mark=none,samples=100,domain=0:3.5] {x^2};
				\addplot [mark=none,samples=100,domain=0:3.5] {0.5*x^2};
				\filldraw[red] (0,0) circle (2pt);
			\end{axis}
			\draw (3,-1) node {Reflection point of the second kind};
		\end{tikzpicture}
		                        & \begin{tikzpicture}[>=latex]
			\begin{axis}[
					axis x line=center,
					axis y line=center,
					xtick={-3,-2,...,3},
					ytick={-3,-2,...,3},
					xlabel={$x$},
					ylabel={$y$},
					xlabel style={below right},
					ylabel style={above left},
					xmin=-3.5,
					xmax=3.5,
					ymin=-3.5,
					ymax=3.5]
				\addplot [mark=none,samples=100,domain=-3.5:3.5] {0.1*x^2};
				\filldraw[red] (0,0) circle (2pt);
			\end{axis}
			\draw (3,-1) node {Normal shape point};\end{tikzpicture}       \\
		\hline
		Odd                     &
		\begin{tikzpicture}[>=latex]
			\begin{axis}[
					axis x line=center,
					axis y line=center,
					xtick={-3,-2,...,3},
					ytick={-3,-2,...,3},
					xlabel={$x$},
					ylabel={$y$},
					xlabel style={below right},
					ylabel style={above left},
					xmin=-3.5,
					xmax=3.5,
					ymin=-3.5,
					ymax=3.5]
				\addplot [mark=none,samples=100,domain=-3.5:3.5] ({x^2},{x^2-x^3});
				\filldraw[red] (0,0) circle (2pt);
			\end{axis}
			\draw (3,-1) node {Reflection point of the first kind};\end{tikzpicture}
		                        & \begin{tikzpicture}[>=latex]
			\begin{axis}[
					axis x line=center,
					axis y line=center,
					xtick={-3,-2,...,3},
					ytick={-3,-2,...,3},
					xlabel={$x$},
					ylabel={$y$},
					xlabel style={below right},
					ylabel style={above left},
					xmin=-3.5,
					xmax=3.5,
					ymin=-3.5,
					ymax=3.5]
				\addplot [mark=none,samples=100,domain=-3.5:3.5] {0.05*x^3};
				\filldraw[red] (0,0) circle (2pt);
			\end{axis}
			\draw (3,-1) node {Inflection point};\end{tikzpicture}       \\
		\hline
	\end{tabular}
\end{table}




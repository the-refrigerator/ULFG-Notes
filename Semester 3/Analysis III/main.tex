\documentclass[a4paper,12pt]{article}

\usepackage{textcomp}
\usepackage{amsmath, amssymb}
\usepackage{bm}
\usepackage{relsize}
\usepackage{parskip}
\usepackage{tasks}
\usepackage{graphicx}
\usepackage{{../NotesTeX}}
\everymath{\displaystyle}

\DeclareRobustCommand{\hamboola}{{\ensuremath{%
  \mathchoice{\includegraphics[height=2ex]{hamboola.png}}
    {\includegraphics[height=2ex]{hamboola.png}}
    {\includegraphics[height=1.5ex]{hamboola.png}}
{\includegraphics[height=1ex]{hamboola.png}}}}}
\newcommand{\La}[1]{\mathcal{L} \left\{ #1 \right\}}
\newcommand{\D}[2][]{\mathrm{D}^{#1}\,#2\,}
\renewcommand{\P}{\mathbb{L}} % This was originally P, hence the command name, but then I decided to change it's name. The set is usually denoted by just a non-mathbb R but I decided on L to avoid confusion.(and the course uses L)

\title{Analysis 3}
\date{}

\begin{document}
\setcounter{figure}{1}
\maketitle
\graphicspath{{./figures}}

\part{Conics and Quadrics}

\section{Conics}
We define a quadric form to be a mapping $q$
\begin{align*}
	q: \mathbb{R}^n & \longrightarrow \mathbb{R}                                                                                  \\
	\va{u}          & \longmapsto q(\va{u}) = \mqty[\rule{5mm}{0.4pt} & ^{\mathsmaller T}\va{u} & \rule{5mm}{0.4pt}]A\mqty[\vline \\\va{u}\\\vline]
	.\end{align*}
Where the matrix $A$ is a symmetric matrix.\mn{symmetric matrices ($A=^{\mathsmaller T}A$) is always diagonalizable}

The conics under study are
\begin{align*}
	 & \frac{x^2}{a^2}+\frac{y^2}{b^2}=1 \quad & \text{ellipse (circle if \(a=b\))}                \\
	 & \frac{x^2}{a^2}+\frac{y^2}{b^2}=-1      & \text{imaginary ellipse}                          \\
	 & \frac{x^2}{a^2}-\frac{y^2}{b^2}=\pm1    & \text{hyperbola with asymptote }y=\pm\frac{b}{a}x \\
	 & \begin{rcases}
		y^2=\pm 2px \quad p>0\\
		x^2=\pm 2py \quad p>0
	\end{rcases}               & \text{parabolas}                                  \\
	 & \frac{x^2}{a^2}-\frac{y^2}{b^2}=0       & \text{union of two straight lines}                \\
	 & \begin{rcases}
		x=\text{const}\\
		y=\text{const}
	\end{rcases}               & \text{straight lines}
\end{align*}

\subsection{Identification of the conics}
Let the general equation of all conics be:
\[
	\Gamma: ax^2+2bxy+cy^2+2dx+2ey+f=0
	.\]
\begin{itemize}
	\item \boxed{\text{if } b=0}: then we simply group together the terms $x^2$ and $x$ as well as $y^2$ and $y$ followed by completing the square to get an equation of a conic.
	\item \boxed{\text{if } b\neq 0}: in this case we have to introduce a new system of reference which eliminates the existence of $xy$\\
	      We do this by first defining a quadratic form $q(x,y)=ax^2+2bxy+cy^2$ using a matrix
	      \[
		      q(x,y) = \mqty(x&y)\mqty(a&b\\b&c)\mqty(x\\y)
		      .\]
	      which we diagonalize in to an or tho normal age-basis which we project our equation in to in order to get rid of the $xy$ term\\

\end{itemize}
\begin{example}

	Find the nature of the conic
	\[
		\Gamma: 5x^2-4xy+8y^2+\frac{20}{\sqrt{5} }x-\frac{80}{\sqrt{5} }y+4=0
		.\]
	\\
	Let $q(x,y)=5x^2-4xy+8y^2 = \smqty(x&y)\smqty(5&-2\\-2&8)\smqty(x\\y)=^{\mathsmaller T}\va{u}A\va{u}$. We find that the matrix $A$ has eigenvalues  $\lambda_1=4$ and $\lambda_2=9$ with eigenvalues $\va{u}_1=\smqty(2\\1)$ and $\va{u}_2=\smqty(1\\-2)$, the age vectors are already orthogonal so we just find $\va{e}_1=\frac{1}{\sqrt{5}} \smqty(2\\1) $ and $\va{e}_2=\frac{1}{\sqrt{5} }\smqty(1\\-2)$, finally
	\[
		P=\mqty(\frac{2}{\sqrt{5} }&-\frac{1}{\sqrt{5} }\\\frac{1}{\sqrt{5} }&\frac{2}{\sqrt{5} })\quad D=\mqty(\dmat{4,9})
		.\]
	We define $\smqty(\alpha\\\beta)$ to be any vector with basis $\{\va{e}_1,\va{e}_2\} $
	\[
		\mqty(x\\y) = P \mqty(\alpha\\\beta)
		.\]

	\begin{align*}
		x & =\frac{2}{\sqrt{5} }\alpha-\frac{1}{\sqrt{5} }\beta = \frac{1}{\sqrt{5} }(2\alpha-\beta) \\
		y & =\frac{1}{\sqrt{5} }\alpha+\frac{2}{\sqrt{5} }\beta = \frac{1}{\sqrt{5} }(\alpha+2\beta)
	\end{align*}

	now we substitute $x$ and $y$ with $\alpha$ and $\beta$ into $\Gamma$ and we manipulate the expression until we get
	\[
		\frac{(x-1)^2}{9}+\frac{(y-2)^2}{4}=1
		.\]

	$\therefore$ $\Gamma$ is an ellipse.
\end{example}

\subsection{Tangent to a conic at point $B$ }
\begin{theorem}
	The normal to vector to a conic $\Gamma$
	\[
		\Gamma: ax^2+2bxy+cy^2+2dx+2ey+f=0
		.\]
	at a point $B\in\Gamma$ is defined to be
	\[
		\nabla  f(B) = \mqty(\pdv{f}{x} \Big|_{(x_B,y_B)}\\\pdv{f}{y} \Big|_{(x_B,y_B)})
		.\]
	where $f(x,y)=ax^2+2bxy+cy^2+2dx+2ey+f=0
	$
\end{theorem}

The equation of a tangent to a conic at a point $B$ is
\[
	a(x-x_B)+b(y-y_B)=0
	.\]
where $a$ and $b$ are respectively the $x$ and $y$ components of the normal vector at $B$


\section{Quadrics}

\begin{definition}
	A quadric is any surface in 3D space with an equation of the form:
	\[
		\underbrace{ax^2+by^2+cz^2+2dyz+2exy+2fxy}_{q(x,y,z):\text{quadratic form of 3 variables}}+\underbrace{gx+hy+iz}_{\text{linear part}}+\underbrace{j}_{\text{constant}}=0
		.\]
\end{definition}
The quadrics under study are\mn{if $a=b$ the surface is a surface of revolution of axis $(Oz)$}
\begin{align*}
	 & \frac{x^2}{a^2}+\frac{y^2}{b^2}+\frac{z^2}{c^2}=1   & \text{Ellipsoid}                \\
	 & \frac{x^2}{a^2}+\frac{y^2}{b^2}-\frac{z^2}{c^2}=1   & \text{Hyperboliod of one sheet} \\
	 & \frac{x^2}{a^2} +\frac{y^2}{b^2}-\frac{z^2}{c^2}=-1 & \text{Hyperboliod of 2 sheets}  \\
	 & \frac{x^2}{a^2}+\frac{y^2}{b^2}-\frac{z^2}{c^2}=0   & \text{Asymptote cone}           \\
	 & \frac{x^2}{a^2}-\frac{y^2}{b^2}=2pz                 & \text{Hyperbolic paraboloid}    \\
	 & \frac{x^2}{a^2}+\frac{y^2}{b^2}=z^2                 & \text{Elliptic cone}
\end{align*}

If a one of variables is missing in the equation then the surface is said to be "(Conic name)-ic Cylinder". For example "Hyperbolic cylinder", "Circular cylinder", and "Elliptical cylinder"

\part{Series of Functions}

\begin{definition}
    Let $f_n(x)$ be sequence of functions defined on $I\subset\mathbb{R}$, we define the series $S(x)$ to be 
    \[
    S(x)=\sum_{n=0}^{\infty} f_n(x)
    .\] 
\end{definition}

\section{Reminder: Convergence of a Series}
In order to prove a series of functions converge we have to prove that it converges for all fixed $x$.
\begin{theorem}
    Suppose there exists a sequence $a_n$ such that $\forall x,n \; |f_n|\le a_n$. The Weierstrass test states that if $\sum a_n$ converges then $\sum f_n(x)$ converges uniformly and absolutely
\end{theorem}

\begin{theorem}
    Let $a_n$ be a sequence of numbers, if $\left| \frac{a_{n+1}}{a_n} \right| =l$ then the sequence is a geometric Series
    \[
        \sum_{n=0}^\infty a_n \begin{cases} \text{converges}&\text{if } |l|<1\\
        \text{diverges}&\text{if }|l|\ge 1\end{cases} 
    .\] 
\end{theorem}

\begin{theorem}
    A harmonic series is defined to be $a_n=\frac{1}{n^p}$
    \[
        \sum_{n=0}^\infty \frac{1}{n^p} \begin{cases} \text{converges}&\text{if } p>1\\
        \text{diverges}&\text{if }p\le 1\end{cases} 
    .\] 

\end{theorem}


\begin{theorem}
    Let $a_n$ be a sequence of numbers. The 2 series $\sum_{n=0}^{\infty} a_n$ and $\sum_{n=0}^{\infty} 2^na_n$ are simultaneously convergent/divergent.
\end{theorem}


\begin{theorem}
    The sequence $\sum_{n=0}^{\infty} (-1)^na_n$ is convergent if $a_n$ is decreasing and $\lim_{n \to \infty}a_n =0$.
\end{theorem}

\begin{theorem}
    Consider the series $S=\sum_{n=0}^{\infty} a_n$
    \[
        \lim_{n \to \infty}\sqrt[n]{|a_n|} = l\qq{such that} \begin{cases}l<1& \text{if $S$ converges}\\
        l>1& \text{if $S$ diverges}\\
l=1& \text{this test cannot help us}\\
\end{cases} 
.\]
        \end{theorem}
\section{Finite Expansion}
The general formula for the finite expansion (Taylor-young formula) is 

\[
    f(x) = f(x-a)+ \frac{x}{1!}f'(x-a)+\frac{x^2}{2!}f''(x-a)+\cdots+\frac{x^n}{n!}f^{(n)}(x-a)+x^no(1) \quad x\to a
.\]
 
Some important expansions to keep in mind are
\[\renewcommand{\arraystretch}{1.75}
\begin{array}{l|l|l}
    e^x=\sum_{n=0}^{\infty} \frac{x^n}{n!}\quad & \sin(x)=\sum_{n=0}^{\infty} (-1)^n \frac{x^{2n+1}}{(2n+1)!}&\cos(x)=\sum_{n=0}^{\infty} (-1)^n \frac{x^{2n}}{(2n)!}\\
            \frac{1}{1-x}=\sum_{n=0}^{\infty} x^n&\sinh(x)=\sum_{n=0}^{\infty} \frac{x^{2n+1}}{(2n+1)!}&\cosh(x)=\sum_{n=0}^{\infty} \frac{x^{2n}}{(2n)!}\\
            \ln(1+x)=\sum_{n=0}^{\infty} (-1)^{n-1}\frac{x^n}{n}&\ln(1-x)=\sum_{n=0}^{\infty} \frac{x^n}{n}&(1+x)^\alpha=\sum_{n=0}^{\infty} \frac{\prod_{k=0}^{n+1}(\alpha-k)  }{n!}x^n
\end{array}
\]

\part{Power Series}

A power series is just a series in the following formula
\begin{align*}
	f(x) & =\sum_{n=0}^{\infty} a_n(x-x_0)^n \\
	     & =\sum_{n=0}^{\infty} U_n
	.\end{align*}

\section{Radius of Convergence}

For some values of $x$ a power series can either diverge or converge, to determine the interval of convergence we employ the ratio test

\begin{theorem}
	Let $r$ be the radius of convergence and $I$ be the domain of convergence. If we compute the limit
	\[
		\Gamma = \lim_{n \to \infty}\left| \frac{U_{n+1}}{U_n} \right|
		.\]
	The ratio test states that
	\[
		\begin{cases}
			\qif* \Gamma = 0      & \qthen* r=\infty\qand I=\mathbb{R} \\
			\qif* \Gamma = \infty & \qthen* r=0\qand I=\{0\}           \\
		\end{cases}
		.\]
	In the case that $\Gamma$ isn't 0 or $\infty$ we set $\Gamma<1$ and then we find $|x|<R$, finally we can say that $r=R$ and $I=]-R,R[$. A special case need to be done for the points  $-R$ and $R$ to determine if they belong in  $I$.
\end{theorem}

\begin{remark}
	The power series $f(x)$ is continous and will always uniformly converge in the interval of convergence $I$
\end{remark}

\begin{theorem}

	If $\sum a_n x^n$ and $\sum b_n x^n$ be 2 power series with radii  $R_1$ and $R_2$. For the power series $\sum (a_n+b_n)x^n$ the radius of convergence $R$
	\[
		R=\min \{R_1,R_2\}
		.\]
\end{theorem}


\begin{theorem}
	If $S(x)=\sum_{n=0}^{\infty} a_n(x-x_0)^n$ is a power series with radius $R$ then $S'(x)=\sum_{n=0}^{\infty} na_n(x-x_0)^n$ as well as $\int_{{x_0}}^{{x}} {S(t)} \: d{t}$ both a radius of $R$
\end{theorem}

The general term $a_k$ of a power series $S(x)=\sum_{n=0}^{\infty} a_n(x-x_0)^n$ is equal to
\[
	a_k = \frac{S^{(k)}(x_0)}{k!}
	.\]

\begin{align*}
    y &= a_0+a_1x+a_2x^2+\cdots+a_nx^n\\
    y' &= a_1+2a_2x+3a_3x^2+\cdots+na_nx^{n-1}\\
    y'' &= 2a_2+6a_3x+\cdots+n(n-1)a_nx^{n-2}+n(n+1)a_{n+1}x^{n-1}\\
\end{align*}

\part{Biot-Savart Law}
\[
	\va{B} = \frac{\mu_0}{4\pi}\int_{C}\frac{I \dd{\va{\ell}}\times \va{r} }{ \| \va{r} \| ^3}
	.\]

\part{Fourier Series}
\begin{definition}
	A Fourier series is a series of functions of general term
	\[
		u_n(x)=a_0 + a_n \cos(nx)+b_n\sin(nx)
		.\]
\end{definition}

\section{Trigonometric Coefficients}
\marginnote{
    \begin{enumerate}
        \item $\sin n\:\pi = 0$
        \item $\sin n\:\pi/2 = (-1)^n$ 
        \item $\cos n\:\pi = (-1)^n$
        \item $\cos n\:\pi/2=0$
    \end{enumerate}
}
The Fourier coefficients of a function defined on an interval $F \subset \mathbb{R}$ of period $T=\frac{2\pi}{\omega}\implies\omega=\frac{2\pi}{T}$ are 
\begin{align*}
    a_0 &= \frac{1}{T}\int_F f(x)\:\dd{x}\\
    a_n&=\frac{2}{T}\int_F f(x) \cos \omega nx \: \dd{x}\\
    b_n&=\frac{2}{T}\int_F f(x) \sin \omega nx\; \dd{x}
\end{align*} 

\subsection{Even Functions}
If a function has a domain $F=[-\ell;\ell]$ and $f(x)=f(-x)\;\forall x\in \mathbb{R}$ the Fourier coefficients ($T=|F|$) become 
\begin{align*}
    a_0&=\frac{1}{\ell}\int_0^\ell f(x)\:\dd{x}\\
    a_n&=\frac{2}{\ell}\int_0^\ell f(x) \cos nx\:\dd{x}\\
    b_n&=0
\end{align*}

\subsection{Odd Functions}
If a function has a domain $F=[-\ell;\ell]$ and $-f(x)=f(-x)\;\forall x\in \mathbb{R}$ the Fourier coefficients ($T=|F|$) become 
\begin{align*}
    a_0&=0\\
    a_n&=0\\
    b_n&=\frac{2}{\ell}\int_0^\ell f(x) \sin nx \: \dd {x}
\end{align*}



\part{Laplace Transforms}
The Laplace transform of a function is defined as
\[
    F(p)=\La{f(t)} = \int_0^\infty e^{-pt} f(t) \dd{t}
.\] 

It only exists if the integral above converges.\\

\section{Transforms of some functions}

\subsection{Unit step function}

Also known as Heaviside's unit step function, it is defined as 
\[
    u(t) = \begin{cases} 1 &\qif t\ge 0\\
    0&\qif t<0\end{cases}
.\] 

\[
    \La{u(t)} = \frac{1}{p} \qfor \Re(p) >0
.\] 

\subsection{Dirac Delta Function}
The Dirac Delta function 
\[
    \delta(t) = \begin{cases} \infty &\qif t= 0\\
    0&\qif t\neq 0\end{cases}
.\] 

\[
    \La{\delta(t)} = 1
.\] 
\subsection{Usual Elementary functions}

\begin{tasks}(2)
    \task $\La{1} = \frac{1}{p}$ 
    \task $\La{t} = \frac{1}{p^2}$ 
    \task $\La{t^n} = \frac{n!}{p^{n+1}}$
    \task $\La{\sin \omega t} = \frac{\omega}{p^2+\omega^2}$
    \task $\La{\cos \omega t} = \frac{p}{p^2+\omega^2}$
    \task $\La{\sinh \omega t} = \frac{\omega}{p^2-\omega^2}$
    \task $\La{\cosh \omega t} = \frac{p}{p^2-\omega^2}$
    \task $\La{e^{at}} = \frac{1}{p-a}$
\end{tasks}

\section{Properties of the Transform}

\begin{enumerate}
    \item Linearity:
        \[
            \La{\lambda f + \mu g} = \lambda \La{f} + \mu \La{g}
        .\] 
    \item Homothety:
        \[
            \La{f(kt)} = \frac{1}{k} F(\frac{p}{k})
        .\] 
    \item Derivation:
        \begin{align*}
            \La{f'(t)} &= p \La{f(t)}-f(0^+)\\
            \La{f''(t)} &= p^2\La{f(t)} - pf(0^+) - f'(0^+)\\
            \La{f^{(n)}(t)} &= p^{n}\La{f(t)}-\sum_{k=1}^{n}p^{n-k}f^{(k-1)}(0^{+})
        \end{align*}
    \item Integration:
        \[
            \La{\int_0^t f(u)\dd {u}} = \frac{F(p)}{p} 
        .\] 
    \item Initial value theorem:
        \[
            f(0^+) = \lim_{p \to \infty} p \La{f(t)} 
        .\] 
    \item Final value theorem:
        \[
            f(\infty) = \lim_{p \to 0} p \La{f(t)} 
        .\] 
\end{enumerate}

\begin{remark}
    \begin{align*}
        \La{tf(t)} &= -\dv{}{p} F(p)\\
        \La{t^2f(t)} &= \dv[2]{}{p} F(p)\\
        \La{t^nf(t)} &= (-1)^n \dv[n]{}{p} F(p)
    \end{align*}
\end{remark}

\begin{remark}
    Convolution over a domain $I\subset \mathbb{R}$ is defined as 
    \[
        f(t)*g(t) = \int_I f(\tau)g(t-\tau)\dd {\tau} = \int_I f(t-\tau)g(\tau)\dd {\tau}
    .\]
    and it's trasform is
    \[
        \La{f(t)*g(t)} = F(p)\cdot G(p)
    .\] 
\end{remark}

\section{Translation}
In the time domain:
\[
    \La{f(t-a)} = e^{-ap} F(p)
.\] 

In the $p$-domain:
 \[
     \La{e^{at} f(t)} = F(p+a)
.\] 


\part{Systems of Differential Equations}

Consider a system of first order differential equations $(S)$
\[
	\begin{cases}
		\dv{x_1}{t} = a_{11}(t)x_1+a_{12}(t)x_2+\cdots+a_{1n}(t)x_n + b_1(t) \\
		\dv{x_2}{t} = a_{21}(t)x_1+a_{22}(t)x_2+\cdots+a_{2n}(t)x_n + b_2(t) \\
		\vdots                                                               \\
		\dv{x_n}{t} = a_{n1}(t)x_1+a_{n2}(t)x_2+\cdots+a_{nn}(t)x_n + b_n(t) \\
	\end{cases}
	.\]

in matrix form the equation can be written as
\[
	\dv{\va{x}}{t} = A\va{x} + \va{b}
	.\]

and the initial condition can be written as
\[
	\va{x}_0(t_0) = \mqty(c_1\\c_2\\\vdots\\c_n)
	.\]

\section{Solving The System of DEs}
There 3 main ways of solving systems of DEs:
\begin{enumerate}
	\item Laplace Transform
	\item Change of Basis
	\item Solving Matrix Formula
\end{enumerate}

\begin{remark}
	Let $A$ be a diagonalizable matrix
	\[
		A=PDP^{-1}
		.\]
	so we define that
	\[
		e^{At}=Pe^{Dt}P^{-1}
		.\]
	or in other words
	\[
		e^{At}=P\mqty(\dmat[0]{e^{\lambda_1t},e^{\lambda_2t},\ddots,e^{\lambda_nt}})P^{-1}
		.\]
\end{remark}

\subsection{Change of Basis}

We consider a new system of DEs to be
\[
	\dv{\va{y}}{t} = P^{-1}AP\va{y}+ P^{-1}\va{b}
	.\]
which simplifies to
\[
	\dv{\va{y}}{t} = D\va{y}+ \va{B}
	.\]

in this new system we can solve for $\va{y}$
\[
	\begin{cases}
		\dv{y_1}{t} = \lambda_1y_1 + B_1 \\
		\dv{y_2}{t} = \lambda_2y_2 + B_2 \\
		\vdots                           \\
		\dv{y_n}{t} = \lambda_ny_n + B_n \\
	\end{cases}
	.\]


\begin{remark}
	The solution to a differential equation of the form
	\[
		\dv{y}{t} = \alpha y + \beta
		.\]
	is
	\[
		y = c_1 e^{\alpha t} - \frac{\beta}{\alpha}
		.\]
\end{remark}

after we find the solution to the new system, we can simply obtain the solution to the original system by
\[
	\va{x} = P\va{y}
	.\]
and by substituting $t_0$ in $\va{x}$ we can solve for the constant terms $(c_1,c_2,\ldots,c_n)$ using $\va{x}_0$.

\subsection{Solving Matrix Formula}

The formula for a system of first order equations is
\[
	\va{x} = \va{x}_h + \va{x}_p
	.\]

Where
\begin{align*}
	\va{x}_h & = V(t,t_0)\va{x}_0                    \\
	\va{x}_p & = \int_{t_0}^t V(t,u)\va{b}(u) \dd{u}
\end{align*}
where
\[
	V(t,t_0) = X(t)X^{-1}(t_0)
	.\]
if $t=0$ then the formula becomes
\[
	\va{x} = e^{At}\va{x}_0 + \int_0^t e^{A(t-u)}\va{b}(u) \dd{u}
	.\]

\section{Fundamental Solutions}
For any given system of homogeneous linear DEs there exists a set of $n$ functions such they for a linearly independent basis for a general solution of said DEs, in other words for a given DE there exists a set of vector functions $(\va*{\zeta}_1,\va*{\zeta}_2,\ldots, \va*{\zeta}_n)$ such that
\[
	\va{x}(t) = c_1\va*{\zeta}_1(t) + c_2\va*{\zeta}_2(t) +\cdots+c_n\va*{\zeta}_n(t) \qq{where} c_{1,2,\ldots,n}\in \mathbb{R}
	.\]
We define the fundamental matrix of the system
\[
	X = \mqty(\va*{\zeta}_1&\va*{\zeta}_2&\ldots& \va*{\zeta}_n) = \mqty(\zeta_{11}& \zeta_{12}&\cdots& \zeta_{1n}\\ \zeta_{21}& \zeta_{22}&\cdots& \zeta_{2n} \\\vdots&&\ddots& \\ \zeta_{n1}& \zeta_{n2}&\cdots& \zeta_{nn})
	.\]
The system can be written in terms of $X$ as
\[
	\dv{X}{t} =AX
	.\]
\begin{remark}
	The fundamental solutions are linearly independent $\implies \det(X)\neq 0$
\end{remark}

\subsection{Wronskian of vector functions}
Consider the vector functions:
\[
	\va*{\phi}_1(t) = \mqty(\phi_{11}(t)\\\phi_{21}(t)\\\vdots\\\phi_{n1}(t))\quad\cdots\quad\va*{\phi}_n(t) = \mqty(\phi_{1n}(t)\\\phi_{2n}(t)\\\vdots\\\phi_{nn}(t))
	.\]
The Wronskian is defined to the determinant:
\[
	W(\va*{\phi}_1,\va*{\phi}_2,\cdots,\va*{\phi}_n)=\mqty|\phi_{11}& \phi_{12}&\cdots& \phi_{1n}\\ \phi_{21}& \phi_{22}&\cdots& \phi_{2n} \\\vdots&&\ddots& \\ \phi_{n1}& \phi_{n2}&\cdots& \phi_{nn}|
	.\]


If the Wronskian $=0$ then the functions $(\va*{\phi}_1,\va*{\phi}_2,\ldots,\va*{\phi}_n)$ are said to be linearly independent.

When dealing with DEs the concept of a Wronskian can be applied to \emph{non-vector functions} as follows

\[
	W(\phi_1,\phi_2,\cdots,\phi_n)=\mqty|\phi_1& \phi_2&\cdots& \phi_n\\ \phi'_1& \phi'_2&\cdots& \phi'_n \\\vdots&&\ddots& \\ \phi_1^{(n-1)}& \phi_2^{(n-1)}&\cdots& \phi_n^{(n-1)}|
	.\]

\section{Solving $n$-th order Homogeneous Linear DE}
We define the notation $\D[n]{x}=\dv[n]{x}{t} $.

An $n$-the order linear DE is any equation of the form:
\[
	\D[n]{x}+a_1(t)\D[n-1]{x}+\cdots+a_{n-1}(t)\D{x}+a_n(t)x=0
	.\]

We can then write the equation in vector form
\begin{align*}
	x          & = x_1                              \\
	\D{x}      & = x_2                              \\
	\D[2]{x}   & = x_3                              \\
	           & \vdots                             \\
	\D[n-1]{x} & = x_n                              \\
	\D[n]{x}   & = -a_nx_1-a_{n-1}x_2-\cdots-a_1x_n
\end{align*}
and we take
\[
	\va{x} = \mqty(x_1\\x_2\\\vdots\\x_n) = \mqty(x\\\D{x}\\\vdots\\\D[n-1]{x})
	.\]

then we can write the system as
\[
	\dv{\va{x}}{t} =A\va{x}\qq{where} A=\mqty(0&1&0&\cdots&0\\0&0&1&\cdots&0\\\vdots&&&\ddots&\\0&0&0&\cdots&1\\-a_n&-a_{n-1}&\cdots&\cdots&-a_1)
	.\]


\subsection{DE from a Set of Fundamental Solutions}

Given a set of fundamental solutions $(\zeta_1,\zeta_2,\ldots,\zeta_{n})$, due to the uniqueness theorem those solutions only satisfy one DE. To find that DE we simply compute
\[
	W(x,\zeta_1,\zeta_2,\ldots,\zeta_n) = \mqty|x&\zeta_1&\cdots&\zeta_n\\\D{x}&\D{\zeta_1}&\cdots&\D{\zeta_n}\\\vdots&&\ddots&\\\D[n]{x}&\D[n]{\zeta_1}&\cdots&\D[n]{\zeta_n}| = 0
	.\]

\begin{example}
	Given the fundamental set of solutions $(e^{\omega t},e^{-\omega t})$, find the second order homogeneous equation for that set of solutions:
	\begin{align*}
		         & W(x,e^{\omega t},e^{-\omega t})=0                                \\
		\implies & \mqty|x                           & e^{\omega t} & e^{-\omega t} \\x'&\omega e^{\omega t}&-\omega e^{-\omega t}\\x''&\omega^2e^{\omega t}&\omega^2e^{-\omega t}|=0\\
		\implies & x''-\omega^2x=0
		.\end{align*}
\end{example}

\subsection{Method of Variation of constants}

Consider a non-homogeneous linear DE
\[
	x^{(n)}(t)+\sum_{i=0}^{n-1}a_i(t)x^{(i)}(t) = b(t)
	.\]
and let $(x_1,x_2,\ldots,x_n)$ be a solution to the corresponding homogeneous differential equation. Then the particular solution to the equation is given by
\[
	x_p = \sum_{i=0}^{n}z_i(t)x_i(t)
	.\]

such that $z_i(t)$ satisfies the condition
\[
	\sum_{i=1}^{n}z_i'(t)x^{(j)}(t)=0 \qfor j=0,1,\ldots,n-2
	.\]
We substitute $x_p$ in to the original DE, and along with the previous condition, we obtain a linear system of equations dependent on $z_i'$. Then we simply find $z_i'$ and integrate to get back $z_i$ then finally obtain a general solution to the non-homogeneous DE.\\

The formula for $z_1$ and $z_2$ are given for 2nd order DEs of the form
\[
	x''+P(t)x'+Q(t)x=g(x)
	.\]

\begin{align*}
	z_1 & =-\int\frac{x_2g}{W(x_1,x_2)}\dd{t} \\
	z_2 & =-\int\frac{x_1g}{W(x_1,x_2)}\dd{t}
\end{align*}

\begin{example}
	Solve:
	\[
		x''+x=\sec(t)=\frac{1}{\cos(x)}
		.\]

	We know the solution to the homogeneous equation is $x_h = c_1\cos(t)+c_2\sin(t)$ (refer to the next chapter for a method for solving the homogeneous equation)\marginnote{Not sure why we're solving systems of DEs before we solve DEs in general, besides half the course is basically redundant}\\
	We substitute the constants with our parameters
	\[
		x = z_1\cos(t) + z_2\sin(t)
		.\]
	Our condition for the parameters becomes
	\[
		z_1'\cos(t) + z_2'\sin(t)=0
		.\]
	and substituting $x$ in the original DE we obtain
	\[
		-z_1'\sin(t)+z_2'\cos(t) = \sec(t)
		.\]
	Solving the system
	\[
		\begin{cases}
			z_1' = -\tan(t) & \Rightarrow z_1 = \ln{|\cos(t)|} + c_1 \\
			z_2' = 1        & \Rightarrow z_2=t+c_2
		\end{cases}
		.\]
	Finally
	\[
		x = (\ln{|\cos(t)|}+c_1)\cos(t) + (t+c_2)\sin(t)
		.\]
\end{example}

\part{Linear Differential Equations}

Let the notation denote

\[
	\D[n]{x} = \dv[n]{x}{t}
	.\]

\section{Differential Operators}
\subsection{Definition}
Let's call the set of differential operators $\P$, a differential operator $A\in\P$ is defined to be
\[
	A = a_0\D[0]{} + a_1\D[1]{} + a_2\D[2]{} + \cdots + a_n\D[n]{}
	.\]

The order of a linear operator is the highest power of D denoted
\[
	\gamma(A) = \max(\text{Power of D})
	.\]

Differential operators form a ring $(\P,+,\cdot)$.\marginnote{They behave just like normal polynomials with the exception of $A\cdot B \neq B\cdot A$ which also has the exception of being equal if $a_k=$ cnst and $b_k=$ cnst}\\

\subsection{Differential Operators as DEs}

We can use differential operators to write DEs in the form
\[
	Ax=b(t)
	.\]

Let $A,B\in\P$ and $\gamma(B)\leq\gamma(A)$ we can prove that $\exists Q,R\in\P\;A=QB+R$ ($Q$: quotient, $R$: remainder)\marginnote{Algebra 1 long division, yum}

\begin{remark}
	When dealing with differential operators, treat D as a variable. Thus differential operators can be though of as polynomials of D.
\end{remark}

If the operators $A$ and $B$ each correspond to a \emph{homogeneous} DE then we can compute $\gcd(A,B)$
\begin{itemize}
	\item $\gcd(A,B)= 1$ then the differential equations do not have a common solution between them.
	\item $\gcd(A,B)\neq 1$ then the differential equations have a common solution, which is the solution to the DE corresponding to $R=\gcd(A,B)$
\end{itemize}

\subsubsection{Case of known $k$ solutions}
If we have a homogeneous DE, whose corresponding operator is $A$ of order $n$, that we know $k$ solutions of we can construct an operator $B$ of order $k$ whose solutions is said $k$ solutions. Then the remaining $n-k$ solutions are the solutions are the solutions for $\gcd(A,B)$. We can construct $B$ using
\[
	W(x,x_1,x_2,\ldots,x_k)=0
	.\]


\subsection{Linear Equations with Constant Coefficients}

The characteristic polynomial of a homogeneous DE is the polynomial where D is substituted with $r$, it is denoted as $P(r)$ and its solutions can help find the solutions to the DE.

If $P(r)$ has a solution in the reals ($\mathbb{R}$) then the solution of the associated DE is
\[
	P(r) = (r-\alpha)^\hamboola =0  \Rightarrow x = C_{\hamboola - 1}e^{\alpha t} \\
	.\]

On the other hand if the solutions to $P(r)$ were complex then the solutions become\marginnote{$\hamboola$ is the order of the root of the polynomial}
\[
	P(r) = \left((r-\omega)(r-\bar{\omega})\right)^\hamboola = 0 \Rightarrow x = e^{\Re(\omega)t}\left( U_{\hamboola -1}\cos(\Im(\omega)t) + i\,V_{\hamboola-1}\sin(\Im(\omega)t) \right)
	.\]

where $C_n$, $U_n$, and $V_n$ are polynomials with constant coefficients $(c_1,c_2,\ldots,c_n)$ and degree $n$. $\left( \sum_{i=0}^{n}c_i t^i \right)$

\begin{remark}
	The principle of super position states that for a linear differential equation of the form
	\[
		Ax=\underbrace{b(t)}_{b_1(t)+b_2(t)}
		.\]
	the particular solutions of the equations with $b_1$ and $b_2$, being $x_1$ and $x_2$ respectively, add up to form the particular solution of the original equation
\end{remark}

Generally, we find the solutions to the homogeneous equations then use variation of parameters/constants to find the particular solution.

\section{Particular Forms of $b(t)$}
\begin{itemize}
	\setlength\itemsep{1em}
	\item $Ax = ke^{\beta t}$: Let $\alpha_1,\alpha_2,\ldots,\alpha_n$ be the roots of the associated $P(r)$ and $\hamboola_1,\hamboola_2,\ldots,\hamboola_n$ be the respective order of those roots.
	      \begin{itemize}
		      \item If $\beta \neq \alpha_i$, then we look for a particular solution of the form $x_p=ce^{\beta t}$ where $c$ is a constant to be determined.
		      \item If $\beta = \alpha_i$, then we look for a particular solution of the form $x_p=ct^{\hamboola_i}e^{\beta t}$ where $c$ is a constant to be determined.
	      \end{itemize}
	\item $a_0x'' +a_1x' + a_2x = k$: We look for a particular solution of the form $x_p=c$
	      \begin{itemize}
		      \item If $a_2=0$ we look for a particular solution of the form $x_p=ct$
		      \item If $a_1=a_2=0\implies x_p=ct^2$
	      \end{itemize}
	\item $a_0x'' +a_1x' + a_2x = I_n$ (Polynomial of degree $n$): We look for a particular solution of the form $x_p=Q_n$ (another polynomial of degree $n$)
	      \begin{itemize}
		      \item If $a_2=0$ we look for a particular solution of the form $x_p=Q_{n+1}$
		      \item If $a_1=a_2=0\implies x_p=Q_{n+2}$
	      \end{itemize}
	\item $a_0x'' +a_1x' + a_2x = I_n e^{\beta t}$:
	      \begin{itemize}
		      \item If $P(\beta)\neq0$ we look for a particular solution of the form $x_p=Q_ne^{\beta t}$
		      \item If $P(\beta)=0$ and the order of $\beta=1$ we look for a particular solution of the form $x_p=Q_{n+1}e^{\beta t}$
		      \item If $P(\beta)=0$ and the order of $\beta=2$ we look for a particular solution of the form $x_p=Q_{n+2}e^{\beta t}$
	      \end{itemize}
	\item $a_0x'' +a_1x' + a_2x = k\sin(\beta t)+h\cos(\beta t)$:
	      \begin{itemize}
		      \item If $P(\pm i\beta)\neq0$ we look for $x_p=c_1\sin(\beta t)+c_2\cos(\beta t)$
		      \item If $P(\pm i\beta)=0$ we look for $x_p=t\left[c_1\sin(\beta t)+h\cos(\beta t)\right]$
	      \end{itemize}
	\item $a_0x'' +a_1x' + a_2x = e^{\lambda t}\left[k\sin(\beta t)+h\cos(\beta t)\right]$:
	      \begin{itemize}
		      \item If $P(\lambda \pm i\beta)\neq0$ we look for $x_p=e^{\lambda t}\left[c_1\sin(\beta t)+h\cos(\beta t)\right]$
		      \item If $P(\lambda \pm i\beta)=0$ we look for $x_p=te^{\lambda t}\left[k\sin(\beta t)+h\cos(\beta t)\right]$
	      \end{itemize}
	\item $a_0x'' +a_1x' + a_2x = I_n\sin(\beta t)+Q_n\cos(\beta t)$:
	      \begin{itemize}
		      \item If $P(\pm i\beta)\neq0$ we look for $x_p=R_n\sin(\beta t)+S_n\cos(\beta t)$
		      \item If $P(\pm i\beta)=0$ we look for $x_p=R_{n+1}\sin(\beta t)+S_{n+1}\cos(\beta t)$
	      \end{itemize}
\end{itemize}

\section{Euler's Equations}
Euler's DE is a DE of the form\marginnote{$a_i\in\mathbb{R}$}
\[
	a_0t^n\D[n]{x}+a_1t^{n-1}\D[n-1]{x} + \cdots + a_nx = 0
	.\]

\begin{itemize}
	\item \textbf{Method One}:\\
	      We assume that $x=t^r$, by substituting $x$ in the equation we get a new equation of the form $I(r)t^r=0$ where $I(r)$ is a polynomial of degree $n$. The solutions to $I(r)$ are $q$ real solutions $r_1,r_2,\ldots,r_n$, thus $x = c_1t^{r_1} + c_2t^{r_2} + \cdots + c_n t^{r_n}$.
	\item \textbf{Method Two}:\\
	      We let $t=e^u$ and $\Delta = \dv{}{u}$. We can prove that
	      \[
		      \D[p]{x} = e^{-p u}\Delta(\Delta-1)\cdots(\Delta-(p+1))
		      .\]
	      After doing this substitution, the equation is transformed in to linear equation with constant coefficients.
\end{itemize}


\part{Non-Linear DEs}
\section{Equations of the Form $F(x,y,y',y'')=0$}
\begin{itemize}
	\item $y$ doesn't appear explicitly in the equation. Let $z=y'$, the equation becomes $F(x,z,z')=0$
	\item $y$ appears in the equation. Let $z=y'$, the equation becomes $F(y,z,z\dv{dz}{dy})=0$
\end{itemize}

\begin{remark}
	Let $y'=f(x,y)$ such that $f(x,y) = f(1,\frac{y}{x})$.\\
	Let $t=\frac{y(x)}{x}\implies y=zx\implies \dv{y}{x}=z+x\dv{z}{x}$, The differential equation becomes
	\[
		\frac{\dd{x}}{x} = \frac{\dd{z}}{f(1,z)-1}
		.\]
\end{remark}

\section{Bernoulli Equation}
\marginnote{$m\in\mathbb{R}^*-\{1\}$}
\[
	y' + P(x)y = Q(x)y^m
	.\]
Let $z=y^{1-m}\implies z'=(1-m)\frac{y'}{y^m}$. Dividing the equation by $y^m$.
\[
	\frac{z'}{1-m}+P(x)z = Q(x)
	.\]

\section{Riccati Equation}
\[
	y'=A(x)y^2+B(x)y+C(x)
	.\]
If we know one solution $y_1$ to the DE, we look for another solution $y=y_1 + u$. With some algebra that I couldn't be bothered to type the equation becomes
\[
	u'=A(x)u^2+(2A(x)y_1 + B(x))u
	.\]
which is a Bernoulli equation $m=2$. Let $z=\frac{1}{u}$
\[
	-z'=\left[2A(x)y_0+B(x)\right]z + A(x)
	.\]

\section{Clairaut's equation}
\[
	xy'-y=\varphi(y')
	.\]
Let $y=cx+k$ by algebra $k=-\varphi$, therefore $y=cx-\varphi(c)$ is general solution of the DE.


\end{document}

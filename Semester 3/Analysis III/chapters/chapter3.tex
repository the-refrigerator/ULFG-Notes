\part{Power Series}

A power series is just a series in the following formula
\begin{align*}
	f(x) & =\sum_{n=0}^{\infty} a_n(x-x_0)^n \\
	     & =\sum_{n=0}^{\infty} U_n
	.\end{align*}

\section{Radius of Convergence}

For some values of $x$ a power series can either diverge or converge, to determine the interval of convergence we employ the ratio test

\begin{theorem}
	Let $r$ be the radius of convergence and $I$ be the domain of convergence. If we compute the limit
	\[
		\Gamma = \lim_{n \to \infty}\left| \frac{U_{n+1}}{U_n} \right|
		.\]
	The ratio test states that
	\[
		\begin{cases}
			\qif* \Gamma = 0      & \qthen* r=\infty\qand I=\mathbb{R} \\
			\qif* \Gamma = \infty & \qthen* r=0\qand I=\{0\}           \\
		\end{cases}
		.\]
	In the case that $\Gamma$ isn't 0 or $\infty$ we set $\Gamma<1$ and then we find $|x|<R$, finally we can say that $r=R$ and $I=]-R,R[$. A special case need to be done for the points  $-R$ and $R$ to determine if they belong in  $I$.
\end{theorem}

\begin{remark}
	The power series $f(x)$ is continous and will always uniformly converge in the interval of convergence $I$
\end{remark}

\begin{theorem}

	If $\sum a_n x^n$ and $\sum b_n x^n$ be 2 power series with radii  $R_1$ and $R_2$. For the power series $\sum (a_n+b_n)x^n$ the radius of convergence $R$
	\[
		R=\min \{R_1,R_2\}
		.\]
\end{theorem}


\begin{theorem}
	If $S(x)=\sum_{n=0}^{\infty} a_n(x-x_0)^n$ is a power series with radius $R$ then $S'(x)=\sum_{n=0}^{\infty} na_n(x-x_0)^n$ as well as $\int_{{x_0}}^{{x}} {S(t)} \: d{t}$ both a radius of $R$
\end{theorem}

The general term $a_k$ of a power series $S(x)=\sum_{n=0}^{\infty} a_n(x-x_0)^n$ is equal to
\[
	a_k = \frac{S^{(k)}(x_0)}{k!}
	.\]

\begin{align*}
    y &= a_0+a_1x+a_2x^2+\cdots+a_nx^n\\
    y' &= a_1+2a_2x+3a_3x^2+\cdots+na_nx^{n-1}\\
    y'' &= 2a_2+6a_3x+\cdots+n(n-1)a_nx^{n-2}+n(n+1)a_{n+1}x^{n-1}\\
\end{align*}

\part{Systems of Differential Equations}

Consider a system of first order differential equations $(S)$
\[
	\begin{cases}
		\dv{x_1}{t} = a_{11}(t)x_1+a_{12}(t)x_2+\cdots+a_{1n}(t)x_n + b_1(t) \\
		\dv{x_2}{t} = a_{21}(t)x_1+a_{22}(t)x_2+\cdots+a_{2n}(t)x_n + b_2(t) \\
		\vdots                                                               \\
		\dv{x_n}{t} = a_{n1}(t)x_1+a_{n2}(t)x_2+\cdots+a_{nn}(t)x_n + b_n(t) \\
	\end{cases}
	.\]

in matrix form the equation can be written as
\[
	\dv{\va{x}}{t} = A\va{x} + \va{b}
	.\]

and the initial condition can be written as
\[
	\va{x}_0(t_0) = \mqty(c_1\\c_2\\\vdots\\c_n)
	.\]

\section{Solving The System of DEs}
There 3 main ways of solving systems of DEs:
\begin{enumerate}
	\item Laplace Transform
	\item Change of Basis
	\item Solving Matrix Formula
\end{enumerate}

\begin{remark}
	Let $A$ be a diagonalizable matrix
	\[
		A=PDP^{-1}
		.\]
	so we define that
	\[
		e^{At}=Pe^{Dt}P^{-1}
		.\]
	or in other words
	\[
		e^{At}=P\mqty(\dmat[0]{e^{\lambda_1t},e^{\lambda_2t},\ddots,e^{\lambda_nt}})P^{-1}
		.\]
\end{remark}

\subsection{Change of Basis}

We consider a new system of DEs to be
\[
	\dv{\va{y}}{t} = P^{-1}AP\va{y}+ P^{-1}\va{b}
	.\]
which simplifies to
\[
	\dv{\va{y}}{t} = D\va{y}+ \va{B}
	.\]

in this new system we can solve for $\va{y}$
\[
	\begin{cases}
		\dv{y_1}{t} = \lambda_1y_1 + B_1 \\
		\dv{y_2}{t} = \lambda_2y_2 + B_2 \\
		\vdots                           \\
		\dv{y_n}{t} = \lambda_ny_n + B_n \\
	\end{cases}
	.\]


\begin{remark}
	The solution to a differential equation of the form
	\[
		\dv{y}{t} = \alpha y + \beta
		.\]
	is
	\[
		y = c_1 e^{\alpha t} - \frac{\beta}{\alpha}
		.\]
\end{remark}

after we find the solution to the new system, we can simply obtain the solution to the original system by
\[
	\va{x} = P\va{y}
	.\]
and by substituting $t_0$ in $\va{x}$ we can solve for the constant terms $(c_1,c_2,\ldots,c_n)$ using $\va{x}_0$.

\subsection{Solving Matrix Formula}

The formula for a system of first order equations is
\[
	\va{x} = \va{x}_h + \va{x}_p
	.\]

Where
\begin{align*}
	\va{x}_h & = V(t,t_0)\va{x}_0                    \\
	\va{x}_p & = \int_{t_0}^t V(t,u)\va{b}(u) \dd{u}
\end{align*}
where
\[
	V(t,t_0) = X(t)X^{-1}(t_0)
	.\]
if $t=0$ then the formula becomes
\[
	\va{x} = e^{At}\va{x}_0 + \int_0^t e^{A(t-u)}\va{b}(u) \dd{u}
	.\]

\section{Fundamental Solutions}
For any given system of homogeneous linear DEs there exists a set of $n$ functions such they for a linearly independent basis for a general solution of said DEs, in other words for a given DE there exists a set of vector functions $(\va*{\zeta}_1,\va*{\zeta}_2,\ldots, \va*{\zeta}_n)$ such that
\[
	\va{x}(t) = c_1\va*{\zeta}_1(t) + c_2\va*{\zeta}_2(t) +\cdots+c_n\va*{\zeta}_n(t) \qq{where} c_{1,2,\ldots,n}\in \mathbb{R}
	.\]
We define the fundamental matrix of the system
\[
	X = \mqty(\va*{\zeta}_1&\va*{\zeta}_2&\ldots& \va*{\zeta}_n) = \mqty(\zeta_{11}& \zeta_{12}&\cdots& \zeta_{1n}\\ \zeta_{21}& \zeta_{22}&\cdots& \zeta_{2n} \\\vdots&&\ddots& \\ \zeta_{n1}& \zeta_{n2}&\cdots& \zeta_{nn})
	.\]
The system can be written in terms of $X$ as
\[
	\dv{X}{t} =AX
	.\]
\begin{remark}
	The fundamental solutions are linearly independent $\implies \det(X)\neq 0$
\end{remark}

\subsection{Wronskian of vector functions}
Consider the vector functions:
\[
	\va*{\phi}_1(t) = \mqty(\phi_{11}(t)\\\phi_{21}(t)\\\vdots\\\phi_{n1}(t))\quad\cdots\quad\va*{\phi}_n(t) = \mqty(\phi_{1n}(t)\\\phi_{2n}(t)\\\vdots\\\phi_{nn}(t))
	.\]
The Wronskian is defined to the determinant:
\[
	W(\va*{\phi}_1,\va*{\phi}_2,\cdots,\va*{\phi}_n)=\mqty|\phi_{11}& \phi_{12}&\cdots& \phi_{1n}\\ \phi_{21}& \phi_{22}&\cdots& \phi_{2n} \\\vdots&&\ddots& \\ \phi_{n1}& \phi_{n2}&\cdots& \phi_{nn}|
	.\]


If the Wronskian $=0$ then the functions $(\va*{\phi}_1,\va*{\phi}_2,\ldots,\va*{\phi}_n)$ are said to be linearly independent.

When dealing with DEs the concept of a Wronskian can be applied to \emph{non-vector functions} as follows

\[
	W(\phi_1,\phi_2,\cdots,\phi_n)=\mqty|\phi_1& \phi_2&\cdots& \phi_n\\ \phi'_1& \phi'_2&\cdots& \phi'_n \\\vdots&&\ddots& \\ \phi_1^{(n-1)}& \phi_2^{(n-1)}&\cdots& \phi_n^{(n-1)}|
	.\]

\section{Solving $n$-th order Homogeneous Linear DE}
We define the notation $\D[n]{x}=\dv[n]{x}{t} $.

An $n$-the order linear DE is any equation of the form:
\[
	\D[n]{x}+a_1(t)\D[n-1]{x}+\cdots+a_{n-1}(t)\D{x}+a_n(t)x=0
	.\]

We can then write the equation in vector form
\begin{align*}
	x          & = x_1                              \\
	\D{x}      & = x_2                              \\
	\D[2]{x}   & = x_3                              \\
	           & \vdots                             \\
	\D[n-1]{x} & = x_n                              \\
	\D[n]{x}   & = -a_nx_1-a_{n-1}x_2-\cdots-a_1x_n
\end{align*}
and we take
\[
	\va{x} = \mqty(x_1\\x_2\\\vdots\\x_n) = \mqty(x\\\D{x}\\\vdots\\\D[n-1]{x})
	.\]

then we can write the system as
\[
	\dv{\va{x}}{t} =A\va{x}\qq{where} A=\mqty(0&1&0&\cdots&0\\0&0&1&\cdots&0\\\vdots&&&\ddots&\\0&0&0&\cdots&1\\-a_n&-a_{n-1}&\cdots&\cdots&-a_1)
	.\]


\subsection{DE from a Set of Fundamental Solutions}

Given a set of fundamental solutions $(\zeta_1,\zeta_2,\ldots,\zeta_{n})$, due to the uniqueness theorem those solutions only satisfy one DE. To find that DE we simply compute
\[
	W(x,\zeta_1,\zeta_2,\ldots,\zeta_n) = \mqty|x&\zeta_1&\cdots&\zeta_n\\\D{x}&\D{\zeta_1}&\cdots&\D{\zeta_n}\\\vdots&&\ddots&\\\D[n]{x}&\D[n]{\zeta_1}&\cdots&\D[n]{\zeta_n}| = 0
	.\]

\begin{example}
	Given the fundamental set of solutions $(e^{\omega t},e^{-\omega t})$, find the second order homogeneous equation for that set of solutions:
	\begin{align*}
		         & W(x,e^{\omega t},e^{-\omega t})=0                                \\
		\implies & \mqty|x                           & e^{\omega t} & e^{-\omega t} \\x'&\omega e^{\omega t}&-\omega e^{-\omega t}\\x''&\omega^2e^{\omega t}&\omega^2e^{-\omega t}|=0\\
		\implies & x''-\omega^2x=0
		.\end{align*}
\end{example}

\subsection{Method of Variation of constants}

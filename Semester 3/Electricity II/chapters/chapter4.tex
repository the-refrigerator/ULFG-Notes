\part{Magnetostatics}
\section{Biot-Savart Law}

The Lorentz force law states that
\[
	\va{F} = q\left( \va{v}\cross\va{B} \right)
	.\]

and the magnitude is
\[
	\left\|\va{F}\right\| = |q|\left\|\va{v}\right\| \left\|\va{B}\right\|\sin\theta
	.\]

The magnetic field of a steady line current is given by the Biot-Savart law:
\[
	\va{B}(\vb{r}) =  \frac{\mu_0}{4\pi}\int_{C}\frac{I \dd{\va{\ell}}\times \vu{r} }{ \| \va{r} \| ^2} = \frac{\mu_0}{4\pi}\int_{C}\frac{I \dd{\va{\ell}}\times \va{r} }{ \| \va{r} \| ^3}
	.\]

The integration is along the current path, in the direction of the flow; $\dd{\ell} $ is an element of length along the wire, and $\va{r}$ is the vector from the source to the point $\vb{r}$. The constant $\mu_0$ is called the permeability of free space
\[
	\mu_0=4\pi\times 10^{-7}
	.\]
The flux $\Phi$ across a surface $S$ is
\[
	\Phi=\oiint_S \va{B}\dd{\va{S}} = \iiint_V \div{\va{B}}\dd{v} =0
	.\]


The induction vector derives from a vector potential
\[
	\va{B}=\curl{\va{A}}\qq{where}\va{A}=\frac{\mu_0I}{2\pi}\int \frac{1}{r}\dd{\va{\ell}}
	.\]
\section{Ampère's circuital law}
Ampère's law is defined \emph{in a closed loop} to be
\[
	\oint \va{B} \dd{\va{\ell}} = \mu_0 I_\text{enc}
	.\]

Recall Coulomb's law and Gauss's law from the previous semester and notice the pattern
\[
	\left\{
	\begin{aligned}
		 & \text{Electrostatics:} & \text{Coulomb}     & \rightarrow\text{Gauss}  \\
		 & \text{Magnetostatics:} & \text{Biot-Savart} & \rightarrow\text{Ampère}
	\end{aligned}
	\right.
	.\]

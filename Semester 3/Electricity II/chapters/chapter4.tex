\part{Magnetism}
\section{Magneto-statics}
\subsection{Biot-Savart Law}

The Lorentz force law states that
\[
	\va{F} = q\left( \va{v}\cross\va{B} \right)
	.\]

and the magnitude is
\[
	\left\|\va{F}\right\| = |q|\left\|\va{v}\right\| \left\|\va{B}\right\|\sin\theta
	.\]

The magnetic field of a steady line current is given by the Biot-Savart law:
\[
	\va{B}(\vb{r}) =  \frac{\mu_0}{4\pi}\int_{C}\frac{I \dd{\va{\ell}}\times \vu{r} }{ \| \va{r} \| ^2} = \frac{\mu_0}{4\pi}\int_{C}\frac{I \dd{\va{\ell}}\times \va{r} }{ \| \va{r} \| ^3}
	.\]

The integration is along the current path, in the direction of the flow; $\dd{\ell} $ is an element of length along the wire, and $\va{r}$ is the vector from the source to the point $\vb{r}$. The constant $\mu_0$ is called the permeability of free space
\[
	\mu_0=4\pi\times 10^{-7}
	.\]
The flux $\Phi$ across a surface $S$ is
\[
	\Phi=\oiint_S \va{B}\dd{\va{S}} = \iiint_V \div{\va{B}}\dd{v} =0
	.\]


The induction vector derives from a vector potential
\[
	\va{B}=\curl{\va{A}}\qq{where}\va{A}=\frac{\mu_0I}{2\pi}\int \frac{1}{r}\dd{\va{\ell}}
	.\]
\subsection{Ampère's circuital law}
Ampère's law is defined \emph{in a closed loop} to be
\[
	\oint \va{B} \dd{\va{\ell}} = \mu_0 I_\text{enc}
	.\]

\section{Laplace’s Force Law}
If a particle moves with a speed $\va{v}$ in an electric field and a magnetic field it will experience Lorentz's force.
\[
	\va{f} = q\va{E}+q\va{v}\cross\va{B}
	.\]

\[
	\dd{\va{f}} = \va{J}\cross\va{B}\dd{\tau} = I(\dd{\va{\ell}}\cross\va{B})
	.\]

The flux going through a surface $S$ in a magnetic field is
\[
	\Phi = \iint_S \va{B}\cdot\dd{\va{S}}
	.\]

Work done by Lorentz force
\[
	W = I(\Phi_2 - \Phi_1)
	.\]

The electromagnetic force
\[
	\va{F} = I \grad{\Phi}
	.\]

Work done if the movement was a rotation
\[
	\dd{W} = \Gamma_x\dd{\alpha} = I \dd{\Phi_\alpha}
	.\]

Where $\alpha, \beta$ and $\gamma$ are components of rotation
\[
	\Gamma_i = I \pdv{\Phi}{\alpha_i}
	.\]

\section{Electromotive Force}
\marginnote{Magnetic field around an infinite wire if $B = \frac{\mu_0I}{2\pi r}$}
The induced emf is given by
\[
	e = -\pdv{\Phi}{t} = \int \va{B}\cdot(\dd{\va{\ell}}\cross\va{v})
	.\]

If a circuit has a induced emf
\[
	E-\dv{\Phi}{t} = RI
	.\]



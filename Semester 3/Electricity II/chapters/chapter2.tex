\part{Ohm's and Joule's Law}
\section{Electrical Mobility and Conductivity}
The velocity of a charge carrier in in an electric field $\va{E}$ is
\begin{equation}
	\va{v}=\mu\va{E}
	.\end{equation}

where $\mu$ is the mobility of the charge carrier.\\

The current density vector in a conductor with conductivity $\sigma$ is said to be
\begin{equation}
	\va{J}=\sigma\va{E}
	.\end{equation}

Ohm's law(doesn't really need an introduction)
\begin{equation}
	U=RI
	.\end{equation}

where we can find $R$ using
\begin{equation}
	R=\frac{\rho l}{A}
	.\end{equation}

where $A$ is the surface area of the resistor and $l$ is the length of the conductor\sn{$\sigma=\frac{1}{\rho}$ }.


\section{Resistance}

The symbol of a resistive conductor is represented by
\begin{figure}[h!]
	\centering
	\begin{circuitikz}[american]
		\draw (0,0) to[R] (2,0);
	\end{circuitikz}
\end{figure}

\begin{align}
	P & =U I \\
	W & = Pt
\end{align}

Power loss due to Joule's law

\begin{align}
	P & =I^2 R=\frac{U^2}{R} \\
	W & =I^2Rt
\end{align}

\section{Electric Ciruits}

\subsection{Power Sources/Generators}
A DC source is characterized by their electromotive force $E$ and internal resistance $r$
\begin{figure}[H]

	\begin{subfigure}[b]{0.4\textwidth}
		\centering
		\begin{circuitikz}[american]
			\draw (0,0) to[battery1,*-*,l=$(E\comma r)$] (3,0);
		\end{circuitikz}
	\end{subfigure}
	\hfill
	\begin{subfigure}[b]{0.4\textwidth}
		\centering
		\begin{circuitikz}[american]
			\draw (0,0) to[battery1,*-,l=$(E\comma 0)$] (1.5,0) to[R=$r$,,-*] (3,0);
		\end{circuitikz}
	\end{subfigure}

\end{figure}

The voltage supplied by the source is
\[
	V_{\text{source}}=E-rI
	.\]

and it's efficiency is
\[
	\eta = 1-\frac{rI}{E}
	.\]

\subsubsection{In Series}
\begin{figure}[H]

	\begin{subfigure}[b]{0.4\textwidth}
		\centering
		\begin{circuitikz}[american]
			\draw (0,0) to[battery1,*-,l=$(E_1\comma r_1)$] (1.5,0) to[battery1,l=$(E_2\comma r_2)$] (3,0);
			\draw[dashed] (3,0) to[short] (4.5,0);
			\draw (4.5,0) to[battery1,-*,l=$(E_n\comma r_n)$] (6,0);
		\end{circuitikz}
	\end{subfigure}
	\hfill
	\begin{subfigure}[b]{0.4\textwidth}
		\centering
		\begin{circuitikz}[american]
			\draw (0,0) to[battery1,*-*,l=$(E_\text{Eq}\comma r_\text{Eq})$] (4.5,0);
		\end{circuitikz}
	\end{subfigure}

\end{figure}

where
\begin{align*}
	E_\text{Eq} & =\sum_{i=1}^{n} E_i \\
	r_\text{Eq} & =\sum_{i=1}^{n} r_i
\end{align*}

\subsubsection{In Parallel}

\begin{figure}[H]
	\centering
	\begin{circuitikz}[american]
		\draw (0,2) to[short,*-] (1,2);

		\draw (1,0) to[short] (1,4);
		\draw (1,0) to[battery1,l=$(E_n\comma r_n)$,i=$I_n$] (5,0);
		\draw [dashed] (3,1) -- (3,2);
		\draw (1,2.5) to[battery1,l=$(E_2\comma r_2)$,i=$I_2$] (5,2.5);
		\draw (1,4) to[battery1,l=$(E_1\comma r_1)$,i=$I_1$] (5,4);
		\draw (5,0) to[short] (5,4);

		\draw (5,2) to[short,-*] (6,2);

		\draw (8,2) to[battery1,*-*,l=$(E_\text{Eq}\comma r_\text{Eq})$] (12.5,2);

	\end{circuitikz}

\end{figure}

In case of \emph{identical} sources:
\begin{align*}
	I                     & =\sum_{i=1}^{n} I_i           \\
	\frac{1}{r_\text{Eq}} & =\sum_{i=1}^{n} \frac{1}{r_i}
\end{align*}
and
\[
	V_\text{source}=E - r_\text{Eq}I
	.\]

\subsection{Loads}
An electrical load is an electrical component or portion of a circuit that consumes electric power.
Electric loads are represented by a counter electromotive force $e$ and internal resistance $r'$ (with the exception of resistors)\\

All formulas for generators are the same as loads with the exception of the efficiency
\[
	\eta = 1-\frac{r'I}{U}
	.\]

\documentclass[a4paper,12pt]{article}

\usepackage{textcomp}
\usepackage{amsmath, amssymb}
\usepackage{bm}
\usepackage{relsize}
\usepackage{parskip}
\usepackage{{../NotesTeX}}
\everymath{\displaystyle}

\DeclareMathOperator{\projection}{proj}
\newcommand{\dotp}[2]{\left\langle #1,#2 \right\rangle}
\newcommand{\proj}[2]{\projection_{#1}\left(#2\right)}

\title{Differential Geometry}
\date{\today}

\begin{document}
\maketitle
\part{Prerequisites}
\section{Matrices}
\begin{theorem}
	To prove a system of vectors $\{\va{u}_1,\va{u}_2,\ldots,\va{u}_n\} $ is free we prove:
	\[
		\det \mqty[\vline&\vline&\vline&&\vline\\\va{u}_1&\va{u}_2&\va{u}_3&\cdots&\va{u}_1\\\vline&\vline&\vline&&\vline] \neq  0
		.\]
\end{theorem}

\begin{theorem}
	A transition matrix $P_{B\to B'}$\marginnote{transition matrices are always square and invertible ($\det P \neq 0$)}between 2 basis $B=\{\va{u}_1,\va{u}_2,\va{u}_3\} $ and $B'=\{\va{v}_1,\va{v}_2,\va{v}_3\} $ we start by solving the system
	\[
		\mqty[\vline&\vline&\vline&&\vline\\\va{u}_1&\va{u}_2&\va{u}_3&\cdots&\va{u}_n\\\vline&\vline&\vline&&\vline] \mqty[\alpha_n\\\beta_n\\\gamma_n] = \mqty[\vline\\\va{v}_n\\\vline]
		.\]
	or in other words finding
	\[
		\begin{cases}
			\va{v}_1 & =\alpha_1\va{u}_1+\beta_1\va{u}_2+\gamma_1\va{u}_3 \\
			\va{v}_2 & =\alpha_2\va{u}_1+\beta_2\va{u}_2+\gamma_2\va{u}_3 \\
			\va{v}_3 & =\alpha_3\va{u}_1+\beta_3\va{u}_2+\gamma_3\va{u}_3 \\
		\end{cases}
		.\]
	Finally we say that
	\[
		P_{B\to B'} = \mqty[\alpha_1&\alpha_2&\alpha_3\\\beta_1&\beta_2&\beta_3\\\gamma_1&\gamma_2&\gamma_3]
		.\]
\end{theorem}
\begin{remark}
	To find the transition matrix in the inverse direction (from $B'$ to $B$ ) we simply do
	\[
		P_{B'\to B} = {P_{B\to B'}}^{-1}
		.\]
\end{remark}
\section{Vectors}
\begin{definition}
	We define an operation called the scalar product (dot product)
	\begin{align*}
		\dotp{\cdot}{\cdot}: \mathbb{R}^n \times \mathbb{R}^n & \longrightarrow \mathbb{R}                                     \\
		\va{u},\va{v}                                         & \longmapsto \dotp{\va{u}}{\va{v}} =\sum_{n=1}^{n} v_i\cdot u_i
		.\end{align*}
\end{definition}

\begin{definition}
	We define the usual norm on $\mathbb{R}$ to be
	\begin{align*}
		\|\cdot \|: \mathbb{R}^n & \longrightarrow \mathbb{R}                            \\
		\va{u}                   & \longmapsto \|\va{u}\| = \sqrt{\dotp{\va{u}}{\va{u}}}
		.\end{align*}
\end{definition}
\begin{theorem}
	The projection of a vector $\va{u}$ on to another vector $\va{v}$ is
	\[
		\proj{\va{v}}{\va{u}} = \frac{\dotp{\va{u}}{\va{v}}}{\|\va{v}\|^2}\va{v}
		.\]
\end{theorem}

\subsection{Gram–Schmidt process}
The aim of this process is to find a new basis $\Gamma=\{\vu{e}_1,\vu{e}_2,\ldots,\vu{e}_n\} $ derived from a basis $B=\{\va{v}_1,\va{v}_2,\ldots,\va{v}_n\} $ such that it is orthonormal or in other words
\[
	\forall \vu{x},\vu{y}\in \Gamma: \dotp{\vu{x}}{\vu{y}}=0 \qq{and} \|\vu{x}\|=1
	.\]
We find it as follows
\begin{align*}
	\va{u}_1 & = \va{v}_1 \quad                                                                                    & \vu{e}_1 = \frac{\va{u}_1}{\|\va{u}_1\|} \\
	\va{u}_2 & = \va{v}_2-\proj{\va{u}_1}{\va{v}_2}                                                                & \vu{e}_2 = \frac{\va{u}_2}{\|\va{u}_2\|} \\
	\va{u}_3 & = \va{v}_3-\proj{\va{u}_1}{\va{v}_3}-\proj{\va{u}_2}{\va{v}_3}                                      & \vu{e}_3 = \frac{\va{u}_3}{\|\va{u}_3\|} \\
	\vdots   &                                                                                                     &                                          \\
	\va{u}_n & = \va{v}_n-\proj{\va{u}_1}{\va{v}_n}-\proj{\va{u}_2}{\va{v}_n}-\ldots-\proj{\va{u}_{n-1}}{\va{v}_n} & \vu{e}_n= \frac{\va{u}_n}{\|\va{u}_n\|}  \\
\end{align*}


\part{Conics and Quadrics}

\section{Quadratic form}
We define a quadric form to be a mapping $q$ 
\begin{align*}
    q: \mathbb{R}^n &\longrightarrow \mathbb{R} \\
    \va{u} &\longmapsto q(\va{u}) = \mqty[\rule{5mm}{0.4pt}&^{\mathsmaller T}\va{u}&\rule{5mm}{0.4pt}]A\mqty[\vline\\\va{u}\\\vline]
.\end{align*}
Where the matrix $A$ is a symmetric matrix.\mn{symmetric matrices ($A=^{\mathsmaller T}A$) is always diagonalizable}

The conics understudy are
\begin{align*}
    &\frac{x^2}{a^2}+\frac{y^2}{b^2}=1 \quad& \text{ellipse (circle if \(a=b\))}\\
    &\frac{x^2}{a^2}+\frac{y^2}{b^2}=-1& \text{imaginary ellipse}\\
    &\frac{x^2}{a^2}-\frac{y^2}{b^2}=\pm1&\text{hyperbola with asymptote }y=\pm\frac{b}{a}x\\
    &\begin{rcases}
        y^2=\pm 2px \quad p>0\\
        x^2=\pm 2py \quad p>0
    \end{rcases}& \text{parabolas}\\
    &\frac{x^2}{a^2}-\frac{y^2}{b^2}=0 &\text{union of two straight lines}\\
    &\begin{rcases}
        x=\text{const}\\
        y=\text{const}
    \end{rcases}&\text{straight lines}
\end{align*}




\end{document}

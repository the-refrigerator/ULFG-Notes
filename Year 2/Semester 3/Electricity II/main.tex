\documentclass[a4paper,12pt]{article}

\usepackage{textcomp}
\usepackage{amsmath, amssymb}
\usepackage{bm}
\usepackage{relsize}
\usepackage{parskip}
\usepackage{esint}
\usepackage{../NotesTeX}

\everymath{\displaystyle}

\title{Electricity 2}
\date{}

\begin{document}
\maketitle

\part{Electric Current}
\section{Current and Current Density}

Electric charge in motion constitutes an electric current. In the steady flow of
charge in a wire of cross-sectional area $A$, the total charge passing through this
area in unit time is defined to be the electric current $I$ at this place. If a total
charge $q$ flows through this area in time $t$, the current $I$ is given by:
\begin{equation}
I = \frac{q}{t}
.\end{equation}

where $I$ is measured in Ampere. Sometimes it is also convenient to look at the density of the current we're observing over a certain area $A$ called \emph{current density}

\begin{equation}
J=\frac{I}{A}
.\end{equation}

If the area in which the current is running through changes at points  1 and 2, the current between those 2 points remain constant ($I_1=I_2$) while the current density changes to become 
\[
\frac{J_1}{J_2}=\frac{A_2}{A_1}
.\] 
If a positively charged cloud of $n$ particles moves through a space with speed $v$ and charge $q$, we find that\sidenote{$q$ can be positive or negative} 

\begin{equation}
    \va{J}=nq\va{v}=\varrho_\tau \va{v}
.\end{equation}

The total current through a \emph{slice} of space mentioned prior it calculated using
\begin{equation}
    I=\iint_A \va{J}\dd{A}
.\end{equation} 
The volume element here is $\va{v}\dd {A}\dd {t}=\dd {\tau}$

\section{Continuity Law}
The continuity law for current is
\begin{equation}
    \div(\va{J}+\va{J}_c)=-\pdv{\varrho_\tau}{t} 
.\end{equation} 
Where
\begin{description}
    \item[$\va{J}$ ] is the conduction current density.
    \item[$\va{J}_c=\varrho_\tau\va{v}_\tau$ ] is the convection current density.
    \item [$\va{v}_\tau$] is the velocity of the volume containing the particles.
\end{description}

There are 2 particular cases for the equation above
\begin{enumerate}
    \item The volume is not moving ($\va{v}_\tau=0 \implies \va{J}_c=0$). The continuity law of static structures is
        \begin{equation}
            \div{\va{J}}=-\pdv{\varrho_\tau}{t} 
        .\end{equation} 
        \item The volume is not moving and we are in a steady state, so the charge density of the particles doesn't change much as they are static so $\pdv*{\varrho_\tau}{t}=0 $ so
            \begin{equation}
            \oiint \va{J}\dd {A}=0
            .\end{equation} 
\end{enumerate}



\end{document}

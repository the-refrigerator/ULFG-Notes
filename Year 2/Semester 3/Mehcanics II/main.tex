\documentclass[a4paper,12pt]{article}

\usepackage{textcomp}
\usepackage{amsmath, amssymb}
\usepackage{bm}
\usepackage{relsize}
\usepackage{parskip}
\usepackage{cancel}
\usepackage{../NotesTeX}
\everymath{\displaystyle}

\pagestyle{fancynotes}

\title{Mechanics 2}
\date{}
\author{Ahmad Abu Zainab}
\affiliation{Lebanese University Faculty of Engineering\\\href{http://www.ulfg.ul.edu.lb/}{Website}}
\emailAdd{ahmad.abuzainab@st.ul.edu.lb}

\newcommand{\tor}[2]{[\mathrm{T}_{#1}(#2)]}
\newcommand{\torc}[2]{\left[ #1 \quad #2 \right] }
\newcommand{\T}{{[\mathrm{T}]}}

\begin{document}
\maketitle
\part{Kinematics of Rigid Bodies}
\section{Mathematical Notions}

\subsection{Vectors}
\begin{definition}
	A double product between 3 vectors $\va{u},\va{v},\va{w}\in \mathbb{R}^n$ is defined as
	\[
		\va{u}\land(\va{v}\land \va{w})= (\va{u}\cdot \va{w})\va{v} - (\va{u}\cdot \va{v})\va{w}
		.\]
\end{definition}

\begin{definition}
	The mixed product of 3 vectors $\va{u},\va{v},\va{w}\in \mathbb{R}^n$ is defined as
	\[
		\va{u}\cdot (\va{v}\land \va{w})= \det \begin{bmatrix} u_x&v_x&w_x\\u_y&v_y&w_y\\u_z&v_z&w_z \end{bmatrix}
		.\]
\end{definition}

\subsection{Antisymmetric mapping}

\begin{definition}
	In Euclidean space $\mathbb{R}^n$ a mapping $f:\mathbb{R}^n\to \mathbb{R}^n$ is symmetric if
	\[
		\forall \va{u},\va{v}\in \mathbb{R}^n: \va{u}\cdot f(\va{v})=\va{v}\cdot f(\va{u})
		.\]
	and antisymmetric if
	\[
		\forall \va{u},\va{v}\in \mathbb{R}^n: \va{u}\cdot f(\va{v})=-\va{v}\cdot f(\va{u})
		.\]
\end{definition}
\begin{remark}
	The (anti)symmetric mapping $f:\mathbb{R}^n\to \mathbb{R}^n$ is linear
	\[
		\forall \va{u},\va{v}\in \mathbb{R}^n: f(\alpha \va{u}+\beta \va{v})=(-)(\alpha f(\va{u})+\beta f(\va{v}))
		.\]
\end{remark}
\begin{theorem}
	The (anti)symmetric mapping $f:\mathbb{R}^n\to \mathbb{R}^n$, there exists a unique vector $R\in \mathbb{R}^n$ which is called the \emph{characteristic} vector of $f$ such that
	\[
		\forall \va{u}\in\mathbb{R}^n\;f(\va{u})=\va{R}\land \va{u}
		.\]
\end{theorem}
\newpage
\subsection{Torsors}
A torsor is a mathematical element used in mechanics to define certain forces.\\

A torsor consists of a resultant vector $\va{R}$ and a moment vector $\va{M}_O$ about a point $O$\sn{to change the moment vector to be about a point $A$ $\va{M}_A=\va{M}_O + \va{R}\land \va{AB}$}, these elements are called \emph{elements of reduction} of the torsor. We represent a torsor like so
\[
	\tor{}{A}=\torc{\va{R}}{\va{M}_O}
	.\]

\begin{definition}
	The scalar invariant of a torsor $\T$, denoted  $I_\T$, is the scalar product of the torsor with the moment at a point $A$.\mn{the invariant is independent of the point the moment is chosen about}
	\[
		I_\T = \va{R}\cdot \va{M}_A
		.\]
\end{definition}
Consider 2 torsors $\tor{1}{A} = \torc{\va{R}_1}{\va{M}_{1A}}$ and  $\tor{2}{A} = \torc{\va{R}_2}{\va{M}_{2A}}$. The properties of a torsor are:

\begin{itemize}
	\item \textbf{Equality}: $\tor{1}{A}=\tor{2}{A} \; \Leftrightarrow \; \va{R}_1 = \va{R}_2 \text{ and } \va{M}_{1A}=\va{M}_{2A}$\vspace{0.5em}
	\item \textbf{Automoment}: $\tor{1}{A}\cdot \tor{1}{A} = 2 I_\T$\vspace{0.5em}
	\item \textbf{Mutiplication by a scalar}: $\forall \lambda\in\mathbb{R}: \tor{1}{A}=\lambda\tor{1}{A}\;\Leftrightarrow\; \va{R}_1=\lambda\va{R}_2\text{ and } \va{M}_{1A}=\lambda\va{M}_{1A}$\vspace{0.5em}
	\item \textbf{Comoment}:\mn{independent of the point $A$} $\tor{1}{A}\cdot \tor{2}{A} = \va{R}_1\cdot \va{M}_{2A}+\va{R}_2+\va{M}_{1A}$\vspace{0.5em}
\end{itemize}
\subsubsection{Central Axis of a Torsor}

\begin{definition}
	The central point $A$ of a torsor is a point where the moment $\va{M}_A$ is collinear with the resultant $\va{R}$:
	\[
		\va{M}_A\land\va{R}=0
		.\]
	and
	\[
		\va{M}_A=k\va{R} \quad k\in \mathbb{R}
		.\]
\end{definition}

\begin{definition}
	The central axis $(\Delta)$ of a torsor is the straight line of the set of central points
	\[
		\Delta =\left\{ A\;/\;\va{M}_A\land\va{R}=0 \right\}
		.\]
\end{definition}

The formula(parametric) for the central axis $\Delta $ is
\[
	(\Delta ) = \frac{\va{R}\land\va{M}_O}{\|\va{R}\|^2} + t\va{R}\quad t\in\mathbb{R}
	.\]
\begin{remark}
	Let $A\in \Delta$,
	\begin{enumerate}
		\item  If $\va{M}_A=0$ or $\va{M}_A\land\va{R}=0$ then $\Delta $ is a straight line passing through $A$ with a direction vector $\va{R}$.
		\item If $\va{M}_A\neq 0$ then $\Delta $ is a straight line passing through $B$ and a direction vector $\va{R}$ such that $\vec{AB} = \frac{\va{R}\land\va{M}_A}{\|\va{R}\|^2}$
	\end{enumerate}
\end{remark}
\subsubsection{Torsor with Scalar Invariant $=0$}

If $\T$ is a torsor with $I_\T=0$, then the torsor is $\begin{cases}\text{null}\\\text{a slider}\\\text{a couple}\end{cases}$.

\begin{itemize}
	\item \textbf{Null}: $\va{R}=\va{M}_A=0$
	\item \textbf{Slider}: $I_\T=0$, $\va{R}\neq 0$ ,and $\forall C\in\Delta \;\va{M}_C=0$
	\item \textbf{Couple}: $I_\T=0,\va{R}=0$, and $\va{M}_A \neq 0$\\
	      A couple torsor is a uniform field $\va{M}_A=\va{M}_B$ and $\cancel{\exists} \Delta$
\end{itemize}

\end{document}

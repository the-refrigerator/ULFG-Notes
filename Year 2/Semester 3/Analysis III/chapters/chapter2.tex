\part{Series of Functions}

\begin{definition}
    Let $f_n(x)$ be sequence of functions defined on $I\subset\mathbb{R}$, we define the series $S(x)$ to be 
    \[
    S(x)=\sum_{n=0}^{\infty} f_n(x)
    .\] 
\end{definition}

\section{Convergence of a Numerical Series}
In order to prove a series of functions converge we have to prove that it converges for all fixed $x$.
\begin{theorem}
    Suppose there exists a sequence $a_n$ such that $\forall x,n \; |f_n|\le a_n$. The Weierstrass test states that if $\sum a_n$ converges then $\sum f_n(x)$ converges uniformly and absolutely
\end{theorem}

\begin{theorem}
    Let $a_n$ be a sequence of numbers, if $\left| \frac{a_{n+1}}{a_n} \right| =l$ then the sequence is a geometric Series
    \[
        \sum_{n=0}^\infty a_n \begin{cases} \text{converges}&\text{if } |l|<1\\
        \text{diverges}&\text{if }|l|\ge 1\end{cases} 
    .\] 
\end{theorem}

\begin{theorem}
    A harmonic series is defined to be $a_n=\frac{1}{n^p}$
    \[
        \sum_{n=0}^\infty \frac{1}{n^p} \begin{cases} \text{converges}&\text{if } p>1\\
        \text{diverges}&\text{if }p\le 1\end{cases} 
    .\] 

\end{theorem}


\begin{theorem}
    Let $a_n$ be a sequence of numbers. The 2 series $\sum_{n=0}^{\infty} a_n$ and $\sum_{n=0}^{\infty} 2^na_n$ are simultaneously convergent/divergent.
\end{theorem}


\begin{theorem}
    The sequence $\sum_{n=0}^{\infty} (-1)^na_n$ is convergent if $a_n$ is decreasing and $\lim_{n \to \infty}a_n =0$.
\end{theorem}

\begin{theorem}
    Consider the series $S=\sum_{n=0}^{\infty} a_n$
    \[
        \lim_{n \to \infty}\sqrt[n]{|a_n|} = l\qq{such that} \begin{cases}l<1& \text{if $S$ converges}\\
        l>1& \text{if $S$ diverges}\\
l=1& \text{this test cannot help us}\\
\end{cases} 
.\]
        \end{theorem}
\section{Finite Expansion}
The general formula for the finite expansion (Taylor-young formula) is 

\[
    f(x) = f(x-a)+ \frac{x}{1!}f'(x-a)+\frac{x^2}{2!}f''(x-a)+\cdots+\frac{x^n}{n!}f^{(n)}(x-a)+x^no(1) \quad x\to a
.\]
 
Some important expansions to keep in mind are
\[\renewcommand{\arraystretch}{2.5}
\begin{array}{l|l|l}
    e^x=\sum_{n=0}^{\infty} \frac{x^n}{n!}\quad & \sin(x)=\sum_{n=0}^{\infty} (-1)^n \frac{x^{2n+1}}{(2n+1)!}&\cos(x)=\sum_{n=0}^{\infty} (-1)^n \frac{x^{2n}}{(2n)!}\\
            \frac{1}{1-x}=\sum_{n=0}^{\infty} x^n&\sinh(x)=\sum_{n=0}^{\infty} \frac{x^{2n+1}}{(2n+1)!}&\cosh(x)=\sum_{n=0}^{\infty} \frac{x^{2n}}{(2n)!}\\
            \ln(1+x)=\sum_{n=1}^{\infty} (-1)^{n-1}\frac{x^n}{n}&\ln(1-x)=\sum_{n=1}^{\infty} -\frac{x^n}{n}&(1+x)^\alpha=\sum_{n=0}^{\infty} \frac{\prod_{k=0}^{n+1}(\alpha-k)  }{n!}x^n
\end{array}
\]

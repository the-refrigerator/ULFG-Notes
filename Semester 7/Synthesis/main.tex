\documentclass{report}

\input{../template/preamble}
\input{../template/macros}
\input{../template/letterfonts}

\title{\Huge{Circuit Synthesis}\\Semester 7}
\author{\huge{Ahmad Abu Zainab}}
\date{}

\newcommand{\Laplace}[1]{\mathcal{L}\left\{#1\right\}}

\newcommand{\lap}[1]{\Laplace{#1}}


\begin{document}

\maketitle
\newpage% or \cleardoublepage
% \pdfbookmark[<level>]{<title>}{<dest>}
\pdfbookmark[section]{\contentsname}{toc}
\tableofcontents
\pagebreak

\chapter{Introduction and Characteristics of Network Functions}

\section{Network Functions}

\dfn{Network Function}{
	A network function is a function that describes the relationship between the input and output of a network.
	It is expressed in terms of the Laplace transform of the input(excitation $E(s)$) and output(response $R(s)$) signals at.

	\[
		H(s) = \frac{R(s)}{E(s)}
		.\]
}

\nt{
	The Laplace transform of a function $f(t)$ is given by
	\[
		F(s) = \lap{f(t)} = \int_{0}^{\infty} f(t) e^{-st} \dd{t}
		.\]

	with $s = \alpha + j\omega$. Where $\alpha$ is called the Neper frequency and $\omega$ is the angular frequency.
}

Network functions are divided into two types:
\begin{enumerate}
	\ii Driving Point Impedance Function: when the input and output are taken at the same port.
	\ii Transfer Function: when the input and output are taken at different ports.
\end{enumerate}

\begin{figure}[H]
	\centering
	\begin{subfigure}{0.29\textwidth}
		\centering
		\begin{circuitikz}[american voltages]
			\draw (0,0) to[short, *-, f={$i(t)$}] ++(1.5,0) to[twoport] ++(0,-1.5) to[short, -*] ++(-1.5,0);
			\draw (0,0) to[open, v^>={$v(t)$}] ++(0,-1.5);
		\end{circuitikz}
		\caption{One-port network}
	\end{subfigure}
	\begin{subfigure}{0.29\textwidth}
		\centering
		\begin{circuitikz}[american voltages]
			\draw (0,0) node[fourport, scale=1.3] (N) {};
			\draw (N.port4) to[short, f_<={$i_1(t)$}, -*] ++(-1.5,0) to[open, v^>={$v_1(t)$}] ($ (N.port1) + (-1.5,0) $) to[short, *-] ++(1.5,0);
			\draw (N.port3) to[short, f^<={$i_2(t)$}, -*] ++(1.5,0) to[open, v^>={$v_2(t)$}] ($ (N.port2) + (1.5,0) $) to[short, *-] ++(-1.5,0);
		\end{circuitikz}
		\caption{Two-port network}
	\end{subfigure}
	% \begin{subfigure}{0.29\textwidth}
	% 	\centering
	% 	\begin{circuitikz}[american voltages]
	%
	% 	\end{circuitikz}
	% 	\caption{Four-port network}
	% \end{subfigure}
\end{figure}


The driving point impedance function is the ratio of the Laplace transform of the voltage across the network port to the Laplace transform of the current through the port.
\[
	Z(s) = \frac{V(s)}{I(s)}
	.\]

The driving point admittance function is the reciprocal of the driving point impedance function.
\[
	Y(s) = \frac{1}{Z(s)} = \frac{I(s)}{V(s)}
	.\]

For a two-port network, we have four driving point impedance functions and eight transfer functions.\\

\begin{itemize}
	\ii Driving Point Impedance Functions:
	\begin{enumerate}
		\ii $Z_{11}(s) = \dfrac{V_1(s)}{I_1(s)}$
		\ii $Z_{22}(s) = \dfrac{V_2(s)}{I_2(s)}$
	\end{enumerate}

	\ii Transfer Functions:
	\begin{enumerate}
		\ii Impedance: $Z_{12}(s) = \dfrac{V_1(s)}{I_2(s)} \qq{} Z_{21}(s) = \dfrac{V_2(s)}{I_1(s)}$
		\ii Admittance: $Y_{12}(s) = \dfrac{I_1(s)}{V_2(s)} \qq{} Y_{21}(s) = \dfrac{I_2(s)}{V_1(s)}$
		\ii Voltage: $G_{12}(s) = \dfrac{V_1(s)}{V_2(s)} \qq{} G_{21}(s) = \dfrac{V_2(s)}{V_1(s)}$
		\ii Current: $\alpha_{12}(s) = \dfrac{I_1(s)}{I_2(s)} \qq{} \alpha_{21}(s) = \dfrac{I_2(s)}{I_1(s)}$
	\end{enumerate}
\end{itemize}

\subsection{General Properties of Network Functions}

The general form of a network function is given by a rational function of $s$.

\begin{align*}
	F(s) = \frac{N(s)}{D(s)} & = \frac{a_ns^n + a_{n-1}s^{n-1} + \ldots + a_1s + a_0}{b_ms^m + b_{m-1}s^{m-1} + \ldots + b_1s + b_0} \qquad a_i, b_i \in \mathbb{R} \\
	                         & = K\frac{\sum_{i=0}^{n} (s-Z_i)}{\sum_{i=0}^{m} (s-P_i)} \qquad K = \frac{a_n}{b_m}
\end{align*}

A network function satisfies the following properties:

\thm{Cauchy Riemann Conditions}{
	If a complex function $f(x+iy) = u(x,y) + jv(x,y)$ is differentiable at a point $z = x + iy$, then the following conditions must be satisfied:
	\[
		\pdv{u}{x} = \pdv{v}{y} \qq{and} \pdv{u}{y} = -\pdv{v}{x}
		.\]
}

\begin{itemize}
	\ii Complex differentiability: $F(s)$ is differentiable at all points in the complex plane except at the poles of $F(s)$.
	\ii Reflection symmetry: $F(\bar{s}) = \overline{F(s)}$
\end{itemize}

\subsubsection{Finding the Real and Imaginary Parts of a Network Function at Real Frequencies $s=j\omega$}

\begin{itemize}
	\ii \textbf{Method 1}:

	\begin{align*}
		\Re\{F(j\omega)\} & = U(\omega) = \frac{1}{2} \lt[ F(s) + F(\bar{s}) \rt]  \\
		\Im\{F(j\omega)\} & = V(\omega) = \frac{1}{2j} \lt[ F(s) - F(\bar{s}) \rt]
	\end{align*}

	\ii \textbf{Method 2}:
	\begin{itemize}
		\ii Arrange the numerator in to the sum of an even polynomial $e_N(s)$ and an odd polynomial $o_N(s)$.
		,		\ii Arrange the denominator in to the sum of an even polynomial $e_D(s)$ and an odd polynomial $o_D(s)$.
	\end{itemize}
	\begin{align*}
		\Re\{F(s)\} & = \lt.\frac{e_N(s)e_D(s) - o_N(s)o_D(s)}{{e_D}^2(s) - {o_D}^2(s)}\rt|_{s=j\omega}             \\
		\Im\{F(s)\} & = \lt.\frac{1}{j} \frac{o_N(s)e_D(s) - e_N(s)o_D(s)}{{e_D}^2(s) - {o_D}^2(s)}\rt|_{s=j\omega}
	\end{align*}

	\ii \textbf{Method 3}:

	\begin{align*}
		\Re\{F(s)\} & = \frac{1}{2} \lt[ F(s) + F(-s) \rt]  \\
		\Im\{F(s)\} & = \frac{1}{2j} \lt[ F(s) - F(-s) \rt]
	\end{align*}
\end{itemize}

\section{System Realizability and Stability}

% A network function is realizable if it can be implemented using passive elements. % idk if copilot hallucinated this
A network function is realizable if it satisfies the following conditions:
\begin{itemize}
	\ii \textbf{Causality}: the system cannot respond before the input is applied. (i.e. $f(t) = 0$ for $t < 0$)
	\ii \textbf{Stability}: the system must be stable for bounded input to produce bounded output.
\end{itemize}

\begin{table}[H]
	\renewcommand{\arraystretch}{2}
	\centering
	\begin{tabular}[c]{ll|ll}
		$\bm{f(t), t \geq 0}$     & $\bm{\lap{f(t)}}$                  & $\bm{f(t), t \geq 0}$                & $\bm{\lap{f(t)}}$                                         \\
		\hline
		$\displaystyle \delta(t)$ & 1                                  & $\displaystyle t^ne^{-at}$           & $\displaystyle \frac{n!}{(s+a)^{n+1}}$                    \\
		$\displaystyle u(t)$      & $\displaystyle \frac{1}{s}$        & $\displaystyle \sin\omega t$         & $\displaystyle \frac{\omega}{s^2 + \omega^2}$             \\
		$\displaystyle t$         & $\displaystyle \frac{1}{s^2}$      & $\displaystyle e^{-at}\sin \omega t$ & $\displaystyle \frac{\omega}{(s+a)^2 + \omega^2}$         \\
		$\displaystyle t^n$       & $\displaystyle \frac{n!}{s^{n+1}}$ & $\displaystyle e^{-at}\cos \omega t$ & $\displaystyle \frac{s+a}{(s+a)^2 + \omega^2}$            \\
		$\displaystyle e^{-at}$   & $\displaystyle \frac{1}{s+a}$      & $\displaystyle t \sin \omega t$      & $\displaystyle \frac{2\omega s}{(s^2 + \omega^2)^2}$      \\
		$\displaystyle te^{-at}$  & $\displaystyle \frac{1}{(s+a)^2}$  & $\displaystyle t \cos \omega t$      & $\displaystyle \frac{s^2 - \omega^2}{(s^2 + \omega^2)^2}$ \\
	\end{tabular}
\end{table}

The necessary and sufficient conditions for a network function to be stable are:
\begin{itemize}
	\ii All poles of the network function must lie in the left half of the complex plane. $\Re\{s\} < 0$.
	\ii The poles of the network function that lie on the imaginary axis must be simple.
	\ii The degree of the numerator $n$ must not differ from the degree of the denominator $m$ by more than one. $n-m \leq 1$.
\end{itemize}

When the poles of a network function are difficult to determine, the system stability can be determined if the denominator is a Hurwitz polynomial.

\dfn{Hurwitz Polynomial}{
	A polynomial is \emph{strictly} Hurwitz if all the roots of the polynomial have negative real parts $(<0)$.
}

\thm{Hurwitz Stability Criterion}{
	A polynomial is Hurwitz (and hence, stable) if it satisfies the \emph{all} of the following conditions:

	\begin{enumerate}
		\ii All coefficients of the polynomial are positive.
		\ii All terms between the highest and lowest degree terms should not be missing, unless they are all even or all odd(refer to the note below).
		\ii The continued fraction should give $n$ quotients before we obtain a zero remainder $(n=\deg(P(s)))$. The continued fraction expansion is obtained as follows:
		\begin{enumerate}
			\ii Arrange the polynomial $P(s) = P_e(s) + P_o(s)$ where $P_e(s)$ is the sum of the even powers of $s$ and $P_o(s)$ is the sum of the odd powers of $s$.
			\ii Evaluate $\begin{cases}
					\frac{P_e(s)}{P_o(s)} & \text{if } \deg(P(s))\text{ is even} \\
					\frac{P_o(s)}{P_e(s)} & \text{if } \deg(P(s))\text{ is odd}
				\end{cases}$
			\ii Divide the divisor(the denominator of the above expression) by the obtained remainder and repeat the process until the remainder is zero.
		\end{enumerate}
		\begin{itemize}
			\ii If the continued fraction terminates (number of terms is less than $n$), then the last obtained remainder $C(s)$ is a common factor of $P_e(s)$ and $P_o(s)$. $C(s)$ should be Hurwitz.
		\end{itemize}
		\ii All coefficients of the obtained quotients should be positive.
	\end{enumerate}

	If any of the above conditions are not satisfied, then the polynomial is not Hurwitz.
}

\nt{
	If the polynomial is entirely even or odd, we can evaluate the continued fraction expansion using the ratio of the polynomial and it's derivative.
}

Some special polynomial forms are:

\begin{enumerate}
	\ii \textbf{All pass function}: An all pass function is a function whose poles and zeros are symmetric about the imaginary axis, and the poles are strictly on the left half of the $s$-plane. If the function doesn't have nay poles or zeros on the imaginary axis then the function is strictly Hurwitz.

	\ii \textbf{Minimum phase function}: A minimum phase function is a function whose poles and zeros all located on the left half of the $s$-plane or at the imaginary axis.
\end{enumerate}

A stable network function $F(s)$ can be written as product of a minimum phase function $H(s)$ and an all pass function $P(s)$.

\[
	F(s) = H(s)P(s)
	.\]

To obtain this factorization, we assume that the set $Z$ is the set of zeros of $F(s)$ located on the right half of the $s$-plane.

\begin{align*}
	P(s) & = \frac{\prod_{\zeta \in Z} (s-\zeta)}{\prod_{\zeta \in Z} (s+\zeta)} \\
	H(s) & = \frac{F(s)}{P(s)}
\end{align*}

\section{Realization of Network Functions}

\subsection{Given its Real Part $U(s)$}

\begin{enumerate}
	\ii We find the even part of the function using $F_e(s) = U(s/j)$.
	\ii Decompose $F_e(s)$ into partial fractions
	\begin{align*}
		F_e(s) & = \frac{A_i}{a-P_i}\quad \text{obtained from poles with } \Re\{P_i\} < 0            \\
		       & + \frac{B_i}{a-Q_i}\quad \text{obtained from poles with } \Re\{Q_i\} > 0            \\
		       & + \frac{C_i}{s^2 + {\omega_i}^2}\quad \text{obtained from poles with no real parts} \\
		       & + K\;\lt(=\lim_{s \to \infty} F_e(s)\rt)
	\end{align*}
	\ii Finally evaluate
	\[
		F(s) = 2 \sum_{i} \frac{A_i}{s-P_i} + \sum_{i} \frac{C_i}{s^2 + {\omega_i}^2} + K
		.\]
\end{enumerate}

\subsection{Given its Imaginary Part $V(s)$}

\begin{enumerate}
	\ii We find the odd part of the function using $F_o(s) = jV(s/j)$.
	\ii Decompose $F_o(s)$ into partial fractions
	\begin{align*}
		F_o(s) & = \frac{A_i}{a-P_i}\quad \text{obtained from poles with } \Re\{P_i\} < 0              \\
		       & + \frac{B_i}{a-Q_i}\quad \text{obtained from poles with } \Re\{Q_i\} > 0              \\
		       & + \frac{C_i s}{s^2 + {\omega_i}^2}\quad \text{obtained from poles with no real parts} \\
		       & + K_\infty s\;\lt(K_\infty=\lim_{s \to \infty} \frac{F_o(s)}{s}\rt)
	\end{align*}
	\ii Finally evaluate
	\[
		F(s) = 2 \sum_{i} \frac{A_i}{s-P_i} + \sum_{i} \frac{C_i s}{s^2 + {\omega_i}^2} + K_\infty s
		.\]
\end{enumerate}

\subsection{Given its Magnitude $A(\omega) = |F(s)|_{s=j\omega}$}

\begin{enumerate}
	\ii Find $A^2(s/j) = F(s)F(-s)$
	\ii Find the poles and zeros of $A^2(s/j)$
	\begin{itemize}
		\ii Poles with $\Re\{P_i\} < 0$ are the poles of $F(s)$.
		\ii Poles with $\Re\{Q_i\} > 0$ are the poles of $F(-s)$.
		\ii Poles with no real parts are the poles of $F(s)$ and $F(-s)$. They should appear as poles of order 2, and should appear as poles of order 1 in $F(s)$ and $F(-s)$.
		\ii Poles at infinity are the poles of $F(s)$ and $F(-s)$. They should appear as poles of order 2, and should appear as poles of order 1 in $F(s)$ and $F(-s)$.
		\ii Zeros with $\Re\{P_i\} < 0$ are the zeros of $F(s)$ \emph{or} $F(-s)$.
		\ii Zeros with no real parts are the poles of $F(s)$ \emph{and} $F(-s)$. They should appear as zeros of an even order $m$, and should appear as poles of order $m/2$ in $F(s)$ and $F(-s)$.
	\end{itemize}
\end{enumerate}

\chapter{Positive Real Functions}

\dfn{Positive Real Function}{
	A network function $F(s)$ is said to be positive real if it satisfies the following conditions:
	\begin{enumerate}
		\ii $F(s)$ is real for all real values of $s=\alpha$.
		\ii $F(s)$ is positive real for all real values of $s$. (i.e. $\Re\{F(s)\}\geq 0$ for $\Re\{s\}\geq 0$)
	\end{enumerate}
}

Some properties of positive real functions are:

\begin{enumerate}
	\ii If $F(s)$ is positive real, then $1/F(s)$ is also positive real.
	\ii If $F(s)$ and $G(s)$ are positive real, then any linear combination of $F(s)$ and $G(s)$ is also positive real. ($aF(s) + bG(s)$ where $a,b \geq 0$).
	\ii If $F(s)$ is positive real, then the poles and zeros of $F(s)$ are located in the left half of the $s$-plane, and are real or occur in conjugate pairs.
	\ii If $F(s)$ is positive real, then its imaginary poles should be simple and their residues should be real positive.
	\begin{itemize}
		\ii A residue is the coefficient of the term $(s-P_i)^{-1}$ in the partial fraction expansion of $F(s)$.
	\end{itemize}
	\ii If $F(s)$ is positive real rational function $= \frac{N(s)}{D(s)}$, then
	\begin{itemize}
		\ii The degrees of both polynomials should not differ by more than one. $\abs{\deg N(s) - \deg D(s)} \leq 1$.
		\ii The lowest degree terms of both polynomials should not differ by more than one. $\abs{\deg_{\min} N(s) - \deg_{\min} D(s)} \leq 1$.
	\end{itemize}
\end{enumerate}

\thm{Necessary and Sufficient Conditions for a Rational Function to be Positive Real}{
	A rational function $F(s) = \frac{N(s)}{D(s)}$ is positive real if and only if the following conditions are satisfied:
	\begin{enumerate}
		\ii All terms of $N(s)$ and $D(s)$ are positive real.
		\ii $\abs{\deg N(s) - \deg D(s)} \leq 1$.
		\ii $\abs{\deg_{\min} N(s) - \deg_{\min} D(s)} \leq 1$.
		\ii $N(s)$ and $D(s)$ are Hurwitz polynomials.
		\ii The residues of the imaginary poles of $F(s)$ are real and positive.
		\ii $\Re\{F(s)_{s=\pm j\omega}\} \geq 0$ for all $\omega$.
	\end{enumerate}
}

\chapter{Elementary Synthesis Concepts of Driving Point Functions}

The basic principle of network synthesis(positive real function) is to realize a given network function using a combination of passive elements, which can be represented as a sum of simpler positive real functions which can be more easily synthesized.

\[
	\underbrace{Z(s) = Z_1(s) + Z_2(s) + \ldots + Z_n(s)}_{\text{Elements in series}} \quad \quad \underbrace{Y(s) = Y_1(s) + Y_2(s) + \ldots + Y_n(s)}_{\text{Elements in parallel}}
\]

\begin{figure}[H]
	\centering
	\begin{subfigure}{0.4\textwidth}
		\centering
		\begin{circuitikz}[american voltages]
			\tikzset{circuitikz/resistors/scale=0.7}
			\draw (0,0) node[fourport, scale=1.3, t={\scalebox{0.76}{$Z_2(s)$}}] (N) {};
			\draw (N.port4) to[generic=$Z_1(s)$, -*] ++(-2.5,0) ($ (N.port1) + (-2.5,0) $) to[short, *-] (N.port1);
			\node (b) at ($($ (N.port4) + (-2.5,0) $)!0.5!($ (N.port1) + (-2.5,0) $)$) {$Z(s)$};
		\end{circuitikz}
		\caption{Elements in series}
	\end{subfigure}
	\begin{subfigure}{0.4\textwidth}
		\centering
		\begin{circuitikz}[american voltages]
			\tikzset{circuitikz/resistors/scale=0.7}
			\draw (0,0) node[fourport, scale=1.3, t={\scalebox{0.76}{$Y_2(s)$}}] (N) {};
			\draw (N.port4) to[short, -*] ++(-2.5,0) ($ (N.port1) + (-2.5,0) $) to[short, *-] (N.port1);
			\node (b) at ($($ (N.port4) + (-2.5,0) $)!0.5!($ (N.port1) + (-2.5,0) $)$) {$Y(s)$};
			\draw (N.port4) ++(-1.25,0) to[generic=$Y_1(s)$] ($ (N.port1) + (-1.25,0) $);
		\end{circuitikz}
		\caption{Elements in parallel}
	\end{subfigure}
\end{figure}

% Some basic removal operations are:
%
% \begin{enumerate}
% 	\ii Removal of a pole at infinity ($F_i(s)=K_\infty s$):
% 	\[
% 		\underbrace{F_i(s) = sL}_{\text{if } F(s) = Z(s)} \quad \quad \underbrace{F_i(s) = sC}_{\text{if } F(s) = Y(s)}
% 	\]
%
% 	\ii Removal of a pole at the origin ($F_i(s)=\frac{K_0}{s}$):
% 	\[
% 		\underbrace{F_i(s) = \frac{1}{sC} = \frac{K_0}{s}}_{\text{if } F(s) = Z(s)} \quad \quad \underbrace{F_i(s) = \frac{1}{sL} = \frac{K_0}{sC}}_{\text{if } F(s) = Y(s)}
% 	\]
%
% 	\ii Removal of conjugate imaginary poles ($F_i(s)=\frac{As}{s^2 + \omega^2}$):
% 	\[
% 		\underbrace{capacitance and an inductance in parallel}_{\text{if } F(s) = Z(s)} \quad \quad \underbrace{capacitance and an inductance in series}_{\text{if } F(s) = Y(s)}
% 	\]
%
% 	\ii Removal of a constant term ($F_i(s)=K$):
% 	\[
% 		\underbrace{F_i(s) = R = K}_{\text{if } F(s) = Z(s)} \quad \quad \underbrace{F_i(s) = G = \frac{1}{K}}_{\text{if } F(s) = Y(s)}
% 	\]
% \end{enumerate}

\section{Basic Removal Operations}

\subsection{Removal of a Pole at Infinity}

\[
	F(s) = \frac{\sum_{i=0}^{n} a_is^i}{\sum_{i=0}^{m} b_is^i}
	.\]

If $m-n=1$, then we have a pole at infinity.

\[
	F(s) = K_\infty s + F_1(s) \qq{with} K_\infty = \lim_{s \to \infty} \frac{F(s)}{s} = \frac{a_n}{b_m}
	.\]


\begin{figure}[H]
	\centering
	\begin{subfigure}{0.4\textwidth}
		\centering
		\begin{circuitikz}[american voltages]
			\tikzset{circuitikz/resistors/scale=0.7}
			\tikzset{circuitikz/inductors/scale=0.7}
			\tikzset{circuitikz/capacitors/scale=0.7}
			\draw (0,0) node[fourport, scale=1.3] (N) {};
			\draw (N.port4) to[L=$L$,mirror, -*] ++(-2.5,0) ($ (N.port1) + (-2.5,0) $) to[short, *-] (N.port1);
			\node (b) at ($($ (N.port4) + (-2.5,0) $)!0.5!($ (N.port1) + (-2.5,0) $)$) {$Z(s)$};
		\end{circuitikz}
		\caption{Impedance Network}
	\end{subfigure}
	\begin{subfigure}{0.4\textwidth}
		\centering
		\begin{circuitikz}[american voltages]
			\tikzset{circuitikz/resistors/scale=0.7}
			\tikzset{circuitikz/inductors/scale=0.7}
			\tikzset{circuitikz/capacitors/scale=0.7}
			\draw (0,0) node[fourport, scale=1.3] (N) {};
			\draw (N.port4) to[short, -*] ++(-2.5,0) ($ (N.port1) + (-2.5,0) $) to[short, *-] (N.port1);
			\node (b) at ($($ (N.port4) + (-2.5,0) $)!0.5!($ (N.port1) + (-2.5,0) $)$) {$Y(s)$};
			\draw (N.port4) ++(-1.25,0) to[C=$C$] ($ (N.port1) + (-1.25,0) $);
		\end{circuitikz}
		\caption{Admittance Network}
	\end{subfigure}
	\caption*{Removal of a pole at infinity $\displaystyle L=C=K_\infty=\frac{a_n}{b_m}$}
\end{figure}

\subsection{Removal of a Pole at the Origin}

\[
	F(s) = \frac{\sum_{i=0}^{n} a_is^i}{\sum_{i=1}^{m} b_is^i}
	.\]

If we are able to factor out an $s$ from the denominator ($b_0=0$) then we have a pole at the origin.

\[
	F(s) = \frac{K_0}{s}  + F_1(s) \qq{with} K_0 = \lim_{s \to 0} sF(s) = \frac{a_0}{b_1}
	.\]


\begin{figure}[H]
	\centering
	\begin{subfigure}{0.4\textwidth}
		\centering
		\begin{circuitikz}[american voltages]
			\tikzset{circuitikz/resistors/scale=0.7}
			\tikzset{circuitikz/inductors/scale=0.7}
			\tikzset{circuitikz/capacitors/scale=0.7}
			\draw (0,0) node[fourport, scale=1.3] (N) {};
			\draw (N.port4) to[C=$C$,mirror, -*] ++(-2.5,0) ($ (N.port1) + (-2.5,0) $) to[short, *-] (N.port1);
			\node (b) at ($($ (N.port4) + (-2.5,0) $)!0.5!($ (N.port1) + (-2.5,0) $)$) {$Z(s)$};
		\end{circuitikz}
		\caption{Impedance Network}
	\end{subfigure}
	\begin{subfigure}{0.4\textwidth}
		\centering
		\begin{circuitikz}[american voltages]
			\tikzset{circuitikz/resistors/scale=0.7}
			\tikzset{circuitikz/inductors/scale=0.7}
			\tikzset{circuitikz/capacitors/scale=0.7}
			\draw (0,0) node[fourport, scale=1.3] (N) {};
			\draw (N.port4) to[short, -*] ++(-2.5,0) ($ (N.port1) + (-2.5,0) $) to[short, *-] (N.port1);
			\node (b) at ($($ (N.port4) + (-2.5,0) $)!0.5!($ (N.port1) + (-2.5,0) $)$) {$Y(s)$};
			\draw (N.port4) ++(-1.25,0) to[L=$L$] ($ (N.port1) + (-1.25,0) $);
		\end{circuitikz}
		\caption{Admittance Network}
	\end{subfigure}
	\caption*{Removal of a pole at the origin $\displaystyle \frac{1}{L}=\frac{1}{C}=K_0 = \frac{a_0}{b_1}$}
\end{figure}

\subsection{Removal of Conjugate Imaginary Poles}

\[
	F(s) = \frac{N(s)}{(s^2 + {\omega_p}^2) D'(s)}
	.\]

If we can factor out $(s^2 + {\omega_p}^2)$ from the denominator, then we have conjugate imaginary poles.

\[
	F(s) = \frac{K_p}{s-j\omega_p} + \frac{K_p}{s+j\omega_p} + F_1(s) = \frac{2K_ps}{s^2 + {\omega_p}^2} + F_1(s) \qq{with} K_p = \lim_{s\to j\omega_p} F(s)
	.\]

\begin{figure}[H]
	\centering
	\begin{subfigure}{0.4\textwidth}
		\centering
		\begin{circuitikz}[american voltages]
			\tikzset{circuitikz/resistors/scale=0.7}
			\tikzset{circuitikz/inductors/scale=0.7}
			\tikzset{circuitikz/capacitors/scale=0.7}
			\draw (0,0) node[fourport, scale=1.3] (N) {};
			\draw (N.port4) -- ++(-0.5,0) coordinate (A) ++(-1.5,0) coordinate (B) to[short, -*] ++(-0.5,0) ($ (N.port1) + (-2.5,0) $) to[short, *-] (N.port1);
			\node (b) at ($($ (N.port4) + (-2.5,0) $)!0.5!($ (N.port1) + (-2.5,0) $)$) {$Z(s)$};

			\draw (A) -- ++(0,0.4) to[L,l_=$L$, mirror] ++(-1.5,0) -- (B);
			\draw (A) -- ++(0,-0.4) to[C=$C$] ++(-1.5,0) -- (B);
		\end{circuitikz}
		\caption{Impedance Network\\
			$\displaystyle L = \frac{2K}{{\omega_p}^2} \qq{and} C = \frac{1}{2K}$}
	\end{subfigure}
	\begin{subfigure}{0.4\textwidth}
		\centering
		\begin{circuitikz}[american voltages]
			\tikzset{circuitikz/resistors/scale=0.5}
			\tikzset{circuitikz/inductors/scale=0.5}
			\tikzset{circuitikz/capacitors/scale=0.5}
			\draw (0,0) node[fourport, scale=1.3] (N) {};
			\draw (N.port4) to[short, -*] ++(-2.5,0) ($ (N.port1) + (-2.5,0) $) to[short, *-] (N.port1);

			\node (b) at ($($ (N.port4) + (-2.5,0) $)!0.5!($ (N.port1) + (-2.5,0) $)$) {$Y(s)$};

			\coordinate (M) at ($($ (N.port4) + (-1.25,0) $)!0.5!($ (N.port1) + (-1.25,0) $)$);

			\draw (N.port4) ++(-1.25,0) to[L=$L$] (M) to[C=$C$] ($ (N.port1) + (-1.25,0) $);
		\end{circuitikz}
		\caption{Admittance Network\\
			$\displaystyle C = \frac{2K}{{\omega_p}^2} \qq{and} L = \frac{1}{2K}$}
	\end{subfigure}
	\caption*{Removal of Conjugate Imaginary Poles}
\end{figure}

\subsection{Removal of a Constant}

\[
	F(j\omega) = U(\omega) + jV(\omega)
	.\]

If $U(\omega)$ is minimal at some $\omega_r$, then we can remove a constant $K\leq K_r$ from $U(\omega)$ so that the remainder is still positive real.

\[
	F(s) = K_r + F_1(s) \qq{with} K_r = U(\omega_r) \qq{such that} \lt.\dv{U(\omega)}{\omega}\rt|_{\omega=\omega_r} = 0
	.\]


\begin{figure}[H]
	\centering
	\begin{subfigure}{0.4\textwidth}
		\centering
		\begin{circuitikz}[american voltages]
			\tikzset{circuitikz/resistors/scale=0.7}
			\tikzset{circuitikz/inductors/scale=0.7}
			\tikzset{circuitikz/capacitors/scale=0.7}
			\draw (0,0) node[fourport, scale=1.3] (N) {};
			\draw (N.port4) to[R=$R$,mirror, -*] ++(-2.5,0) ($ (N.port1) + (-2.5,0) $) to[short, *-] (N.port1);
			\node (b) at ($($ (N.port4) + (-2.5,0) $)!0.5!($ (N.port1) + (-2.5,0) $)$) {$Z(s)$};
		\end{circuitikz}
		\caption{Impedance Network}
	\end{subfigure}
	\begin{subfigure}{0.4\textwidth}
		\centering
		\begin{circuitikz}[american voltages]
			\tikzset{circuitikz/resistors/scale=0.7}
			\tikzset{circuitikz/inductors/scale=0.7}
			\tikzset{circuitikz/capacitors/scale=0.7}
			\draw (0,0) node[fourport, scale=1.3] (N) {};
			\draw (N.port4) to[short, -*] ++(-2.5,0) ($ (N.port1) + (-2.5,0) $) to[short, *-] (N.port1);
			\node (b) at ($($ (N.port4) + (-2.5,0) $)!0.5!($ (N.port1) + (-2.5,0) $)$) {$Y(s)$};
			\draw (N.port4) ++(-1.25,0) to[R=$G$] ($ (N.port1) + (-1.25,0) $);
		\end{circuitikz}
		\caption{Admittance Network}
	\end{subfigure}
	\caption*{Removal of a pole at the origin $\displaystyle R=K\quad G=\frac{1}{K}$}
\end{figure}


\end{document}

\documentclass{report}

\input{../template/preamble}
\input{../template/macros}
\input{../template/letterfonts}

\title{\Huge{Digital Signal Processing}\\Semester 7}
\author{\huge{Ahmad Abu Zainab}}
\date{}

\newcommand{\scalar}[1]{\lt\langle #1 \rt\rangle}
\newcommand{\Ft}[1]{\SF_t\!\lt\{ #1 \rt\}}
\newcommand{\Fti}[1]{\SF_t^{-1}\!\lt\{ #1 \rt\}}
\newcommand{\Ht}[1]{\SH\!\lt\{ #1 \rt\}}

\renewcommand{\EE}[1]{\mathbb{E}\lt[ #1 \rt]}

\DeclareMathOperator{\rect}{rect}
\DeclareMathOperator{\tri}{tri}
\DeclareMathOperator{\sinc}{sinc}
\DeclareMathOperator{\sgn}{sgn}

\usepgfplotslibrary{fillbetween}

\pgfplotsset{compat=1.18}

\pgfmathdeclarefunction{u}{1}{
\pgfmathparse{
((#1<0)  * 0 )  +
((#1>=0) * 1 )
}
}

\pgfmathdeclarefunction{rect}{2}{
\pgfmathparse{
((#1>= -#2/2) * (#1<=#2/2) * 1 )
}
}

\pgfmathdeclarefunction{tri}{2}{
\pgfmathparse{
((#1>= -#2) * (#1<=#2) * (1-abs(#1)/#2) ) 
}
}

\pgfmathdeclarefunction{sinc}{1}{
\pgfmathparse{
    sin(deg(pi*#1))/(pi*#1)
}
}

\pgfmathdeclarefunction{sgn}{1}{
\pgfmathparse{
    (#1>0)  -  (#1<0)
}
}

\pgfmathdeclarefunction{delta}{1}{
\pgfmathparse{
    (#1==0) ? 1 : 0
}
}


\begin{document}

\maketitle
\newpage% or \cleardoublepage
% \pdfbookmark[<level>]{<title>}{<dest>}
\pdfbookmark[section]{\contentsname}{toc}
\tableofcontents
\pagebreak

\chapter{General Definitions for Digital Signals}

\dfn{Digital Signal}{
	A digital signal is a sequence of numbers $x[n]$ where $n\in\ZZ$. The signal is said to be discrete in time.
	The signal can be represented as a function of time $x(t)$ where $t=nT_s$ and $T_s$ is the sampling period $x[n]=x(nT_s)$.

	\begin{center}
		\begin{tikzpicture}
			\begin{axis}[
					% axis lines = middle,
					xlabel = {$t$},
					ylabel = {$x[n]$},
					% xmin = -5,
					% xmax = 5,
					% ymin = 0,
					% ymax = 1,
					% xtick = {-4,-3,-2,-1,0,1,2,3,4},
					% ytick = {-1,0,1},
				]

				% add sine wave
				\addplot[domain=-5:5, samples=100, red, dashed] {sin(deg(pi*x))};

				% add dots on the same sine wave
				\addplot[domain=-5:5, samples=22, only marks, mark=*, mark size=2pt, blue] {sin(deg(pi*x))};

				\addlegendentry{$x(t)$};
				\addlegendentry{$x[n]$};
			\end{axis}
		\end{tikzpicture}
	\end{center}
}

\subsubsection{Power and Energy}

The energy of a signal is defined as

\[
	E_x = \sum_{n = -\infty}^{\infty} \abs{x[n]}^2
	.\]

And the power as

\[
	P_x = \lim_{N\to\infty} \frac{\sum_{n = -N}^{N} \abs{x[n]}^2}{2N+1}
	.\]

\subsubsection{Periodic Signal}

A digital periodic signal is a signal $x[n]$ with a period $N\in\ZZ$ such that $\forall n\in\ZZ x(n+N)=x[n]$.

\ex{}{
	For a periodic signal $\cos(\pi n_0 n)$ the condition for periodicity becomes

	\[
		\cos(\pi n_0 n) = \cos(\pi n_0 (n+N))
		.\]

	\begin{align*}
		\pi n_0 N & = 2k\pi          \\
		N         & = \frac{2k}{n_0}
	\end{align*}

	We choose the smallest possible $k\in\ZZ$ to obtain an $N\in\ZZ$.
}

\subsubsection{Even and Odd}

\begin{align*}
	x[n] & = x[-n] \quad \text{Even} \\
	x[n] & = -x[-n] \quad \text{Odd}
\end{align*}

\[
	x[n] = \underbrace{\frac{x[n] + x[-n]}{2}}_{\text{Even}} + \underbrace{\frac{x[n] - x[-n]}{2}}_{\text{Odd}}
	.\]

\subsubsection{Usually Used Signals}

\[
	\delta[n] = \begin{cases}
		1 & n = 0    \\
		0 & n \neq 0
	\end{cases}      \qq{;}
	u[n] = \begin{cases}
		1 & n \geq 0 \\
		0 & n < 0
	\end{cases}   \qq{;}
	\sgn[n] = \begin{cases}
		1  & n \geq 0 \\
		-1 & n < 0
	\end{cases}     \qq{;}
	x[n] = a^n \quad a\in\CC
\]

\section{Linear Time Invariant Systems}

For a LTI system, an input $x[n]$ will produce an output $y[n]$ such that the relation can be expressed as

\begin{enumerate}
	\ii Analytically: E.g. $y[n] = x[n] - 2x[n-1]$.
	\ii As a function of the impulse response $h[n]$: $y[n] = f\lt( x[n], x[n-1],\ldots,h[n], h[n-1], \ldots \rt)$.
\end{enumerate}

\nt{
	A system is said to be relaxed if the output is zero for zero input. (i.e. the only energy source is the input)
}

An input $x[n]$ can be decomposed into a sum of shifted impulses

\[
	x[n] = \sum_{k=-\infty}^{\infty} x[k]\delta[n-k]
	.\]

If a system is linear, then the output of the system to the input $x[n]$ can be expressed as a sum of the outputs to the shifted impulses

\[
	y[n] = \sum_{k=-\infty}^{\infty} x[k]h[n-k] = \underbrace{x[n]*h[n]}_{\text{Convolution}}
	.\]

\nt{
The convolution of two signals $x[n]$ and $h[n]$ is commutative
\[
	x[n]*h[n] = h[n]*x[n]
	.\]
}

If the system is causal, then the impulse response $h[n]$ is zero for $n<0$. The output of the system can be expressed as

\[
	y[n] = \sum_{k=0}^{\infty} h[k]x[n-k] = \sum_{k=-\infty}^{n} x[k]h[n-k]
	.\]

If both the input and the impulse response are causal, then the output can be expressed as

\[
	y[n] = \sum_{k=0}^{n} x[k]h[n-k] = \sum_{k=0}^{n} h[k]x[n-k]
	.\]

\subsection{Stability}

A system is said to be stable if the output is bounded for a bounded input.

\[
	\abs{x[n]} \leq \alpha \quad \forall n \implies \abs{y[n]} \leq \beta
	.\]

\[
	\abs{y[n]} \leq \sum_{k=-\infty}^{\infty} \alpha\abs{h[n-k]}
	.\]

So a system is stable if it satisfies the sufficient condition

\[
	\sum_{n=-\infty}^{\infty} \abs{h[n]} \leq \gamma
	.\]

\dfn{Difference Equation}{
	A difference equation is a relation between the input $x[n]$, the output $y[n]$ and the impulse response $h[n]$ of a system. The relation can be expressed as

	\[
		y[n] = - \sum_{k=0}^{N} a_k y[n-k] + \sum_{k=0}^{M} b_k x[n-k]
		.\]

	where $a_k$ and $b_k$ are constants $\in\CC$.

	\begin{itemize}
		\ii If $N=0$, the system is said to be a moving average system.
		\ii If $M=0$, the system is said to be an autoregressive system.
	\end{itemize}
}

\section{Response of LTI Systems to an input $x[n]$}

There are three methods to find the output of an LTI system to an input $x[n]$.

\begin{enumerate}
	\ii Integration (point by point)
	\ii Convolution (using the impulse response)
	\[
		y[n] = \sum_{k=-\infty}^{\infty} x[k]h[n-k]
		.\]
	\ii Finding the analytical expression for the output
\end{enumerate}

\subsection{Analytical Expression}

The analytical expression of $y[n]$ is the sum of the terms

\[
	y[n] = y_\text{ZI}[n] + y_\text{ZS}[n]
	.\]

Where $y_\text{ZI}[n]$ is the zero input response (the response of the system to the input $x[n]=0$) and $y_\text{ZS}[n]$ is the zero state response (the response of the system to the input $x[n]$ when the system is relaxed).

\subsubsection{Characteristic Equation}

The characteristic equation of a difference equation is the equation obtained by setting the output to zero and substituting $y[n] = \lambda^n$.
The roots $\{\lambda_1, \lambda_2, \ldots, \lambda_N\}$ of the characteristic equation are the poles of the system, and are useful for computing the ZIR and ZSR.

\[
	\lambda^n = \sum_{k=1}^{N} a_k \lambda^{n-k}
	.\]

\subsubsection{Zero Input Response}

The zero input response is the response to the system when the input is zero.
Due to the superposition principle, the zero input response is always present in the output of the system regardless of the input.
To find the zero input response, we set $x[n]=0$ in the difference equation.

\[
	y_\text{ZI}[n] = \sum_{k=1}^{N} C_k \lambda_k^n \qq{where} C_k \text{ is a $k$-th order polynomial} = c_0 + c_1n + \cdots + c_k n^k = \sum_{j=0}^{k} c_jn^j
	.\]

The coefficients $c_j$ can be found by substituting the initial conditions $y[0], y[1], \ldots, y[N-1]$ into the difference equation.

\subsubsection{Zero State Response}

The zero state response is the response of the system to the input $x[n]$ when the system is relaxed. I.E. initial conditions are zero.

\[
	y_\text{ZS}[n] = y_h[n] + y_p[n]
	.\]

Where $y_h[n]$ shares the same form as the zero input response, and $y_p[n]$ is the particular solution to the difference equation for the input $x[n]$.

\[
	y_h[n] = \sum_{k=1}^{N} C_k \lambda_k^n
	.\]

The coefficients $C_k$ can be found by substituting $n$ into the difference equation and solving the system of equations.\\

The zero state response can be found by solving the difference equation with the input $x[n]$ and substituting $y[n] = kx[n]$.

\[
	y_p[n] = kx[n] = -k \sum_{k=1}^{N} a_k x[n-k] + \sum_{k=0}^{M} b_k x[n-k]
	.\]

Finding $k$ is done by substituting $n$ and solving for $k$.

\subsubsection{Multiple Inputs}

If the input is a sum of signals $x[n] = \sum_{i=1}^{N} x_i[n]$, then the output can be expressed as

\[
	y[n] = y_\text{ZI}[n] + \sum_{i=1}^{N} y_{\text{ZS}_i}[n]
	.\]


\ex{}{
	Consider a system with initial conditions $y[-1] = 1$ and $y[-2] = -1$, an input $x[n] = (0.2)^n$, and a difference equation

	\[
		y[n] = 0.6y[n-1] - 0.09y[n-2] + x[n]
		.\]

	Find the output $y[n]$.\\

	\textbf{Solution:}
	\begin{enumerate}
		\ii Find the characteristic equation and $\lambda_1$ and $\lambda_2$.

		\begin{align*}
			\lambda^n & = 0.6\lambda^{n-1} - 0.09\lambda^{n-2} \\
			\lambda^2 & = 0.6\lambda - 0.09
		\end{align*}
		\[
			\therefore \lambda_1 = \lambda_2 = 0.3
			.\]

		\ii Find the zero input response $y_\text{ZI}[n]$.

		\[
			y_\text{ZI}[n] = (C_1 + C_2n)0.3^n
			.\]
		Substituting the initial conditions

		\begin{align*}
			y[-1] & : (C_1 - C_2)0.3^{-1} = 1    \\
			y[-1] & : (C_1 - 2 C_2)0.3^{-2} = -1
		\end{align*}

		Solving the system of equations we obtain

		\[
			\therefore y_\text{ZI}[n] = (0.69 + 0.39n)0.3^n
			.\]

		\ii Find the zero state response $y_\text{ZS}[n]$.
		\begin{enumerate}
			\ii Find the particular solution $y_h[n]$.
			\[
				y_h[n] = (C_1 + C_2n)0.3^n
				.\]

			\begin{align*}
				y_h[0] & : C_1 = 0.6(0) - 0.09(0) + (0.2)^0            \\
				y_h[1] & : (C_1 + C_2)0.3 = 0.6(1) - 0.09(0) + (0.2)^1
			\end{align*}

			\[
				\therefore y_h[n] = \lt(1 + \frac{5}{3}n\rt)0.3^n
				.\]

			\ii Find the particular solution $y_p[n]$.
			\[
				y_p[n] = k(0.2)^n
				.\]

			\begin{align*}
				k(0.2)^n & = 0.6k(0.2)^{n-1} - 0.09k(0.2)^{n-2} + (0.2)^n     \\
				k        & = 0.6k(0.2)^{-1} - 0.09k(0.2)^{-2} + 1  \qq{;} n=0 \\
				k        & = 4
			\end{align*}

			\[
				\therefore y_p[n] = 4(0.2)^n
				.\]
		\end{enumerate}

		\[
			\therefore y_\text{ZS}[n] = \lt(1 + \frac{5}{3}n\rt)0.3^n + 4(0.2)^n
			.\]
	\end{enumerate}

	Finally, the output is
	\[
		y[n] = (0.69 + 0.39n)0.3^n + \lt(1 + \frac{5}{3}n\rt)0.3^n + 4(0.2)^n
		.\]
}

\section{Multiple LTI Systems}

\begin{figure}[H]
	\centering
	\begin{circuitikz}
		\draw (0,0) to[twoport,>,t=\shortstack{$S_1$\\$h_1$}] (2,0) to[twoport,>,t=\shortstack{$S_2$\\$h_2$}] (4,0);
		\node at (5,0) {$\equiv$};
		\node[below] at (5,0) {Series};
		\draw (6,0) to[twoport,>,t=\shortstack{$S$\\$h_1\!*\!h_2$}] (8,0);

		\begin{scope}[yshift=-3cm]
			\draw (0,0) -| (0.5,1) to[twoport, >,t=\shortstack{$S_1$\\$h_1$}] (3,1) |- (4,0);
			\draw (0,0) -| (0.5,-1) to[twoport, >,t=\shortstack{$S_2$\\$h_2$}] (3,-1) |- (4,0);

			\node[adder, fill=white, scale=0.7] at (3,0) (m) {};
			\draw (m.n) node[inputarrow, rotate=-90] {};
			\draw (m.s) node[inputarrow, rotate=90] {};

			\node at (5,0) {$\equiv$};
			\node[below] at (5,0) {Parallel};

			\draw (6,0) to[twoport,>,t=\shortstack{$S$\\$h_1\!\!+\!h_2$}] (8,0);
		\end{scope}
	\end{circuitikz}
\end{figure}

\subsubsection{Types of systems}

Denote by $\mathcal{S}$ a system with impulse response $h[n]$. Then

\begin{itemize}
	\ii $\mathcal{S}$ is said to be \emph{causal} if $h[n]=0$ for $n<0$ or if $y[n] = f\lt\{ x[k], x[k-1], \ldots \rt\}$ where $n\geq k$.
	\ii $\mathcal{S}$ is said to be \emph{stable} if $|x[n]| \leq \alpha \implies \abs{\mathcal{S}\lt\{ x[n] \rt\}} \leq \beta$.
	\ii $\mathcal{S}$ is said to be \emph{invertible} if there exists a system $\mathcal{S}^{-1}$ such that $\mathcal{S}^{-1}\mathcal{S} = \delta[n]$.
	\ii $\mathcal{S}$ is said to be \emph{linear} if $\mathcal{S}\lt\{ a_1x_1[n] + a_2x_2[n] \rt\} = a_1\mathcal{S}\lt\{ x_1[n] \rt\} + a_2\mathcal{S}\lt\{ x_2[n] \rt\}$.
	\ii $\mathcal{S}$ is said to be \emph{time invariant} if $\mathcal{S}\lt\{ x[n] \rt\} = y[n] \implies \mathcal{S}\lt\{ x[n-k] \rt\} = y[n-k]$.
\end{itemize}

\chapter{Frequency Analysis of Signals}

\section{Fourier Transform}

\dfn{Discrete Fourier Transform}{
The Fourier Transform of a signal $x[n]$ is defined as

\[
	X(f) = \Ft{x[n]} = \sum_{n=-\infty}^{\infty} x[n]e^{-j\omega n}
	.\]
}

The Fourier Transform of a signal $x[n]$ exists if the signal is absolutely summable. I.e.
\[
	\sum_{n=-\infty}^{\infty} \abs{x[n]} < \infty
	.\]

\subsubsection{Properties}

\begin{itemize}
	\ii Linearity: $\SF\lt\{ a_1x_1[n] + a_2x_2[n] \rt\} = a_1X_1(f) + a_2X_2(f)$.
	\ii Time Shifting: $\SF\lt\{ x[n-k] \rt\} = e^{-j\omega k}X(f)$.
	\ii Inversion: $\SF\lt\{ x[-n] \rt\} = X(-f)$.
	\ii Frequency Shifting: $\SF\lt\{ e^{j\omega_0 n}x[n] \rt\} = X(f-f_0)$.
	\ii Convolution: $\SF\lt\{ x[n]*h[n] \rt\} = X(f)\cdot H(f)$.
	\ii Multiplication: $\SF\lt\{ x[n]\cdot h[n] \rt\} = X(f)*H(f)$.
	\ii Derivative: $\displaystyle\SF\lt\{ nx[n] \rt\} = \frac{j}{2\pi}\dv{X(f)}{f}$.
\end{itemize}

\subsubsection{Power Spectral Density}

The power spectral density of a signal $x[n]$ is defined as

\[
	S_x(f) = \abs{X(f)}^2 = \dv{E}{f}
	.\]

\section{System Response}

Let $H(f)$ be the Transfer Function of a system.

\[
	H(f) = \frac{Y(f)}{X(f)} = \frac{\SF\lt\{ y[n] \rt\}}{\SF\lt\{ x[n] \rt\}}
	.\]

Take for example the difference equation

\begin{align*}
	y[n]              & = - \sum_{k=1}^{N} a_k y[n-k] + \sum_{k=0}^{M} b_k x[n-k]                         \\
	Y(f)              & = - \sum_{k=1}^{N} a_k e^{-j\omega k}Y(f) + \sum_{k=0}^{M} b_k e^{-j\omega k}X(f) \\
	\frac{Y(f)}{X(f)} & = \frac{\sum_{k=0}^{M} b_k e^{-j\omega k}}{1 + \sum_{k=1}^{N} a_k e^{-j\omega k}}
\end{align*}

\subsection{Frequency Response}

The frequency response of a system to an input $x[n]$ is defined as

\[
	|y[n]| = |x[n]| \cdot |H(f_0)| \qand \arg\lt\{ y[n] \rt\} = \arg\lt\{ x[n] \rt\} + \arg\lt\{ H(f_0) \rt\}
	.\]

Where $f_0$ is the frequency of the input signal.\\

For example, if $x[n] = 3 \sin(\frac{\pi}{5}n + \frac{2\pi}3)$ and $H(f)|_{s=\frac{1}{10}} = 2e^{j\frac{\pi}{3}}$, then the output will be

\[
	y[n] = 6 \sin\lt( \frac{\pi}{5}n + \frac{2\pi}{3} + \frac{\pi}{3} \rt)
	.\]

\nt{
	The frequency of a constant signal is $0$.
}


\end{document}

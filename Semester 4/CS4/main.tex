\documentclass{report}

\input{../template/preamble}
\input{../template/macros}
\input{../template/letterfonts}


\title{\Huge{Computer Science 4}\\Semester 4}
\author{}
\date{}

\begin{document}

\maketitle
\newpage% or \cleardoublepage
% \pdfbookmark[<level>]{<title>}{<dest>}
\pdfbookmark[section]{\contentsname}{toc}
\tableofcontents
\pagebreak


\chapter{The Fall of Communism}
The most basic C\# program (equivelant to \mintinline{c}|int main(void)| in C) is
\inputminted{csharp}{./code/basic.cs}


\section{Similarities with C}

\begin{minipage}{0.4\linewidth}
	\begin{minted}{csharp}
/* Variable assignment */
int x = 5, i = 0;
float y = 4.5;
double z = 3.1415;
char c = 'A';

/* Basic Operations */
x + y;
x * y;
x++;
etc...

/* Conditionals */
if (x == y) {
    /*...*/
} else if (x < y) {
    /*..*/
} else {
    /*...*/
}
    \end{minted}
\end{minipage}
\hfill
\begin{minipage}{0.4\linewidth}
	\begin{minted}{csharp}
/* Ternary Operator */
int w = (x < y)? y : x;

/* Loops */
while (i < 6) {
    /*...*/
}

do {
    /*...*/
} while (i < 6);

for (int i = 0; i < 6; i++) {
    /*...*/
}

    \end{minted}
\end{minipage}


\section{Basic C\#}
\subsection{Input Output}

Basically how to take input from the user and give an output back.

In C we have the \mintinline{c}|scanf()|, which is used to get input from the user and place it in to a variable. in C\# we have the \mintinline{csharp}|Console.ReadLine()| which is similar however it \emph{returns} the user input as a string rather than placing it in a variable so if we want user input in C we do
\begin{minted}{csharp}
    int x;
    scanf("%d",&x);
\end{minted}

While in C\# we have to read the input as a string then convert it to an integer like so

\begin{minted}{csharp}
    int x = System.Int32.Parse(Console.ReadLine());
\end{minted}

Likewise, in C we use \mintinline{c}|printf()| to print something to the screen, but in C\# we use \mintinline{csharp}|Console.WriteLine()| to do the same thing, however the difference is that C forces you to use a string to format before it is printed while C\# can handle the formatting for you, for example the C code
\begin{minted}{c}
    int x = 5;
    printf("%d\n",x);
\end{minted}
is equivelant to the following C\# code

\begin{minted}{csharp}
    int x = 5;
    Console.WriteLine(x); // print directly, no need to specify `%d` or `\n`
\end{minted}

\subsection{Format Strings}
Strings in C\# can be defined using \mintinline{csharp}|string name = "World"|. If we want to format strings when printing we do \mintinline{c}|printf("Hello, %d!\n",name)|, but in C\# this can be done in a much simpler fashion using 

\begin{minted}{csharp}
    Console.WriteLine($"Hello, {name}!")
\end{minted}
or alternatively using
\begin{minted}{csharp}
    Console.WriteLine("Hello, " + name + "!")
\end{minted}


\nt{
	2 strings can be \emph{concatenated} together in C\#(added together) like so
	\begin{center}
		\mint{csharp}|  string firstName = "Hamboola ";|
		\mint{csharp}|  string lastName = "Habooling";|
		\mint{csharp}|  string fullName = firstName + lastName; /* fullName => "Hamboola Hambooling" */|
	\end{center}
}




\end{document}

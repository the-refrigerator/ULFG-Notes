\documentclass{report}

\input{../template/preamble}
\input{../template/macros}
\input{../template/letterfonts}

\title{\Huge{Thermodynamics}\\Semester 4}
\author{}
\date{}

\begin{document}

\maketitle
\newpage% or \cleardoublepage
% \pdfbookmark[<level>]{<title>}{<dest>}
\pdfbookmark[section]{\contentsname}{toc}
\tableofcontents
\pagebreak


\chapter{Introductory concepts and definitions}
\section{Pressure}

When we place an item in a fluid, the fluid exerts a pressure on the item. At a point on the item, the pressure is the same from all directions, however the pressure can different from one point to another.
\\
A manometer is used to measure pressure, it is a U shaped pipe connected to the medium we are trying to measure pressure in.

\begin{figure}[H]
	\centering
	\includegraphics[width=0.2\textwidth]{figs/man.png}
\end{figure}

\[
	P_\text{Tank} =  P_\text{atm} + \rho_\text{Fluid}gL
	.\]

\nt{
	Reminder of the relation between force and pressure
	\[
		F = \frac{P}{A}
		.\]
}

\section{Work and Energy}

The work of a force is defined to be
\[
	W = \int_{\ell_1}^{\ell_2} F \dd{\ell} = \int_{V_1}^{V_2}p\dd{V}
	.\]

As well as the relation
\[
	\frac{1}{2}m\lt({v_2}^2 - {v_1}^2\rt) = \int_{\ell_1}^{\ell_2} F \dd{\ell}
	.\]

Power is defined to me the change in work
\[
	\dot{W} = F\cdot v
	.\]

The change energy in a system $\Delta E$ is the sum of the changes in kinetic energy, potential energy, and internal energy $\Delta U$
\[
	\Delta E = \Delta\text{KE} + \Delta\text{PE} + \Delta U
	.\]

\subsection{Heat transfer}

The heat energy in the system is denoted as $Q$
\[
	Q = \int_{t_1}^{t_2}\dot{Q}\dd{t}
	.\]

the heat flux in the system $\dot{q}$ can be used to calculate the heat transfer rate in the system
\[
	\dot{Q} = \int_A \dot{q}\dd{A}
	.\]

\subsubsection{Conduction (Fourier’s law)}

Conduction is the transfer of energy from the more energetic particles of a substance to the adjacent less energetic ones as a result of interactions between the particles
\[
	\dot{Q}_\text{cond} =-\kappa A \dv{T}{x}
	.\]
Where $\kappa$ is constant called thermal conductivity.

\subsubsection{Convection (Newton’s law of cooling)}
Convection is the mode of energy transfer between a solid surface and the adjacent liquid or gas that is in motion, and it involves the combined effects of conduction and fluid motion.

\[
	\dot{Q}_\text{conv} = hA(T_0-T_1)
	.\]
Where $h$ is constant called the heat transfer coefficient.

\subsubsection{Radiation}
Radiation is the energy emitted by matter in the form of electromagnetic waves (or photons) as a result of the changes in the electronic configurations of the atoms or molecules. Unlike conduction and convection, the transfer of energy by radiation does not require the presence of an intervening medium.

\[
	\dot{Q}_\text{emitted} = \varepsilon\sigma A{T_0}^4
	.\]
where $\varepsilon$ is the emissivity of the surface and $\sigma$ is the Stefan-Boltzmann constant

\[
	\sigma = 5.670367 \times 10^{−8}
	.\]

\section{First Law of Thermodynamics}

In a closed system the following relation holds
\[
	\Delta E = Q - W
	.\]

In a cycle the energy does not change so
\[
	W_\text{cycle} = Q_\text{in} - Q_\text{out}
	.\]

The formual for thermal effeciency is
\[
	\eta = \frac{W_\text{cycle}}{Q_\text{in}} = 1-\frac{Q_\text{out}}{Q_\text{in}}
	.\]

The effectiveness or coefficient of performance $\beta$ of a refrigerator is
\[
	\beta = \frac{Q_\text{in}}{W_\text{cycle}} = \frac{Q_\text{in}}{Q_\text{out} - Q_\text{in}}
	.\]
The effectiveness or coefficient of performance $\gamma$ of a heat pump is
\[
	\gamma = \frac{Q_\text{out}}{W_\text{cycle}} = \frac{Q_\text{out}}{Q_\text{out} - Q_\text{in}}
	.\]

\chapter{Evaluating Properties} % 2ism il chapter hek, doesn't make any sense bas ok

The quality of a liquid-vapor substance is defined by
\[
	x=\frac{m_\text{vapor}}{m_\text{liquid} + m_\text{vapor}}
	.\]

We define the specific volume $\nu$ to be the inverse of density $\rho$
\[
	\nu = \frac{1}{\rho}=\frac{V}{m}
	.\]

The specific volume of a liquid-vapor substance is
\[
	\nu = (1-x)\nu_\ell + x\nu_g
	.\]
where $\nu_\ell$ and $\nu_g$ are the specific volume of the liquid portion and the gas portion of the substance respectively.\\

We define specific internal energy to be

\[
	u = \frac{U}{m}
	.\]

Similarly to specific volume

\[
	u = (1-x)u_\ell + xu_g
	.\]

We define enthalpy to be
\[
	H=U+pV
	.\]

specific enthalpy to be
\[
	h=u+p\nu
	.\]
and similarly to both previous specific quantities
\[
	h = (1-x)h_\ell + xh_g
	.\]


We define the specific heats to be

\begin{align*}
	C_V & = \pdv{u}{T}\Big|_V \\
	C_p & = \pdv{h}{T}\Big|_p
\end{align*}

from these equation we can deduce the formulas
\begin{align*}
	\Delta U & = mC_V\Delta T \\
	\Delta H & = mC_p\Delta T
\end{align*}
We also define the specific heat ratio as
\[
	k=\frac{C_p}{C_V}
	.\]
In case of incompressible fluids/solids, we have $C_V = C_p$
\begin{align*}
	\Delta u & = C\Delta T               \\
	\Delta h & = C\Delta T + \nu\Delta p
\end{align*}

\section{Ideal gas model}

The universal gas constant is
\[
	\mathcal{R} = 8.314471\times\mathrm{kJ}\cdot\mathrm{kmol}^{-1}\cdot\mathrm{K}^{-1}
	.\]

We define the compressibility factor to be
\[
	Z = \frac{p\bar{\nu}}{\mathcal{R}T}
	.\]

such that $\bar{\nu} = M\nu$ where $M$ is the atomic/molecular mass

\[
	Z = \frac{p\nu}{RT}
	.\]

where $R = \frac{M}{\mathcal{R}}$ and it is specific to every gas.\\

The ideal gas equation is
\[
	pV = mRT
	.\]

In an ideal gas, enthalpy only depends on temperature
\[
	h = u + RT
	.\]

\begin{align*}
	\Delta u(T) & = \int_{T_1}^{T_2}C_V(T)\dd{T} \\
	\Delta h(T) & = \int_{T_1}^{T_2}C_p(T)\dd{T} \\
\end{align*}

and

\begin{align*}
	C_V(T) & =\frac{R}{k-1}  \\
	C_p(T) & =\frac{kR}{k-1}
\end{align*}

\end{document}

\documentclass{report}

\input{../template/preamble}
\input{../template/macros}
\input{../template/letterfonts}

\title{\Huge{Statistics}\\Semester 4}
\author{\huge{}}
\date{}

\begin{document}

\maketitle
\newpage% or \cleardoublepage
% \pdfbookmark[<level>]{<title>}{<dest>}
\pdfbookmark[section]{\contentsname}{toc}
\tableofcontents
\pagebreak

\chapter{Revision of Probability}

I'm simply gonna list rules.
\begin{align*}
	\EE(X)    & = \mu =\sum_{i\in\Omega}X_i\Pr{X_i}                            \\
	\EE(g(X)) & = \sum_{i\in\Omega}g(X_i)\Pr{X_i}                              \\
	\EE(aX+b) & = a\EE(X) + b                                                  \\
	\EE(X+Y)  & = \EE(X) + \EE(Y)\quad\text{if both variables are independent}
\end{align*}

\begin{align*}
	\Var{X}     & = \sigma^2 = \EE(X^2) - \mu^2          \\
	\Var{aX+bY} & = a^2\Var{X} + b^2\Var{Y}+2ab\cov{X,Y}
\end{align*}

where
\[
	\cov{X,Y} = \EE(XY) - \EE(X)\cdot\EE(Y)
	.\]

\section{Discrete Distributions}

\begin{enumerate}
	\ii \textbf{Uniform discrete law}
	\begin{align*}
		 & X(\Omega)= \{1,2,3,\dots,n\}                        \\
		 & \Pr{X=k} = \frac{1}{n}\quad\forall k =1,2,3,\dots,n \\
		 & \begin{cases}
			\EE(X) = \frac{n+1}{2} \\
			\Var{X} = \frac{n^2-1}{12}
		\end{cases}
	\end{align*}
	\ii \textbf{Bernoulli law of parameters $p$} $(0<p<1)$
	\begin{align*}
		 & X \sim \mathrm{B}(p)              \\
		 & X(\Omega) = \{0,1\}               \\
		 & \Pr{X=1} = p \quad \Pr{X=0} = 1-p \\
		 & \begin{cases}
			\EE(X) = p \\
			\Var{X} = p(1-p)
		\end{cases}
	\end{align*}
	\ii \textbf{Binomial law of parameters $n$ and $p$}
	\begin{align*}
		 & X \sim \mathrm{Bin}(n,p)                                      \\
		 & X(\Omega) = \{1,2,\dots,n\}                                   \\
		 & \Pr{X=1} = C^k_n p^kq^{n-k}\quad\forall k\in\{0,1,2,\dots,n\} \\
		 & \begin{cases}
			\EE(X) = np \\
			\Var{X} = np(1-p)
		\end{cases}
	\end{align*}
	\ii \textbf{Hypergeometric law}
	\begin{align*}
		 & X \sim \mathcal{H}(N,n,p)                                                     \\
		 & X(\Omega) = [\max\{0,n-N+M\},\min\{M,n\}]                                     \\
		 & \Pr{X=k} = \frac{C^k_M \cdot C^{n-k}_{N-M}}{C^n_N}\quad\forall k\in X(\Omega) \\
		 & \begin{cases}
			\EE(X) = np \\
			\Var{X} = np(1-p)\lt(\frac{N-n}{N-1}\rt)
		\end{cases}
	\end{align*}
	\ii \textbf{Geometric law}
	\begin{align*}
		 & X \sim \mathrm{G}(p)                          \\
		 & X(\Omega) = \NN^*                             \\
		 & \Pr{X=k} =p(1-p)^{k-1}\quad\forall k\in \NN^* \\
		 & \begin{cases}
			\EE(X) = \frac{1}{p} \\
			\Var{X} = \frac{1-p}{p^2}
		\end{cases}
	\end{align*}
	\ii \textbf{Poisson's law of parameter} $\lambda$ $(\lambda\in\RR_+^*)$
	\begin{align*}
		 & X \sim \mathcal{P}(\lambda)                                      \\
		 & X(\Omega) = \NN                                                  \\
		 & \Pr{X=k} = e^{-\lambda}\frac{\lambda^k}{k!}\quad\forall k\in \NN \\
		 & \begin{cases}
			\EE(X) = \lambda \\
			\Var{X} = \lambda
		\end{cases}
	\end{align*}

\end{enumerate}

\section{Continuous Distributions}

\begin{enumerate}
	\ii \textbf{Uniform law}
	\begin{align*}
		 & f(x) = \begin{cases}
			\frac{1}{b-a} & \text{if } x\in[a,b] \\
			0             & \text{else}
		\end{cases} \\
		 & \begin{cases}
			\EE(x) =  \frac{a+b}{2} \\
			\Var{x} = \frac{(b-a)^2}{12}
		\end{cases}
	\end{align*}
	\ii \textbf{Continuous law}
	\begin{align*}
		 & x \sim \xi(\lambda)               \\
		 & f(x) = \begin{cases}
			\lambda e^{-\lambda x} & \text{if } x>0 \\
			0                      & \text{else}
		\end{cases} \\
		 & \begin{cases}
			\EE(x) =  \frac{1}{\lambda} \\
			\Var{x} = \frac{1}{\lambda^2}
		\end{cases}
	\end{align*}

	\ii \textbf{Normal law}
	\begin{align*}
		 & x \sim \mathcal{N}(\mu,\sigma)                                     \\
		 & f(x) = \frac{1}{\sigma\sqrt{2\pi}}e^{-\frac{(x-\mu)^2}{2\sigma^2}} \\
		 & \begin{cases}
			\EE(x) =  \mu \\
			\Var{x} = \sigma^2
		\end{cases}
	\end{align*}
	For $\mathcal{N}(0,1)$
	\[
		\Phi(z) = \int_{-\infty}^z\frac{1}{\sqrt{2\pi}}e^{-\frac{x^2}{2}}\dd{x}
		.\]
	\[
		\pi(z) = \Phi(z) - 0.5 = \int_0^z\frac{1}{\sqrt{2\pi}}e^{-\frac{x^2}{2}}\dd{x}
		.\]

\end{enumerate}

\section{Convergence}

\thm{Chebyshev's inequality}{
	Let $X$ be a random variable of expectation $\EE(X)$ and variance $\Var{X}$. Then $\forall \varepsilon$
	\[
		\Pr{\lt| X-\EE(X) \rt|\geq \varepsilon} \leq \frac{\Var{X}}{\varepsilon^2}
		.\]
	it can also be stated as
	\[
		\Pr{\lt| X-\EE(X) \rt|< \varepsilon} \geq 1-\frac{\Var{X}}{\varepsilon^2}
		.\]
}

We say a sequence of random variables $X_n$ converges to $a$ $(X_n)\xrightarrow{\mathrm{Pr}}a$ if $\forall \varepsilon$
\[
	\lim_{n\to+\infty}\Pr{|X_n-a|>\varepsilon}=0
	.\]
or
\[
	\lim_{n\to+\infty}\Pr{|X_n-a|\leq\varepsilon}=1
	.\]

\thm{Weak law of large numbers}{
	Consider a random variable $(X_n)$ of mean $\mu$ and variance $\sigma^2$. Consider the random variable $\tilde{X}_n = \frac{X_1+X_2+\cdots+X_n}{n}$. It can be shown that $\tilde{X}_n$ converges to $\mu$ meaning $\forall \varepsilon$
	\[
		\lim_{n\to+\infty}\Pr{|\tilde{X}_n-\mu|>\varepsilon}=0
		.\]
}

\section{Approximations}
\thm{Binomial by a Poisson}{
	\[
		\mathrm{Bin}(n,p)\sim\mathcal{P}(np)\quad \text{if }\begin{cases}
			n\geq30  \\
			p\leq0.1 \\
			np<15
		\end{cases}
		.\]
}

\thm{Hypergeometric by a Binomial}{
	\[
		\mathcal{H}(N,n,p)\sim\mathrm{Bin}(n,p)\quad \text{if } n\leq0.05N
		.\]
}

\thm{De Moivre–Laplace theorem}{
	\[
		\mathrm{Bin}(n,p)\sim\mathcal{N}\lt(np,\sqrt{np(1-p)}\rt)\quad \text{if }\begin{cases}
			n\geq30 \\
			np\geq5 \\
			n(1-p)\geq5
		\end{cases}
		.\]
	In this case the event $X=k$ can be replaced by $k-0.5<X<l+0.5$
}

\thm{Central limit theorem}{
	Let $(X_n)$ be a sequence of independent random variables following the same law of expectation $\mu$ and of standard deviation $\sigma$. Let $S_n = \sum_{i=1}^n X_i$ and $S_n^* = \frac{S_n-n\mu}{\sigma\sqrt{n}}$. It can be shown that $S_n^*$ converges in law to $\mathcal{N}(0,1)$.
	\begin{align*}
		\EE(S_n)  & = n\mu      \\
		\Var{S_n} & = n\sigma^2
	\end{align*}
}
\section{Further laws}

\thm{Chi square law}{
Let $X_1,X_2,\dots,X_n$ be $n$ independent random variables following the standard normal law $\mathcal{N}(0,1)$. Let $Y={X_1}^2 + {X_2}^2+\dots+{X_n}^2$. We say that $Y$ follows a chi-square law with $n$ degrees of freedom. $Y\sim{\chi_n}^2$.

\begin{align*}
	\EE(Y)  & = n  \\
	\Var{Y} & = 2n
\end{align*}

It can be shown that the density function of $Y$ is
\[
	f(x)=\begin{cases}
		\dfrac{1}{2^{\frac{n}{2}}\Gamma\lt(\frac{n}{2}\rt)}x^{\frac{n}{2}-1}e^{-\frac{x}{2}} & \text{if }x>0 \\
		0                                                                                    & \text{else}
	\end{cases}
	.\]
where $\Gamma$ is the gamma function
\[
	\Gamma(x)=\int_0^{+\infty}t^{x-1}e^{-t}\dd{t}\quad\forall x>0
	.\]
}

\thm{Student law(t-distribution)}{
Let $X$, $Z$ be two independent random random variables such that $X\sim\mathcal{N}(0,1)$ and $Z\sim{\chi_n}^2$. Hence the random variable
\[
	T=\frac{X}{\sqrt{\frac{Z}{n}}}
	.\]
is said to be following a student law. $T\sim\mathcal{T}_n$
\[
	f(t)=\frac{1}{\sqrt{n\pi}}\frac{\Gamma\lt(\dfrac{n+1}{2}\rt)}{\Gamma\lt(\dfrac{n}{2}\rt)}\lt(1+\frac{t^2}{n}\rt)^{-\frac{n+1}{2}}
	.\]
}

\end{document}

\documentclass{report}

%%%%%%%%%%%%%%%%%%%%%%%%%%%%%%%%%
% PACKAGE IMPORTS
%%%%%%%%%%%%%%%%%%%%%%%%%%%%%%%%%


\usepackage[tmargin=2cm,rmargin=1in,lmargin=1in,margin=0.85in,bmargin=2cm,footskip=.2in]{geometry}
\usepackage{amsmath,amsfonts,amsthm,amssymb,mathtools}
\usepackage[varbb]{newpxmath}
\usepackage{xfrac}
\usepackage[makeroom]{cancel}
\usepackage{mathtools}
\usepackage{bookmark}
\usepackage{enumitem}
\usepackage{hyperref,theoremref}
\usepackage{xparse}
\hypersetup{
	pdftitle={Hamboola my beloved},
	colorlinks=true, linkcolor=doc!90,
	bookmarksnumbered=true,
	bookmarksopen=true
}
\usepackage[most,many,breakable]{tcolorbox}
\usepackage{xcolor}
\usepackage{varwidth}
\usepackage{varwidth}
\usepackage{etoolbox}
\usepackage{bm}
%\usepackage{authblk}
\usepackage{pgfplots}
\usepackage{nameref}
\usepackage{multicol,array}
\usepackage{tikz-cd}
\usepackage[ruled,vlined,linesnumbered]{algorithm2e}
\usepackage{comment} % enables the use of multi-line comments (\ifx \fi) 
\usepackage{import}
\usepackage{xifthen}
\usepackage{pdfpages}
\usepackage{transparent}
\usepackage{minted}
\usepackage{fontspec}
\usepackage{tasks}
\usepackage{chemfig}
\usepackage[version=4]{mhchem}
\usepackage{suffix}
\usepackage{tabularx}
\usepackage{subcaption}
\usepackage{physics}


% \setmonofont{SpaceMono Nerd Font}
\setminted{fontsize=\footnotesize}

\newcommand\mycommfont[1]{\footnotesize\ttfamily\textcolor{blue}{#1}}
\SetCommentSty{mycommfont}
\newcommand{\incfig}[1]{%
	\def\svgwidth{\columnwidth}
	\import{./figures/}{#1.pdf_tex}
}

\usepackage{tikzsymbols}
% \renewcommand\qedsymbol{$\Laughey$}
\pgfplotsset{compat=1.18}

%\usepackage{import}
%\usepackage{xifthen}
%\usepackage{pdfpages}
%\usepackage{transparent}


%%%%%%%%%%%%%%%%%%%%%%%%%%%%%%
% SELF MADE COLORS
%%%%%%%%%%%%%%%%%%%%%%%%%%%%%%



\definecolor{myg}{RGB}{56, 140, 70}
\definecolor{myb}{RGB}{45, 111, 177}
\definecolor{myr}{RGB}{199, 68, 64}
\definecolor{mytheorembg}{HTML}{F2F2F9}
\definecolor{mytheoremfr}{HTML}{00007B}
\definecolor{mylenmabg}{HTML}{FFFAF8}
\definecolor{mylenmafr}{HTML}{983b0f}
\definecolor{mypropbg}{HTML}{f2fbfc}
\definecolor{mypropfr}{HTML}{191971}
\definecolor{myexamplebg}{HTML}{F2FBF8}
\definecolor{myexamplefr}{HTML}{88D6D1}
\definecolor{myexampleti}{HTML}{2A7F7F}
\definecolor{mydefinitbg}{HTML}{E5E5FF}
\definecolor{mydefinitfr}{HTML}{3F3FA3}
\definecolor{notesgreen}{RGB}{0,162,0}
\definecolor{myp}{RGB}{197, 92, 212}
\definecolor{mygr}{HTML}{2C3338}
\definecolor{myred}{RGB}{127,0,0}
\definecolor{myyellow}{RGB}{169,121,69}
\definecolor{myexercisebg}{HTML}{F2FBF8}
\definecolor{myexercisefg}{HTML}{88D6D1}
\definecolor{codebg}{HTML}{0D1117}

%%%%%%%%%%%%%%%%%%%%%%%%%%%%
% TCOLORBOX SETUPS
%%%%%%%%%%%%%%%%%%%%%%%%%%%%

\setlength{\parindent}{0cm}
%================================
% THEOREM BOX
%================================

\tcbuselibrary{theorems,skins,hooks}
\newtcbtheorem[number within=section]{Theorem}{Theorem}
{%
	enhanced,
	breakable,
	colback = mytheorembg,
	frame hidden,
	boxrule = 0sp,
	borderline west = {2pt}{0pt}{mytheoremfr},
	sharp corners,
	detach title,
	before upper = \tcbtitle\par\smallskip,
	coltitle = mytheoremfr,
	fonttitle = \bfseries\sffamily,
	description font = \mdseries,
	separator sign none,
	segmentation style={solid, mytheoremfr},
}
{th}

\tcbuselibrary{theorems,skins,hooks}
\newtcbtheorem[number within=chapter]{theorem}{Theorem}
{%
	enhanced,
	breakable,
	colback = mytheorembg,
	frame hidden,
	boxrule = 0sp,
	borderline west = {2pt}{0pt}{mytheoremfr},
	sharp corners,
	detach title,
	before upper = \tcbtitle\par\smallskip,
	coltitle = mytheoremfr,
	fonttitle = \bfseries\sffamily,
	description font = \mdseries,
	separator sign none,
	segmentation style={solid, mytheoremfr},
}
{th}


\tcbuselibrary{theorems,skins,hooks}
\newtcolorbox{Theoremcon}
{%
	enhanced
	,breakable
	,colback = mytheorembg
	,frame hidden
	,boxrule = 0sp
	,borderline west = {2pt}{0pt}{mytheoremfr}
	,sharp corners
	,description font = \mdseries
	,separator sign none
}

%================================
% Corollery
%================================
\tcbuselibrary{theorems,skins,hooks}
\newtcbtheorem[number within=section]{Corollary}{Corollary}
{%
	enhanced
	,breakable
	,colback = myp!10
	,frame hidden
	,boxrule = 0sp
	,borderline west = {2pt}{0pt}{myp!85!black}
	,sharp corners
	,detach title
	,before upper = \tcbtitle\par\smallskip
	,coltitle = myp!85!black
	,fonttitle = \bfseries\sffamily
	,description font = \mdseries
	,separator sign none
	,segmentation style={solid, myp!85!black}
}
{th}
\tcbuselibrary{theorems,skins,hooks}
\newtcbtheorem[number within=chapter]{corollary}{Corollary}
{%
	enhanced
	,breakable
	,colback = myp!10
	,frame hidden
	,boxrule = 0sp
	,borderline west = {2pt}{0pt}{myp!85!black}
	,sharp corners
	,detach title
	,before upper = \tcbtitle\par\smallskip
	,coltitle = myp!85!black
	,fonttitle = \bfseries\sffamily
	,description font = \mdseries
	,separator sign none
	,segmentation style={solid, myp!85!black}
}
{th}


%================================
% LENMA
%================================

\tcbuselibrary{theorems,skins,hooks}
\newtcbtheorem[number within=section]{Lenma}{Lenma}
{%
	enhanced,
	breakable,
	colback = mylenmabg,
	frame hidden,
	boxrule = 0sp,
	borderline west = {2pt}{0pt}{mylenmafr},
	sharp corners,
	detach title,
	before upper = \tcbtitle\par\smallskip,
	coltitle = mylenmafr,
	fonttitle = \bfseries\sffamily,
	description font = \mdseries,
	separator sign none,
	segmentation style={solid, mylenmafr},
}
{th}

\tcbuselibrary{theorems,skins,hooks}
\newtcbtheorem[number within=chapter]{lenma}{Lenma}
{%
	enhanced,
	breakable,
	colback = mylenmabg,
	frame hidden,
	boxrule = 0sp,
	borderline west = {2pt}{0pt}{mylenmafr},
	sharp corners,
	detach title,
	before upper = \tcbtitle\par\smallskip,
	coltitle = mylenmafr,
	fonttitle = \bfseries\sffamily,
	description font = \mdseries,
	separator sign none,
	segmentation style={solid, mylenmafr},
}
{th}


%================================
% PROPOSITION
%================================

\tcbuselibrary{theorems,skins,hooks}
\newtcbtheorem[number within=section]{Prop}{Proposition}
{%
	enhanced,
	breakable,
	colback = mypropbg,
	frame hidden,
	boxrule = 0sp,
	borderline west = {2pt}{0pt}{mypropfr},
	sharp corners,
	detach title,
	before upper = \tcbtitle\par\smallskip,
	coltitle = mypropfr,
	fonttitle = \bfseries\sffamily,
	description font = \mdseries,
	separator sign none,
	segmentation style={solid, mypropfr},
}
{th}

\tcbuselibrary{theorems,skins,hooks}
\newtcbtheorem[number within=chapter]{prop}{Proposition}
{%
	enhanced,
	breakable,
	colback = mypropbg,
	frame hidden,
	boxrule = 0sp,
	borderline west = {2pt}{0pt}{mypropfr},
	sharp corners,
	detach title,
	before upper = \tcbtitle\par\smallskip,
	coltitle = mypropfr,
	fonttitle = \bfseries\sffamily,
	description font = \mdseries,
	separator sign none,
	segmentation style={solid, mypropfr},
}
{th}


%================================
% CLAIM
%================================

\tcbuselibrary{theorems,skins,hooks}
\newtcbtheorem[number within=section]{claim}{Claim}
{%
	enhanced
	,breakable
	,colback = myg!10
	,frame hidden
	,boxrule = 0sp
	,borderline west = {2pt}{0pt}{myg}
	,sharp corners
	,detach title
	,before upper = \tcbtitle\par\smallskip
	,coltitle = myg!85!black
	,fonttitle = \bfseries\sffamily
	,description font = \mdseries
	,separator sign none
	,segmentation style={solid, myg!85!black}
}
{th}



%================================
% Exercise
%================================

\tcbuselibrary{theorems,skins,hooks}
\newtcbtheorem[number within=section]{Exercise}{Exercise}
{%
	enhanced,
	breakable,
	colback = myexercisebg,
	frame hidden,
	boxrule = 0sp,
	borderline west = {2pt}{0pt}{myexercisefg},
	sharp corners,
	detach title,
	before upper = \tcbtitle\par\smallskip,
	coltitle = myexercisefg,
	fonttitle = \bfseries\sffamily,
	description font = \mdseries,
	separator sign none,
	segmentation style={solid, myexercisefg},
}
{th}

\tcbuselibrary{theorems,skins,hooks}
\newtcbtheorem[number within=chapter]{exercise}{Exercise}
{%
	enhanced,
	breakable,
	colback = myexercisebg,
	frame hidden,
	boxrule = 0sp,
	borderline west = {2pt}{0pt}{myexercisefg},
	sharp corners,
	detach title,
	before upper = \tcbtitle\par\smallskip,
	coltitle = myexercisefg,
	fonttitle = \bfseries\sffamily,
	description font = \mdseries,
	separator sign none,
	segmentation style={solid, myexercisefg},
}
{th}

%================================
% EXAMPLE BOX
%================================

\newtcbtheorem[number within=section]{Example}{Example}
{%
	colback = myexamplebg
	,breakable
	,colframe = myexamplefr
	,coltitle = myexampleti
	,boxrule = 1pt
	,sharp corners
	,detach title
	,before upper=\tcbtitle\par\smallskip
	,fonttitle = \bfseries
	,description font = \mdseries
	,separator sign none
	,description delimiters parenthesis
}
{ex}

\newtcbtheorem[number within=chapter]{example}{Example}
{%
	colback = myexamplebg
	,breakable
	,colframe = myexamplefr
	,coltitle = myexampleti
	,boxrule = 1pt
	,sharp corners
	,detach title
	,before upper=\tcbtitle\par\smallskip
	,fonttitle = \bfseries
	,description font = \mdseries
	,separator sign none
	,description delimiters parenthesis
}
{ex}

%================================
% DEFINITION BOX
%================================

\newtcbtheorem[number within=section]{Definition}{Definition}{enhanced,
	before skip=2mm,after skip=2mm, colback=red!5,colframe=red!80!black,boxrule=0.5mm,
	attach boxed title to top left={xshift=1cm,yshift*=1mm-\tcboxedtitleheight}, varwidth boxed title*=-3cm,
	boxed title style={frame code={
					\path[fill=tcbcolback]
					([yshift=-1mm,xshift=-1mm]frame.north west)
					arc[start angle=0,end angle=180,radius=1mm]
					([yshift=-1mm,xshift=1mm]frame.north east)
					arc[start angle=180,end angle=0,radius=1mm];
					\path[left color=tcbcolback!60!black,right color=tcbcolback!60!black,
						middle color=tcbcolback!80!black]
					([xshift=-2mm]frame.north west) -- ([xshift=2mm]frame.north east)
					[rounded corners=1mm]-- ([xshift=1mm,yshift=-1mm]frame.north east)
					-- (frame.south east) -- (frame.south west)
					-- ([xshift=-1mm,yshift=-1mm]frame.north west)
					[sharp corners]-- cycle;
				},interior engine=empty,
		},
	fonttitle=\bfseries,
	title={#2},#1}{def}
\newtcbtheorem[number within=chapter]{definition}{Definition}{enhanced,
	before skip=2mm,after skip=2mm, colback=red!5,colframe=red!80!black,boxrule=0.5mm,
	attach boxed title to top left={xshift=1cm,yshift*=1mm-\tcboxedtitleheight}, varwidth boxed title*=-3cm,
	boxed title style={frame code={
					\path[fill=tcbcolback]
					([yshift=-1mm,xshift=-1mm]frame.north west)
					arc[start angle=0,end angle=180,radius=1mm]
					([yshift=-1mm,xshift=1mm]frame.north east)
					arc[start angle=180,end angle=0,radius=1mm];
					\path[left color=tcbcolback!60!black,right color=tcbcolback!60!black,
						middle color=tcbcolback!80!black]
					([xshift=-2mm]frame.north west) -- ([xshift=2mm]frame.north east)
					[rounded corners=1mm]-- ([xshift=1mm,yshift=-1mm]frame.north east)
					-- (frame.south east) -- (frame.south west)
					-- ([xshift=-1mm,yshift=-1mm]frame.north west)
					[sharp corners]-- cycle;
				},interior engine=empty,
		},
	fonttitle=\bfseries,
	title={#2},#1}{def}



%================================
% Solution BOX
%================================

\makeatletter
\newtcbtheorem{question}{Question}{enhanced,
	breakable,
	colback=white,
	colframe=myb!80!black,
	attach boxed title to top left={yshift*=-\tcboxedtitleheight},
	fonttitle=\bfseries,
	title={#2},
	boxed title size=title,
	boxed title style={%
			sharp corners,
			rounded corners=northwest,
			colback=tcbcolframe,
			boxrule=0pt,
		},
	underlay boxed title={%
			\path[fill=tcbcolframe] (title.south west)--(title.south east)
			to[out=0, in=180] ([xshift=5mm]title.east)--
			(title.center-|frame.east)
			[rounded corners=\kvtcb@arc] |-
			(frame.north) -| cycle;
		},
	#1
}{def}
\makeatother

%================================
% SOLUTION BOX
%================================

\makeatletter
\newtcolorbox{solution}{enhanced,
	breakable,
	colback=white,
	colframe=myg!80!black,
	attach boxed title to top left={yshift*=-\tcboxedtitleheight},
	title=Solution,
	boxed title size=title,
	boxed title style={%
			sharp corners,
			rounded corners=northwest,
			colback=tcbcolframe,
			boxrule=0pt,
		},
	underlay boxed title={%
			\path[fill=tcbcolframe] (title.south west)--(title.south east)
			to[out=0, in=180] ([xshift=5mm]title.east)--
			(title.center-|frame.east)
			[rounded corners=\kvtcb@arc] |-
			(frame.north) -| cycle;
		},
}
\makeatother

%================================
% Question BOX
%================================

\makeatletter
\newtcbtheorem{qstion}{Question}{enhanced,
	breakable,
	colback=white,
	colframe=mygr,
	attach boxed title to top left={yshift*=-\tcboxedtitleheight},
	fonttitle=\bfseries,
	title={#2},
	boxed title size=title,
	boxed title style={%
			sharp corners,
			rounded corners=northwest,
			colback=tcbcolframe,
			boxrule=0pt,
		},
	underlay boxed title={%
			\path[fill=tcbcolframe] (title.south west)--(title.south east)
			to[out=0, in=180] ([xshift=5mm]title.east)--
			(title.center-|frame.east)
			[rounded corners=\kvtcb@arc] |-
			(frame.north) -| cycle;
		},
	#1
}{def}
\makeatother

\newtcbtheorem[number within=chapter]{wconc}{Wrong Concept}{
	breakable,
	enhanced,
	colback=white,
	colframe=myr,
	arc=0pt,
	outer arc=0pt,
	fonttitle=\bfseries\sffamily\large,
	colbacktitle=myr,
	attach boxed title to top left={},
	boxed title style={
			enhanced,
			skin=enhancedfirst jigsaw,
			arc=3pt,
			bottom=0pt,
			interior style={fill=myr}
		},
	#1
}{def}



%================================
% NOTE BOX
%================================

\usetikzlibrary{hobby}
\usetikzlibrary{arrows,calc,shadows.blur}
\tcbuselibrary{skins}
\newtcolorbox{note}[1][]{%
	enhanced jigsaw,
	colback=gray!20!white,%
	colframe=gray!80!black,
	size=small,
	boxrule=1pt,
	title=\textbf{Note:-},
	halign title=flush center,
	coltitle=black,
	breakable,
	drop shadow=black!50!white,
	attach boxed title to top left={xshift=1cm,yshift=-\tcboxedtitleheight/2,yshifttext=-\tcboxedtitleheight/2},
	minipage boxed title=1.5cm,
	boxed title style={%
			colback=white,
			size=fbox,
			boxrule=1pt,
			boxsep=2pt,
			underlay={%
					\coordinate (dotA) at ($(interior.west) + (-0.5pt,0)$);
					\coordinate (dotB) at ($(interior.east) + (0.5pt,0)$);
					\begin{scope}
						\clip (interior.north west) rectangle ([xshift=3ex]interior.east);
						\filldraw [white, blur shadow={shadow opacity=60, shadow yshift=-.75ex}, rounded corners=2pt] (interior.north west) rectangle (interior.south east);
					\end{scope}
					\begin{scope}[gray!80!black]
						\fill (dotA) circle (2pt);
						\fill (dotB) circle (2pt);
					\end{scope}
				},
		},
	#1,
}

%%%%%%%%%%%%%%%%%%%%%%%%%%%%%%
% SELF MADE COMMANDS
%%%%%%%%%%%%%%%%%%%%%%%%%%%%%%


\newcommand{\thm}[2]{\begin{Theorem}{#1}{}#2\end{Theorem}}
\newcommand{\cor}[2]{\begin{Corollary}{#1}{}#2\end{Corollary}}
\newcommand{\mlenma}[2]{\begin{Lenma}{#1}{}#2\end{Lenma}}
\newcommand{\mprop}[2]{\begin{Prop}{#1}{}#2\end{Prop}}
\newcommand{\clm}[3]{\begin{claim}{#1}{#2}#3\end{claim}}
\newcommand{\wc}[2]{\begin{wconc}{#1}{}\setlength{\parindent}{1cm}#2\end{wconc}}
\newcommand{\thmcon}[1]{\begin{Theoremcon}{#1}\end{Theoremcon}}
\newcommand{\ex}[2]{\begin{Example}{#1}{}#2\end{Example}}
\newcommand{\dfn}[2]{\begin{Definition}[colbacktitle=red!75!black]{#1}{}#2\end{Definition}}
\newcommand{\dfnc}[2]{\begin{definition}[colbacktitle=red!75!black]{#1}{}#2\end{definition}}
\newcommand{\qs}[2]{\begin{question}{#1}{}#2\end{question}}
\newcommand{\pf}[2]{\begin{myproof}[#1]#2\end{myproof}}
\newcommand{\nt}[1]{\begin{note}#1\end{note}}

\newcommand*\circled[1]{\tikz[baseline=(char.base)]{
		\node[shape=circle,draw,inner sep=1pt] (char) {#1};}}
\newcommand\getcurrentref[1]{%
	\ifnumequal{\value{#1}}{0}
	{??}
	{\the\value{#1}}%
}
\newcommand{\getCurrentSectionNumber}{\getcurrentref{section}}
\newenvironment{myproof}[1][\proofname]{%
	\proof[\bfseries #1: ]%
}{\endproof}

\newcommand{\mclm}[2]{\begin{myclaim}[#1]#2\end{myclaim}}
\newenvironment{myclaim}[1][\claimname]{\proof[\bfseries #1: ]}{}

\newcounter{mylabelcounter}

\makeatletter
\newcommand{\setword}[2]{%
	\phantomsection
	#1\def\@currentlabel{\unexpanded{#1}}\label{#2}%
}
\makeatother




\tikzset{
	symbol/.style={
			draw=none,
			every to/.append style={
					edge node={node [sloped, allow upside down, auto=false]{$#1$}}}
		}
}


% deliminators
% \DeclarePairedDelimiter{\abs}{\lvert}{\rvert}
% \DeclarePairedDelimiter{\norm}{\lVert}{\rVert}

\DeclarePairedDelimiter{\ceil}{\lceil}{\rceil}
\DeclarePairedDelimiter{\floor}{\lfloor}{\rfloor}
\DeclarePairedDelimiter{\round}{\lfloor}{\rceil}

\newsavebox\diffdbox
\newcommand{\slantedromand}{{\mathpalette\makesl{d}}}
\newcommand{\makesl}[2]{%
	\begingroup
	\sbox{\diffdbox}{$\mathsurround=0pt#1\mathrm{#2}$}%
	\pdfsave
	\pdfsetmatrix{1 0 0.2 1}%
	\rlap{\usebox{\diffdbox}}%
	\pdfrestore
	\hskip\wd\diffdbox
	\endgroup
}
% \newcommand{\dd}[1][]{\ensuremath{\mathop{}\!\ifstrempty{#1}{%
% 			\slantedromand\@ifnextchar^{\hspace{0.2ex}}{\hspace{0.1ex}}}%
% 		{\slantedromand\hspace{0.2ex}^{#1}}}}
\ProvideDocumentCommand\dv{o m g}{%
	\ensuremath{%
		\IfValueTF{#3}{%
			\IfNoValueTF{#1}{%
				\frac{\dd #2}{\dd #3}%
			}{%
				\frac{\dd^{#1} #2}{\dd #3^{#1}}%
			}%
		}{%
			\IfNoValueTF{#1}{%
				\frac{\dd}{\dd #2}%
			}{%
				\frac{\dd^{#1}}{\dd #2^{#1}}%
			}%
		}%
	}%
}
\DeclareDocumentCommand\pdv{ s o m g g d() }
{ % Partial derivative
	% s: star for \flatfrac flat derivative
	% o: optional n for nth derivative
	% m: mandatory (x in df/dx)
	% g: optional (f in df/dx)
	% g: optional (y in d^2f/dxdy)
	% d: long-form d/dx(...)
	\IfBooleanTF{#1}
	{\let\fractype\flatfrac}
	{\let\fractype\frac}
	\IfNoValueTF{#4}
	{
		\IfNoValueTF{#6}
		{\fractype{\partial \IfNoValueTF{#2}{}{^{#2}}}{\partial #3\IfNoValueTF{#2}{}{^{#2}}}}
		{\fractype{\partial \IfNoValueTF{#2}{}{^{#2}}}{\partial #3\IfNoValueTF{#2}{}{^{#2}}} \argopen(#6\argclose)}
	}
	{
		\IfNoValueTF{#5}
		{\fractype{\partial \IfNoValueTF{#2}{}{^{#2}} #3}{\partial #4\IfNoValueTF{#2}{}{^{#2}}}}
		{\fractype{\partial^2 #3}{\partial #4 \partial #5}}
	}
}
% \providecommand*{\pdv}[3][]{\frac{\partial^{#1}#2}{\partial#3^{#1}}}
%  - others
\DeclareMathOperator{\Lap}{\mathcal{L}}
\DeclareMathOperator{\Var}{Var} % varience
\DeclareMathOperator{\Cov}{Cov} % covarience
\DeclareMathOperator{\E}{E} % expected

% Since the amsthm package isn't loaded

% I prefer the slanted \leq
\let\oldleq\leq % save them in case they're every wanted
\let\oldgeq\geq
\renewcommand{\leq}{\leqslant}
\renewcommand{\geq}{\geqslant}

% % redefine matrix env to allow for alignment, use r as default
% \renewcommand*\env@matrix[1][r]{\hskip -\arraycolsep
%     \let\@ifnextchar\new@ifnextchar
%     \array{*\c@MaxMatrixCols #1}}


%\usepackage{framed}
%\usepackage{titletoc}
%\usepackage{etoolbox}
%\usepackage{lmodern}


%\patchcmd{\tableofcontents}{\contentsname}{\sffamily\contentsname}{}{}

%\renewenvironment{leftbar}
%{\def\FrameCommand{\hspace{6em}%
%		{\color{myyellow}\vrule width 2pt depth 6pt}\hspace{1em}}%
%	\MakeFramed{\parshape 1 0cm \dimexpr\textwidth-6em\relax\FrameRestore}\vskip2pt%
%}
%{\endMakeFramed}

%\titlecontents{chapter}
%[0em]{\vspace*{2\baselineskip}}
%{\parbox{4.5em}{%
%		\hfill\Huge\sffamily\bfseries\color{myred}\thecontentspage}%
%	\vspace*{-2.3\baselineskip}\leftbar\textsc{\small\chaptername~\thecontentslabel}\\\sffamily}
%{}{\endleftbar}
%\titlecontents{section}
%[8.4em]
%{\sffamily\contentslabel{3em}}{}{}
%{\hspace{0.5em}\nobreak\itshape\color{myred}\contentspage}
%\titlecontents{subsection}
%[8.4em]
%{\sffamily\contentslabel{3em}}{}{}  
%{\hspace{0.5em}\nobreak\itshape\color{myred}\contentspage}



%%%%%%%%%%%%%%%%%%%%%%%%%%%%%%%%%%%%%%%%%%%
% TABLE OF CONTENTS
%%%%%%%%%%%%%%%%%%%%%%%%%%%%%%%%%%%%%%%%%%%

\usepackage{tikz}
\definecolor{doc}{RGB}{0,60,110}
\usepackage{titletoc}
\contentsmargin{0cm}
\titlecontents{chapter}[3.7pc]
{\addvspace{30pt}%
	\begin{tikzpicture}[remember picture, overlay]%
		\draw[fill=doc!60,draw=doc!60] (-7,-.1) rectangle (-0.9,.5);%
		\pgftext[left,x=-3.5cm,y=0.2cm]{\color{white}\Large\sc\bfseries Chapter\ \thecontentslabel};%
	\end{tikzpicture}\color{doc!60}\large\sc\bfseries}%
{}
{}
{\;\titlerule\;\large\sc\bfseries Page \thecontentspage
	\begin{tikzpicture}[remember picture, overlay]
		\draw[fill=doc!60,draw=doc!60] (2pt,0) rectangle (4,0.1pt);
	\end{tikzpicture}}%
\titlecontents{section}[3.7pc]
{\addvspace{2pt}}
{\contentslabel[\thecontentslabel]{2pc}}
{}
{\hfill\small \thecontentspage}
[]
\titlecontents*{subsection}[3.7pc]
{\addvspace{-1pt}\small}
{}
{}
{\ --- \small\thecontentspage}
[ \textbullet\ ][]

\makeatletter
\renewcommand{\tableofcontents}{%
	\chapter*{%
	  \vspace*{-20\p@}%
	  \begin{tikzpicture}[remember picture, overlay]%
		  \pgftext[right,x=15cm,y=0.2cm]{\color{doc!60}\Huge\sc\bfseries \contentsname};%
		  \draw[fill=doc!60,draw=doc!60] (13,-.75) rectangle (20,1);%
		  \clip (13,-.75) rectangle (20,1);
		  \pgftext[right,x=15cm,y=0.2cm]{\color{white}\Huge\sc\bfseries \contentsname};%
	  \end{tikzpicture}}%
	\@starttoc{toc}}
\makeatother

%From M275 "Topology" at SJSU
\newcommand{\id}{\mathrm{id}}
\newcommand{\taking}[1]{\xrightarrow{#1}}
\newcommand{\inv}{^{-1}}

%From M170 "Introduction to Graph Theory" at SJSU
\DeclareMathOperator{\diam}{diam}
\DeclareMathOperator{\ord}{ord}
\newcommand{\defeq}{\overset{\mathrm{def}}{=}}

%From the USAMO .tex files
\newcommand{\ts}{\textsuperscript}
\newcommand{\dg}{^\circ}
\newcommand{\ii}{\item}

% % From Math 55 and Math 145 at Harvard
% \newenvironment{subproof}[1][Proof]{%
% \begin{proof}[#1] \renewcommand{\qedsymbol}{$\blacksquare$}}%
% {\end{proof}}

\newcommand{\liff}{\leftrightarrow}
\newcommand{\lthen}{\rightarrow}
\newcommand{\opname}{\operatorname}
\newcommand{\surjto}{\twoheadrightarrow}
\newcommand{\injto}{\hookrightarrow}
\newcommand{\On}{\mathrm{On}} % ordinals
% \newcommand{\EE}{\mathbb{E}} % Expectance
\DeclareMathOperator{\img}{im} % Image
\DeclareMathOperator{\Img}{Im} % Image
\DeclareMathOperator{\coker}{coker} % Cokernel
\DeclareMathOperator{\Coker}{Coker} % Cokernel
\DeclareMathOperator{\Ker}{Ker} % Kernel
\DeclareMathOperator{\rank}{rank}
\DeclareMathOperator{\Spec}{Spec} % spectrum
\DeclareMathOperator{\Tr}{Tr} % trace
\DeclareMathOperator{\pr}{pr} % projection
\DeclareMathOperator{\ext}{ext} % extension
\DeclareMathOperator{\pred}{pred} % predecessor
\DeclareMathOperator{\dom}{dom} % domain
\DeclareMathOperator{\ran}{ran} % range
\DeclareMathOperator{\Hom}{Hom} % homomorphism
\DeclareMathOperator{\Mor}{Mor} % morphisms
\DeclareMathOperator{\End}{End} % endomorphism
% \DeclareMathOperator{\Pr}{Pr} % probability
% \DeclareMathOperator{\Var}{Var} % variance

\newcommand{\eps}{\epsilon}
\newcommand{\veps}{\varepsilon}
\newcommand{\ol}{\overline}
\newcommand{\ul}{\underline}
\newcommand{\wt}{\widetilde}
\newcommand{\wh}{\widehat}
\newcommand{\vocab}[1]{\textbf{\color{blue} #1}}
\providecommand{\half}{\frac{1}{2}}
\newcommand{\dang}{\measuredangle} %% Directed angle
\newcommand{\ray}[1]{\overrightarrow{#1}}
\newcommand{\seg}[1]{\overline{#1}}
\newcommand{\arc}[1]{\wideparen{#1}}
\DeclareMathOperator{\cis}{cis}
\DeclareMathOperator*{\lcm}{lcm}
\DeclareMathOperator*{\argmin}{arg min}
\DeclareMathOperator*{\argmax}{arg max}
\newcommand{\cycsum}{\sum_{\mathrm{cyc}}}
\newcommand{\symsum}{\sum_{\mathrm{sym}}}
\newcommand{\cycprod}{\prod_{\mathrm{cyc}}}
\newcommand{\symprod}{\prod_{\mathrm{sym}}}
\newcommand{\Qed}{\begin{flushright}\qed\end{flushright}}
\newcommand{\parinn}{\setlength{\parindent}{1cm}}
\newcommand{\parinf}{\setlength{\parindent}{0cm}}
% \newcommand{\norm}{\|\cdot\|}
\newcommand{\inorm}{\norm_{\infty}}
\newcommand{\opensets}{\{V_{\alpha}\}_{\alpha\in I}}
\newcommand{\oset}{V_{\alpha}}
\newcommand{\opset}[1]{V_{\alpha_{#1}}}
\newcommand{\lub}{\text{lub}}
\newcommand{\del}[2]{\frac{\partial #1}{\partial #2}}
\newcommand{\Del}[3]{\frac{\partial^{#1} #2}{\partial^{#1} #3}}
\newcommand{\deld}[2]{\dfrac{\partial #1}{\partial #2}}
\newcommand{\Deld}[3]{\dfrac{\partial^{#1} #2}{\partial^{#1} #3}}
\newcommand{\lm}{\lambda}
\newcommand{\uin}{\mathbin{\rotatebox[origin=c]{90}{$\in$}}}
\newcommand{\usubset}{\mathbin{\rotatebox[origin=c]{90}{$\subset$}}}
\newcommand{\lt}{\left}
\newcommand{\rt}{\right}
\newcommand{\bs}[1]{\boldsymbol{#1}}
\newcommand{\exs}{\exists}
\newcommand{\st}{\strut}
\newcommand{\dps}[1]{\displaystyle{#1}}
\newcommand{\va}[1]{\vec{\bm{\mathrm{#1}}}}
\WithSuffix\newcommand\va*[1]{\vec{\bm{#1}}}
\newcommand{\vb}[1]{\bm{\mathrm{#1}}}
\WithSuffix\newcommand\vb*[1]{\bm{#1}}
\newcommand{\vu}[1]{\hat{\bm{\mathrm{#1}}}}
\WithSuffix\newcommand\vu*[1]{\hat{\bm{#1}}}
\renewcommand{\dd}[1]{\mathrm{d}#1}
\renewcommand{\Re}{\mathrm{Re}}
\renewcommand{\Im}{\mathrm{Im}}
\DeclareMathOperator{\tr}{tr}
\newcommand{\csin}[1]{\mintinline{csharp}|#1|}
\renewcommand{\Pr}[1]{\mathrm{Pr}\lt( #1 \rt)}
\renewcommand{\Var}[1]{\mathrm{Var}\lt( #1 \rt)}
\newcommand{\cov}[1]{\mathrm{cov}\lt( #1 \rt)}

\newcommand{\sol}{\setlength{\parindent}{0cm}\textbf{\textit{Solution:}}\setlength{\parindent}{1cm} }
\newcommand{\solve}[1]{\setlength{\parindent}{0cm}\textbf{\textit{Solution: }}\setlength{\parindent}{1cm}#1 \Qed}

% Things Lie
\newcommand{\kb}{\mathfrak b}
\newcommand{\kg}{\mathfrak g}
\newcommand{\kh}{\mathfrak h}
\newcommand{\kn}{\mathfrak n}
\newcommand{\ku}{\mathfrak u}
\newcommand{\kz}{\mathfrak z}
\DeclareMathOperator{\Ext}{Ext} % Ext functor
\DeclareMathOperator{\Tor}{Tor} % Tor functor
\newcommand{\gl}{\opname{\mathfrak{gl}}} % frak gl group
\renewcommand{\sl}{\opname{\mathfrak{sl}}} % frak sl group chktex 6

% More script letters etc.
\newcommand{\SA}{\mathcal A}
\newcommand{\SB}{\mathcal B}
\newcommand{\SC}{\mathcal C}
\newcommand{\SF}{\mathcal F}
\newcommand{\SG}{\mathcal G}
\newcommand{\SH}{\mathcal H}
\newcommand{\OO}{\mathcal O}

\newcommand{\SCA}{\mathscr A}
\newcommand{\SCB}{\mathscr B}
\newcommand{\SCC}{\mathscr C}
\newcommand{\SCD}{\mathscr D}
\newcommand{\SCE}{\mathscr E}
\newcommand{\SCF}{\mathscr F}
\newcommand{\SCG}{\mathscr G}
\newcommand{\SCH}{\mathscr H}

% Mathfrak primes
\newcommand{\km}{\mathfrak m}
\newcommand{\kp}{\mathfrak p}
\newcommand{\kq}{\mathfrak q}

% number sets
\newcommand{\RR}[1][]{\ensuremath{\ifstrempty{#1}{\mathbb{R}}{\mathbb{R}^{#1}}}}
\newcommand{\NN}[1][]{\ensuremath{\ifstrempty{#1}{\mathbb{N}}{\mathbb{N}^{#1}}}}
\newcommand{\ZZ}[1][]{\ensuremath{\ifstrempty{#1}{\mathbb{Z}}{\mathbb{Z}^{#1}}}}
\newcommand{\QQ}[1][]{\ensuremath{\ifstrempty{#1}{\mathbb{Q}}{\mathbb{Q}^{#1}}}}
\newcommand{\CC}[1][]{\ensuremath{\ifstrempty{#1}{\mathbb{C}}{\mathbb{C}^{#1}}}}
\newcommand{\PP}[1][]{\ensuremath{\ifstrempty{#1}{\mathbb{P}}{\mathbb{P}^{#1}}}}
\newcommand{\HH}[1][]{\ensuremath{\ifstrempty{#1}{\mathbb{H}}{\mathbb{H}^{#1}}}}
\newcommand{\FF}[1][]{\ensuremath{\ifstrempty{#1}{\mathbb{F}}{\mathbb{F}^{#1}}}}
% expected value
\newcommand{\EE}{\ensuremath{\mathbb{E}}}
\newcommand{\charin}{\text{ char }}
\DeclareMathOperator{\sign}{sign}
\DeclareMathOperator{\Aut}{Aut}
\DeclareMathOperator{\Inn}{Inn}
\DeclareMathOperator{\Syl}{Syl}
\DeclareMathOperator{\Gal}{Gal}
\DeclareMathOperator{\GL}{GL} % General linear group
\DeclareMathOperator{\SL}{SL} % Special linear group

%---------------------------------------
% BlackBoard Math Fonts :-
%---------------------------------------

%Captital Letters
\newcommand{\bbA}{\mathbb{A}}	\newcommand{\bbB}{\mathbb{B}}
\newcommand{\bbC}{\mathbb{C}}	\newcommand{\bbD}{\mathbb{D}}
\newcommand{\bbE}{\mathbb{E}}	\newcommand{\bbF}{\mathbb{F}}
\newcommand{\bbG}{\mathbb{G}}	\newcommand{\bbH}{\mathbb{H}}
\newcommand{\bbI}{\mathbb{I}}	\newcommand{\bbJ}{\mathbb{J}}
\newcommand{\bbK}{\mathbb{K}}	\newcommand{\bbL}{\mathbb{L}}
\newcommand{\bbM}{\mathbb{M}}	\newcommand{\bbN}{\mathbb{N}}
\newcommand{\bbO}{\mathbb{O}}	\newcommand{\bbP}{\mathbb{P}}
\newcommand{\bbQ}{\mathbb{Q}}	\newcommand{\bbR}{\mathbb{R}}
\newcommand{\bbS}{\mathbb{S}}	\newcommand{\bbT}{\mathbb{T}}
\newcommand{\bbU}{\mathbb{U}}	\newcommand{\bbV}{\mathbb{V}}
\newcommand{\bbW}{\mathbb{W}}	\newcommand{\bbX}{\mathbb{X}}
\newcommand{\bbY}{\mathbb{Y}}	\newcommand{\bbZ}{\mathbb{Z}}

%---------------------------------------
% MathCal Fonts :-
%---------------------------------------

%Captital Letters
\newcommand{\mcA}{\mathcal{A}}	\newcommand{\mcB}{\mathcal{B}}
\newcommand{\mcC}{\mathcal{C}}	\newcommand{\mcD}{\mathcal{D}}
\newcommand{\mcE}{\mathcal{E}}	\newcommand{\mcF}{\mathcal{F}}
\newcommand{\mcG}{\mathcal{G}}	\newcommand{\mcH}{\mathcal{H}}
\newcommand{\mcI}{\mathcal{I}}	\newcommand{\mcJ}{\mathcal{J}}
\newcommand{\mcK}{\mathcal{K}}	\newcommand{\mcL}{\mathcal{L}}
\newcommand{\mcM}{\mathcal{M}}	\newcommand{\mcN}{\mathcal{N}}
\newcommand{\mcO}{\mathcal{O}}	\newcommand{\mcP}{\mathcal{P}}
\newcommand{\mcQ}{\mathcal{Q}}	\newcommand{\mcR}{\mathcal{R}}
\newcommand{\mcS}{\mathcal{S}}	\newcommand{\mcT}{\mathcal{T}}
\newcommand{\mcU}{\mathcal{U}}	\newcommand{\mcV}{\mathcal{V}}
\newcommand{\mcW}{\mathcal{W}}	\newcommand{\mcX}{\mathcal{X}}
\newcommand{\mcY}{\mathcal{Y}}	\newcommand{\mcZ}{\mathcal{Z}}


%---------------------------------------
% Bold Math Fonts :-
%---------------------------------------

%Captital Letters
\newcommand{\bmA}{\boldsymbol{A}}	\newcommand{\bmB}{\boldsymbol{B}}
\newcommand{\bmC}{\boldsymbol{C}}	\newcommand{\bmD}{\boldsymbol{D}}
\newcommand{\bmE}{\boldsymbol{E}}	\newcommand{\bmF}{\boldsymbol{F}}
\newcommand{\bmG}{\boldsymbol{G}}	\newcommand{\bmH}{\boldsymbol{H}}
\newcommand{\bmI}{\boldsymbol{I}}	\newcommand{\bmJ}{\boldsymbol{J}}
\newcommand{\bmK}{\boldsymbol{K}}	\newcommand{\bmL}{\boldsymbol{L}}
\newcommand{\bmM}{\boldsymbol{M}}	\newcommand{\bmN}{\boldsymbol{N}}
\newcommand{\bmO}{\boldsymbol{O}}	\newcommand{\bmP}{\boldsymbol{P}}
\newcommand{\bmQ}{\boldsymbol{Q}}	\newcommand{\bmR}{\boldsymbol{R}}
\newcommand{\bmS}{\boldsymbol{S}}	\newcommand{\bmT}{\boldsymbol{T}}
\newcommand{\bmU}{\boldsymbol{U}}	\newcommand{\bmV}{\boldsymbol{V}}
\newcommand{\bmW}{\boldsymbol{W}}	\newcommand{\bmX}{\boldsymbol{X}}
\newcommand{\bmY}{\boldsymbol{Y}}	\newcommand{\bmZ}{\boldsymbol{Z}}
%Small Letters
\newcommand{\bma}{\boldsymbol{a}}	\newcommand{\bmb}{\boldsymbol{b}}
\newcommand{\bmc}{\boldsymbol{c}}	\newcommand{\bmd}{\boldsymbol{d}}
\newcommand{\bme}{\boldsymbol{e}}	\newcommand{\bmf}{\boldsymbol{f}}
\newcommand{\bmg}{\boldsymbol{g}}	\newcommand{\bmh}{\boldsymbol{h}}
\newcommand{\bmi}{\boldsymbol{i}}	\newcommand{\bmj}{\boldsymbol{j}}
\newcommand{\bmk}{\boldsymbol{k}}	\newcommand{\bml}{\boldsymbol{l}}
\newcommand{\bmm}{\boldsymbol{m}}	\newcommand{\bmn}{\boldsymbol{n}}
\newcommand{\bmo}{\boldsymbol{o}}	\newcommand{\bmp}{\boldsymbol{p}}
\newcommand{\bmq}{\boldsymbol{q}}	\newcommand{\bmr}{\boldsymbol{r}}
\newcommand{\bms}{\boldsymbol{s}}	\newcommand{\bmt}{\boldsymbol{t}}
\newcommand{\bmu}{\boldsymbol{u}}	\newcommand{\bmv}{\boldsymbol{v}}
\newcommand{\bmw}{\boldsymbol{w}}	\newcommand{\bmx}{\boldsymbol{x}}
\newcommand{\bmy}{\boldsymbol{y}}	\newcommand{\bmz}{\boldsymbol{z}}

%---------------------------------------
% Scr Math Fonts :-
%---------------------------------------

\newcommand{\sA}{{\mathscr{A}}}   \newcommand{\sB}{{\mathscr{B}}}
\newcommand{\sC}{{\mathscr{C}}}   \newcommand{\sD}{{\mathscr{D}}}
\newcommand{\sE}{{\mathscr{E}}}   \newcommand{\sF}{{\mathscr{F}}}
\newcommand{\sG}{{\mathscr{G}}}   \newcommand{\sH}{{\mathscr{H}}}
\newcommand{\sI}{{\mathscr{I}}}   \newcommand{\sJ}{{\mathscr{J}}}
\newcommand{\sK}{{\mathscr{K}}}   \newcommand{\sL}{{\mathscr{L}}}
\newcommand{\sM}{{\mathscr{M}}}   \newcommand{\sN}{{\mathscr{N}}}
\newcommand{\sO}{{\mathscr{O}}}   \newcommand{\sP}{{\mathscr{P}}}
\newcommand{\sQ}{{\mathscr{Q}}}   \newcommand{\sR}{{\mathscr{R}}}
\newcommand{\sS}{{\mathscr{S}}}   \newcommand{\sT}{{\mathscr{T}}}
\newcommand{\sU}{{\mathscr{U}}}   \newcommand{\sV}{{\mathscr{V}}}
\newcommand{\sW}{{\mathscr{W}}}   \newcommand{\sX}{{\mathscr{X}}}
\newcommand{\sY}{{\mathscr{Y}}}   \newcommand{\sZ}{{\mathscr{Z}}}


%---------------------------------------
% Math Fraktur Font
%---------------------------------------

%Captital Letters
\newcommand{\mfA}{\mathfrak{A}}	\newcommand{\mfB}{\mathfrak{B}}
\newcommand{\mfC}{\mathfrak{C}}	\newcommand{\mfD}{\mathfrak{D}}
\newcommand{\mfE}{\mathfrak{E}}	\newcommand{\mfF}{\mathfrak{F}}
\newcommand{\mfG}{\mathfrak{G}}	\newcommand{\mfH}{\mathfrak{H}}
\newcommand{\mfI}{\mathfrak{I}}	\newcommand{\mfJ}{\mathfrak{J}}
\newcommand{\mfK}{\mathfrak{K}}	\newcommand{\mfL}{\mathfrak{L}}
\newcommand{\mfM}{\mathfrak{M}}	\newcommand{\mfN}{\mathfrak{N}}
\newcommand{\mfO}{\mathfrak{O}}	\newcommand{\mfP}{\mathfrak{P}}
\newcommand{\mfQ}{\mathfrak{Q}}	\newcommand{\mfR}{\mathfrak{R}}
\newcommand{\mfS}{\mathfrak{S}}	\newcommand{\mfT}{\mathfrak{T}}
\newcommand{\mfU}{\mathfrak{U}}	\newcommand{\mfV}{\mathfrak{V}}
\newcommand{\mfW}{\mathfrak{W}}	\newcommand{\mfX}{\mathfrak{X}}
\newcommand{\mfY}{\mathfrak{Y}}	\newcommand{\mfZ}{\mathfrak{Z}}
%Small Letters
\newcommand{\mfa}{\mathfrak{a}}	\newcommand{\mfb}{\mathfrak{b}}
\newcommand{\mfc}{\mathfrak{c}}	\newcommand{\mfd}{\mathfrak{d}}
\newcommand{\mfe}{\mathfrak{e}}	\newcommand{\mff}{\mathfrak{f}}
\newcommand{\mfg}{\mathfrak{g}}	\newcommand{\mfh}{\mathfrak{h}}
\newcommand{\mfi}{\mathfrak{i}}	\newcommand{\mfj}{\mathfrak{j}}
\newcommand{\mfk}{\mathfrak{k}}	\newcommand{\mfl}{\mathfrak{l}}
\newcommand{\mfm}{\mathfrak{m}}	\newcommand{\mfn}{\mathfrak{n}}
\newcommand{\mfo}{\mathfrak{o}}	\newcommand{\mfp}{\mathfrak{p}}
\newcommand{\mfq}{\mathfrak{q}}	\newcommand{\mfr}{\mathfrak{r}}
\newcommand{\mfs}{\mathfrak{s}}	\newcommand{\mft}{\mathfrak{t}}
\newcommand{\mfu}{\mathfrak{u}}	\newcommand{\mfv}{\mathfrak{v}}
\newcommand{\mfw}{\mathfrak{w}}	\newcommand{\mfx}{\mathfrak{x}}
\newcommand{\mfy}{\mathfrak{y}}	\newcommand{\mfz}{\mathfrak{z}}


\title{\Huge{Statistics}\\Semester 4}
\author{\huge{}}
\date{}

\begin{document}

\maketitle
\newpage% or \cleardoublepage
% \pdfbookmark[<level>]{<title>}{<dest>}
\pdfbookmark[section]{\contentsname}{toc}
\tableofcontents
\pagebreak

\chapter{Revision of Probability}

I'm simply gonna list rules.
\begin{align*}
	\EE(X)    & = \mu =\sum_{i\in\Omega}X_i\Pr{X_i}                            \\
	\EE(g(X)) & = \sum_{i\in\Omega}g(X_i)\Pr{X_i}                              \\
	\EE(aX+b) & = a\EE(X) + b                                                  \\
	\EE(X+Y)  & = \EE(X) + \EE(Y)\quad\text{if both variables are independent}
\end{align*}

\begin{align*}
	\Var{X}     & = \sigma^2 = \EE(X^2) - \mu^2          \\
	\Var{aX+bY} & = a^2\Var{X} + b^2\Var{Y}+2ab\cov{X,Y}
\end{align*}

where
\[
	\cov{X,Y} = \EE(XY) - \EE(X)\cdot\EE(Y)
	.\]

\section{Discrete Distributions}

\begin{enumerate}
	\ii \textbf{Uniform discrete law}
	\begin{align*}
		 & X(\Omega)= \{1,2,3,\dots,n\}                        \\
		 & \Pr{X=k} = \frac{1}{n}\quad\forall k =1,2,3,\dots,n \\
		 & \begin{cases}
			   \EE(X) = \frac{n+1}{2} \\
			   \Var{X} = \frac{n^2-1}{12}
		   \end{cases}
	\end{align*}
	\ii \textbf{Bernoulli law of parameters $p$} $(0<p<1)$
	\begin{align*}
		 & X \sim \mathrm{B}(p)              \\
		 & X(\Omega) = \{0,1\}               \\
		 & \Pr{X=1} = p \quad \Pr{X=0} = 1-p \\
		 & \begin{cases}
			   \EE(X) = p \\
			   \Var{X} = p(1-p)
		   \end{cases}
	\end{align*}
	\ii \textbf{Binomial law of parameters $n$ and $p$}
	\begin{align*}
		 & X \sim \mathrm{Bin}(n,p)                                      \\
		 & X(\Omega) = \{1,2,\dots,n\}                                   \\
		 & \Pr{X=1} = C^k_n p^kq^{n-k}\quad\forall k\in\{0,1,2,\dots,n\} \\
		 & \begin{cases}
			   \EE(X) = np \\
			   \Var{X} = np(1-p)
		   \end{cases}
	\end{align*}
	\ii \textbf{Hypergeometric law}
	\begin{align*}
		 & X \sim \mathcal{H}(N,n,p)                                                     \\
		 & X(\Omega) = [\max\{0,n-N+M\},\min\{M,n\}]                                     \\
		 & \Pr{X=k} = \frac{C^k_M \cdot C^{n-k}_{N-M}}{C^n_N}\quad\forall k\in X(\Omega) \\
		 & \begin{cases}
			   \EE(X) = np \\
			   \Var{X} = np(1-p)\lt(\frac{N-n}{N-1}\rt)
		   \end{cases}
	\end{align*}
	\ii \textbf{Geometric law}
	\begin{align*}
		 & X \sim \mathrm{G}(p)                          \\
		 & X(\Omega) = \NN^*                             \\
		 & \Pr{X=k} =p(1-p)^{k-1}\quad\forall k\in \NN^* \\
		 & \begin{cases}
			   \EE(X) = \frac{1}{p} \\
			   \Var{X} = \frac{1-p}{p^2}
		   \end{cases}
	\end{align*}
	\ii \textbf{Poisson's law of parameter} $\lambda$ $(\lambda\in\RR_+^*)$
	\begin{align*}
		 & X \sim \mathcal{P}(\lambda)                                      \\
		 & X(\Omega) = \NN                                                  \\
		 & \Pr{X=k} = e^{-\lambda}\frac{\lambda^k}{k!}\quad\forall k\in \NN \\
		 & \begin{cases}
			   \EE(X) = \lambda \\
			   \Var{X} = \lambda
		   \end{cases}
	\end{align*}

\end{enumerate}

\section{Continuous Distributions}

\begin{enumerate}
	\ii \textbf{Uniform law}
	\begin{align*}
		 & f(x) = \begin{cases}
			          \frac{1}{b-a} & \text{if } x\in[a,b] \\
			          0             & \text{else}
		          \end{cases} \\
		 & \begin{cases}
			   \EE(x) =  \frac{a+b}{2} \\
			   \Var{x} = \frac{(b-a)^2}{12}
		   \end{cases}
	\end{align*}
	\ii \textbf{Exponential law}
	\begin{align*}
		 & x \sim \xi(\lambda)                            \\
		 & f(x) = \begin{cases}
			          \lambda e^{-\lambda x} & \text{if } x>0 \\
			          0                      & \text{else}
		          \end{cases} \\
		 & \begin{cases}
			   \EE(x) =  \frac{1}{\lambda} \\
			   \Var{x} = \frac{1}{\lambda^2}
		   \end{cases}
	\end{align*}

	\ii \textbf{Normal law}
	\begin{align*}
		 & x \sim \mathcal{N}(\mu,\sigma)                                     \\
		 & f(x) = \frac{1}{\sigma\sqrt{2\pi}}e^{-\frac{(x-\mu)^2}{2\sigma^2}} \\
		 & \begin{cases}
			   \EE(x) =  \mu \\
			   \Var{x} = \sigma^2
		   \end{cases}
	\end{align*}
	For $\mathcal{N}(0,1)$
	\[
		\Phi(z) = \int_{-\infty}^z\frac{1}{\sqrt{2\pi}}e^{-\frac{x^2}{2}}\dd{x}
		.\]
	\[
		\pi(z) = \Phi(z) - 0.5 = \int_0^z\frac{1}{\sqrt{2\pi}}e^{-\frac{x^2}{2}}\dd{x}
		.\]

\end{enumerate}

\section{Convergence}

\thm{Chebyshev's inequality}{
	Let $X$ be a random variable of expectation $\EE(X)$ and variance $\Var{X}$. Then $\forall \varepsilon$
	\[
		\Pr{\lt| X-\EE(X) \rt|\geq \varepsilon} \leq \frac{\Var{X}}{\varepsilon^2}
		.\]
	it can also be stated as
	\[
		\Pr{\lt| X-\EE(X) \rt|< \varepsilon} \geq 1-\frac{\Var{X}}{\varepsilon^2}
		.\]
}

We say a sequence of random variables $X_n$ converges to $a$ $(X_n)\xrightarrow{\mathrm{Pr}}a$ if $\forall \varepsilon$
\[
	\lim_{n\to+\infty}\Pr{|X_n-a|>\varepsilon}=0
	.\]
or
\[
	\lim_{n\to+\infty}\Pr{|X_n-a|\leq\varepsilon}=1
	.\]

\thm{Weak law of large numbers}{
	Consider a random variable $(X_n)$ of mean $\mu$ and variance $\sigma^2$. Consider the random variable $\tilde{X}_n = \frac{X_1+X_2+\cdots+X_n}{n}$. It can be shown that $\tilde{X}_n$ converges to $\mu$ meaning $\forall \varepsilon$
	\[
		\lim_{n\to+\infty}\Pr{|\tilde{X}_n-\mu|>\varepsilon}=0
		.\]
}

\section{Approximations}
\thm{Binomial by a Poisson}{
	\[
		\mathrm{Bin}(n,p)\sim\mathcal{P}(np)\quad \text{if }\begin{cases}
			n\geq30  \\
			p\leq0.1 \\
			np<15
		\end{cases}
		.\]
}

\thm{Hypergeometric by a Binomial}{
	\[
		\mathcal{H}(N,n,p)\sim\mathrm{Bin}(n,p)\quad \text{if } n\leq0.05N
		.\]
}

\thm{De Moivre–Laplace theorem}{
	\[
		\mathrm{Bin}(n,p)\sim\mathcal{N}\lt(np,\sqrt{np(1-p)}\rt)\quad \text{if }\begin{cases}
			n\geq30 \\
			np\geq5 \\
			n(1-p)\geq5
		\end{cases}
		.\]
	In this case the event $X=k$ can be replaced by $k-0.5<X<l+0.5$
}

\thm{Central limit theorem}{
	Let $(X_n)$ be a sequence of independent random variables following the same law of expectation $\mu$ and of standard deviation $\sigma$. Let $S_n = \sum_{i=1}^n X_i$ and $S_n^* = \frac{S_n-n\mu}{\sigma\sqrt{n}}$. It can be shown that $S_n^*$ converges in law to $\mathcal{N}(0,1)$.
	\begin{align*}
		\EE(S_n)  & = n\mu      \\
		\Var{S_n} & = n\sigma^2
	\end{align*}
}
\section{Further laws}

\thm{Chi square law}{
Let $X_1,X_2,\dots,X_n$ be $n$ independent random variables following the standard normal law $\mathcal{N}(0,1)$. Let $Y={X_1}^2 + {X_2}^2+\dots+{X_n}^2$. We say that $Y$ follows a chi-square law with $n$ degrees of freedom. $Y\sim{\chi_n}^2$.

\begin{align*}
	\EE(Y)  & = n  \\
	\Var{Y} & = 2n
\end{align*}

It can be shown that the density function of $Y$ is
\[
	f(x)=\begin{cases}
		\dfrac{1}{2^{\frac{n}{2}}\Gamma\lt(\frac{n}{2}\rt)}x^{\frac{n}{2}-1}e^{-\frac{x}{2}} & \text{if }x>0 \\
		0                                                                                    & \text{else}
	\end{cases}
	.\]
where $\Gamma$ is the gamma function
\[
	\Gamma(x)=\int_0^{+\infty}t^{x-1}e^{-t}\dd{t}\quad\forall x>0
	.\]
}

\thm{Student law(t-distribution)}{
Let $X$, $Z$ be two independent random random variables such that $X\sim\mathcal{N}(0,1)$ and $Z\sim{\chi_n}^2$. Hence the random variable
\[
	T=\frac{X}{\sqrt{\frac{Z}{n}}}
	.\]
is said to be following a student law. $T\sim\mathcal{T}_n$
\[
	f(t)=\frac{1}{\sqrt{n\pi}}\frac{\Gamma\lt(\dfrac{n+1}{2}\rt)}{\Gamma\lt(\dfrac{n}{2}\rt)}\lt(1+\frac{t^2}{n}\rt)^{-\frac{n+1}{2}}
	.\]
}

\chapter{Estimators}

Let $\theta$ be a certain characteristic of a population $P$ of $N$ individuals, for exmaple letting $\theta$ be the expectation of a certain random variable $X$ defined over the population. We take a sample of size $n<N$ of the population to estimate the value of $\theta$.\\

Let $Y_n$ be a function of the random variables $X_1,X_2,\dots,X_n$. $Y_n$ is called an estimator of $\theta$ if
\[
	\lim_{n\to+\infty}\EE(Y_n)=\theta
	.\]

a consistent estimator if
\[
	\lim_{n\to+\infty}\Var{Y_n}=0
	.\]
and an unbiased estimator if
\[
	\EE(Y_n)=\theta\quad\forall n\in\NN^*
	.\]
the value $y_n$ of $Y_n$ computed from any observed sample is called point estimation of $\theta$

\section{Point estimation of the mean}
Let $X$ be a random variable defined over the population $P$ of the expected value $\mu$ and standard deviation $\sigma$. Consider a sample $(X_1,X_2,\dots,X_n)$ of size $n$, randomly selected from $P$ such that $X_i$ are independent and follow the same law.\\

Consider the statistic $\bar{X}_n = \frac{X_1+X_2+\dots+X_n}{n}$, it is a random variable whose distribution is called the sample distribution of the mean.
\begin{align*}
	\EE(\bar{X}_n)  & = \mu                \\
	\Var{\bar{X}_n} & = \frac{\sigma^2}{n}
\end{align*}

Since $\Var{\bar{X}_n}\xrightarrow[n\to+\infty]{}0$ then $\bar{X}_n$ is a consistent unbiased estimator of the mean $\mu$.
\nt{
	The standard deviation of $\bar{X}_n$ is called standard error of the mean
	\[
		\sigma(\bar{X}_n)=\frac{\sigma}{\sqrt{n}}
		.\]
}

Due to the central limit theorem, as the sample size gets larger and larger $\bar{X}_n$ approaches a normal distribution $\bar{X}_n\sim\mathcal{N}\lt(\mu,\frac{\sigma}{\sqrt{n}}\rt)$.

\section{Point estimator of the variance}

\subsection{Suppose $\mu$ is unknown}

Consider the random variable $S^2$ (estimator of $\sigma^2$)
\[
	S^2 = \frac{1}{n}\sum_{i=1}^n \lt(X_i-\bar{X}_n\rt)^2
	.\]

The expectation of $S^2$ can be proved to be
\[
	\EE(S^2)=\frac{n-1}{n}\sigma^2
	.\]
Since $\EE(S^2)\xrightarrow[n\to+\infty]{}\sigma^2$ then $S^2$ is a biased estimator of $\sigma^2$.\\

Consider the random variable $S'^2$
\[
	S'^2 = \frac{n}{n-1}S^2 = \frac{1}{n-1}\sum_{i=1}^n \lt(X_i-\bar{X}_n\rt)^2
	.\]
Since $\EE(S'^2)=\sigma^2$ then $S'^2$ is an unbiased estimator of $\sigma^2$.\\

Hence $\sigma$ can be estimated by
\[
	S' = \sqrt{\frac{n}{n-1}}S
	.\]
and
\[
	\sigma(\bar{X}_n) = \frac{S}{\sqrt{n-1}}
	.\]
where
\begin{description}
	\ii[$\sigma^2$] variance of the population.
	\ii[$S^2$] variance of the sample.
	\ii[$\sigma^2(\bar{X}_n)$] variance of the distribution of the sample mean.
	\ii[$S'^2$] corrected variance of the sample.
\end{description}
\nt{
	It is better to estimate $\sigma^2$ using $S'^2$ than $S^2$ since $S^2$ is a biased estimator. However, if $n$ (sample size) is big enough $\lt(\frac{n}{n-1}\approx1\rt)$, then $\sigma^2$ can be estimated by $S^2$
}
\subsection{Suppose $\mu$ is known}
Consider the random variable $Z^2$ (not the variance of the sample)
\[
	Z^2=\frac{1}{n}\sum_{i=1}^n\lt(X_i-\mu\rt)^2
	.\]

Since $\EE(Z^2)=\sigma^2$ then $Z^2$ is an unbiased estimator of $\sigma^2$ thus the value $z^2 = \frac{1}{n}\sum_{i=1}^n(x_i-\mu)^2$ is a point estimation of the variance $\sigma^2$ of the population.

\nt{
	If $n>0.05N$ and if the sample is selected without replacement then the value of the variance changes to become
	\[
		\Var{\bar{X}_n}=\lt(\frac{N-n}{N-1}\rt)\frac{\sigma^2}{n}
		.\]
	and the standard error becomes
	\[
		\sigma(\bar{X}_n)=\frac{\sigma}{\sqrt{n}}\sqrt{\frac{N-n}{N-1}}
		.\]
	If the variance of the population is not known then we can use use $S^2$ or $Z^2$ to estimate $\Var{\bar{X}_n}$
	\[
		\Var{\bar{X}_n}=\lt(\frac{N-n}{N-1}\rt)\frac{S^2}{n-1}
		.\]
	and the standard error with
	\[
		\sigma(\bar{X}_n)=\frac{S}{\sqrt{n-1}}\sqrt{\frac{N-n}{N-1}}
		.\]
}

\section{Point estimation of a proportion (percentage)}

Consider a population $P$ of individuals with a proportion $p$ if individuals having a certain characteristic $\theta$. Let $(a_1,a_2,\dots,a_n)$ be a sample randomly selected $P$. We define for each individual $a_i$ the Bernoulli random variable $X_i$ as follows
\[
	\begin{cases}
		X_i=1 & \text{if } a_i \text{ has the characteristic }\theta \text{ with probability }p \\
		X_i=0 & \text{else}
	\end{cases}
	.\]

Let $Y_n=\frac{X_1+X_2+\dots+X_n}{n}=\frac{1}{n}\sum_{i=1}^nX_i$. $Y_n$ is the random variable giving the proportion of individuals of the sample that have the characteristic $\theta$.
\begin{align*}
	\Pr{X_i=1} & =\frac{\text{number of individuals of the population having }\theta}{\text{total number of individuals}} = p \\
	\Pr{X_i=0} & =1-p
\end{align*}

Thus $X_1+X_2+\dots+X_n\sim\mathrm{Bin}(n,p)$
\begin{align*}
	\EE(X_1+X_2+\dots+X_n)  & = np      \\
	\Var{X_1+X_2+\dots+X_n} & = np(1-p)
\end{align*}
\begin{align*}
	\EE(Y_n)  & = p                \\
	\Var{Y_n} & = \frac{p(1-p)}{n}
\end{align*}

Hence $Y_n$ is a consistent unbiased estimator of $p$. Therefore any observed value $y_n$ of $Y_n$ is a point estimator of $P$, meaning $p$ is estimated by the proportion of the sample.

\section{Confidence interval}

\subsection{Confidence interval for the mean}

\begin{enumerate}

	\ii \textbf{Suppose that $n\geq30$, the population is normally distributed, and $\sigma$ is known}\\
	Let $X$ be a random variable defined over a population $P$ of mean $\EE(X)=\mu$ and of variance $\Var{X}=\sigma^2$.

	Here we consider that $\bar{X}_n\sim\mathcal{N}\lt(\mu,\frac{\sigma}{\sqrt{n}}\rt)$. Hence $\frac{\bar{X}_n-\mu}{\sigma_{\bar{X}_n}}\sim\mathcal{N}(0,1)$.

	Given the probability $\gamma$ (level of confidence), we can find $t$ such that
	\begin{align*}
		\Pr{-t\leq\frac{\bar{X}_n-\mu}{\sigma_{\bar{X}_n}}\leq t}                   & =\gamma \\
		\Pr{\bar{X}_n-t\sigma_{\bar{X}_n}\leq\mu\leq \bar{X}_n+t\sigma_{\bar{X}_n}} & =\gamma
	\end{align*}
	where $\pi(t)=\frac{\gamma}{2}$. Knowing $\gamma$ we get $t$. Therefore a $\gamma\%$ confidence interval for the mean $\mu$ is given by
	\[
		\mathrm{IC}_\gamma(\mu)=[\bar{x}_n-t\sigma_{\bar{X}_n},\bar{x}_n+t\sigma_{\bar{X}_n}]
		.\]

	where
	\[
		\sigma_{\bar{X}_n} = \begin{cases}
			\frac{\sigma}{\sqrt{n}} & \text{if }\sigma\text{ is known}                                           \\
			\frac{S}{\sqrt{n-1}}    & \text{if }\sigma\text{ is unknown (estimated by }S'=\sqrt{\frac{n}{n-1}}S) \\
		\end{cases}
		.\]
	\ii \textbf{Suppose that $n<30$, the population is normally distributed, and $\sigma$ is unknown}:\\

	Using the table of student distributed knowing $\gamma$, we determine $t$ such that
	\[
		\Pr{\bar{X}_n-t\frac{S}{\sqrt{n-1}}\leq \mu\leq \bar{X}_n+t\frac{S}{\sqrt{n-1}}} = \gamma
		.\]

	hence the confidence inteval for the mean $\mu$ is
	\[
		\mathrm{IC}_\gamma(\mu) = \lt[\bar{X}_n-t\frac{S}{\sqrt{n-1}},\bar{X}_n+t\frac{S}{\sqrt{n-1}}\rt]
		.\]

	\thm{}{
		\begin{enumerate}
			\ii $\bar{X}_n$ and $S^2$ are two independent random variance.
			\ii The random variable $n\frac{S^2}{\sigma^2}$ follows a chi-square law with $n-1$ degrees of freedom.
		\end{enumerate}
	}

	\thm{}{
		The random variable
		\[
			\tilde{T} = \frac{\bar{X}_n-\mu}{\frac{S'}{\sqrt{n}}} = \frac{\bar{X}_n-\mu}{\frac{S}{\sqrt{n-1}}}
			.\]
		follows a student law (t-distribution) with $n-1$ degrees of freedom
	}
	\ii \textbf{Suppose that $n<30$, the population is not normally distributed}:\\
	In this case we cannot use the normal distributed nor the student distribution. However we can use Chebyshev's inequality.
	\[
		\Pr{|\bar{X}_n-\mu|\leq\varepsilon}\geq 1 -\frac{\sigma_{\bar{X}_n}^2}{\varepsilon^2}
		.\]
	Take $\varepsilon=t\sigma_{\bar{X}_n}$
	\[
		\Pr{\bar{X}_n-t\sigma_{\bar{X}_n} \leq \mu \leq \bar{X}_n+t\sigma_{\bar{X}_n}} \geq 1-\frac{1}{t^2}
		.\]
	Then we set $1-\frac{1}{t^2}$ equal to $\gamma$ solve for $t$ and find the interval as follows
	\[
		\mathrm{IC}_\gamma = [\bar{x}_n-t\sigma_{\bar{X}_n},\bar{x}_n+t\sigma_{\bar{X}_n}]
		.\]
	\nt{
		\begin{itemize}
			\ii if $\sigma$ is known then $\sigma_{\bar{X}_n}=\frac{\sigma}{\sqrt{n}}$
			\ii if $\sigma$ is unknown then we replace $\sigma_{\bar{X}_n}$ by its point estimator $\frac{S'}{\sqrt{n}}=\frac{S}{\sqrt{n-1}}$
		\end{itemize}
	}
\end{enumerate}

\subsection{Confidence interval for a proportion (precentage)}

same setup as last time. If we assume this time that $\mathrm{Bin}(n,p)\approx \mathcal{N}\lt(np,\sqrt{np(1-p)}\rt)$ if ($n\geq30,\;np,n(1-p)\geq5$) then we can say $Y_n\sim\mathcal{N}\lt(p,\sqrt{\frac{p(1-p)}{n}}\rt)$. Knowing $\gamma$ we can determine $t$ such that
\[
	\Pr{-t\leq \frac{Y_n-p}{\sigma_{Y_n}}\leq t}=\gamma
	.\]
The confidence interval becomes
\[
	[y_n-t\sigma_{Y_n}, y_n+t\sigma_{Y_n}]
	.\]

where $\sigma_{Y_n}=\sqrt{\frac{p(1-p)}{n}}$ estimated by
\[
	\sqrt{\frac{n}{n-1}}\sqrt{\frac{y_n(1-y_n)}{n}}=\sqrt{\frac{y_n(1-y_n)}{n-1}}
	.\]

Therefore the confidence interval becomes
\[
	\mathrm{IC}_\gamma(p)=\lt[y_n-t\sqrt{\frac{y_n(1-y_n)}{n-1}}, y_n+t\sqrt{\frac{y_n(1-y_n)}{n-1}}\rt]
	.\]

\nt{
	If $n\geq 100$ then $\frac{n}{n-1}\approx 1$, then the confidence interval is
	\[
		\lt[y_n-t\sqrt{\frac{y_n(1-y_n)}{n}},y_n+t\sqrt{\frac{y_n(1-y_n)}{n}}\rt]
		.\]
}
\nt{
	If the sample is selected without replace and if $n>0.05N$ then we shall put a correcting factor $\frac{N-n}{N-1}$ to $\sigma_{Y_n}=\sqrt{\frac{p(1-p)}{n}}$, thus the confidence interval for proportion $p$ becomes
	\[
		\lt[ y_n-t\sqrt{\frac{N-n}{N-1}}\sqrt{\frac{y_n(1-y_n)}{n-1}}, y_n +t\sqrt{\frac{N-n}{N-1}}\sqrt{\frac{y_n(1-y_n)}{n-1}} \rt]
		.\]
}
\subsection{Confidence interval for the variance}
Assume $X\sim\mathcal{N}(\mu,\sigma)$ and $X_1,X_2,\dots,X_n$ $n$ independent random variables and identically distributed as $X$. We set the variables
\begin{align*}
	\bar{X}_n & = \frac{1}{n}\sum_{i=1}^nX_i                                          \\
	S^2       & = \frac{1}{n}\sum_{i=1}^n\lt(X_i-\bar{X}_n\rt)^2                      \\
	S'^2      & = \frac{n}{n-1}S^2 = \frac{1}{n-1}\sum_{i=1}^n\lt(X_i-\bar{X}_n\rt)^2 \\
	Z^2       & =  \frac{1}{n}\sum_{i=1}^n\lt(X_i-\mu\rt)^2                           \\
\end{align*}

we have
\begin{align*}
	\bar{X}_n             & \sim\mathcal{N}\lt(\mu,\frac{\sigma}{\sqrt{n}}\rt) \\
	n\frac{S^2}{\sigma^2} & \sim{\chi_{n-1}}^2                                 \\
	n\frac{Z^2}{\sigma^2} & \sim{\chi_n}^2                                     \\
\end{align*}

\begin{enumerate}
	\ii Suppose $\mu$ is unknown\\

	Since $n\frac{S^2}{\sigma^2}\sim{\chi_{n-1}}^2$, then we determine the values $v_{\alpha/2}$ and $v_{1-\alpha/2}$ from the chi-square table such that
	\[
		\Pr{v_{\alpha/2}\leq \frac{nS^2}{\sigma^2}\leq v_{1-\alpha/2}} = \gamma=1-\alpha
		.\]
	therefore a confidence interval of level $\gamma$ (risk $\alpha$) is given by
	\[
		IC_\gamma(\sigma^2)=\lt[\frac{nS^2}{v_{1-\alpha/2}}, \frac{nS^2}{v_{\alpha/2}} \rt] = \lt[\frac{(n-1)S'^2}{v_{1-\alpha/2}}, \frac{(n-1)S'^2}{v_{\alpha/2}} \rt]
		.\]

	\ii Suppose $\mu$ is known\\

	From the chi-square table, we determine the values of the quantities $v_{\alpha/2}$ and $v_{1-\alpha/2}$ for the law ${\chi_n}^2$ such that
	\[
		\Pr{v_{\alpha/2}\leq \frac{nZ^2}{\sigma^2}\leq v_{1-\alpha/2}}=\gamma
		.\]

	Therefore the confidence interval of level $\gamma$ is given by
	\[
		\mathrm{IC}_\gamma(\sigma^2)=\lt[\frac{nz^2}{v_{1-\alpha/2}},\frac{nz^2}{v_{\alpha/2}}\rt]
		.\]

\end{enumerate}

\section{Mean Squared Error}

Consider the estimator $y_n = f(x_1,x_2,\dots,x_n)$ of $\theta$. The bias of $y_n$ relative to $\theta$ is
\[
	\mathrm{Bias}(y_n)=\EE(y_n)-\theta
	.\]
The mean squared error of $y_n$ with respect to $\theta$ is
\[
	\mathrm{MSE}(y_n)=\EE\lt[(y_n-\theta)^2\rt]
	.\]
It can also be shown that
\[
	\mathrm{MSE}(y_n)=\Var{y_n} + \mathrm{Bias}(y_n)^2
	.\]
If $y_n$ is an unbiased estimator of $\theta$ then $\mathrm{Bias}(y_n)=\EE(y_n)-\theta=0\Rightarrow$

\[
	\mathrm{MSE}(y_n)=\Var{y_n}
	.\]
Assume $y_n$ and $z_n$ are two estimators of the same parameter $\theta$. We say $y_n$ is more efficient than $z_n$ if
\[
	\mathrm{MSE}(y_n)<\mathrm{MSE}(z_n)
	.\]

Assume $y_n$ and $z_n$ are two unbiased estimators of $\theta$, then $y_n$ is more efficient then $z_n$ if and only if $\Var{y_n}<\Var{z_n}$

\chapter{Hypothesis Testing}

\section{Introduction}

In hypothesis testing, we have a null-hypothesis $H_0$ on a sample space $\Omega$ and an alternative hypothesis $H_1$ on $\Omega$. We want to test $H_0$ against $H_1$.\\

To do so we consider a random sample size $n$ and calculate the probability that $H_0$ is within a certain \emph{significane level} and deduce if we can accpet $H_0$.

As an example consider $H_0 = \mu\neq m$, then $H_1$ is

\begin{align*}
	 & \begin{rcases}
		   H_1 = \mu > m \\
		   H_1 = \mu < m \\
	   \end{rcases}\, \text{One sided test} \\
	 & \begin{rcases}
		   H_1 = \mu \neq m
	   \end{rcases}\, \text{Two sided test}
\end{align*}

\begin{table}
	\begin{center}
		\begin{tabular}[c]{|l|l|l|}
			\hline
			             & $H_0$ is true    & $H_0$ is false   \\
			\hline
			Accept $H_0$ & Correct decision & Type II error    \\
			\hline
			Reject $H_0$ & Type I error     & Correct decision \\
			\hline
		\end{tabular}
	\end{center}
\end{table}

Type I error

\[
	\alpha = \Pr{\text{Type I error}} = \Pr{\text{Reject }H_0|H_0\text{ is true}}
	.\]

Type II error

\[
	\beta = \Pr{\text{Type II error}} = \Pr{\text{Accept }H_0|H_0\text{ is false}}
	.\]

Power of the test

\[
	\pi = \begin{cases}
		\alpha  & \text{if }H_0\text{ is true} \\
		1-\beta & \text{if }H_1\text{ is true} \\
	\end{cases}
\]


\begin{figure}[h]
	\centering
	\begin{subfigure}[b]{\textwidth}
		\centering
		\begin{tikzpicture}
			\begin{axis}[
					no markers, domain=-3:3, samples=100,
					axis lines*=left, xlabel={}, ylabel=$y$,
					height=5cm, width=12cm,
					xtick={0}, ytick=\empty,
					enlargelimits=false, clip=false, axis on top,
					grid = major
				]
				\addplot [fill=cyan!20, draw=none, domain=-3:-1.96] {exp(-x^2/2)/sqrt(2*pi)} \closedcycle;
				\addplot [fill=cyan!20, draw=none, domain=1.96:3] {exp(-x^2/2)/sqrt(2*pi)} \closedcycle;
				\addplot [very thick,black] {exp(-x^2/2)/sqrt(2*pi)} node [pos=0.8, anchor=south west] {$\mu_0$};
				\draw [yshift=-0.6cm, latex-latex](axis cs:-1.96,0) -- node [fill=white] {$1-\alpha$} (axis cs:1.96,0);
			\end{axis}
		\end{tikzpicture}
		\caption*{Rejection region for a 1 (left) sided test}
	\end{subfigure}

	\begin{subfigure}[b]{\textwidth}
		\centering
		\begin{tikzpicture}
			\begin{axis}[
					no markers, domain=-3:3, samples=100,
					axis lines*=left, xlabel={}, ylabel=$y$,
					height=5cm, width=12cm,
					xtick={0}, ytick=\empty,
					enlargelimits=false, clip=false, axis on top,
					grid = major
				]
				\addplot [fill=cyan!20, draw=none, domain=1.96:3] {exp(-x^2/2)/sqrt(2*pi)} \closedcycle;
				\addplot [very thick,black] {exp(-x^2/2)/sqrt(2*pi)} node [pos=0.8, anchor=south west] {$\mu_0$};
				\draw [yshift=-0.6cm, latex-latex](axis cs:-3,0) -- node [fill=white] {$1-\alpha$} (axis cs:1.96,0);
			\end{axis}
		\end{tikzpicture}
		\caption*{Rejection region for a 2 sided test}
	\end{subfigure}
\end{figure}


\section{Comparison between a mean and a reference value}

\subsection{Two sided test}

\begin{align*}
	H_0 & : \mu_0 = \mu    \\
	H_1 & : \mu_0 \neq \mu
\end{align*}

\begin{itemize}
	\ii Assume $\sigma$ is known. Test statistic $T = \frac{\bar{x}_n - \mu_0}{\frac{\sigma}{\sqrt{n}}} $.\\

	Rule of rejection
	\[
		\lt|T\rt| > t \quad \begin{cases}
			\Pr{|T|>t} = \alpha \\
			T\sim \mathcal{N}(0,1)
		\end{cases}
		.\]

	\ii Assume $\sigma$ is unknown. Test statistic $T = \frac{\bar{x}_n - \mu_0}{\frac{S'}{\sqrt{n}}} = \frac{\bar{x}_n - \mu_0}{\frac{S}{\sqrt{n-1}}}$.\\

	Rule of rejection
	\[
		\lt|T\rt| > t \quad \begin{cases}
			\Pr{|T|>t} = \alpha \\
			T\sim \mathcal{T}_{n-1}
		\end{cases}
		.\]

\end{itemize}

\subsection{One sided test}

\subsubsection{Left sided test}

\begin{align*}
	H_0 & : \mu_0 = \mu \\
	H_1 & : \mu_0 < \mu
\end{align*}

\begin{itemize}
	\ii Assume $\sigma$ is known. Test statistic $T = \frac{\bar{x}_n - \mu_0}{\frac{\sigma}{\sqrt{n}}} $.\\

	Rule of rejection
	\[
		T > t \quad \begin{cases}
			\Pr{T>t} = \alpha \\
			T\sim \mathcal{N}(0,1)
		\end{cases}
		.\]

	\ii Assume $\sigma$ is unknown. Test statistic $T = \frac{\bar{x}_n - \mu_0}{\frac{S'}{\sqrt{n}}} = \frac{\bar{x}_n - \mu_0}{\frac{S}{\sqrt{n-1}}}$.\\

	Rule of rejection
	\[
		T > t \quad \begin{cases}
			\Pr{T>t} = \alpha \\
			T\sim \mathcal{T}_{n-1}
		\end{cases}
		.\]
\end{itemize}

\subsubsection{Right sided test}

Same as left sided test but with $T < -t$.

\section{Comparison between a proportion and a reference value}

\subsection{Two sided test}

\begin{align*}
	H_0 & : p_0 = p    \\
	H_1 & : p_0 \neq p
\end{align*}

Test statistic $X \sim \mathrm{Bin}(n,p)$.\\

Rule of rejection
\[
	\begin{cases}
		\Pr{X>b_{n,p_0,1-\alpha/2}} = \Pr{X>b_{n,p_0,\alpha/2}} = \frac{\alpha}{2} \\
		X\sim \mathrm{Bin}(n, p_0)
	\end{cases}
	.\]

Acceptance region $[b_{n,p_0,\alpha/2}, b_{n,p_0,1-\alpha/2}]$.

\[
	\Pr{X\in [b_{n,p_0,\alpha/2}, b_{n,p_0,1-\alpha/2}]} = 1-\alpha
	.\]

However if $n\geq 30$ we can use the normal approximation.


Test statistic $T = \frac{X - np_0}{\sqrt{np_0(1-p_0)}}$.\\

Rule of rejection
\[
	\lt|T\rt| > t \quad \begin{cases}
		\Pr{|T|>t} = \alpha \\
		T\sim \mathcal{N}(0,1)
	\end{cases}
	.\]

\subsection{One sided test}

\begin{align*}
	H_0 & : p_0 = p \\
	H_1 & : p_0 < p
\end{align*}

Test statistic $X \sim \mathrm{Bin}(n,p)$.\\

Rule of rejection
\[
	\Pr{X\geq b_{n,p_0,1-\alpha}} = \alpha
	.\]

Acceptance region $[-\infty, b_{n,p_0,1-\alpha}]$.\\

Normal approximation
Test statistic $T = \frac{X - np_0}{\sqrt{np_0(1-p_0)}}$.\\

Rule of rejection
\[
	T > t \quad \begin{cases}
		\Pr{T>t} = \alpha \\
		T\sim \mathcal{N}(0,1)
	\end{cases}
	.\]

\section{Comparison between a variance and a reference value}

\subsection{Two sided test}

\begin{align*}
	H_0 & : \sigma_0^2 = \sigma^2    \\
	H_1 & : \sigma_0^2 \neq \sigma^2
\end{align*}

Test statistic
\[
	T = \begin{cases}
		n\frac{Z^2}{\sigma_0^2} \sim {\chi_{n}}^2   & \text{if } \mu \text{ is known}   \\
		n\frac{S^2}{\sigma_0^2} \sim {\chi_{n-1}}^2 & \text{if } \mu \text{ is unknown}
	\end{cases}
	.\]

We reject $H_0$ if $T \notin [v_{\alpha/2}, v_{1-\alpha/2}]$.

\[
	\Pr{T \notin [v_{\alpha/2}, v_{1-\alpha/2}]} = \frac{\alpha}{2}\\
	.\]

\subsection{One sided test}

\begin{align*}
	H_0 & : \sigma^2 = \sigma_0^2 \\
	H_1 & : \sigma^2 > \sigma_0^2
\end{align*}

Test statistic

\[
	T = \begin{cases}
		n\frac{Z^2}{\sigma_0^2} \sim {\chi_{n}}^2   & \text{if } \mu \text{ is known}   \\
		n\frac{S^2}{\sigma_0^2} \sim {\chi_{n-1}}^2 & \text{if } \mu \text{ is unknown}
	\end{cases}
	.\]

We reject $H_0$ if $T > v_{1-\alpha}$.

\[
	\Pr{T > v_{1-\alpha}} = \alpha
	.\]

\section{Critical Probability}

\[
	\mathrm{P}_\text{c} = \begin{cases}
		\Pr{|T|\geq|t_0|/\theta=\theta_0} & \text{for one sided tests }        \\
		\Pr{T\geq t_0/\theta=\theta_0}    & \text{for } H_1: \theta > \theta_0 \\
		\Pr{T\leq t_0/\theta=\theta_0}    & \text{for } H_1: \theta < \theta_0
	\end{cases}
	.\]


\section{Comparison between two means}

\subsection{Two sided test}

\begin{align*}
	H_0 & : \mu_1 = \mu_2    \\
	H_1 & : \mu_1 \neq \mu_2
\end{align*}

Test statistic $T = \frac{\bar{X}_1 - \bar{X}_2}{\sqrt{\frac{\sigma_1^2}{n_1^2} + \frac{\sigma_2^2}{n_2^2}}}$.\\

Rule of rejection

\[
	\begin{cases}
		\Pr{|T|>t} = \alpha \\
		T\sim \mathcal{N}(0,1)
	\end{cases}
	.\]

\nt{
	\begin{align*}
		\bar{X}_1 - \bar{X}_2       & \sim \mathcal{N}\lt(\mu_1 - \mu_2, \sqrt{\frac{\sigma_1^2}{n_1} + \frac{\sigma_2^2}{n_2}} \rt)
		\EE(\bar{X}_1 - \bar{X}_2)  & = \mu_1 - \mu_2                                                                                \\
		\Var{\bar{X}_1 - \bar{X}_2} & = \frac{\sigma_1^2}{n_1} + \frac{\sigma_2^2}{n_2}
	\end{align*}
}

\subsection{One sided test}

\begin{align*}
	H_0 & : \mu_1 = \mu_2 \\
	H_1 & : \mu_1 > \mu_2
\end{align*}

Test statistic same as before.\\

Rule of rejection

\[
	\begin{cases}
		\Pr{T>t} = \alpha \\
		T\sim \mathcal{N}(0,1)
	\end{cases}
	.\]

\nt{
	If $\sigma_1$ and $\sigma_2$ are unknown, and $\sigma_1=\sigma_2$ we perform all the same preceding tests but this a new test statistic
	\[
		T = \frac{\bar{X}_1 - \bar{X}_2}{S\sqrt{\frac{1}{n_1} + \frac{1}{n_2}}} \sim \mathcal{T}_{n_1+n_2-2}
		.\]
	if $\sigma_1 \neq \sigma_2$ we use the preceding tests only if $n_1$ and $n_2$ are sufficiently large ($>30$).
}

\section{Comparison between 2 proportions}

Consider the following 2 random variables

\[
	P = \frac{X+Y}{n_1+n_2}\quad S_d^2 = P(1-P)\lt(\frac{1}{n_1} + \frac{1}{n_2}\rt)
	.\]

\subsection{Two sided test}

\begin{align*}
	H_0 & : p_1 = p_2    \\
	H_1 & : p_1 \neq p_2
\end{align*}

Test statistic $T = \frac{\frac{X}{n_1} - \frac{Y}{n_2}}{S_d}$.\\

Rule of rejection

\[
	\begin{cases}
		\Pr{|T|>t} = \alpha \\
		T\sim \mathcal{N}(0,1)
	\end{cases}
	.\]

\subsection{One sided test}

\begin{align*}
	H_0 & : p_1 = p_2 \\
	H_1 & : p_1 > p_2
\end{align*}

Same job as before.

\section{Comparison between two variances}

\subsection{Two sided test}

\begin{align*}
	H_0 & : {\sigma_1}^2 = {\sigma_2}^2    \\
	H_1 & : {\sigma_1}^2 \neq {\sigma_2}^2
\end{align*}

Test statistic $F = \frac{S_1^{'2}}{S_2^{'2}}$.\\

\[
	F \sim \mathcal{F}_{n_1-1,n_2-1}
	.\]

where $\mathcal{F}$ is the Fisher distribution.\\

Rule of rejection

\[
	\begin{cases}
		\Pr{F<f_{n_1-1,n_2-1,1-\alpha/2}} = \Pr{F>f_{n_1-1,n_2-1,\alpha/2}} = \frac{\alpha}{2} \\
		F\sim \mathcal{F}_{n_1-1,n_2-1}
	\end{cases}
	.\]

\subsection{One sided test}

\begin{align*}
	H_0 & : {\sigma_1}^2 = {\sigma_2}^2 \\
	H_1 & : {\sigma_1}^2 < {\sigma_2}^2
\end{align*}

Rule of rejection

\[
	\begin{cases}
		\Pr{F>f_{n_1-1,n_2-1,1-\alpha}} = \frac{\alpha}{2} \\
		F\sim \mathcal{F}_{n_1-1,n_2-1}
	\end{cases}
	.\]

\end{document}

\documentclass{report}

\input{../template/preamble}
\input{../template/macros}
\input{../template/letterfonts}

\title{\Huge{Numerical Analysis}\\Semester 4}
\author{}
\date{}

\begin{document}

\maketitle
\newpage% or \cleardoublepage
% \pdfbookmark[<level>]{<title>}{<dest>}
\pdfbookmark[section]{\contentsname}{toc}
\tableofcontents
\pagebreak

\chapter{Interpolation}

\section{Linear Interpolation}

\begin{minipage}{0.4\linewidth}
	\begin{figure}[H]
		\centering
		\begin{tikzpicture}
			\begin{axis}[
					legend pos=outer north east,
					axis lines = box,
					xlabel = $x$,
					ylabel = $f(x)$,
					variable = t,
					trig format plots = rad,
					axis x line=bottom,
					axis y line=left,
					xmin=0,xmax=5.5,
					ymin=0,ymax=3.9,
				]

				\addplot[
					color=blue,
				]
				coordinates {
						(0,0)(1,2)(2,3)(3,1)(4,3)(5,3.5)
					};
				\addplot[
					color=red,
					only marks,
					mark=*,
				]
				coordinates {
						(0,0)(1,2)(2,3)(3,1)(4,3)(5,3.5)
					};

			\end{axis}
		\end{tikzpicture}
	\end{figure}
\end{minipage}
\hfill
\begin{minipage}{0.4\linewidth}
	\begin{tabular}{l|l}
		$x$ & $f(x)$ \\
		\hline
		0   & 0      \\
		1   & 2      \\
		2   & 3      \\
		3   & 1      \\
		4   & 3      \\
		5   & 3.5    \\
	\end{tabular}
\end{minipage}

Linear interpolation is just drawing lines between the data points.
\dfn{Linear Interpolation(lerp) equation}{
	The equation of the lines between data points is

	\[
		y = \frac{y_{i+1} - y_i}{x_{i+1} - x_i}(x-x_i) + y_i
		.\]
}
\thm{Error due to linear interpolation}{
	Let $f$ be a continuous and differentiable on $[a,b]$. We define the error $z(x)$ to be
	\[
		|z(x)| \leq \frac{(b-a)^2 }{8 }\sup_{a\leq x \leq b}\lt|f''(x)\rt|
		.\]
}

\section{Polynomial Interpolation}

\subsection{Lagrange Polynomials}

Really nice video \href{https://www.youtube.com/watch?v=bzp_q7NDdd4}{here} explaining Lagrange polynomials.


\begin{figure}[H]
	\centering
	\begin{tikzpicture}
		\begin{axis}[
				legend pos=outer north east,
				axis lines = box,
				xlabel = $x$,
				ylabel = $f(x)$,
				axis x line=bottom,
				axis y line=left,
				xmin=0,xmax=5.5,
			]

			\addplot[
				only marks,
				color=red,
				mark=*,
			]
			coordinates {
					(0,0)(1,2)(2,3)(3,1)(4,3)(5,3.5)
				};

			\addplot [
				domain=0:5,
				samples=100,
				color=blue,
			]
			{-0.17916*x^5+(13/6)*x^4+(-425/48)*x^3+(163/12)*x^2+(-283/60)*x};
			% \addlegendentry{$-\frac{43}{240}x^5 + \frac{13}{6}x^4 - \frac{425}{48}x^3 + \frac{163 }{12}x^2 - \frac{283}{60}x$}
		\end{axis}
	\end{tikzpicture}
\end{figure}


\thm{Lagrange polynomial equation}{
	Consider a set of $n$ points $(x_1,y_1),(x_2,y_2),\dots,(x_n,y_n)$. The Lagrange polynomial for this set of data is
	\[
		L(x) = \sum_{k=0}^{n} y_k \ell_k(x)
		.\]

	where
	\[
		\ell_k(x) = \prod_{\substack{i=1\\i\neq k}}^{n}\frac{x-x_i }{x_k-x_i}
		.\]
}

\subsection{Newton Polynomial}

\dfn{Newton Polynomial equation}{
	Consider a set of $n$ points $(x_1,y_1),(x_2,y_2),\dots,(x_n,y_n)$. The Newton polynomial for this set of data is
	\[
		p_n(x) = a_0 + a_1(x-x_0) + a_2(x-x_0)(x-x_1) + \dots + a_n \prod_{i=0}^{n-1}(x-x_i)
		.\]
	where
	\[
		a_i = f[x_0,x_1,\dots,x_i]
		.\]
	Here $f[\dots]$ is the divided difference of the inputted data.
}

The divided difference has 2 formulas, the recurrence formula
\[
	f[x_0,x_1,\dots,x_{n+1}] = \frac{f[x_1,x_2,\dots,x_{n+1}]-f[x_0,x_1,\dots,x_{n}]}{x_{n+1}-x_0}
	.\]

and a general formula
\[
	f[x_0,x_1,\dots,x_n] = \sum_{i=1}^{n}\frac{y_i }{\prod_{\substack{k=0\\k\neq i}}^{n} (x_i - x_k)}
	.\]
Now forget you ever saw those cause there is an easier method to finding the divided difference.

\subsubsection{Divided Difference Table}
\[
	\begin{array}{cccccc}
		x_0 & y_0                                                                                              \\
		    &     & \frac{y_1-y_0}{x_1-x_0}=f[x_0,x_1]                                                         \\
		x_1 & y_1 &                                    & \frac{f[x_1,x_2]-f[x_0,x_1]}{x_2-x_0}                 \\
		    &     & \frac{y_2-y_1}{x_2-x_1}=f[x_1,x_2] &                                       & \dots         \\
		x_2 & y_2 &                                    & \frac{f[x_2,x_3]-f[x_1,x_2]}{x_3-x_1} &       & \dots \\
		    &     & \frac{y_3-y_2}{x_3-x_2}=f[x_2,x_3] &                                       & \dots         \\
		x_3 & y_3 &                                    & \frac{f[x_3,x_4]-f[x_2,x_3]}{x_4-x_2}                 \\
		    &     & \frac{y_4-y_3}{x_4-x_3}=f[x_3,x_4]                                                         \\
		x_4 & y_4
	\end{array}
\]
After we have constructed the table we can find the divided difference we want by looking at the top diagonal

\[
	\begin{array}{cccccc}
		x_0 & y_0                                                                             \\
		    &     & f[x_0,x_1]                                                                \\
		x_1 & y_1 &            & f[x_0,x_1,x_2]                                               \\
		    &     & \dots      &                & f[x_0,x_1,x_2,x_3]                          \\
		x_2 & y_2 &            & \dots          &                    & f[x_0,x_1,x_2,x_3,x_4] \\
		    &     & \dots      &                & \dots                                       \\
		x_3 & y_3 &            & \dots                                                        \\
		    &     & \dots                                                                     \\
		x_4 & y_4
	\end{array}
\]

\ex{Yes}{
	I wanna do an example for the data I used at the top but it's 1AM so cowabunga it is
}

\end{document}

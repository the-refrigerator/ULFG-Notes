\documentclass{report}

%%%%%%%%%%%%%%%%%%%%%%%%%%%%%%%%%
% PACKAGE IMPORTS
%%%%%%%%%%%%%%%%%%%%%%%%%%%%%%%%%


\usepackage[tmargin=2cm,rmargin=1in,lmargin=1in,margin=0.85in,bmargin=2cm,footskip=.2in]{geometry}
\usepackage{amsmath,amsfonts,amsthm,amssymb,mathtools}
\usepackage[varbb]{newpxmath}
\usepackage{xfrac}
\usepackage[makeroom]{cancel}
\usepackage{mathtools}
\usepackage{bookmark}
\usepackage{enumitem}
\usepackage{hyperref,theoremref}
\usepackage{xparse}
\hypersetup{
	pdftitle={Hamboola my beloved},
	colorlinks=true, linkcolor=doc!90,
	bookmarksnumbered=true,
	bookmarksopen=true
}
\usepackage[most,many,breakable]{tcolorbox}
\usepackage{xcolor}
\usepackage{varwidth}
\usepackage{varwidth}
\usepackage{etoolbox}
\usepackage{bm}
%\usepackage{authblk}
\usepackage{pgfplots}
\usepackage{nameref}
\usepackage{multicol,array}
\usepackage{tikz-cd}
\usepackage[ruled,vlined,linesnumbered]{algorithm2e}
\usepackage{comment} % enables the use of multi-line comments (\ifx \fi) 
\usepackage{import}
\usepackage{xifthen}
\usepackage{pdfpages}
\usepackage{transparent}
\usepackage{minted}
\usepackage{fontspec}
\usepackage{tasks}
\usepackage{chemfig}
\usepackage[version=4]{mhchem}
\usepackage{suffix}
\usepackage{tabularx}
\usepackage{subcaption}
\usepackage{physics}


% \setmonofont{SpaceMono Nerd Font}
\setminted{fontsize=\footnotesize}

\newcommand\mycommfont[1]{\footnotesize\ttfamily\textcolor{blue}{#1}}
\SetCommentSty{mycommfont}
\newcommand{\incfig}[1]{%
	\def\svgwidth{\columnwidth}
	\import{./figures/}{#1.pdf_tex}
}

\usepackage{tikzsymbols}
% \renewcommand\qedsymbol{$\Laughey$}
\pgfplotsset{compat=1.18}

%\usepackage{import}
%\usepackage{xifthen}
%\usepackage{pdfpages}
%\usepackage{transparent}


%%%%%%%%%%%%%%%%%%%%%%%%%%%%%%
% SELF MADE COLORS
%%%%%%%%%%%%%%%%%%%%%%%%%%%%%%



\definecolor{myg}{RGB}{56, 140, 70}
\definecolor{myb}{RGB}{45, 111, 177}
\definecolor{myr}{RGB}{199, 68, 64}
\definecolor{mytheorembg}{HTML}{F2F2F9}
\definecolor{mytheoremfr}{HTML}{00007B}
\definecolor{mylenmabg}{HTML}{FFFAF8}
\definecolor{mylenmafr}{HTML}{983b0f}
\definecolor{mypropbg}{HTML}{f2fbfc}
\definecolor{mypropfr}{HTML}{191971}
\definecolor{myexamplebg}{HTML}{F2FBF8}
\definecolor{myexamplefr}{HTML}{88D6D1}
\definecolor{myexampleti}{HTML}{2A7F7F}
\definecolor{mydefinitbg}{HTML}{E5E5FF}
\definecolor{mydefinitfr}{HTML}{3F3FA3}
\definecolor{notesgreen}{RGB}{0,162,0}
\definecolor{myp}{RGB}{197, 92, 212}
\definecolor{mygr}{HTML}{2C3338}
\definecolor{myred}{RGB}{127,0,0}
\definecolor{myyellow}{RGB}{169,121,69}
\definecolor{myexercisebg}{HTML}{F2FBF8}
\definecolor{myexercisefg}{HTML}{88D6D1}
\definecolor{codebg}{HTML}{0D1117}

%%%%%%%%%%%%%%%%%%%%%%%%%%%%
% TCOLORBOX SETUPS
%%%%%%%%%%%%%%%%%%%%%%%%%%%%

\setlength{\parindent}{0cm}
%================================
% THEOREM BOX
%================================

\tcbuselibrary{theorems,skins,hooks}
\newtcbtheorem[number within=section]{Theorem}{Theorem}
{%
	enhanced,
	breakable,
	colback = mytheorembg,
	frame hidden,
	boxrule = 0sp,
	borderline west = {2pt}{0pt}{mytheoremfr},
	sharp corners,
	detach title,
	before upper = \tcbtitle\par\smallskip,
	coltitle = mytheoremfr,
	fonttitle = \bfseries\sffamily,
	description font = \mdseries,
	separator sign none,
	segmentation style={solid, mytheoremfr},
}
{th}

\tcbuselibrary{theorems,skins,hooks}
\newtcbtheorem[number within=chapter]{theorem}{Theorem}
{%
	enhanced,
	breakable,
	colback = mytheorembg,
	frame hidden,
	boxrule = 0sp,
	borderline west = {2pt}{0pt}{mytheoremfr},
	sharp corners,
	detach title,
	before upper = \tcbtitle\par\smallskip,
	coltitle = mytheoremfr,
	fonttitle = \bfseries\sffamily,
	description font = \mdseries,
	separator sign none,
	segmentation style={solid, mytheoremfr},
}
{th}


\tcbuselibrary{theorems,skins,hooks}
\newtcolorbox{Theoremcon}
{%
	enhanced
	,breakable
	,colback = mytheorembg
	,frame hidden
	,boxrule = 0sp
	,borderline west = {2pt}{0pt}{mytheoremfr}
	,sharp corners
	,description font = \mdseries
	,separator sign none
}

%================================
% Corollery
%================================
\tcbuselibrary{theorems,skins,hooks}
\newtcbtheorem[number within=section]{Corollary}{Corollary}
{%
	enhanced
	,breakable
	,colback = myp!10
	,frame hidden
	,boxrule = 0sp
	,borderline west = {2pt}{0pt}{myp!85!black}
	,sharp corners
	,detach title
	,before upper = \tcbtitle\par\smallskip
	,coltitle = myp!85!black
	,fonttitle = \bfseries\sffamily
	,description font = \mdseries
	,separator sign none
	,segmentation style={solid, myp!85!black}
}
{th}
\tcbuselibrary{theorems,skins,hooks}
\newtcbtheorem[number within=chapter]{corollary}{Corollary}
{%
	enhanced
	,breakable
	,colback = myp!10
	,frame hidden
	,boxrule = 0sp
	,borderline west = {2pt}{0pt}{myp!85!black}
	,sharp corners
	,detach title
	,before upper = \tcbtitle\par\smallskip
	,coltitle = myp!85!black
	,fonttitle = \bfseries\sffamily
	,description font = \mdseries
	,separator sign none
	,segmentation style={solid, myp!85!black}
}
{th}


%================================
% LENMA
%================================

\tcbuselibrary{theorems,skins,hooks}
\newtcbtheorem[number within=section]{Lenma}{Lenma}
{%
	enhanced,
	breakable,
	colback = mylenmabg,
	frame hidden,
	boxrule = 0sp,
	borderline west = {2pt}{0pt}{mylenmafr},
	sharp corners,
	detach title,
	before upper = \tcbtitle\par\smallskip,
	coltitle = mylenmafr,
	fonttitle = \bfseries\sffamily,
	description font = \mdseries,
	separator sign none,
	segmentation style={solid, mylenmafr},
}
{th}

\tcbuselibrary{theorems,skins,hooks}
\newtcbtheorem[number within=chapter]{lenma}{Lenma}
{%
	enhanced,
	breakable,
	colback = mylenmabg,
	frame hidden,
	boxrule = 0sp,
	borderline west = {2pt}{0pt}{mylenmafr},
	sharp corners,
	detach title,
	before upper = \tcbtitle\par\smallskip,
	coltitle = mylenmafr,
	fonttitle = \bfseries\sffamily,
	description font = \mdseries,
	separator sign none,
	segmentation style={solid, mylenmafr},
}
{th}


%================================
% PROPOSITION
%================================

\tcbuselibrary{theorems,skins,hooks}
\newtcbtheorem[number within=section]{Prop}{Proposition}
{%
	enhanced,
	breakable,
	colback = mypropbg,
	frame hidden,
	boxrule = 0sp,
	borderline west = {2pt}{0pt}{mypropfr},
	sharp corners,
	detach title,
	before upper = \tcbtitle\par\smallskip,
	coltitle = mypropfr,
	fonttitle = \bfseries\sffamily,
	description font = \mdseries,
	separator sign none,
	segmentation style={solid, mypropfr},
}
{th}

\tcbuselibrary{theorems,skins,hooks}
\newtcbtheorem[number within=chapter]{prop}{Proposition}
{%
	enhanced,
	breakable,
	colback = mypropbg,
	frame hidden,
	boxrule = 0sp,
	borderline west = {2pt}{0pt}{mypropfr},
	sharp corners,
	detach title,
	before upper = \tcbtitle\par\smallskip,
	coltitle = mypropfr,
	fonttitle = \bfseries\sffamily,
	description font = \mdseries,
	separator sign none,
	segmentation style={solid, mypropfr},
}
{th}


%================================
% CLAIM
%================================

\tcbuselibrary{theorems,skins,hooks}
\newtcbtheorem[number within=section]{claim}{Claim}
{%
	enhanced
	,breakable
	,colback = myg!10
	,frame hidden
	,boxrule = 0sp
	,borderline west = {2pt}{0pt}{myg}
	,sharp corners
	,detach title
	,before upper = \tcbtitle\par\smallskip
	,coltitle = myg!85!black
	,fonttitle = \bfseries\sffamily
	,description font = \mdseries
	,separator sign none
	,segmentation style={solid, myg!85!black}
}
{th}



%================================
% Exercise
%================================

\tcbuselibrary{theorems,skins,hooks}
\newtcbtheorem[number within=section]{Exercise}{Exercise}
{%
	enhanced,
	breakable,
	colback = myexercisebg,
	frame hidden,
	boxrule = 0sp,
	borderline west = {2pt}{0pt}{myexercisefg},
	sharp corners,
	detach title,
	before upper = \tcbtitle\par\smallskip,
	coltitle = myexercisefg,
	fonttitle = \bfseries\sffamily,
	description font = \mdseries,
	separator sign none,
	segmentation style={solid, myexercisefg},
}
{th}

\tcbuselibrary{theorems,skins,hooks}
\newtcbtheorem[number within=chapter]{exercise}{Exercise}
{%
	enhanced,
	breakable,
	colback = myexercisebg,
	frame hidden,
	boxrule = 0sp,
	borderline west = {2pt}{0pt}{myexercisefg},
	sharp corners,
	detach title,
	before upper = \tcbtitle\par\smallskip,
	coltitle = myexercisefg,
	fonttitle = \bfseries\sffamily,
	description font = \mdseries,
	separator sign none,
	segmentation style={solid, myexercisefg},
}
{th}

%================================
% EXAMPLE BOX
%================================

\newtcbtheorem[number within=section]{Example}{Example}
{%
	colback = myexamplebg
	,breakable
	,colframe = myexamplefr
	,coltitle = myexampleti
	,boxrule = 1pt
	,sharp corners
	,detach title
	,before upper=\tcbtitle\par\smallskip
	,fonttitle = \bfseries
	,description font = \mdseries
	,separator sign none
	,description delimiters parenthesis
}
{ex}

\newtcbtheorem[number within=chapter]{example}{Example}
{%
	colback = myexamplebg
	,breakable
	,colframe = myexamplefr
	,coltitle = myexampleti
	,boxrule = 1pt
	,sharp corners
	,detach title
	,before upper=\tcbtitle\par\smallskip
	,fonttitle = \bfseries
	,description font = \mdseries
	,separator sign none
	,description delimiters parenthesis
}
{ex}

%================================
% DEFINITION BOX
%================================

\newtcbtheorem[number within=section]{Definition}{Definition}{enhanced,
	before skip=2mm,after skip=2mm, colback=red!5,colframe=red!80!black,boxrule=0.5mm,
	attach boxed title to top left={xshift=1cm,yshift*=1mm-\tcboxedtitleheight}, varwidth boxed title*=-3cm,
	boxed title style={frame code={
					\path[fill=tcbcolback]
					([yshift=-1mm,xshift=-1mm]frame.north west)
					arc[start angle=0,end angle=180,radius=1mm]
					([yshift=-1mm,xshift=1mm]frame.north east)
					arc[start angle=180,end angle=0,radius=1mm];
					\path[left color=tcbcolback!60!black,right color=tcbcolback!60!black,
						middle color=tcbcolback!80!black]
					([xshift=-2mm]frame.north west) -- ([xshift=2mm]frame.north east)
					[rounded corners=1mm]-- ([xshift=1mm,yshift=-1mm]frame.north east)
					-- (frame.south east) -- (frame.south west)
					-- ([xshift=-1mm,yshift=-1mm]frame.north west)
					[sharp corners]-- cycle;
				},interior engine=empty,
		},
	fonttitle=\bfseries,
	title={#2},#1}{def}
\newtcbtheorem[number within=chapter]{definition}{Definition}{enhanced,
	before skip=2mm,after skip=2mm, colback=red!5,colframe=red!80!black,boxrule=0.5mm,
	attach boxed title to top left={xshift=1cm,yshift*=1mm-\tcboxedtitleheight}, varwidth boxed title*=-3cm,
	boxed title style={frame code={
					\path[fill=tcbcolback]
					([yshift=-1mm,xshift=-1mm]frame.north west)
					arc[start angle=0,end angle=180,radius=1mm]
					([yshift=-1mm,xshift=1mm]frame.north east)
					arc[start angle=180,end angle=0,radius=1mm];
					\path[left color=tcbcolback!60!black,right color=tcbcolback!60!black,
						middle color=tcbcolback!80!black]
					([xshift=-2mm]frame.north west) -- ([xshift=2mm]frame.north east)
					[rounded corners=1mm]-- ([xshift=1mm,yshift=-1mm]frame.north east)
					-- (frame.south east) -- (frame.south west)
					-- ([xshift=-1mm,yshift=-1mm]frame.north west)
					[sharp corners]-- cycle;
				},interior engine=empty,
		},
	fonttitle=\bfseries,
	title={#2},#1}{def}



%================================
% Solution BOX
%================================

\makeatletter
\newtcbtheorem{question}{Question}{enhanced,
	breakable,
	colback=white,
	colframe=myb!80!black,
	attach boxed title to top left={yshift*=-\tcboxedtitleheight},
	fonttitle=\bfseries,
	title={#2},
	boxed title size=title,
	boxed title style={%
			sharp corners,
			rounded corners=northwest,
			colback=tcbcolframe,
			boxrule=0pt,
		},
	underlay boxed title={%
			\path[fill=tcbcolframe] (title.south west)--(title.south east)
			to[out=0, in=180] ([xshift=5mm]title.east)--
			(title.center-|frame.east)
			[rounded corners=\kvtcb@arc] |-
			(frame.north) -| cycle;
		},
	#1
}{def}
\makeatother

%================================
% SOLUTION BOX
%================================

\makeatletter
\newtcolorbox{solution}{enhanced,
	breakable,
	colback=white,
	colframe=myg!80!black,
	attach boxed title to top left={yshift*=-\tcboxedtitleheight},
	title=Solution,
	boxed title size=title,
	boxed title style={%
			sharp corners,
			rounded corners=northwest,
			colback=tcbcolframe,
			boxrule=0pt,
		},
	underlay boxed title={%
			\path[fill=tcbcolframe] (title.south west)--(title.south east)
			to[out=0, in=180] ([xshift=5mm]title.east)--
			(title.center-|frame.east)
			[rounded corners=\kvtcb@arc] |-
			(frame.north) -| cycle;
		},
}
\makeatother

%================================
% Question BOX
%================================

\makeatletter
\newtcbtheorem{qstion}{Question}{enhanced,
	breakable,
	colback=white,
	colframe=mygr,
	attach boxed title to top left={yshift*=-\tcboxedtitleheight},
	fonttitle=\bfseries,
	title={#2},
	boxed title size=title,
	boxed title style={%
			sharp corners,
			rounded corners=northwest,
			colback=tcbcolframe,
			boxrule=0pt,
		},
	underlay boxed title={%
			\path[fill=tcbcolframe] (title.south west)--(title.south east)
			to[out=0, in=180] ([xshift=5mm]title.east)--
			(title.center-|frame.east)
			[rounded corners=\kvtcb@arc] |-
			(frame.north) -| cycle;
		},
	#1
}{def}
\makeatother

\newtcbtheorem[number within=chapter]{wconc}{Wrong Concept}{
	breakable,
	enhanced,
	colback=white,
	colframe=myr,
	arc=0pt,
	outer arc=0pt,
	fonttitle=\bfseries\sffamily\large,
	colbacktitle=myr,
	attach boxed title to top left={},
	boxed title style={
			enhanced,
			skin=enhancedfirst jigsaw,
			arc=3pt,
			bottom=0pt,
			interior style={fill=myr}
		},
	#1
}{def}



%================================
% NOTE BOX
%================================

\usetikzlibrary{hobby}
\usetikzlibrary{arrows,calc,shadows.blur}
\tcbuselibrary{skins}
\newtcolorbox{note}[1][]{%
	enhanced jigsaw,
	colback=gray!20!white,%
	colframe=gray!80!black,
	size=small,
	boxrule=1pt,
	title=\textbf{Note:-},
	halign title=flush center,
	coltitle=black,
	breakable,
	drop shadow=black!50!white,
	attach boxed title to top left={xshift=1cm,yshift=-\tcboxedtitleheight/2,yshifttext=-\tcboxedtitleheight/2},
	minipage boxed title=1.5cm,
	boxed title style={%
			colback=white,
			size=fbox,
			boxrule=1pt,
			boxsep=2pt,
			underlay={%
					\coordinate (dotA) at ($(interior.west) + (-0.5pt,0)$);
					\coordinate (dotB) at ($(interior.east) + (0.5pt,0)$);
					\begin{scope}
						\clip (interior.north west) rectangle ([xshift=3ex]interior.east);
						\filldraw [white, blur shadow={shadow opacity=60, shadow yshift=-.75ex}, rounded corners=2pt] (interior.north west) rectangle (interior.south east);
					\end{scope}
					\begin{scope}[gray!80!black]
						\fill (dotA) circle (2pt);
						\fill (dotB) circle (2pt);
					\end{scope}
				},
		},
	#1,
}

%%%%%%%%%%%%%%%%%%%%%%%%%%%%%%
% SELF MADE COMMANDS
%%%%%%%%%%%%%%%%%%%%%%%%%%%%%%


\newcommand{\thm}[2]{\begin{Theorem}{#1}{}#2\end{Theorem}}
\newcommand{\cor}[2]{\begin{Corollary}{#1}{}#2\end{Corollary}}
\newcommand{\mlenma}[2]{\begin{Lenma}{#1}{}#2\end{Lenma}}
\newcommand{\mprop}[2]{\begin{Prop}{#1}{}#2\end{Prop}}
\newcommand{\clm}[3]{\begin{claim}{#1}{#2}#3\end{claim}}
\newcommand{\wc}[2]{\begin{wconc}{#1}{}\setlength{\parindent}{1cm}#2\end{wconc}}
\newcommand{\thmcon}[1]{\begin{Theoremcon}{#1}\end{Theoremcon}}
\newcommand{\ex}[2]{\begin{Example}{#1}{}#2\end{Example}}
\newcommand{\dfn}[2]{\begin{Definition}[colbacktitle=red!75!black]{#1}{}#2\end{Definition}}
\newcommand{\dfnc}[2]{\begin{definition}[colbacktitle=red!75!black]{#1}{}#2\end{definition}}
\newcommand{\qs}[2]{\begin{question}{#1}{}#2\end{question}}
\newcommand{\pf}[2]{\begin{myproof}[#1]#2\end{myproof}}
\newcommand{\nt}[1]{\begin{note}#1\end{note}}

\newcommand*\circled[1]{\tikz[baseline=(char.base)]{
		\node[shape=circle,draw,inner sep=1pt] (char) {#1};}}
\newcommand\getcurrentref[1]{%
	\ifnumequal{\value{#1}}{0}
	{??}
	{\the\value{#1}}%
}
\newcommand{\getCurrentSectionNumber}{\getcurrentref{section}}
\newenvironment{myproof}[1][\proofname]{%
	\proof[\bfseries #1: ]%
}{\endproof}

\newcommand{\mclm}[2]{\begin{myclaim}[#1]#2\end{myclaim}}
\newenvironment{myclaim}[1][\claimname]{\proof[\bfseries #1: ]}{}

\newcounter{mylabelcounter}

\makeatletter
\newcommand{\setword}[2]{%
	\phantomsection
	#1\def\@currentlabel{\unexpanded{#1}}\label{#2}%
}
\makeatother




\tikzset{
	symbol/.style={
			draw=none,
			every to/.append style={
					edge node={node [sloped, allow upside down, auto=false]{$#1$}}}
		}
}


% deliminators
% \DeclarePairedDelimiter{\abs}{\lvert}{\rvert}
% \DeclarePairedDelimiter{\norm}{\lVert}{\rVert}

\DeclarePairedDelimiter{\ceil}{\lceil}{\rceil}
\DeclarePairedDelimiter{\floor}{\lfloor}{\rfloor}
\DeclarePairedDelimiter{\round}{\lfloor}{\rceil}

\newsavebox\diffdbox
\newcommand{\slantedromand}{{\mathpalette\makesl{d}}}
\newcommand{\makesl}[2]{%
	\begingroup
	\sbox{\diffdbox}{$\mathsurround=0pt#1\mathrm{#2}$}%
	\pdfsave
	\pdfsetmatrix{1 0 0.2 1}%
	\rlap{\usebox{\diffdbox}}%
	\pdfrestore
	\hskip\wd\diffdbox
	\endgroup
}
% \newcommand{\dd}[1][]{\ensuremath{\mathop{}\!\ifstrempty{#1}{%
% 			\slantedromand\@ifnextchar^{\hspace{0.2ex}}{\hspace{0.1ex}}}%
% 		{\slantedromand\hspace{0.2ex}^{#1}}}}
\ProvideDocumentCommand\dv{o m g}{%
	\ensuremath{%
		\IfValueTF{#3}{%
			\IfNoValueTF{#1}{%
				\frac{\dd #2}{\dd #3}%
			}{%
				\frac{\dd^{#1} #2}{\dd #3^{#1}}%
			}%
		}{%
			\IfNoValueTF{#1}{%
				\frac{\dd}{\dd #2}%
			}{%
				\frac{\dd^{#1}}{\dd #2^{#1}}%
			}%
		}%
	}%
}
\DeclareDocumentCommand\pdv{ s o m g g d() }
{ % Partial derivative
	% s: star for \flatfrac flat derivative
	% o: optional n for nth derivative
	% m: mandatory (x in df/dx)
	% g: optional (f in df/dx)
	% g: optional (y in d^2f/dxdy)
	% d: long-form d/dx(...)
	\IfBooleanTF{#1}
	{\let\fractype\flatfrac}
	{\let\fractype\frac}
	\IfNoValueTF{#4}
	{
		\IfNoValueTF{#6}
		{\fractype{\partial \IfNoValueTF{#2}{}{^{#2}}}{\partial #3\IfNoValueTF{#2}{}{^{#2}}}}
		{\fractype{\partial \IfNoValueTF{#2}{}{^{#2}}}{\partial #3\IfNoValueTF{#2}{}{^{#2}}} \argopen(#6\argclose)}
	}
	{
		\IfNoValueTF{#5}
		{\fractype{\partial \IfNoValueTF{#2}{}{^{#2}} #3}{\partial #4\IfNoValueTF{#2}{}{^{#2}}}}
		{\fractype{\partial^2 #3}{\partial #4 \partial #5}}
	}
}
% \providecommand*{\pdv}[3][]{\frac{\partial^{#1}#2}{\partial#3^{#1}}}
%  - others
\DeclareMathOperator{\Lap}{\mathcal{L}}
\DeclareMathOperator{\Var}{Var} % varience
\DeclareMathOperator{\Cov}{Cov} % covarience
\DeclareMathOperator{\E}{E} % expected

% Since the amsthm package isn't loaded

% I prefer the slanted \leq
\let\oldleq\leq % save them in case they're every wanted
\let\oldgeq\geq
\renewcommand{\leq}{\leqslant}
\renewcommand{\geq}{\geqslant}

% % redefine matrix env to allow for alignment, use r as default
% \renewcommand*\env@matrix[1][r]{\hskip -\arraycolsep
%     \let\@ifnextchar\new@ifnextchar
%     \array{*\c@MaxMatrixCols #1}}


%\usepackage{framed}
%\usepackage{titletoc}
%\usepackage{etoolbox}
%\usepackage{lmodern}


%\patchcmd{\tableofcontents}{\contentsname}{\sffamily\contentsname}{}{}

%\renewenvironment{leftbar}
%{\def\FrameCommand{\hspace{6em}%
%		{\color{myyellow}\vrule width 2pt depth 6pt}\hspace{1em}}%
%	\MakeFramed{\parshape 1 0cm \dimexpr\textwidth-6em\relax\FrameRestore}\vskip2pt%
%}
%{\endMakeFramed}

%\titlecontents{chapter}
%[0em]{\vspace*{2\baselineskip}}
%{\parbox{4.5em}{%
%		\hfill\Huge\sffamily\bfseries\color{myred}\thecontentspage}%
%	\vspace*{-2.3\baselineskip}\leftbar\textsc{\small\chaptername~\thecontentslabel}\\\sffamily}
%{}{\endleftbar}
%\titlecontents{section}
%[8.4em]
%{\sffamily\contentslabel{3em}}{}{}
%{\hspace{0.5em}\nobreak\itshape\color{myred}\contentspage}
%\titlecontents{subsection}
%[8.4em]
%{\sffamily\contentslabel{3em}}{}{}  
%{\hspace{0.5em}\nobreak\itshape\color{myred}\contentspage}



%%%%%%%%%%%%%%%%%%%%%%%%%%%%%%%%%%%%%%%%%%%
% TABLE OF CONTENTS
%%%%%%%%%%%%%%%%%%%%%%%%%%%%%%%%%%%%%%%%%%%

\usepackage{tikz}
\definecolor{doc}{RGB}{0,60,110}
\usepackage{titletoc}
\contentsmargin{0cm}
\titlecontents{chapter}[3.7pc]
{\addvspace{30pt}%
	\begin{tikzpicture}[remember picture, overlay]%
		\draw[fill=doc!60,draw=doc!60] (-7,-.1) rectangle (-0.9,.5);%
		\pgftext[left,x=-3.5cm,y=0.2cm]{\color{white}\Large\sc\bfseries Chapter\ \thecontentslabel};%
	\end{tikzpicture}\color{doc!60}\large\sc\bfseries}%
{}
{}
{\;\titlerule\;\large\sc\bfseries Page \thecontentspage
	\begin{tikzpicture}[remember picture, overlay]
		\draw[fill=doc!60,draw=doc!60] (2pt,0) rectangle (4,0.1pt);
	\end{tikzpicture}}%
\titlecontents{section}[3.7pc]
{\addvspace{2pt}}
{\contentslabel[\thecontentslabel]{2pc}}
{}
{\hfill\small \thecontentspage}
[]
\titlecontents*{subsection}[3.7pc]
{\addvspace{-1pt}\small}
{}
{}
{\ --- \small\thecontentspage}
[ \textbullet\ ][]

\makeatletter
\renewcommand{\tableofcontents}{%
	\chapter*{%
	  \vspace*{-20\p@}%
	  \begin{tikzpicture}[remember picture, overlay]%
		  \pgftext[right,x=15cm,y=0.2cm]{\color{doc!60}\Huge\sc\bfseries \contentsname};%
		  \draw[fill=doc!60,draw=doc!60] (13,-.75) rectangle (20,1);%
		  \clip (13,-.75) rectangle (20,1);
		  \pgftext[right,x=15cm,y=0.2cm]{\color{white}\Huge\sc\bfseries \contentsname};%
	  \end{tikzpicture}}%
	\@starttoc{toc}}
\makeatother

%From M275 "Topology" at SJSU
\newcommand{\id}{\mathrm{id}}
\newcommand{\taking}[1]{\xrightarrow{#1}}
\newcommand{\inv}{^{-1}}

%From M170 "Introduction to Graph Theory" at SJSU
\DeclareMathOperator{\diam}{diam}
\DeclareMathOperator{\ord}{ord}
\newcommand{\defeq}{\overset{\mathrm{def}}{=}}

%From the USAMO .tex files
\newcommand{\ts}{\textsuperscript}
\newcommand{\dg}{^\circ}
\newcommand{\ii}{\item}

% % From Math 55 and Math 145 at Harvard
% \newenvironment{subproof}[1][Proof]{%
% \begin{proof}[#1] \renewcommand{\qedsymbol}{$\blacksquare$}}%
% {\end{proof}}

\newcommand{\liff}{\leftrightarrow}
\newcommand{\lthen}{\rightarrow}
\newcommand{\opname}{\operatorname}
\newcommand{\surjto}{\twoheadrightarrow}
\newcommand{\injto}{\hookrightarrow}
\newcommand{\On}{\mathrm{On}} % ordinals
% \newcommand{\EE}{\mathbb{E}} % Expectance
\DeclareMathOperator{\img}{im} % Image
\DeclareMathOperator{\Img}{Im} % Image
\DeclareMathOperator{\coker}{coker} % Cokernel
\DeclareMathOperator{\Coker}{Coker} % Cokernel
\DeclareMathOperator{\Ker}{Ker} % Kernel
\DeclareMathOperator{\rank}{rank}
\DeclareMathOperator{\Spec}{Spec} % spectrum
\DeclareMathOperator{\Tr}{Tr} % trace
\DeclareMathOperator{\pr}{pr} % projection
\DeclareMathOperator{\ext}{ext} % extension
\DeclareMathOperator{\pred}{pred} % predecessor
\DeclareMathOperator{\dom}{dom} % domain
\DeclareMathOperator{\ran}{ran} % range
\DeclareMathOperator{\Hom}{Hom} % homomorphism
\DeclareMathOperator{\Mor}{Mor} % morphisms
\DeclareMathOperator{\End}{End} % endomorphism
% \DeclareMathOperator{\Pr}{Pr} % probability
% \DeclareMathOperator{\Var}{Var} % variance

\newcommand{\eps}{\epsilon}
\newcommand{\veps}{\varepsilon}
\newcommand{\ol}{\overline}
\newcommand{\ul}{\underline}
\newcommand{\wt}{\widetilde}
\newcommand{\wh}{\widehat}
\newcommand{\vocab}[1]{\textbf{\color{blue} #1}}
\providecommand{\half}{\frac{1}{2}}
\newcommand{\dang}{\measuredangle} %% Directed angle
\newcommand{\ray}[1]{\overrightarrow{#1}}
\newcommand{\seg}[1]{\overline{#1}}
\newcommand{\arc}[1]{\wideparen{#1}}
\DeclareMathOperator{\cis}{cis}
\DeclareMathOperator*{\lcm}{lcm}
\DeclareMathOperator*{\argmin}{arg min}
\DeclareMathOperator*{\argmax}{arg max}
\newcommand{\cycsum}{\sum_{\mathrm{cyc}}}
\newcommand{\symsum}{\sum_{\mathrm{sym}}}
\newcommand{\cycprod}{\prod_{\mathrm{cyc}}}
\newcommand{\symprod}{\prod_{\mathrm{sym}}}
\newcommand{\Qed}{\begin{flushright}\qed\end{flushright}}
\newcommand{\parinn}{\setlength{\parindent}{1cm}}
\newcommand{\parinf}{\setlength{\parindent}{0cm}}
% \newcommand{\norm}{\|\cdot\|}
\newcommand{\inorm}{\norm_{\infty}}
\newcommand{\opensets}{\{V_{\alpha}\}_{\alpha\in I}}
\newcommand{\oset}{V_{\alpha}}
\newcommand{\opset}[1]{V_{\alpha_{#1}}}
\newcommand{\lub}{\text{lub}}
\newcommand{\del}[2]{\frac{\partial #1}{\partial #2}}
\newcommand{\Del}[3]{\frac{\partial^{#1} #2}{\partial^{#1} #3}}
\newcommand{\deld}[2]{\dfrac{\partial #1}{\partial #2}}
\newcommand{\Deld}[3]{\dfrac{\partial^{#1} #2}{\partial^{#1} #3}}
\newcommand{\lm}{\lambda}
\newcommand{\uin}{\mathbin{\rotatebox[origin=c]{90}{$\in$}}}
\newcommand{\usubset}{\mathbin{\rotatebox[origin=c]{90}{$\subset$}}}
\newcommand{\lt}{\left}
\newcommand{\rt}{\right}
\newcommand{\bs}[1]{\boldsymbol{#1}}
\newcommand{\exs}{\exists}
\newcommand{\st}{\strut}
\newcommand{\dps}[1]{\displaystyle{#1}}
\newcommand{\va}[1]{\vec{\bm{\mathrm{#1}}}}
\WithSuffix\newcommand\va*[1]{\vec{\bm{#1}}}
\newcommand{\vb}[1]{\bm{\mathrm{#1}}}
\WithSuffix\newcommand\vb*[1]{\bm{#1}}
\newcommand{\vu}[1]{\hat{\bm{\mathrm{#1}}}}
\WithSuffix\newcommand\vu*[1]{\hat{\bm{#1}}}
\renewcommand{\dd}[1]{\mathrm{d}#1}
\renewcommand{\Re}{\mathrm{Re}}
\renewcommand{\Im}{\mathrm{Im}}
\DeclareMathOperator{\tr}{tr}
\newcommand{\csin}[1]{\mintinline{csharp}|#1|}
\renewcommand{\Pr}[1]{\mathrm{Pr}\lt( #1 \rt)}
\renewcommand{\Var}[1]{\mathrm{Var}\lt( #1 \rt)}
\newcommand{\cov}[1]{\mathrm{cov}\lt( #1 \rt)}

\newcommand{\sol}{\setlength{\parindent}{0cm}\textbf{\textit{Solution:}}\setlength{\parindent}{1cm} }
\newcommand{\solve}[1]{\setlength{\parindent}{0cm}\textbf{\textit{Solution: }}\setlength{\parindent}{1cm}#1 \Qed}

% Things Lie
\newcommand{\kb}{\mathfrak b}
\newcommand{\kg}{\mathfrak g}
\newcommand{\kh}{\mathfrak h}
\newcommand{\kn}{\mathfrak n}
\newcommand{\ku}{\mathfrak u}
\newcommand{\kz}{\mathfrak z}
\DeclareMathOperator{\Ext}{Ext} % Ext functor
\DeclareMathOperator{\Tor}{Tor} % Tor functor
\newcommand{\gl}{\opname{\mathfrak{gl}}} % frak gl group
\renewcommand{\sl}{\opname{\mathfrak{sl}}} % frak sl group chktex 6

% More script letters etc.
\newcommand{\SA}{\mathcal A}
\newcommand{\SB}{\mathcal B}
\newcommand{\SC}{\mathcal C}
\newcommand{\SF}{\mathcal F}
\newcommand{\SG}{\mathcal G}
\newcommand{\SH}{\mathcal H}
\newcommand{\OO}{\mathcal O}

\newcommand{\SCA}{\mathscr A}
\newcommand{\SCB}{\mathscr B}
\newcommand{\SCC}{\mathscr C}
\newcommand{\SCD}{\mathscr D}
\newcommand{\SCE}{\mathscr E}
\newcommand{\SCF}{\mathscr F}
\newcommand{\SCG}{\mathscr G}
\newcommand{\SCH}{\mathscr H}

% Mathfrak primes
\newcommand{\km}{\mathfrak m}
\newcommand{\kp}{\mathfrak p}
\newcommand{\kq}{\mathfrak q}

% number sets
\newcommand{\RR}[1][]{\ensuremath{\ifstrempty{#1}{\mathbb{R}}{\mathbb{R}^{#1}}}}
\newcommand{\NN}[1][]{\ensuremath{\ifstrempty{#1}{\mathbb{N}}{\mathbb{N}^{#1}}}}
\newcommand{\ZZ}[1][]{\ensuremath{\ifstrempty{#1}{\mathbb{Z}}{\mathbb{Z}^{#1}}}}
\newcommand{\QQ}[1][]{\ensuremath{\ifstrempty{#1}{\mathbb{Q}}{\mathbb{Q}^{#1}}}}
\newcommand{\CC}[1][]{\ensuremath{\ifstrempty{#1}{\mathbb{C}}{\mathbb{C}^{#1}}}}
\newcommand{\PP}[1][]{\ensuremath{\ifstrempty{#1}{\mathbb{P}}{\mathbb{P}^{#1}}}}
\newcommand{\HH}[1][]{\ensuremath{\ifstrempty{#1}{\mathbb{H}}{\mathbb{H}^{#1}}}}
\newcommand{\FF}[1][]{\ensuremath{\ifstrempty{#1}{\mathbb{F}}{\mathbb{F}^{#1}}}}
% expected value
\newcommand{\EE}{\ensuremath{\mathbb{E}}}
\newcommand{\charin}{\text{ char }}
\DeclareMathOperator{\sign}{sign}
\DeclareMathOperator{\Aut}{Aut}
\DeclareMathOperator{\Inn}{Inn}
\DeclareMathOperator{\Syl}{Syl}
\DeclareMathOperator{\Gal}{Gal}
\DeclareMathOperator{\GL}{GL} % General linear group
\DeclareMathOperator{\SL}{SL} % Special linear group

%---------------------------------------
% BlackBoard Math Fonts :-
%---------------------------------------

%Captital Letters
\newcommand{\bbA}{\mathbb{A}}	\newcommand{\bbB}{\mathbb{B}}
\newcommand{\bbC}{\mathbb{C}}	\newcommand{\bbD}{\mathbb{D}}
\newcommand{\bbE}{\mathbb{E}}	\newcommand{\bbF}{\mathbb{F}}
\newcommand{\bbG}{\mathbb{G}}	\newcommand{\bbH}{\mathbb{H}}
\newcommand{\bbI}{\mathbb{I}}	\newcommand{\bbJ}{\mathbb{J}}
\newcommand{\bbK}{\mathbb{K}}	\newcommand{\bbL}{\mathbb{L}}
\newcommand{\bbM}{\mathbb{M}}	\newcommand{\bbN}{\mathbb{N}}
\newcommand{\bbO}{\mathbb{O}}	\newcommand{\bbP}{\mathbb{P}}
\newcommand{\bbQ}{\mathbb{Q}}	\newcommand{\bbR}{\mathbb{R}}
\newcommand{\bbS}{\mathbb{S}}	\newcommand{\bbT}{\mathbb{T}}
\newcommand{\bbU}{\mathbb{U}}	\newcommand{\bbV}{\mathbb{V}}
\newcommand{\bbW}{\mathbb{W}}	\newcommand{\bbX}{\mathbb{X}}
\newcommand{\bbY}{\mathbb{Y}}	\newcommand{\bbZ}{\mathbb{Z}}

%---------------------------------------
% MathCal Fonts :-
%---------------------------------------

%Captital Letters
\newcommand{\mcA}{\mathcal{A}}	\newcommand{\mcB}{\mathcal{B}}
\newcommand{\mcC}{\mathcal{C}}	\newcommand{\mcD}{\mathcal{D}}
\newcommand{\mcE}{\mathcal{E}}	\newcommand{\mcF}{\mathcal{F}}
\newcommand{\mcG}{\mathcal{G}}	\newcommand{\mcH}{\mathcal{H}}
\newcommand{\mcI}{\mathcal{I}}	\newcommand{\mcJ}{\mathcal{J}}
\newcommand{\mcK}{\mathcal{K}}	\newcommand{\mcL}{\mathcal{L}}
\newcommand{\mcM}{\mathcal{M}}	\newcommand{\mcN}{\mathcal{N}}
\newcommand{\mcO}{\mathcal{O}}	\newcommand{\mcP}{\mathcal{P}}
\newcommand{\mcQ}{\mathcal{Q}}	\newcommand{\mcR}{\mathcal{R}}
\newcommand{\mcS}{\mathcal{S}}	\newcommand{\mcT}{\mathcal{T}}
\newcommand{\mcU}{\mathcal{U}}	\newcommand{\mcV}{\mathcal{V}}
\newcommand{\mcW}{\mathcal{W}}	\newcommand{\mcX}{\mathcal{X}}
\newcommand{\mcY}{\mathcal{Y}}	\newcommand{\mcZ}{\mathcal{Z}}


%---------------------------------------
% Bold Math Fonts :-
%---------------------------------------

%Captital Letters
\newcommand{\bmA}{\boldsymbol{A}}	\newcommand{\bmB}{\boldsymbol{B}}
\newcommand{\bmC}{\boldsymbol{C}}	\newcommand{\bmD}{\boldsymbol{D}}
\newcommand{\bmE}{\boldsymbol{E}}	\newcommand{\bmF}{\boldsymbol{F}}
\newcommand{\bmG}{\boldsymbol{G}}	\newcommand{\bmH}{\boldsymbol{H}}
\newcommand{\bmI}{\boldsymbol{I}}	\newcommand{\bmJ}{\boldsymbol{J}}
\newcommand{\bmK}{\boldsymbol{K}}	\newcommand{\bmL}{\boldsymbol{L}}
\newcommand{\bmM}{\boldsymbol{M}}	\newcommand{\bmN}{\boldsymbol{N}}
\newcommand{\bmO}{\boldsymbol{O}}	\newcommand{\bmP}{\boldsymbol{P}}
\newcommand{\bmQ}{\boldsymbol{Q}}	\newcommand{\bmR}{\boldsymbol{R}}
\newcommand{\bmS}{\boldsymbol{S}}	\newcommand{\bmT}{\boldsymbol{T}}
\newcommand{\bmU}{\boldsymbol{U}}	\newcommand{\bmV}{\boldsymbol{V}}
\newcommand{\bmW}{\boldsymbol{W}}	\newcommand{\bmX}{\boldsymbol{X}}
\newcommand{\bmY}{\boldsymbol{Y}}	\newcommand{\bmZ}{\boldsymbol{Z}}
%Small Letters
\newcommand{\bma}{\boldsymbol{a}}	\newcommand{\bmb}{\boldsymbol{b}}
\newcommand{\bmc}{\boldsymbol{c}}	\newcommand{\bmd}{\boldsymbol{d}}
\newcommand{\bme}{\boldsymbol{e}}	\newcommand{\bmf}{\boldsymbol{f}}
\newcommand{\bmg}{\boldsymbol{g}}	\newcommand{\bmh}{\boldsymbol{h}}
\newcommand{\bmi}{\boldsymbol{i}}	\newcommand{\bmj}{\boldsymbol{j}}
\newcommand{\bmk}{\boldsymbol{k}}	\newcommand{\bml}{\boldsymbol{l}}
\newcommand{\bmm}{\boldsymbol{m}}	\newcommand{\bmn}{\boldsymbol{n}}
\newcommand{\bmo}{\boldsymbol{o}}	\newcommand{\bmp}{\boldsymbol{p}}
\newcommand{\bmq}{\boldsymbol{q}}	\newcommand{\bmr}{\boldsymbol{r}}
\newcommand{\bms}{\boldsymbol{s}}	\newcommand{\bmt}{\boldsymbol{t}}
\newcommand{\bmu}{\boldsymbol{u}}	\newcommand{\bmv}{\boldsymbol{v}}
\newcommand{\bmw}{\boldsymbol{w}}	\newcommand{\bmx}{\boldsymbol{x}}
\newcommand{\bmy}{\boldsymbol{y}}	\newcommand{\bmz}{\boldsymbol{z}}

%---------------------------------------
% Scr Math Fonts :-
%---------------------------------------

\newcommand{\sA}{{\mathscr{A}}}   \newcommand{\sB}{{\mathscr{B}}}
\newcommand{\sC}{{\mathscr{C}}}   \newcommand{\sD}{{\mathscr{D}}}
\newcommand{\sE}{{\mathscr{E}}}   \newcommand{\sF}{{\mathscr{F}}}
\newcommand{\sG}{{\mathscr{G}}}   \newcommand{\sH}{{\mathscr{H}}}
\newcommand{\sI}{{\mathscr{I}}}   \newcommand{\sJ}{{\mathscr{J}}}
\newcommand{\sK}{{\mathscr{K}}}   \newcommand{\sL}{{\mathscr{L}}}
\newcommand{\sM}{{\mathscr{M}}}   \newcommand{\sN}{{\mathscr{N}}}
\newcommand{\sO}{{\mathscr{O}}}   \newcommand{\sP}{{\mathscr{P}}}
\newcommand{\sQ}{{\mathscr{Q}}}   \newcommand{\sR}{{\mathscr{R}}}
\newcommand{\sS}{{\mathscr{S}}}   \newcommand{\sT}{{\mathscr{T}}}
\newcommand{\sU}{{\mathscr{U}}}   \newcommand{\sV}{{\mathscr{V}}}
\newcommand{\sW}{{\mathscr{W}}}   \newcommand{\sX}{{\mathscr{X}}}
\newcommand{\sY}{{\mathscr{Y}}}   \newcommand{\sZ}{{\mathscr{Z}}}


%---------------------------------------
% Math Fraktur Font
%---------------------------------------

%Captital Letters
\newcommand{\mfA}{\mathfrak{A}}	\newcommand{\mfB}{\mathfrak{B}}
\newcommand{\mfC}{\mathfrak{C}}	\newcommand{\mfD}{\mathfrak{D}}
\newcommand{\mfE}{\mathfrak{E}}	\newcommand{\mfF}{\mathfrak{F}}
\newcommand{\mfG}{\mathfrak{G}}	\newcommand{\mfH}{\mathfrak{H}}
\newcommand{\mfI}{\mathfrak{I}}	\newcommand{\mfJ}{\mathfrak{J}}
\newcommand{\mfK}{\mathfrak{K}}	\newcommand{\mfL}{\mathfrak{L}}
\newcommand{\mfM}{\mathfrak{M}}	\newcommand{\mfN}{\mathfrak{N}}
\newcommand{\mfO}{\mathfrak{O}}	\newcommand{\mfP}{\mathfrak{P}}
\newcommand{\mfQ}{\mathfrak{Q}}	\newcommand{\mfR}{\mathfrak{R}}
\newcommand{\mfS}{\mathfrak{S}}	\newcommand{\mfT}{\mathfrak{T}}
\newcommand{\mfU}{\mathfrak{U}}	\newcommand{\mfV}{\mathfrak{V}}
\newcommand{\mfW}{\mathfrak{W}}	\newcommand{\mfX}{\mathfrak{X}}
\newcommand{\mfY}{\mathfrak{Y}}	\newcommand{\mfZ}{\mathfrak{Z}}
%Small Letters
\newcommand{\mfa}{\mathfrak{a}}	\newcommand{\mfb}{\mathfrak{b}}
\newcommand{\mfc}{\mathfrak{c}}	\newcommand{\mfd}{\mathfrak{d}}
\newcommand{\mfe}{\mathfrak{e}}	\newcommand{\mff}{\mathfrak{f}}
\newcommand{\mfg}{\mathfrak{g}}	\newcommand{\mfh}{\mathfrak{h}}
\newcommand{\mfi}{\mathfrak{i}}	\newcommand{\mfj}{\mathfrak{j}}
\newcommand{\mfk}{\mathfrak{k}}	\newcommand{\mfl}{\mathfrak{l}}
\newcommand{\mfm}{\mathfrak{m}}	\newcommand{\mfn}{\mathfrak{n}}
\newcommand{\mfo}{\mathfrak{o}}	\newcommand{\mfp}{\mathfrak{p}}
\newcommand{\mfq}{\mathfrak{q}}	\newcommand{\mfr}{\mathfrak{r}}
\newcommand{\mfs}{\mathfrak{s}}	\newcommand{\mft}{\mathfrak{t}}
\newcommand{\mfu}{\mathfrak{u}}	\newcommand{\mfv}{\mathfrak{v}}
\newcommand{\mfw}{\mathfrak{w}}	\newcommand{\mfx}{\mathfrak{x}}
\newcommand{\mfy}{\mathfrak{y}}	\newcommand{\mfz}{\mathfrak{z}}


\title{\Huge{Numerical Analysis}\\Semester 4}
\author{}
\date{}

\begin{document}

\maketitle
\newpage% or \cleardoublepage
% \pdfbookmark[<level>]{<title>}{<dest>}
\pdfbookmark[section]{\contentsname}{toc}
\tableofcontents
\pagebreak

\chapter{Interpolation}

\section{Linear Interpolation}

\begin{minipage}{0.4\linewidth}
	\begin{figure}[H]
		\centering
		\begin{tikzpicture}
			\begin{axis}[
					legend pos=outer north east,
					axis lines = box,
					xlabel = $x$,
					ylabel = $f(x)$,
					variable = t,
					trig format plots = rad,
					axis x line=bottom,
					axis y line=left,
					xmin=0,xmax=5.5,
					ymin=0,ymax=3.9,
				]

				\addplot[
					color=blue,
				]
				coordinates {
						(0,0)(1,2)(2,3)(3,1)(4,3)(5,3.5)
					};
				\addplot[
					color=red,
					only marks,
					mark=*,
				]
				coordinates {
						(0,0)(1,2)(2,3)(3,1)(4,3)(5,3.5)
					};

			\end{axis}
		\end{tikzpicture}
	\end{figure}
\end{minipage}
\hfill
\begin{minipage}{0.4\linewidth}
	\begin{tabular}{l|l}
		$x$ & $f(x)$ \\
		\hline
		0   & 0      \\
		1   & 2      \\
		2   & 3      \\
		3   & 1      \\
		4   & 3      \\
		5   & 3.5    \\
	\end{tabular}
\end{minipage}

Linear interpolation is just drawing lines between the data points.
\dfn{Linear Interpolation(lerp) equation}{
	The equation of the lines between data points is

	\[
		y = \frac{y_{i+1} - y_i}{x_{i+1} - x_i}(x-x_i) + y_i
		.\]
}
\thm{Error due to linear interpolation}{
	Let $f$ be a continuous and differentiable on $[a,b]$. We define the error $z(x)$ to be
	\[
		|z(x)| \leq \frac{(b-a)^2 }{8 }\sup_{a\leq x \leq b}\lt|f''(x)\rt|
		.\]
}

\section{Polynomial Interpolation}

\subsection{Lagrange Polynomials}

Really nice video \href{https://www.youtube.com/watch?v=bzp_q7NDdd4}{here} explaining Lagrange polynomials.


\begin{figure}[H]
	\centering
	\begin{tikzpicture}
		\begin{axis}[
				legend pos=outer north east,
				axis lines = box,
				xlabel = $x$,
				ylabel = $f(x)$,
				axis x line=bottom,
				axis y line=left,
				xmin=0,xmax=5.5,
			]

			\addplot[
				only marks,
				color=red,
				mark=*,
			]
			coordinates {
					(0,0)(1,2)(2,3)(3,1)(4,3)(5,3.5)
				};

			\addplot [
				domain=0:5,
				samples=100,
				color=blue,
			]
			{-0.17916*x^5+(13/6)*x^4+(-425/48)*x^3+(163/12)*x^2+(-283/60)*x};
			% \addlegendentry{$-\frac{43}{240}x^5 + \frac{13}{6}x^4 - \frac{425}{48}x^3 + \frac{163 }{12}x^2 - \frac{283}{60}x$}
		\end{axis}
	\end{tikzpicture}
\end{figure}


\thm{Lagrange polynomial equation}{
	Consider a set of $n$ points $(x_1,y_1),(x_2,y_2),\dots,(x_n,y_n)$. The Lagrange polynomial for this set of data is
	\[
		L(x) = \sum_{k=0}^{n} y_k \ell_k(x)
		.\]

	where
	\[
		\ell_k(x) = \prod_{\substack{i=1\\i\neq k}}^{n}\frac{x-x_i }{x_k-x_i}
		.\]
}

\subsubsection{Case of equidistant points}

If the set of $x_i$ are equidistant from each other with a distance of $h=x_{i+1}-x_i$, then we can represent any point as $x_k = x_0 + kh$ where $k\in\mathbb{N}$ and any number $x = x_0 + sh$ where $s\in\mathbb{R}$. We can rewrite the formula as
\[
	Q(s) = \sum_{k=0}^{n} \ell_k(s)f(x_k)
	.\]
where
\[
	\ell_k(s) = \prod_{\substack{j=0\\j\neq k}}^n\frac{s-j}{k-j}
	.\]

by substitution
\[
	s = \frac{x-x_0}{h}
	.\]

\subsubsection{Existence}

\begin{myproof}
	$P(x)$ belongs to the vectorial space of polynomial of degree of, at most, $n$. Now, we must fins a basis for this vectorial space. Find the polynomial $\ell_k$ of degree $\leq n$ such that
	\[
		\ell_k(x_i) = \delta_{ki} = \begin{cases}
			1 & \text{if }i=k     \\
			0 & \text{if }i\neq k
		\end{cases}
		.\]
	Then, $\ell_k(x) = \lambda(x-x_0)\dots(x-x_{k-1})(x-x_{k+1})\dots(x-x_n)$
	where
	\[
		\lambda = \frac{1}{(x_k-x_0)\dots(x_k-x_{k-1})(x_k-x_{k+1})\dots(x_k-x_n)}
		.\]
	The $(n+1)$ polynomials $\ell_k(x)$ for a system of generators in the vectorial space of polynomials of degree at most $n$.

	\[
		\lambda_0\ell_0(x) + \lambda_1\ell_1(x) + \dots + \lambda_k\ell_k(x) + \dots + \lambda_n\ell_n(x) = 0
		.\]

	for $x=x_k$
	\begin{align*}
		\lambda_0\ell_0(x_k) + \lambda_1\ell_1(x_k) + \dots + \lambda_k\ell_k(x_k) + \dots + \lambda_n\ell_n(x_k) & = 0 \\
		0+0+\dots+\lambda_k1+\dots+0                                                                              & = 0 \\
		\lambda_k                                                                                                 & = 0
		.\end{align*}
	$\therefore$ the set of $\ell_k$ for a basis in the vector space $\Rightarrow$ there has to exist a polynomial passing through the given set of points.

\end{myproof}


\subsubsection{Uniqueness}

\begin{myproof}
	Let $P$ and $Q$ be 2 Lagrange polynomials of degrees $\leq n\Big/P(x_i)=Q(x_i)=f(x_i)\quad\forall i = 0,1,\dots,n$.\\
	Let
	\[
		\begin{rcases}
			R=P-Q \text{ of degree }\leq n \\
			R=0 \; (n+1) \text{ times}
		\end{rcases}R\equiv0\Rightarrow P=Q\;\forall x
		.\]
\end{myproof}

\subsection{Newton Polynomial}

\dfn{Newton Polynomial equation}{
	Consider a set of $n$ points $(x_1,y_1),(x_2,y_2),\dots,(x_n,y_n)$. The Newton polynomial for this set of data is
	\[
		p_n(x) = \underbrace{a_0}_{A_0} + \underbrace{a_1(x-x_0)}_{A_1} + \underbrace{a_2(x-x_0)(x-x_1)}_{A_2} + \dots + \underbrace{a_n \prod_{i=0}^{n-1}(x-x_i)}_{A_n}
		.\]
	where
	\[
		a_i = f[x_0,x_1,\dots,x_i]
		.\]
	Here $f[\dots]$ is the divided difference of the inputted data.
}

\dfn{Backwards formula}{
\[
	P_n(x) = f_n + A_1 + A_2 + \dots+ A_n
	.\]

where
\[
	A_i = f[x_n,x_{n-1},\dots,x_{n-i}]\prod_{j=n-i+1}^{n}(x-x_j)
	.\]
}

The divided difference has 2 formulas, the recurrence formula
\[
	f[x_0,x_1,\dots,x_{n+1}] = \frac{f[x_1,x_2,\dots,x_{n+1}]-f[x_0,x_1,\dots,x_{n}]}{x_{n+1}-x_0}
	.\]

and a general formula
\[
	f[x_0,x_1,\dots,x_n] = \sum_{i=1}^{n}\frac{y_i }{\prod_{\substack{k=0\\k\neq i}}^{n} (x_i - x_k)}
	.\]
Now forget you ever saw those cause there is an easier method to finding the divided difference.

\subsubsection{Divided Difference Table}
\[
	\begin{array}{cccccc}
		x_0 & y_0                                                                                              \\
		    &     & \frac{y_1-y_0}{x_1-x_0}=f[x_0,x_1]                                                         \\
		x_1 & y_1 &                                    & \frac{f[x_1,x_2]-f[x_0,x_1]}{x_2-x_0}                 \\
		    &     & \frac{y_2-y_1}{x_2-x_1}=f[x_1,x_2] &                                       & \dots         \\
		x_2 & y_2 &                                    & \frac{f[x_2,x_3]-f[x_1,x_2]}{x_3-x_1} &       & \dots \\
		    &     & \frac{y_3-y_2}{x_3-x_2}=f[x_2,x_3] &                                       & \dots         \\
		x_3 & y_3 &                                    & \frac{f[x_3,x_4]-f[x_2,x_3]}{x_4-x_2}                 \\
		    &     & \frac{y_4-y_3}{x_4-x_3}=f[x_3,x_4]                                                         \\
		x_4 & y_4
	\end{array}
\]
After we have constructed the table we can find the divided difference we want by looking at the top diagonal

\[
	\begin{array}{cccccc}
		x_0 & y_0                                                                             \\
		    &     & f[x_0,x_1]                                                                \\
		x_1 & y_1 &            & f[x_0,x_1,x_2]                                               \\
		    &     & \dots      &                & f[x_0,x_1,x_2,x_3]                          \\
		x_2 & y_2 &            & \dots          &                    & f[x_0,x_1,x_2,x_3,x_4] \\
		    &     & \dots      &                & \dots                                       \\
		x_3 & y_3 &            & \dots                                                        \\
		    &     & \dots                                                                     \\
		x_4 & y_4
	\end{array}
\]


\subsubsection{Case of equidistant points}
Bla bla bla the formula becomes

\[
	P(t) = a_0 + a_1(t-0) + a_2(t-0)(t-1) + \dots + a_n \prod_{i=0}^{n-1}(t-i)
	.\]

where in this case
\[
	a_k = \frac{\nabla^k[y](x_k)}{k!}
	.\]

and
\[
	x = x_0 +th
	.\]

where $\nabla^k[y]$ is the discrete difference.

\[
	\nabla[y](x_i) = y(x_i+h)-y(x_i)
	.\]

and the backwards formula is
\[
	P(t) = f_n + A_1 + A_2 + \dots + A_n
	.\]

where
\[
	A_i = \frac{\bar{\nabla}^if_n}{i!}\prod_{j=n-i+1}^n (t-j)
	.\]

\dfn{Discrete Difference}{
	Forward discrete difference:
	\begin{align*}
		\nabla[y](x_i)   & = y(x_i+h) - y(x_i)                    \\
		\nabla^2[y](x_i) & = \nabla[y](x_i + h) - \nabla [y](x_i) \\
		                 & = y(x_i+2h)-2y(x_i+h)+y(x_i)           \\
		\nabla^k[y](x_i) & = \nabla\lt(\nabla^{k-1}[y](x_i)\rt)
	\end{align*}


	Backwards discrete difference:
	\begin{align*}
		\bar{\nabla}[y](x_i)   & = y(x_i) - y(x_i - h)                            \\
		\bar{\nabla}^k[y](x_i) & = \bar{\nabla}\lt(\bar{\nabla}^{k-1}[y](x_i)\rt)
	\end{align*}
}
\[
	\begin{array}{cccccc}
		x_0 & y_0                                                                           \\
		    &     & \nabla[y](x_i)                                                          \\
		x_1 & y_1 &                & \nabla^2[y](x_i)                                       \\
		    &     & \dots          &                  & \nabla^3[y](x_i)                    \\
		x_2 & y_2 &                & \dots            &                  & \nabla^4[y](x_i) \\
		    &     & \dots          &                  & \dots                               \\
		x_3 & y_3 &                & \dots                                                  \\
		    &     & \dots                                                                   \\
		x_4 & y_4
	\end{array}
\]

\subsection{Error due to polynomial interpolation}

Let $f(x)$ be of class $C^{n+1} \quad \forall x \in [a,b]$ and let the polynomial $P(x)$ interpolate it. \\

The error function is bounded by
\[
	|\text{Error}| = |f(x) - P(x)| \leq \frac{\lt| \prod_{i=0}^n (x-x_i) \rt|}{(n+1)!} \sup_{x\in[a,b]} \lt|f^{(n+1)}(x)\rt|
	.\]

\subsection{Hermite Interpolation}

\dfn{Hermite interpolation formula}{
	Consider $(n+1)$ sets of point $(x_i,y_i,y'_i)$ representing $f(x)$ ($y_i = f(x_i)$ and $y_i' = f'(x_i)$), the hermite polynomial $P(x)$ interpolates $f(x)$ such that $P'(x) = f'(x)$.

	\[
		P(x) = \sum_{i=0}^n h_i(x)y_i + \sum_{i=0}^{n}k_i(x)y_i'
		.\]

	where
	\begin{align*}
		h_i(x)    & =\lt(1-2(x-x_i)\ell_i'(x_i)\rt)\ell^2_i(x) \\
		k_i(x)    & =(x-x_i)\ell^2_i(x)                        \\
		\ell_i(x) & = \prod_{\substack{j=0                     \\j\neq i}}^n \frac{x-x_j}{x_i-x_j}
	\end{align*}
}

\thm{Error due to Hermite interpolation}{
\[
	|\text{Error}| = |f(x) - P(x)| \leq \frac{\lt| \prod_{i=0}^n (x-x_i)^2 \rt|}{(2n+2)!} \sup_{x\in[a,b]} |f^{(2n+2)}(x)|
	.\]
}

\subsubsection{Existence}

\begin{myproof}

	\[
		P(x) = \sum_{i=0}^n h_i(x)y_i + \sum_{i=0}^{n}k_i(x)y_i'
		.\]

	where
	\begin{align*}
		h_i(x)    & =\lt(1-2(x-x_i)\ell_i'(x_i)\rt)\ell^2_i(x) \\
		k_i(x)    & =(x-x_i)\ell^2_i(x)                        \\
		\ell_i(x) & = \prod_{\substack{j=0                     \\j\neq i}}^n \frac{x-x_j}{x_i-x_j}
	\end{align*}

	Let $i\neq j$.
	\begin{align*}
		k_i(x_j) & = (x_j-x_i)\ell_i^2(x_j)=0 \\
		k_i(x_i) & = (x_i-x_i)\ell_i^2(x_i)=0
	\end{align*}
	and
	\begin{align*}
		h_i(x_j) & = (1-2(x_j-x_i)\ell_i'(x_i))\ell_i^2(x_j)=0 \\
		h_i(x_j) & = (1-2(x_i-x_i)\ell_i'(x_i))\ell_i^2(x_i)=1
	\end{align*}
	We conclude that $P(x_i) = f(x_i)$\\

	Now we have to prove that $P'(x_i) = f'(x_i)$
	\begin{align*}
		h_i'(x) & = -2\ell_i'(x_i)\ell_i^2(x)+2(1-2(x-x_i)\ell_i'(x_i))\ell_i(x)\ell_i'(x) \\
		k_i'(x) & = \ell^2_i(x) + 2(x-x_i)\ell_i(x)\ell_i'(x)
	\end{align*}

	\begin{align*}
		h_i'(x_j) & = -2\ell_i'(x_i)\ell_i^2(x_j)+2(1-2(x_j-x_i)\ell_i'(x_i))\ell_i(x_j)\ell_i'(x_j) = 0 \\
		h_i'(x_i) & = -2\ell_i'(x_i)\ell_i^2(x_i)+2(1-2(x_i-x_i)\ell_i'(x_i))\ell_i(x_i)\ell_i'(x_i) =0  \\
	\end{align*}

	\begin{align*}
		k_i'(x_j) & = \ell^2_i(x-j) + 2(x_j-x_i)\ell_i(x_j)\ell_i'(x_j) = 0 \\
		k_i'(x_j) & = \ell^2_i(x-i) + 2(x_i-x_i)\ell_i(x_i)\ell_i'(x_i) = 1 \\
	\end{align*}
	$\therefore P'(x_i) = f'(x_i)$
\end{myproof}

\subsubsection{Uniqueness}

\begin{myproof}
	Suppose that there exists 2 polynomials $P$ and $Q$ of degree $n\leq 2n+1$ such that $P(x_i)=Q(x_i)=f(x_i)$ and $P'(x_i)=Q'(x_i)=f'(x_i)\;\forall i = 0,1,\dots,n$.\\
	Let $R(x)=P(x)-Q(x)$.\\
	$R=0\;(n+1)$ times $\Rightarrow$ according to Rolle's theorem $\exists\,n$ points $\neq x_i\Big/R'=0$\\
	$R'=0\;n$ times as a consequence of Rolle's theorem then
	\[
		\begin{rcases}
			R'(x)=0\;(2n+1)\text{ times} \\
			R'(x)\text{ is of degree }2n
		\end{rcases}
		R'(x)=0 \;\forall x
		.\]

	\[
		R'(x)=0\Rightarrow R(x)=\text{cnst}\quad\text{and}\quad R(x_i)=P(x_i)-Q(x_i)=0 \Rightarrow \text{cnst } = 0
		.\]

	\[
		R(x) = P(x)-Q(x)=0 \;\forall x
		.\]
	$\therefore P(x) = Q(x)$
\end{myproof}


\chapter{Finding $f(x)=0$}

We will assume that every function is defined in the interval $I = [a,b]$ and that every $x_0\in I$

\section{Bisection Method}

Suppose that $f$ is a continuous and monotone function on the domain $I = [a,b]$ such that $f(a)f(b)<0\Rightarrow \exists r \in]a,b[:\;f(r)=0$.\\

At each step in the algorithm, in an iteration we let $c = (a+b)/2$, then we check the value of $f(c)$, if it is 0 then we are done.\\
However when it is not, then we define a new interval such that

\[
	I = \begin{cases}
		[a,c] & \text{if }f(c)f(a)<0 \\
		[c,b] & \text{if }f(c)f(b)<0
	\end{cases}
	.\]

We repeat this step until the length of the interval reaches a certain number (for example $|b-a|<10^{-5}$), at this point we stop and the best guess for the root would be $(a+b)/2$

\subsubsection{Error of the Bisection Method}

After $n$ iterations, the error of the approximated root would be
\[
	\text{Error} \leq \frac{|b-a|}{2^{n+1}}
	.\]

\section{Lagrange Method}

Suppose that $f$ is a continuous and monotone function on the domain $I = [a,b]$ such that $f(a)f(b)<0\Rightarrow \exists r \in]a,b[:\;f(r)=0$.\\

The starting value of $x_0$ depends on the value of $f$
\[
	x_0 = \begin{cases}
		a & \text{if }f(a)f''(a)<0 \\
		b & \text{if }f(b)f''(b)<0
	\end{cases}
	.\]

then we can find a new guess $x$ depending on the value of $x_0$
\begin{itemize}
	\ii if $x_0 = a$

	\[
		x_1 = x_0 - \frac{b-x_0}{f(b)-f(x_0)}f(x_0)
		.\]
	\ii if $x_0 = b$
	\[
		x_1 = x_0-\frac{a-x_0}{f(a)-f(x_0)}f(x_0)
		.\]
\end{itemize}

\subsubsection{Error from Lagrange Method}
For the first iteration
\[
	\text{Error} \leq \sup_{x\in[a,b]}|f''(x)|\frac{(b-a)^2}{8}
	.\]
For the second iteration
\[
	M_2 = \sup_{x\in[a,b]}|f''(x)|
	.\]
\begin{itemize}
	\ii if $x_0 = a$
	\[
		\text{Error} \leq \frac{M_2}{8}\lt|\frac{(b-x_0)^3}{f(b)-f(x_0)}\rt|
		.\]
	\ii if $x_0 = b$
	\[
		\text{Error} \leq \frac{M_2}{8}\lt|\frac{(a-x_0)^3}{f(a)-f(x_0)}\rt|
		.\]
\end{itemize}

\section{Newton Method}

Suppose that $f$ is a continuous and monotone function on the domain $I = [a,b]$ such that $f(a)f(b)<0\Rightarrow \exists r \in]a,b[:\;f(r)=0$.\\

The starting value of $x_0$ depends on the value of $f$
\[
	x_0 = \begin{cases}
		a & \text{if }f(a)f''(a)>0 \\
		b & \text{if }f(b)f''(b)>0
	\end{cases}
	.\]

Then the new guess for the root would be
\[
	x_1 = x_0 - \frac{f(x_0)}{f'(x_0)}
	.\]

\subsection{Improved Newton Method}
To improve the method we first let $\eta = b-a$, and we define condition
\[
	\frac{\eta M_2}{2|f'(x_0)|}<1
	.\]
if the condition is not satisfied we need to choose another interval $[a_1,b_1]\subset I$ where $f(a_1)f(b_1)<0$

\subsubsection{Error due to Newton Method}

For one iteration
\[
	\text{Error} = \leq \frac{\eta^2 M_2}{2|f'(x_0)|} \quad\text{where}\quad M_2 = \sup_{x\in[x_0-\eta,x_0+\eta]} |f''(x)|
	.\]

\section{Fixed Point Iteration Method}
If a function can be converted to the form $x=g(x)$ along with the sequence $x_{n+1} = g(x_n)$ with initial guess $x_0$, then it is called a fixed point scheme.

The scheme converges if
\begin{itemize}
	\ii $\forall x \in [a,b]:\; g(x)\in[a,b]$
	\ii $g$ is strictly contracting meaning that $\exists \varepsilon \in \RR\; 0\leq \varepsilon < 1$
	\[
		\forall x,y \in[a,b],\; |g(x)-g(y)|\leq \varepsilon|x-y|
		.\]
\end{itemize}
then $\forall x_0$ the sequence converges to $l\in[a,b]$

\nt{
	\[
		\sup_{x\in[a,b]}|g'(x)| = L<1\Rightarrow g(x)\text{ is strictly contracting}
		.\]
}

\nt{
	Let $l$ be the solution to $g(l)=l$
	\begin{itemize}
		\ii If $|g'(l)| <1$ then there exists an interval $I$ containing $l$ for which the sequence converges to $l$
		\ii If $|g'(l)| >1$ then the sequence diverges
	\end{itemize}
}

\section{Order of Convergence}
Order of convergence (Rate of convergence) tells us how the error decreases between 2 iterations. The order of convergence $p$ of a sequence is defined to be
\[
	\lim_{n\to+\infty}\lt|\frac{x_{n+1}-l}{(x_n-l)}\rt|\in \RR^*_+
	.\]

\nt{
	The order of convergence of
	\begin{itemize}
		\ii Lagrange Method
		\[
			g'(l) = \frac{(b-l)^2}{2f(b)}f''(c)
			.\]
		If $f''(c)\neq0$ then $g'(l)\neq0$ then the order is 1.
		\ii Newton method, if $g'(l)=0$ then the order is at least 2.
	\end{itemize}
}

\nt{
	We stop the iteration method when
	\begin{itemize}
		\ii First case $g'(x)<0$, then we stop iteration when
		\[
			|x_{n+1}-r|<\varepsilon
			.\]
		\ii Second case $g'(x)>0$, then we stop iteration when
		\[
			|f(x_n)|<\eta
			.\]
		where
		\[
			\eta = \varepsilon\inf|f'(x)|
			.\]
	\end{itemize}
}

\section{Polynomial Shenanigans}
\subsection{Roots of $x^3 + px + q = 0$}

Let $y_1(x) = x^3 + px$ and $y_2(x)=-q$

\begin{itemize}
	\ii $p\geq 0\Rightarrow \exists\,1$ root
	\ii $p<0$ then we have 3 separate cases
	\[
		27q^2 + 4p^3 \begin{cases}
			=0 & \text{we have 2 separate real roots (one double and one single)} \\
			>0 & \text{we have one real root}                                     \\
			<0 & \text{we have 3 separate real roots}
		\end{cases}
		.\]
\end{itemize}

\subsection{Roots of $x^3+ax^2 + bx +c = 0$}

If we replace $x$ with $X+h$ where $h = -\frac{a}{3}$, we can get the cubic in the form
\[
	X^3 + PX + Q = 0
	.\]

where
\begin{align*}
	P & = -\frac{a^2}{3}+b                 \\
	Q & = \frac{2a^3}{27} - \frac{ab}{3}+c
\end{align*}

\subsection{Roots of $x^4+ax^3+bx^2+cx+d=0$}

If we replace $x$ with $X+h$ where $h = -\frac{a}{4}$, we can get the quartic in the form
\[
	X^4 + PX^2 + QX + R = 0
	.\]

where
\begin{align*}
	P & = -\frac{3a^2}{8}+b                   \\
	Q & = \frac{a^3}{8} -\frac{ab}{2} + c     \\
	R & = -\frac{3a^4}{256} -\frac{ac}{4} + d
\end{align*}

Let the circle $C$ be the circle of radius $\displaystyle\lt(-\frac{Q}{2},\frac{1-P}{2}\rt)$ and of radius $\displaystyle\sqrt{\lt(\frac{P-1}{2}\rt)^2 + \frac{Q^2}{4} - R}$.\\

The roots of the polynomial $X^4 + PX^2 + QX + R = 0$ are the intersection of the circle $C$ and the parabola $Y=X^2$


\chapter{Numerical Intergration}


Let $f$ be a continuous function on $[a,b]$ and $I=\int_a^b f(x)\,\dd{x}$

\section{Rectangle method}

We sample the domain of $f$ in to $n$ equal subintervals ($x_i - x_{i+1}=\frac{b-a}{n}=h$). The approximated value of $I$ becomes
\[
	\int_a^b f(x)\,\dd{x} \approx \frac{b-a}{n}\lt[ \sum_{i=0}^{n-1} f(x_i) \rt]
	.\]

such that $x_0=a$ and $x_1=b$\\

The error associated with this approximation is
\[
	|\varepsilon|\leq \frac{M_1}{2n}(b-a)^2
	.\]

where
\[
	M_1 = \sup_{[a,b]}\lt|f'(x)\rt|
	.\]

\section{Trapezoid method}

Same sampling as before.

\[
	\int_a^b f(x)\,\dd{x} \approx \frac{b-a}{2n}\lt(f(a)+f(b)+2\sum_{i=1}^{n-1}f(x_i)\rt)
	.\]
The error associated with this approximation is
\[
	|\varepsilon|\leq \frac{M-2}{12n^2}\lt|(b-a)^3\rt|
	.\]

where
\[
	M_2 = \sup_{[a,b]}\lt|f''(x)\rt|
	.\]

\section{Simpson's rule}

You get the point by now

\[
	\int_a^b f(x)\,\dd{x} \approx \frac{b-a}{6n}\lt(f(a)+f(b)+2\sum_{i=1}^{n-1}f(x_i) + 4\lt(
	f\lt(\frac{a+x_1}{2}\rt) +
	f\lt(\frac{x_1+x_2}{2}\rt) +
	\cdots
	f\lt(\frac{x_{n-1}+b}{2}\rt)
	\rt)
	\rt)
	.\]

The error associated with this approximation is
\[
	|\varepsilon|\leq \frac{\lt|(b-a)^5\rt|}{n^4}\frac{M_4}{2880}
	.\]

where
\[
	M_4 = \sup_{[a,b]}\lt|f^{(4)}(x)\rt|
	.\]


\section{Newton Cote's method}

Let $P$ be the Lagrange polynomial that interpolates the function $f$ at $(n+1)$ points $(x_0,f_0),(x_1,f_1),\dots,(x_n,f_n)$.
\[
	P(s) = \sum_{i=0}^{n}\ell_i(s)f(x_i)
	.\]

The approximated value of $I$ is
\[
	\int_a^b f(x)\,\dd{x} \approx (b-a)\sum_{i=0}^n H_i f_i
	.\]

where
\[
	H_i = \frac{1}{n}\int_0^n \ell_i(s)\,\dd{s}
	.\]

\end{document}

\documentclass{report}

\input{../template/preamble}
\input{../template/macros}
\input{../template/letterfonts}

\title{\Huge{Mechanics of Materials}\\Semester 4}
\author{}
\date{}

\begin{document}

\maketitle
\newpage% or \cleardoublepage
% \pdfbookmark[<level>]{<title>}{<dest>}
\pdfbookmark[section]{\contentsname}{toc}
\tableofcontents
\pagebreak

\chapter{Mathematical Concepts}
\section{Tensors}
\dfn{Einstein Notation}{
	Also known as summation notation, says that if we have a repeated index then we are summing over that index. For example
	\[
		y = c_i \vu{e}_i
		.\]
	implies that
	\[
		y = \sum_{i=1}^{3} c_i \vu{e}_i = c_1\vu{e}_1 + c_2\vu{e}_2 + c_3\vu{e}_3
		.\]
	same thing with
	\[
		a_i\cdot b_i = a_1\cdot b_1 + a_2\cdot b_2 + a_3\cdot b_3
		.\]
}



\dfn{}{
	Kronecker delta is defined to be
	\[
		\delta_{ij} = \begin{cases}
			0 & \text{if }i \neq j \\
			1 & \text{if }i = j
		\end{cases}
		.\]
	and the permutation symbol
	\[
		\varepsilon _{ijk}=\begin{cases}
			+1           & {\text{if }}(i,j,k){\text{ is }}(1,2,3),(2,3,1),{\text{ or }}(3,1,2), \\
			-1           & {\text{if }}(i,j,k){\text{ is }}(3,2,1),(1,3,2),{\text{ or }}(2,1,3), \\
			\phantom{+}0 & {\text{if }}i=j,{\text{ or }}j=k,{\text{ or }}k=i
		\end{cases}
		.\]
	And they appear in
	\begin{align*}
		\vu{e}_i \cdot \vu{e}_j  & = \delta_{ij}               \\
		\vu{e}_i \times \vu{e}_j & = \varepsilon_{ijk}\vu{e}_k
	\end{align*}
}

\dfn{Tensors}{
	In an $m$-dimensional space, a tensor of rank $n$ is a mathematical object that has $n$ indices, $m^n$ components, and obeys certain \emph{transformation rules}\\
}
\nt{
	Typically $m=3$ corresponding to the 3D space.
}
\ex{}{
	\begin{itemize}
		\item A rank 0 tensor is a scalar
		      \[
			      A
			      .\]
		\item A rank 1 tensor is a vector
		      \[
			      A \vu{x} = A_i x_i = A_1 x_1 + A_2 x_2 + A_3 x_3 = \begin{bmatrix}
				      A_1 \\A_2\\A_3
			      \end{bmatrix}
			      .\]
		\item A rank 2 tensor is a matrix
		      \[
			      A (\vu{x},\vu{y}) = A_{ij} x_i y_j = \begin{bmatrix}
				      A_{11} & A_{12} & A_{13} \\
				      A_{21} & A_{22} & A_{23} \\
				      A_{31} & A_{32} & A_{33}
			      \end{bmatrix}
			      .\]
	\end{itemize}
}
Some notable tensors are:
\begin{enumerate}
	\item Symmetric tensors
	      \[
		      A_{ij} = A_{ji}
		      .\]
	\item Anti-symmetric tensors
	      \[
		      A_{ij} = -A_{ji}
		      .\]
	\item General tensor. It can be represented using a symmetric and an anti symmetric tensor
	      \[
		      A = A^S + A^A
		      .\]
	      where
	      \begin{align*}
		      A^S & = \frac{1}{2}(A+A^T)      \\
		      A^A & = \frac{1}{2}(A\cdot A^T)
	      \end{align*}
\end{enumerate}

The identity tensor is the tensor whose components $I_{ij} = \delta_{ij}$
\[
	I = \begin{bmatrix}
		1 & 0 & 0 \\
		0 & 1 & 0 \\
		0 & 0 & 1
	\end{bmatrix}
	.\]

The scalar invariants of a tensor
\begin{enumerate}
	\ii $I_1 = \tr(A) = A_{ii} = A_{11} + A_{22} + A_{33}$
	\ii $I_2 = \frac{1}{2} \lt[\tr(A)^2 - tr(A^2)\rt] = \frac{1}{2} \lt(A_{ii}A_{jj} - A_{ij}A_{ji}\rt)$
	\ii $I_3 = \det(A) = \varepsilon_{iij}T_{i1}T_{j2}T_{k3}$
\end{enumerate}

The characteristic polynomial of a tensor $\det(A-\lambda I)$ can be expressed as
\[
	\det(A-\lambda I) = -\lambda^3 + I_1 \lambda^2 - I_2 \lambda + I_3
	.\]

\dfn{Tensor Product}{
	We define the tensor product between 2 tensors $X$ and $Y$ of order 3 to be
	\[
		(X\otimes Y)_{ij} = X_iY_j
		.\]
	and with a tensor of order 2 $T$
	\[
		(T\otimes X)_{ij} = T_{ij}X_k
		.\]

}

\ex{Tensor Product}{
	\begin{align*}
		\begin{bmatrix}
			1 & \alpha \\\alpha^* & 1
		\end{bmatrix}
		\otimes
		\begin{bmatrix}
			1 & \beta \\\beta^* & 1
		\end{bmatrix} & =
		\begin{bmatrix}
			1
			\begin{bmatrix}
				1 & \beta \\\beta^* & 1
			\end{bmatrix}
			 &
			\alpha
			\begin{bmatrix}
				1 & \beta \\\beta^* & 1
			\end{bmatrix}
			\\
			\alpha^*
			\begin{bmatrix}
				1 & \beta \\\beta^* & 1
			\end{bmatrix}
			 &
			1 \begin{bmatrix}
				1 & \beta \\\beta^* & 1
			\end{bmatrix}
		\end{bmatrix}
		\\
		                           & =
		\begin{bmatrix}
			1               & \beta         & \alpha        & \alpha\beta \\
			\beta^*         & 1             & \alpha\beta^* & \alpha      \\
			\alpha^*        & \alpha^*\beta & 1             & \beta       \\
			\alpha^*\beta^* & \alpha^*      & \beta^*       & 1           \\
		\end{bmatrix}
	\end{align*}
}

\url{https://www.math3ma.com/blog/the-tensor-product-demystified}

\nt{
	The order of a tensor product $X\otimes Y$ is the sum of the orders of $X$ and $Y$.
}
\dfn{Contraction}{
	We define the tensor product between 2 tensors $X$ and $Y$ to be
	\[
		X\cdot Y = X_iY_j
		.\]
}

\nt{
	From what I understand, a tensor product is the outer product and a contraction is an inner product
	\begin{align*}
		X\otimes Y & = X \times Y^T \\
		X\cdot Y   & = X^T \times Y
	\end{align*}
}

\section{Tensor Calculus}

\dfn{Gradient operator}{
	The gradient operator on a scalar tensor is defined to be
	\[
		\nabla f = \pdv{f}{x_i}\vu{e}_i = \pdv{f}{x_1}\vu{e}_1 + \pdv{f}{x_2}\vu{e}_2 + \pdv{f}{x_3}\vu{e}_3
		.\]
	in cylindrical coordinates
	\[
		\nabla f = \pdv{f}{r}\vu{e}_r + \frac{1}{r}\pdv{f}{\theta}\vu{e}_\theta + \pdv{f}{z}\vu{e}_z
		.\]
}

\dfn{Gradient of a vector}{
	The gradient of a vector tensor is
	\[
		\nabla \va{a} = \begin{bmatrix}
			\pdv{a_1}{x_1} & \pdv{a_1}{x_2} & \pdv{a_1}{x_3} \\
			\pdv{a_2}{x_1} & \pdv{a_2}{x_2} & \pdv{a_2}{x_3} \\
			\pdv{a_3}{x_1} & \pdv{a_3}{x_2} & \pdv{a_3}{x_3}
		\end{bmatrix}
		.\]
	in cylindrical coordinates
	\[
		\nabla \va{a} = \begin{bmatrix}
			\pdv{a_r}{r}      & \frac{1}{r}\lt(\pdv{a_r}{\theta}-a_\theta\rt) & \pdv{a_r}{z}      \\
			\pdv{a_\theta}{r} & \frac{1}{r}\lt(\pdv{a_\theta}{\theta}+a_r\rt) & \pdv{a_\theta}{z} \\
			\pdv{a_z}{r}      & \frac{1}{r}\pdv{a_z}{\theta}                  & \pdv{a_z}{z}
		\end{bmatrix}
		.\]
}
\nt{
	The order of a gradient tensor is 1 order higher than the tensor it operates on.
}

\dfn{Divergence}{
	The divergence is defined to be
	\[
		\nabla\cdot \va{a} = \tr(\nabla \va{a})
		.\]
	unlike the gradient, it reduces the order of the tensor.
}
\dfn{Laplacian}{
	The Laplacian is is the composition of a divergence and a gradient. It keeps the same order of the tensor
	\[
		\Delta f = \nabla\cdot(\nabla f) =\pdv[2]{f}{x_1} + \pdv[2]{f}{x_2} + \pdv[2]{f}{x_3}
		.\]
}

\dfn{Rotation}{
	Rotation mostly applies to vector tensors and retains the same order as it
	\[
		(\nabla\times \va{a})_i = \varepsilon_{ijk}\pdv{a_k}{x_j}
		.\]
	\[
		\nabla \times \va{a} =
		\lt(\pdv{a_3}{x_2} - \pdv{a_2}{x_3}\rt)\vu{e}_1 +
		\lt(\pdv{a_1}{x_3} - \pdv{a_3}{x_1}\rt)\vu{e}_2 +
		\lt(\pdv{a_2}{x_1} - \pdv{a_1}{x_2}\rt)\vu{e}_3
		.\]
	in cylindrical coordinates
	\[
		\nabla \times \va{a} =
		\lt(\frac{1}{r}\pdv{a_z}{\theta} - \pdv{a_\theta}{z}\rt)\vu{e}_r +
		\lt(\pdv{a_r}{r} - \pdv{a_z}{x_r}\rt)\vu{e}_\theta +
		\lt(\pdv{a_\theta}{r} - \frac{1}{r}\pdv{a_r}{\theta}+\frac{a_\theta}{r}\rt)\vu{e}_z
		.\]
}

\nt{
	\begin{align*}
		\nabla\times (\nabla \va{a})      & =0                          \\
		\nabla \cdot(\nabla\times \va{a}) & =0                          \\
		\nabla (ab)                       & = a(\nabla b) + b(\nabla a)
	\end{align*}
}

\thm{Ostrogradsky's theorem}{
	Denote $\iiint_D \dots\dd{V}$ as a volume integral and $\iint_S \dots\vu{n}\dd{S}$ as a surface integral.
	\begin{align*}
		\iiint_D \nabla f \dd{V}           & = \iint_S f\vu{n}\dd{S}      \\
		\iiint_D \nabla\cdot \va{U} \dd{V} & = \iint_S \va{U}\vu{n}\dd{S} \\
		\iiint_D \nabla \cdot T \dd{V}     & = \iint_S T\vu{n}\dd{S}      \\
	\end{align*} % TODO make all 2nd order tensors \vb{}
}

\chapter{Deformation}

We consider a body under some deformation, at time $t=0$, a point $P$ on that body can be described as
\[
	\va{X} = X_k\vu{e}_k
	.\]

after some time $t$ the object has deformed and the position of the point $P$ is now $\va{x}$. The relation between it's initial position and it's new position is
\[
	\va{x} = \va*{\Phi}(\va{X},t)
	.\]
where $\va*{\Phi}$ is a bijective transformation($\forall \va*{\Phi}, \exists \va*{\Phi}^{-1}$). The vector $\va{x}$ is a function of the initial position and time.

The displacement vector is
\[
	\va{u}\lt(\va{X},t\rt) = \va{x}-\va{X}
	.\]
velocity vector
\[
	\va{v}\lt(\va{X},t\rt) = \pdv{\va{x}}{t}
	.\]
and acceleration vector
\[
	\va{a} = \pdv{\va{v}}{t}
	.\]

We consider a point $P$ on a body and 2 points on the same body $Q_1$ and $Q_2$ described with respect to the point $P$. The differentials of $Q_1$ and $Q_2$ are
\begin{align*}
	\dd{\va{X}_1} & = \va{X}_{Q_1} - \va{X}_P \\
	\dd{\va{X}_2} & = \va{X}_{Q_2} - \va{X}_P
\end{align*}
and after the deformation
\begin{align*}
	\dd{\va{x}_1} & = \va*{\Phi}\lt(\va{X}_P+\dd{\va{X}_1},t\rt) - \va*{\Phi}\lt(\va{X}_P,t\rt) \\
	\dd{\va{x}_2} & = \va*{\Phi}\lt(\va{X}_P+\dd{\va{X}_2},t\rt) - \va*{\Phi}\lt(\va{X}_P,t\rt) \\
\end{align*}

we define a differential tensor of the transformation
\[
	\vb{F}\lt(\va{X},t\rt) = \pdv{\va*{\Phi}}{\va{X}}
	.\]

aka the Jacobian matrix
\[
	\vb{F} = \begin{bmatrix}
		\pdv{x_1}{X_1} & \pdv{x_1}{X_2} & \pdv{x_1}{X_3} \\
		\pdv{x_2}{X_1} & \pdv{x_2}{X_2} & \pdv{x_2}{X_3} \\
		\pdv{x_3}{X_1} & \pdv{x_3}{X_2} & \pdv{x_3}{X_3}
	\end{bmatrix}
	.\]

The differential can be written as
\begin{align*}
	\dd{\va{x}_1} & =\vb{F}\lt(\va{X}_P,t\rt)\cdot\dd{\va{X}_1} \\
	\dd{\va{x}_2} & =\vb{F}\lt(\va{X}_P,t\rt)\cdot\dd{\va{X}_2} \\
\end{align*}

The Jacobian is also useful for a change of reference when integrating
\[
	\int_v a(\va{x})\dd{v} = \int_V a\lt(\va{x}\lt(\va{X},t\rt)\rt)\det(\vb{F})\dd{V}
	.\]

The relation between vectors before and after deformation
\[
	\dd{\va{x}_1}\cdot\dd{\va{x}_2} = \dd{\va{X}_1}\cdot\vb{C}\cdot\dd{\va{X}_2}
	.\]
where $\vb{C}$ is the Cauchy–Green deformation tensor
\[
	\vb{C} = \vb{F}^T \cdot \vb{F}
	.\]

The elongation after deformation in a given direction
\[
	\delta(\dd{\va{X}})=\dv{\va{x}}{\va{X}}-1 = \dv{\sqrt{\dd{\va{X}}\cdot\vb{C}\cdot\dd{\va{X}}}}{\va{X}} - 1
	.\]

We define
\[
	\lambda = \dv{\sqrt{\dd{\va{X}}\cdot\vb{C}\cdot\dd{\va{X}}}}{\va{X}} = \delta + 1
	.\]

\[
	\delta \begin{cases}
		>0 & \text{elongation in the direction of }\dd{\va{x}}  \\
		<0 & \text{contraction in the direction of }\dd{\va{x}}
	\end{cases}
	.\]

Consider 2 orthogonal vectors, $X_1$ and $X_2$. The new angle formed $\alpha = \frac{\pi}{2}-\gamma$ is calculated using the formula
\[
	\sin(\gamma) = \frac{\dd{\va{X}}_1\cdot\vb{C}\cdot\dd{\va{X}_2}}{\sqrt{\dd{\va{X}}_1\cdot\vb{C}\cdot\dd{\va{X}_1}}\cdot\sqrt{\dd{\va{X}}_2\cdot\vb{C}\cdot\dd{\va{X}_2}}}
	.\]

We define the Green-Lagrangian strain tensor to be
\[
	\vb{E}=\frac{1}{2}(\vb{C}-\vb{I}) = \frac{1}{2}(\vb{F}^T\vb{F}-\vb{I})
	.\]

The diagonal elements of $\vb{E}$ represent scaling of basis vectors of the body, while non-diagonal elements represent the change in angle (rotation).\\

We define $\dd{L}$ to the magnitude of $\dd{\va{X}}$ and $\vu{N}$ to be its direction vector.
\[
	\dd{\va{X}} = \dd{L}\vu{N}
	.\]
similarly for $\dd{\va{x}}$
\[
	\dd{\va{x}} = \dd{l}\vu{n}
	.\]

it follows that
\[
	\frac{1}{2}\lt(\frac{\dd{l}^2-\dd{L}^2}{\dd{L}^2}\rt) = \vu{N}\cdot\vb{E}\cdot\vu{N}
	.\]

and the angle between the 2 transformed vectors becomes $\alpha = \frac{\pi}{2}-\gamma$
\[
	\frac{1}{2}\sin(\gamma)\dv{l_1}{L_1}\dv{l_2}{L_2}=\vu{N}_1\cdot \vb{C}\cdot\vu{N}_2
	.\]

We can decompose the gradient tensor $\vb{F}$ in to 2 other tensors where $\vb{R}$ is an orthogonal matrix ($\vb{R}^T = \vb{R}^{-1}$) and $\vb{U}$ is a symmetric matrix
\[
	\vb{F}=\vb{R}\cdot\vb{U}
	.\]

\[
	\vb{C} = \vb{U}\cdot\vb{U}
	.\]
and
\[
	\vb{E} = \frac{1}{2}(\vb{U}^2-\vb{I})
	.\]

in a small displacement
\[
	\vb{E} = \frac{1}{2}\lt(\pdv{\va{u}}{\va{X}} + \lt(\pdv{\va{u}}{\va{X}}\rt)^T + \lt(\pdv{\va{u}}{\va{X}}\rt)^T\cdot \pdv{\va{u}}{\va{X}}\rt)
	.\]

We can ignore the quadratic terms to obtain the strain tensor for small displacement $\vb*{\varepsilon}$
\[
	\vb*{\varepsilon} \approx \frac{1}{2}\lt(\pdv{\va{u}}{\va{X}} + \lt(\pdv{\va{u}}{\va{X}}\rt)^T\rt)
	.\]
\[
	\varepsilon_{ij} = \frac{1}{2}\lt(\pdv{u_i}{X_j} + \pdv{u_j}{X_i}\rt)
	.\]
Using the above definition we can explicitly define the matrix
\[
	\vb{\varepsilon} = \begin{bmatrix}
		\pdv{u_1}{X_1}                                     & \frac{1}{2}\lt(\pdv{u_1}{X_2} + \pdv{u_2}{X_1}\rt) & \frac{1}{2}\lt(\pdv{u_1}{X_3} + \pdv{u_3}{X_1}\rt) \\
		\frac{1}{2}\lt(\pdv{u_1}{X_2} + \pdv{u_2}{X_1}\rt) & \pdv{u_2}{X_2}                                     & \frac{1}{2}\lt(\pdv{u_2}{X_3} + \pdv{u_3}{X_2}\rt) \\
		\frac{1}{2}\lt(\pdv{u_1}{X_3} + \pdv{u_3}{X_1}\rt) & \frac{1}{2}\lt(\pdv{u_2}{X_3} + \pdv{u_3}{X_2}\rt) & \pdv{u_3}{X_3}
	\end{bmatrix}
	.\]

We can also prove that $\gamma$ between the 2 base vectors $\vu{e}_1$ and $\vu{e}_2$ is
\[
	\frac{\gamma}{2} = \varepsilon_{12}
	.\]

\end{document}

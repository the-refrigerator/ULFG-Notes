\documentclass{report}

%%%%%%%%%%%%%%%%%%%%%%%%%%%%%%%%%
% PACKAGE IMPORTS
%%%%%%%%%%%%%%%%%%%%%%%%%%%%%%%%%


\usepackage[tmargin=2cm,rmargin=1in,lmargin=1in,margin=0.85in,bmargin=2cm,footskip=.2in]{geometry}
\usepackage{amsmath,amsfonts,amsthm,amssymb,mathtools}
\usepackage[varbb]{newpxmath}
\usepackage{xfrac}
\usepackage[makeroom]{cancel}
\usepackage{mathtools}
\usepackage{bookmark}
\usepackage{enumitem}
\usepackage{hyperref,theoremref}
\usepackage{xparse}
\hypersetup{
	pdftitle={Hamboola my beloved},
	colorlinks=true, linkcolor=doc!90,
	bookmarksnumbered=true,
	bookmarksopen=true
}
\usepackage[most,many,breakable]{tcolorbox}
\usepackage{xcolor}
\usepackage{varwidth}
\usepackage{varwidth}
\usepackage{etoolbox}
\usepackage{bm}
%\usepackage{authblk}
\usepackage{pgfplots}
\usepackage{nameref}
\usepackage{multicol,array}
\usepackage{tikz-cd}
\usepackage[ruled,vlined,linesnumbered]{algorithm2e}
\usepackage{comment} % enables the use of multi-line comments (\ifx \fi) 
\usepackage{import}
\usepackage{xifthen}
\usepackage{pdfpages}
\usepackage{transparent}
\usepackage{minted}
\usepackage{fontspec}
\usepackage{tasks}
\usepackage{chemfig}
\usepackage[version=4]{mhchem}
\usepackage{suffix}
\usepackage{tabularx}
\usepackage{subcaption}
\usepackage{physics}


% \setmonofont{SpaceMono Nerd Font}
\setminted{fontsize=\footnotesize}

\newcommand\mycommfont[1]{\footnotesize\ttfamily\textcolor{blue}{#1}}
\SetCommentSty{mycommfont}
\newcommand{\incfig}[1]{%
	\def\svgwidth{\columnwidth}
	\import{./figures/}{#1.pdf_tex}
}

\usepackage{tikzsymbols}
% \renewcommand\qedsymbol{$\Laughey$}
\pgfplotsset{compat=1.18}

%\usepackage{import}
%\usepackage{xifthen}
%\usepackage{pdfpages}
%\usepackage{transparent}


%%%%%%%%%%%%%%%%%%%%%%%%%%%%%%
% SELF MADE COLORS
%%%%%%%%%%%%%%%%%%%%%%%%%%%%%%



\definecolor{myg}{RGB}{56, 140, 70}
\definecolor{myb}{RGB}{45, 111, 177}
\definecolor{myr}{RGB}{199, 68, 64}
\definecolor{mytheorembg}{HTML}{F2F2F9}
\definecolor{mytheoremfr}{HTML}{00007B}
\definecolor{mylenmabg}{HTML}{FFFAF8}
\definecolor{mylenmafr}{HTML}{983b0f}
\definecolor{mypropbg}{HTML}{f2fbfc}
\definecolor{mypropfr}{HTML}{191971}
\definecolor{myexamplebg}{HTML}{F2FBF8}
\definecolor{myexamplefr}{HTML}{88D6D1}
\definecolor{myexampleti}{HTML}{2A7F7F}
\definecolor{mydefinitbg}{HTML}{E5E5FF}
\definecolor{mydefinitfr}{HTML}{3F3FA3}
\definecolor{notesgreen}{RGB}{0,162,0}
\definecolor{myp}{RGB}{197, 92, 212}
\definecolor{mygr}{HTML}{2C3338}
\definecolor{myred}{RGB}{127,0,0}
\definecolor{myyellow}{RGB}{169,121,69}
\definecolor{myexercisebg}{HTML}{F2FBF8}
\definecolor{myexercisefg}{HTML}{88D6D1}
\definecolor{codebg}{HTML}{0D1117}

%%%%%%%%%%%%%%%%%%%%%%%%%%%%
% TCOLORBOX SETUPS
%%%%%%%%%%%%%%%%%%%%%%%%%%%%

\setlength{\parindent}{0cm}
%================================
% THEOREM BOX
%================================

\tcbuselibrary{theorems,skins,hooks}
\newtcbtheorem[number within=section]{Theorem}{Theorem}
{%
	enhanced,
	breakable,
	colback = mytheorembg,
	frame hidden,
	boxrule = 0sp,
	borderline west = {2pt}{0pt}{mytheoremfr},
	sharp corners,
	detach title,
	before upper = \tcbtitle\par\smallskip,
	coltitle = mytheoremfr,
	fonttitle = \bfseries\sffamily,
	description font = \mdseries,
	separator sign none,
	segmentation style={solid, mytheoremfr},
}
{th}

\tcbuselibrary{theorems,skins,hooks}
\newtcbtheorem[number within=chapter]{theorem}{Theorem}
{%
	enhanced,
	breakable,
	colback = mytheorembg,
	frame hidden,
	boxrule = 0sp,
	borderline west = {2pt}{0pt}{mytheoremfr},
	sharp corners,
	detach title,
	before upper = \tcbtitle\par\smallskip,
	coltitle = mytheoremfr,
	fonttitle = \bfseries\sffamily,
	description font = \mdseries,
	separator sign none,
	segmentation style={solid, mytheoremfr},
}
{th}


\tcbuselibrary{theorems,skins,hooks}
\newtcolorbox{Theoremcon}
{%
	enhanced
	,breakable
	,colback = mytheorembg
	,frame hidden
	,boxrule = 0sp
	,borderline west = {2pt}{0pt}{mytheoremfr}
	,sharp corners
	,description font = \mdseries
	,separator sign none
}

%================================
% Corollery
%================================
\tcbuselibrary{theorems,skins,hooks}
\newtcbtheorem[number within=section]{Corollary}{Corollary}
{%
	enhanced
	,breakable
	,colback = myp!10
	,frame hidden
	,boxrule = 0sp
	,borderline west = {2pt}{0pt}{myp!85!black}
	,sharp corners
	,detach title
	,before upper = \tcbtitle\par\smallskip
	,coltitle = myp!85!black
	,fonttitle = \bfseries\sffamily
	,description font = \mdseries
	,separator sign none
	,segmentation style={solid, myp!85!black}
}
{th}
\tcbuselibrary{theorems,skins,hooks}
\newtcbtheorem[number within=chapter]{corollary}{Corollary}
{%
	enhanced
	,breakable
	,colback = myp!10
	,frame hidden
	,boxrule = 0sp
	,borderline west = {2pt}{0pt}{myp!85!black}
	,sharp corners
	,detach title
	,before upper = \tcbtitle\par\smallskip
	,coltitle = myp!85!black
	,fonttitle = \bfseries\sffamily
	,description font = \mdseries
	,separator sign none
	,segmentation style={solid, myp!85!black}
}
{th}


%================================
% LENMA
%================================

\tcbuselibrary{theorems,skins,hooks}
\newtcbtheorem[number within=section]{Lenma}{Lenma}
{%
	enhanced,
	breakable,
	colback = mylenmabg,
	frame hidden,
	boxrule = 0sp,
	borderline west = {2pt}{0pt}{mylenmafr},
	sharp corners,
	detach title,
	before upper = \tcbtitle\par\smallskip,
	coltitle = mylenmafr,
	fonttitle = \bfseries\sffamily,
	description font = \mdseries,
	separator sign none,
	segmentation style={solid, mylenmafr},
}
{th}

\tcbuselibrary{theorems,skins,hooks}
\newtcbtheorem[number within=chapter]{lenma}{Lenma}
{%
	enhanced,
	breakable,
	colback = mylenmabg,
	frame hidden,
	boxrule = 0sp,
	borderline west = {2pt}{0pt}{mylenmafr},
	sharp corners,
	detach title,
	before upper = \tcbtitle\par\smallskip,
	coltitle = mylenmafr,
	fonttitle = \bfseries\sffamily,
	description font = \mdseries,
	separator sign none,
	segmentation style={solid, mylenmafr},
}
{th}


%================================
% PROPOSITION
%================================

\tcbuselibrary{theorems,skins,hooks}
\newtcbtheorem[number within=section]{Prop}{Proposition}
{%
	enhanced,
	breakable,
	colback = mypropbg,
	frame hidden,
	boxrule = 0sp,
	borderline west = {2pt}{0pt}{mypropfr},
	sharp corners,
	detach title,
	before upper = \tcbtitle\par\smallskip,
	coltitle = mypropfr,
	fonttitle = \bfseries\sffamily,
	description font = \mdseries,
	separator sign none,
	segmentation style={solid, mypropfr},
}
{th}

\tcbuselibrary{theorems,skins,hooks}
\newtcbtheorem[number within=chapter]{prop}{Proposition}
{%
	enhanced,
	breakable,
	colback = mypropbg,
	frame hidden,
	boxrule = 0sp,
	borderline west = {2pt}{0pt}{mypropfr},
	sharp corners,
	detach title,
	before upper = \tcbtitle\par\smallskip,
	coltitle = mypropfr,
	fonttitle = \bfseries\sffamily,
	description font = \mdseries,
	separator sign none,
	segmentation style={solid, mypropfr},
}
{th}


%================================
% CLAIM
%================================

\tcbuselibrary{theorems,skins,hooks}
\newtcbtheorem[number within=section]{claim}{Claim}
{%
	enhanced
	,breakable
	,colback = myg!10
	,frame hidden
	,boxrule = 0sp
	,borderline west = {2pt}{0pt}{myg}
	,sharp corners
	,detach title
	,before upper = \tcbtitle\par\smallskip
	,coltitle = myg!85!black
	,fonttitle = \bfseries\sffamily
	,description font = \mdseries
	,separator sign none
	,segmentation style={solid, myg!85!black}
}
{th}



%================================
% Exercise
%================================

\tcbuselibrary{theorems,skins,hooks}
\newtcbtheorem[number within=section]{Exercise}{Exercise}
{%
	enhanced,
	breakable,
	colback = myexercisebg,
	frame hidden,
	boxrule = 0sp,
	borderline west = {2pt}{0pt}{myexercisefg},
	sharp corners,
	detach title,
	before upper = \tcbtitle\par\smallskip,
	coltitle = myexercisefg,
	fonttitle = \bfseries\sffamily,
	description font = \mdseries,
	separator sign none,
	segmentation style={solid, myexercisefg},
}
{th}

\tcbuselibrary{theorems,skins,hooks}
\newtcbtheorem[number within=chapter]{exercise}{Exercise}
{%
	enhanced,
	breakable,
	colback = myexercisebg,
	frame hidden,
	boxrule = 0sp,
	borderline west = {2pt}{0pt}{myexercisefg},
	sharp corners,
	detach title,
	before upper = \tcbtitle\par\smallskip,
	coltitle = myexercisefg,
	fonttitle = \bfseries\sffamily,
	description font = \mdseries,
	separator sign none,
	segmentation style={solid, myexercisefg},
}
{th}

%================================
% EXAMPLE BOX
%================================

\newtcbtheorem[number within=section]{Example}{Example}
{%
	colback = myexamplebg
	,breakable
	,colframe = myexamplefr
	,coltitle = myexampleti
	,boxrule = 1pt
	,sharp corners
	,detach title
	,before upper=\tcbtitle\par\smallskip
	,fonttitle = \bfseries
	,description font = \mdseries
	,separator sign none
	,description delimiters parenthesis
}
{ex}

\newtcbtheorem[number within=chapter]{example}{Example}
{%
	colback = myexamplebg
	,breakable
	,colframe = myexamplefr
	,coltitle = myexampleti
	,boxrule = 1pt
	,sharp corners
	,detach title
	,before upper=\tcbtitle\par\smallskip
	,fonttitle = \bfseries
	,description font = \mdseries
	,separator sign none
	,description delimiters parenthesis
}
{ex}

%================================
% DEFINITION BOX
%================================

\newtcbtheorem[number within=section]{Definition}{Definition}{enhanced,
	before skip=2mm,after skip=2mm, colback=red!5,colframe=red!80!black,boxrule=0.5mm,
	attach boxed title to top left={xshift=1cm,yshift*=1mm-\tcboxedtitleheight}, varwidth boxed title*=-3cm,
	boxed title style={frame code={
					\path[fill=tcbcolback]
					([yshift=-1mm,xshift=-1mm]frame.north west)
					arc[start angle=0,end angle=180,radius=1mm]
					([yshift=-1mm,xshift=1mm]frame.north east)
					arc[start angle=180,end angle=0,radius=1mm];
					\path[left color=tcbcolback!60!black,right color=tcbcolback!60!black,
						middle color=tcbcolback!80!black]
					([xshift=-2mm]frame.north west) -- ([xshift=2mm]frame.north east)
					[rounded corners=1mm]-- ([xshift=1mm,yshift=-1mm]frame.north east)
					-- (frame.south east) -- (frame.south west)
					-- ([xshift=-1mm,yshift=-1mm]frame.north west)
					[sharp corners]-- cycle;
				},interior engine=empty,
		},
	fonttitle=\bfseries,
	title={#2},#1}{def}
\newtcbtheorem[number within=chapter]{definition}{Definition}{enhanced,
	before skip=2mm,after skip=2mm, colback=red!5,colframe=red!80!black,boxrule=0.5mm,
	attach boxed title to top left={xshift=1cm,yshift*=1mm-\tcboxedtitleheight}, varwidth boxed title*=-3cm,
	boxed title style={frame code={
					\path[fill=tcbcolback]
					([yshift=-1mm,xshift=-1mm]frame.north west)
					arc[start angle=0,end angle=180,radius=1mm]
					([yshift=-1mm,xshift=1mm]frame.north east)
					arc[start angle=180,end angle=0,radius=1mm];
					\path[left color=tcbcolback!60!black,right color=tcbcolback!60!black,
						middle color=tcbcolback!80!black]
					([xshift=-2mm]frame.north west) -- ([xshift=2mm]frame.north east)
					[rounded corners=1mm]-- ([xshift=1mm,yshift=-1mm]frame.north east)
					-- (frame.south east) -- (frame.south west)
					-- ([xshift=-1mm,yshift=-1mm]frame.north west)
					[sharp corners]-- cycle;
				},interior engine=empty,
		},
	fonttitle=\bfseries,
	title={#2},#1}{def}



%================================
% Solution BOX
%================================

\makeatletter
\newtcbtheorem{question}{Question}{enhanced,
	breakable,
	colback=white,
	colframe=myb!80!black,
	attach boxed title to top left={yshift*=-\tcboxedtitleheight},
	fonttitle=\bfseries,
	title={#2},
	boxed title size=title,
	boxed title style={%
			sharp corners,
			rounded corners=northwest,
			colback=tcbcolframe,
			boxrule=0pt,
		},
	underlay boxed title={%
			\path[fill=tcbcolframe] (title.south west)--(title.south east)
			to[out=0, in=180] ([xshift=5mm]title.east)--
			(title.center-|frame.east)
			[rounded corners=\kvtcb@arc] |-
			(frame.north) -| cycle;
		},
	#1
}{def}
\makeatother

%================================
% SOLUTION BOX
%================================

\makeatletter
\newtcolorbox{solution}{enhanced,
	breakable,
	colback=white,
	colframe=myg!80!black,
	attach boxed title to top left={yshift*=-\tcboxedtitleheight},
	title=Solution,
	boxed title size=title,
	boxed title style={%
			sharp corners,
			rounded corners=northwest,
			colback=tcbcolframe,
			boxrule=0pt,
		},
	underlay boxed title={%
			\path[fill=tcbcolframe] (title.south west)--(title.south east)
			to[out=0, in=180] ([xshift=5mm]title.east)--
			(title.center-|frame.east)
			[rounded corners=\kvtcb@arc] |-
			(frame.north) -| cycle;
		},
}
\makeatother

%================================
% Question BOX
%================================

\makeatletter
\newtcbtheorem{qstion}{Question}{enhanced,
	breakable,
	colback=white,
	colframe=mygr,
	attach boxed title to top left={yshift*=-\tcboxedtitleheight},
	fonttitle=\bfseries,
	title={#2},
	boxed title size=title,
	boxed title style={%
			sharp corners,
			rounded corners=northwest,
			colback=tcbcolframe,
			boxrule=0pt,
		},
	underlay boxed title={%
			\path[fill=tcbcolframe] (title.south west)--(title.south east)
			to[out=0, in=180] ([xshift=5mm]title.east)--
			(title.center-|frame.east)
			[rounded corners=\kvtcb@arc] |-
			(frame.north) -| cycle;
		},
	#1
}{def}
\makeatother

\newtcbtheorem[number within=chapter]{wconc}{Wrong Concept}{
	breakable,
	enhanced,
	colback=white,
	colframe=myr,
	arc=0pt,
	outer arc=0pt,
	fonttitle=\bfseries\sffamily\large,
	colbacktitle=myr,
	attach boxed title to top left={},
	boxed title style={
			enhanced,
			skin=enhancedfirst jigsaw,
			arc=3pt,
			bottom=0pt,
			interior style={fill=myr}
		},
	#1
}{def}



%================================
% NOTE BOX
%================================

\usetikzlibrary{hobby}
\usetikzlibrary{arrows,calc,shadows.blur}
\tcbuselibrary{skins}
\newtcolorbox{note}[1][]{%
	enhanced jigsaw,
	colback=gray!20!white,%
	colframe=gray!80!black,
	size=small,
	boxrule=1pt,
	title=\textbf{Note:-},
	halign title=flush center,
	coltitle=black,
	breakable,
	drop shadow=black!50!white,
	attach boxed title to top left={xshift=1cm,yshift=-\tcboxedtitleheight/2,yshifttext=-\tcboxedtitleheight/2},
	minipage boxed title=1.5cm,
	boxed title style={%
			colback=white,
			size=fbox,
			boxrule=1pt,
			boxsep=2pt,
			underlay={%
					\coordinate (dotA) at ($(interior.west) + (-0.5pt,0)$);
					\coordinate (dotB) at ($(interior.east) + (0.5pt,0)$);
					\begin{scope}
						\clip (interior.north west) rectangle ([xshift=3ex]interior.east);
						\filldraw [white, blur shadow={shadow opacity=60, shadow yshift=-.75ex}, rounded corners=2pt] (interior.north west) rectangle (interior.south east);
					\end{scope}
					\begin{scope}[gray!80!black]
						\fill (dotA) circle (2pt);
						\fill (dotB) circle (2pt);
					\end{scope}
				},
		},
	#1,
}

%%%%%%%%%%%%%%%%%%%%%%%%%%%%%%
% SELF MADE COMMANDS
%%%%%%%%%%%%%%%%%%%%%%%%%%%%%%


\newcommand{\thm}[2]{\begin{Theorem}{#1}{}#2\end{Theorem}}
\newcommand{\cor}[2]{\begin{Corollary}{#1}{}#2\end{Corollary}}
\newcommand{\mlenma}[2]{\begin{Lenma}{#1}{}#2\end{Lenma}}
\newcommand{\mprop}[2]{\begin{Prop}{#1}{}#2\end{Prop}}
\newcommand{\clm}[3]{\begin{claim}{#1}{#2}#3\end{claim}}
\newcommand{\wc}[2]{\begin{wconc}{#1}{}\setlength{\parindent}{1cm}#2\end{wconc}}
\newcommand{\thmcon}[1]{\begin{Theoremcon}{#1}\end{Theoremcon}}
\newcommand{\ex}[2]{\begin{Example}{#1}{}#2\end{Example}}
\newcommand{\dfn}[2]{\begin{Definition}[colbacktitle=red!75!black]{#1}{}#2\end{Definition}}
\newcommand{\dfnc}[2]{\begin{definition}[colbacktitle=red!75!black]{#1}{}#2\end{definition}}
\newcommand{\qs}[2]{\begin{question}{#1}{}#2\end{question}}
\newcommand{\pf}[2]{\begin{myproof}[#1]#2\end{myproof}}
\newcommand{\nt}[1]{\begin{note}#1\end{note}}

\newcommand*\circled[1]{\tikz[baseline=(char.base)]{
		\node[shape=circle,draw,inner sep=1pt] (char) {#1};}}
\newcommand\getcurrentref[1]{%
	\ifnumequal{\value{#1}}{0}
	{??}
	{\the\value{#1}}%
}
\newcommand{\getCurrentSectionNumber}{\getcurrentref{section}}
\newenvironment{myproof}[1][\proofname]{%
	\proof[\bfseries #1: ]%
}{\endproof}

\newcommand{\mclm}[2]{\begin{myclaim}[#1]#2\end{myclaim}}
\newenvironment{myclaim}[1][\claimname]{\proof[\bfseries #1: ]}{}

\newcounter{mylabelcounter}

\makeatletter
\newcommand{\setword}[2]{%
	\phantomsection
	#1\def\@currentlabel{\unexpanded{#1}}\label{#2}%
}
\makeatother




\tikzset{
	symbol/.style={
			draw=none,
			every to/.append style={
					edge node={node [sloped, allow upside down, auto=false]{$#1$}}}
		}
}


% deliminators
% \DeclarePairedDelimiter{\abs}{\lvert}{\rvert}
% \DeclarePairedDelimiter{\norm}{\lVert}{\rVert}

\DeclarePairedDelimiter{\ceil}{\lceil}{\rceil}
\DeclarePairedDelimiter{\floor}{\lfloor}{\rfloor}
\DeclarePairedDelimiter{\round}{\lfloor}{\rceil}

\newsavebox\diffdbox
\newcommand{\slantedromand}{{\mathpalette\makesl{d}}}
\newcommand{\makesl}[2]{%
	\begingroup
	\sbox{\diffdbox}{$\mathsurround=0pt#1\mathrm{#2}$}%
	\pdfsave
	\pdfsetmatrix{1 0 0.2 1}%
	\rlap{\usebox{\diffdbox}}%
	\pdfrestore
	\hskip\wd\diffdbox
	\endgroup
}
% \newcommand{\dd}[1][]{\ensuremath{\mathop{}\!\ifstrempty{#1}{%
% 			\slantedromand\@ifnextchar^{\hspace{0.2ex}}{\hspace{0.1ex}}}%
% 		{\slantedromand\hspace{0.2ex}^{#1}}}}
\ProvideDocumentCommand\dv{o m g}{%
	\ensuremath{%
		\IfValueTF{#3}{%
			\IfNoValueTF{#1}{%
				\frac{\dd #2}{\dd #3}%
			}{%
				\frac{\dd^{#1} #2}{\dd #3^{#1}}%
			}%
		}{%
			\IfNoValueTF{#1}{%
				\frac{\dd}{\dd #2}%
			}{%
				\frac{\dd^{#1}}{\dd #2^{#1}}%
			}%
		}%
	}%
}
\DeclareDocumentCommand\pdv{ s o m g g d() }
{ % Partial derivative
	% s: star for \flatfrac flat derivative
	% o: optional n for nth derivative
	% m: mandatory (x in df/dx)
	% g: optional (f in df/dx)
	% g: optional (y in d^2f/dxdy)
	% d: long-form d/dx(...)
	\IfBooleanTF{#1}
	{\let\fractype\flatfrac}
	{\let\fractype\frac}
	\IfNoValueTF{#4}
	{
		\IfNoValueTF{#6}
		{\fractype{\partial \IfNoValueTF{#2}{}{^{#2}}}{\partial #3\IfNoValueTF{#2}{}{^{#2}}}}
		{\fractype{\partial \IfNoValueTF{#2}{}{^{#2}}}{\partial #3\IfNoValueTF{#2}{}{^{#2}}} \argopen(#6\argclose)}
	}
	{
		\IfNoValueTF{#5}
		{\fractype{\partial \IfNoValueTF{#2}{}{^{#2}} #3}{\partial #4\IfNoValueTF{#2}{}{^{#2}}}}
		{\fractype{\partial^2 #3}{\partial #4 \partial #5}}
	}
}
% \providecommand*{\pdv}[3][]{\frac{\partial^{#1}#2}{\partial#3^{#1}}}
%  - others
\DeclareMathOperator{\Lap}{\mathcal{L}}
\DeclareMathOperator{\Var}{Var} % varience
\DeclareMathOperator{\Cov}{Cov} % covarience
\DeclareMathOperator{\E}{E} % expected

% Since the amsthm package isn't loaded

% I prefer the slanted \leq
\let\oldleq\leq % save them in case they're every wanted
\let\oldgeq\geq
\renewcommand{\leq}{\leqslant}
\renewcommand{\geq}{\geqslant}

% % redefine matrix env to allow for alignment, use r as default
% \renewcommand*\env@matrix[1][r]{\hskip -\arraycolsep
%     \let\@ifnextchar\new@ifnextchar
%     \array{*\c@MaxMatrixCols #1}}


%\usepackage{framed}
%\usepackage{titletoc}
%\usepackage{etoolbox}
%\usepackage{lmodern}


%\patchcmd{\tableofcontents}{\contentsname}{\sffamily\contentsname}{}{}

%\renewenvironment{leftbar}
%{\def\FrameCommand{\hspace{6em}%
%		{\color{myyellow}\vrule width 2pt depth 6pt}\hspace{1em}}%
%	\MakeFramed{\parshape 1 0cm \dimexpr\textwidth-6em\relax\FrameRestore}\vskip2pt%
%}
%{\endMakeFramed}

%\titlecontents{chapter}
%[0em]{\vspace*{2\baselineskip}}
%{\parbox{4.5em}{%
%		\hfill\Huge\sffamily\bfseries\color{myred}\thecontentspage}%
%	\vspace*{-2.3\baselineskip}\leftbar\textsc{\small\chaptername~\thecontentslabel}\\\sffamily}
%{}{\endleftbar}
%\titlecontents{section}
%[8.4em]
%{\sffamily\contentslabel{3em}}{}{}
%{\hspace{0.5em}\nobreak\itshape\color{myred}\contentspage}
%\titlecontents{subsection}
%[8.4em]
%{\sffamily\contentslabel{3em}}{}{}  
%{\hspace{0.5em}\nobreak\itshape\color{myred}\contentspage}



%%%%%%%%%%%%%%%%%%%%%%%%%%%%%%%%%%%%%%%%%%%
% TABLE OF CONTENTS
%%%%%%%%%%%%%%%%%%%%%%%%%%%%%%%%%%%%%%%%%%%

\usepackage{tikz}
\definecolor{doc}{RGB}{0,60,110}
\usepackage{titletoc}
\contentsmargin{0cm}
\titlecontents{chapter}[3.7pc]
{\addvspace{30pt}%
	\begin{tikzpicture}[remember picture, overlay]%
		\draw[fill=doc!60,draw=doc!60] (-7,-.1) rectangle (-0.9,.5);%
		\pgftext[left,x=-3.5cm,y=0.2cm]{\color{white}\Large\sc\bfseries Chapter\ \thecontentslabel};%
	\end{tikzpicture}\color{doc!60}\large\sc\bfseries}%
{}
{}
{\;\titlerule\;\large\sc\bfseries Page \thecontentspage
	\begin{tikzpicture}[remember picture, overlay]
		\draw[fill=doc!60,draw=doc!60] (2pt,0) rectangle (4,0.1pt);
	\end{tikzpicture}}%
\titlecontents{section}[3.7pc]
{\addvspace{2pt}}
{\contentslabel[\thecontentslabel]{2pc}}
{}
{\hfill\small \thecontentspage}
[]
\titlecontents*{subsection}[3.7pc]
{\addvspace{-1pt}\small}
{}
{}
{\ --- \small\thecontentspage}
[ \textbullet\ ][]

\makeatletter
\renewcommand{\tableofcontents}{%
	\chapter*{%
	  \vspace*{-20\p@}%
	  \begin{tikzpicture}[remember picture, overlay]%
		  \pgftext[right,x=15cm,y=0.2cm]{\color{doc!60}\Huge\sc\bfseries \contentsname};%
		  \draw[fill=doc!60,draw=doc!60] (13,-.75) rectangle (20,1);%
		  \clip (13,-.75) rectangle (20,1);
		  \pgftext[right,x=15cm,y=0.2cm]{\color{white}\Huge\sc\bfseries \contentsname};%
	  \end{tikzpicture}}%
	\@starttoc{toc}}
\makeatother

%From M275 "Topology" at SJSU
\newcommand{\id}{\mathrm{id}}
\newcommand{\taking}[1]{\xrightarrow{#1}}
\newcommand{\inv}{^{-1}}

%From M170 "Introduction to Graph Theory" at SJSU
\DeclareMathOperator{\diam}{diam}
\DeclareMathOperator{\ord}{ord}
\newcommand{\defeq}{\overset{\mathrm{def}}{=}}

%From the USAMO .tex files
\newcommand{\ts}{\textsuperscript}
\newcommand{\dg}{^\circ}
\newcommand{\ii}{\item}

% % From Math 55 and Math 145 at Harvard
% \newenvironment{subproof}[1][Proof]{%
% \begin{proof}[#1] \renewcommand{\qedsymbol}{$\blacksquare$}}%
% {\end{proof}}

\newcommand{\liff}{\leftrightarrow}
\newcommand{\lthen}{\rightarrow}
\newcommand{\opname}{\operatorname}
\newcommand{\surjto}{\twoheadrightarrow}
\newcommand{\injto}{\hookrightarrow}
\newcommand{\On}{\mathrm{On}} % ordinals
% \newcommand{\EE}{\mathbb{E}} % Expectance
\DeclareMathOperator{\img}{im} % Image
\DeclareMathOperator{\Img}{Im} % Image
\DeclareMathOperator{\coker}{coker} % Cokernel
\DeclareMathOperator{\Coker}{Coker} % Cokernel
\DeclareMathOperator{\Ker}{Ker} % Kernel
\DeclareMathOperator{\rank}{rank}
\DeclareMathOperator{\Spec}{Spec} % spectrum
\DeclareMathOperator{\Tr}{Tr} % trace
\DeclareMathOperator{\pr}{pr} % projection
\DeclareMathOperator{\ext}{ext} % extension
\DeclareMathOperator{\pred}{pred} % predecessor
\DeclareMathOperator{\dom}{dom} % domain
\DeclareMathOperator{\ran}{ran} % range
\DeclareMathOperator{\Hom}{Hom} % homomorphism
\DeclareMathOperator{\Mor}{Mor} % morphisms
\DeclareMathOperator{\End}{End} % endomorphism
% \DeclareMathOperator{\Pr}{Pr} % probability
% \DeclareMathOperator{\Var}{Var} % variance

\newcommand{\eps}{\epsilon}
\newcommand{\veps}{\varepsilon}
\newcommand{\ol}{\overline}
\newcommand{\ul}{\underline}
\newcommand{\wt}{\widetilde}
\newcommand{\wh}{\widehat}
\newcommand{\vocab}[1]{\textbf{\color{blue} #1}}
\providecommand{\half}{\frac{1}{2}}
\newcommand{\dang}{\measuredangle} %% Directed angle
\newcommand{\ray}[1]{\overrightarrow{#1}}
\newcommand{\seg}[1]{\overline{#1}}
\newcommand{\arc}[1]{\wideparen{#1}}
\DeclareMathOperator{\cis}{cis}
\DeclareMathOperator*{\lcm}{lcm}
\DeclareMathOperator*{\argmin}{arg min}
\DeclareMathOperator*{\argmax}{arg max}
\newcommand{\cycsum}{\sum_{\mathrm{cyc}}}
\newcommand{\symsum}{\sum_{\mathrm{sym}}}
\newcommand{\cycprod}{\prod_{\mathrm{cyc}}}
\newcommand{\symprod}{\prod_{\mathrm{sym}}}
\newcommand{\Qed}{\begin{flushright}\qed\end{flushright}}
\newcommand{\parinn}{\setlength{\parindent}{1cm}}
\newcommand{\parinf}{\setlength{\parindent}{0cm}}
% \newcommand{\norm}{\|\cdot\|}
\newcommand{\inorm}{\norm_{\infty}}
\newcommand{\opensets}{\{V_{\alpha}\}_{\alpha\in I}}
\newcommand{\oset}{V_{\alpha}}
\newcommand{\opset}[1]{V_{\alpha_{#1}}}
\newcommand{\lub}{\text{lub}}
\newcommand{\del}[2]{\frac{\partial #1}{\partial #2}}
\newcommand{\Del}[3]{\frac{\partial^{#1} #2}{\partial^{#1} #3}}
\newcommand{\deld}[2]{\dfrac{\partial #1}{\partial #2}}
\newcommand{\Deld}[3]{\dfrac{\partial^{#1} #2}{\partial^{#1} #3}}
\newcommand{\lm}{\lambda}
\newcommand{\uin}{\mathbin{\rotatebox[origin=c]{90}{$\in$}}}
\newcommand{\usubset}{\mathbin{\rotatebox[origin=c]{90}{$\subset$}}}
\newcommand{\lt}{\left}
\newcommand{\rt}{\right}
\newcommand{\bs}[1]{\boldsymbol{#1}}
\newcommand{\exs}{\exists}
\newcommand{\st}{\strut}
\newcommand{\dps}[1]{\displaystyle{#1}}
\newcommand{\va}[1]{\vec{\bm{\mathrm{#1}}}}
\WithSuffix\newcommand\va*[1]{\vec{\bm{#1}}}
\newcommand{\vb}[1]{\bm{\mathrm{#1}}}
\WithSuffix\newcommand\vb*[1]{\bm{#1}}
\newcommand{\vu}[1]{\hat{\bm{\mathrm{#1}}}}
\WithSuffix\newcommand\vu*[1]{\hat{\bm{#1}}}
\renewcommand{\dd}[1]{\mathrm{d}#1}
\renewcommand{\Re}{\mathrm{Re}}
\renewcommand{\Im}{\mathrm{Im}}
\DeclareMathOperator{\tr}{tr}
\newcommand{\csin}[1]{\mintinline{csharp}|#1|}
\renewcommand{\Pr}[1]{\mathrm{Pr}\lt( #1 \rt)}
\renewcommand{\Var}[1]{\mathrm{Var}\lt( #1 \rt)}
\newcommand{\cov}[1]{\mathrm{cov}\lt( #1 \rt)}

\newcommand{\sol}{\setlength{\parindent}{0cm}\textbf{\textit{Solution:}}\setlength{\parindent}{1cm} }
\newcommand{\solve}[1]{\setlength{\parindent}{0cm}\textbf{\textit{Solution: }}\setlength{\parindent}{1cm}#1 \Qed}

% Things Lie
\newcommand{\kb}{\mathfrak b}
\newcommand{\kg}{\mathfrak g}
\newcommand{\kh}{\mathfrak h}
\newcommand{\kn}{\mathfrak n}
\newcommand{\ku}{\mathfrak u}
\newcommand{\kz}{\mathfrak z}
\DeclareMathOperator{\Ext}{Ext} % Ext functor
\DeclareMathOperator{\Tor}{Tor} % Tor functor
\newcommand{\gl}{\opname{\mathfrak{gl}}} % frak gl group
\renewcommand{\sl}{\opname{\mathfrak{sl}}} % frak sl group chktex 6

% More script letters etc.
\newcommand{\SA}{\mathcal A}
\newcommand{\SB}{\mathcal B}
\newcommand{\SC}{\mathcal C}
\newcommand{\SF}{\mathcal F}
\newcommand{\SG}{\mathcal G}
\newcommand{\SH}{\mathcal H}
\newcommand{\OO}{\mathcal O}

\newcommand{\SCA}{\mathscr A}
\newcommand{\SCB}{\mathscr B}
\newcommand{\SCC}{\mathscr C}
\newcommand{\SCD}{\mathscr D}
\newcommand{\SCE}{\mathscr E}
\newcommand{\SCF}{\mathscr F}
\newcommand{\SCG}{\mathscr G}
\newcommand{\SCH}{\mathscr H}

% Mathfrak primes
\newcommand{\km}{\mathfrak m}
\newcommand{\kp}{\mathfrak p}
\newcommand{\kq}{\mathfrak q}

% number sets
\newcommand{\RR}[1][]{\ensuremath{\ifstrempty{#1}{\mathbb{R}}{\mathbb{R}^{#1}}}}
\newcommand{\NN}[1][]{\ensuremath{\ifstrempty{#1}{\mathbb{N}}{\mathbb{N}^{#1}}}}
\newcommand{\ZZ}[1][]{\ensuremath{\ifstrempty{#1}{\mathbb{Z}}{\mathbb{Z}^{#1}}}}
\newcommand{\QQ}[1][]{\ensuremath{\ifstrempty{#1}{\mathbb{Q}}{\mathbb{Q}^{#1}}}}
\newcommand{\CC}[1][]{\ensuremath{\ifstrempty{#1}{\mathbb{C}}{\mathbb{C}^{#1}}}}
\newcommand{\PP}[1][]{\ensuremath{\ifstrempty{#1}{\mathbb{P}}{\mathbb{P}^{#1}}}}
\newcommand{\HH}[1][]{\ensuremath{\ifstrempty{#1}{\mathbb{H}}{\mathbb{H}^{#1}}}}
\newcommand{\FF}[1][]{\ensuremath{\ifstrempty{#1}{\mathbb{F}}{\mathbb{F}^{#1}}}}
% expected value
\newcommand{\EE}{\ensuremath{\mathbb{E}}}
\newcommand{\charin}{\text{ char }}
\DeclareMathOperator{\sign}{sign}
\DeclareMathOperator{\Aut}{Aut}
\DeclareMathOperator{\Inn}{Inn}
\DeclareMathOperator{\Syl}{Syl}
\DeclareMathOperator{\Gal}{Gal}
\DeclareMathOperator{\GL}{GL} % General linear group
\DeclareMathOperator{\SL}{SL} % Special linear group

%---------------------------------------
% BlackBoard Math Fonts :-
%---------------------------------------

%Captital Letters
\newcommand{\bbA}{\mathbb{A}}	\newcommand{\bbB}{\mathbb{B}}
\newcommand{\bbC}{\mathbb{C}}	\newcommand{\bbD}{\mathbb{D}}
\newcommand{\bbE}{\mathbb{E}}	\newcommand{\bbF}{\mathbb{F}}
\newcommand{\bbG}{\mathbb{G}}	\newcommand{\bbH}{\mathbb{H}}
\newcommand{\bbI}{\mathbb{I}}	\newcommand{\bbJ}{\mathbb{J}}
\newcommand{\bbK}{\mathbb{K}}	\newcommand{\bbL}{\mathbb{L}}
\newcommand{\bbM}{\mathbb{M}}	\newcommand{\bbN}{\mathbb{N}}
\newcommand{\bbO}{\mathbb{O}}	\newcommand{\bbP}{\mathbb{P}}
\newcommand{\bbQ}{\mathbb{Q}}	\newcommand{\bbR}{\mathbb{R}}
\newcommand{\bbS}{\mathbb{S}}	\newcommand{\bbT}{\mathbb{T}}
\newcommand{\bbU}{\mathbb{U}}	\newcommand{\bbV}{\mathbb{V}}
\newcommand{\bbW}{\mathbb{W}}	\newcommand{\bbX}{\mathbb{X}}
\newcommand{\bbY}{\mathbb{Y}}	\newcommand{\bbZ}{\mathbb{Z}}

%---------------------------------------
% MathCal Fonts :-
%---------------------------------------

%Captital Letters
\newcommand{\mcA}{\mathcal{A}}	\newcommand{\mcB}{\mathcal{B}}
\newcommand{\mcC}{\mathcal{C}}	\newcommand{\mcD}{\mathcal{D}}
\newcommand{\mcE}{\mathcal{E}}	\newcommand{\mcF}{\mathcal{F}}
\newcommand{\mcG}{\mathcal{G}}	\newcommand{\mcH}{\mathcal{H}}
\newcommand{\mcI}{\mathcal{I}}	\newcommand{\mcJ}{\mathcal{J}}
\newcommand{\mcK}{\mathcal{K}}	\newcommand{\mcL}{\mathcal{L}}
\newcommand{\mcM}{\mathcal{M}}	\newcommand{\mcN}{\mathcal{N}}
\newcommand{\mcO}{\mathcal{O}}	\newcommand{\mcP}{\mathcal{P}}
\newcommand{\mcQ}{\mathcal{Q}}	\newcommand{\mcR}{\mathcal{R}}
\newcommand{\mcS}{\mathcal{S}}	\newcommand{\mcT}{\mathcal{T}}
\newcommand{\mcU}{\mathcal{U}}	\newcommand{\mcV}{\mathcal{V}}
\newcommand{\mcW}{\mathcal{W}}	\newcommand{\mcX}{\mathcal{X}}
\newcommand{\mcY}{\mathcal{Y}}	\newcommand{\mcZ}{\mathcal{Z}}


%---------------------------------------
% Bold Math Fonts :-
%---------------------------------------

%Captital Letters
\newcommand{\bmA}{\boldsymbol{A}}	\newcommand{\bmB}{\boldsymbol{B}}
\newcommand{\bmC}{\boldsymbol{C}}	\newcommand{\bmD}{\boldsymbol{D}}
\newcommand{\bmE}{\boldsymbol{E}}	\newcommand{\bmF}{\boldsymbol{F}}
\newcommand{\bmG}{\boldsymbol{G}}	\newcommand{\bmH}{\boldsymbol{H}}
\newcommand{\bmI}{\boldsymbol{I}}	\newcommand{\bmJ}{\boldsymbol{J}}
\newcommand{\bmK}{\boldsymbol{K}}	\newcommand{\bmL}{\boldsymbol{L}}
\newcommand{\bmM}{\boldsymbol{M}}	\newcommand{\bmN}{\boldsymbol{N}}
\newcommand{\bmO}{\boldsymbol{O}}	\newcommand{\bmP}{\boldsymbol{P}}
\newcommand{\bmQ}{\boldsymbol{Q}}	\newcommand{\bmR}{\boldsymbol{R}}
\newcommand{\bmS}{\boldsymbol{S}}	\newcommand{\bmT}{\boldsymbol{T}}
\newcommand{\bmU}{\boldsymbol{U}}	\newcommand{\bmV}{\boldsymbol{V}}
\newcommand{\bmW}{\boldsymbol{W}}	\newcommand{\bmX}{\boldsymbol{X}}
\newcommand{\bmY}{\boldsymbol{Y}}	\newcommand{\bmZ}{\boldsymbol{Z}}
%Small Letters
\newcommand{\bma}{\boldsymbol{a}}	\newcommand{\bmb}{\boldsymbol{b}}
\newcommand{\bmc}{\boldsymbol{c}}	\newcommand{\bmd}{\boldsymbol{d}}
\newcommand{\bme}{\boldsymbol{e}}	\newcommand{\bmf}{\boldsymbol{f}}
\newcommand{\bmg}{\boldsymbol{g}}	\newcommand{\bmh}{\boldsymbol{h}}
\newcommand{\bmi}{\boldsymbol{i}}	\newcommand{\bmj}{\boldsymbol{j}}
\newcommand{\bmk}{\boldsymbol{k}}	\newcommand{\bml}{\boldsymbol{l}}
\newcommand{\bmm}{\boldsymbol{m}}	\newcommand{\bmn}{\boldsymbol{n}}
\newcommand{\bmo}{\boldsymbol{o}}	\newcommand{\bmp}{\boldsymbol{p}}
\newcommand{\bmq}{\boldsymbol{q}}	\newcommand{\bmr}{\boldsymbol{r}}
\newcommand{\bms}{\boldsymbol{s}}	\newcommand{\bmt}{\boldsymbol{t}}
\newcommand{\bmu}{\boldsymbol{u}}	\newcommand{\bmv}{\boldsymbol{v}}
\newcommand{\bmw}{\boldsymbol{w}}	\newcommand{\bmx}{\boldsymbol{x}}
\newcommand{\bmy}{\boldsymbol{y}}	\newcommand{\bmz}{\boldsymbol{z}}

%---------------------------------------
% Scr Math Fonts :-
%---------------------------------------

\newcommand{\sA}{{\mathscr{A}}}   \newcommand{\sB}{{\mathscr{B}}}
\newcommand{\sC}{{\mathscr{C}}}   \newcommand{\sD}{{\mathscr{D}}}
\newcommand{\sE}{{\mathscr{E}}}   \newcommand{\sF}{{\mathscr{F}}}
\newcommand{\sG}{{\mathscr{G}}}   \newcommand{\sH}{{\mathscr{H}}}
\newcommand{\sI}{{\mathscr{I}}}   \newcommand{\sJ}{{\mathscr{J}}}
\newcommand{\sK}{{\mathscr{K}}}   \newcommand{\sL}{{\mathscr{L}}}
\newcommand{\sM}{{\mathscr{M}}}   \newcommand{\sN}{{\mathscr{N}}}
\newcommand{\sO}{{\mathscr{O}}}   \newcommand{\sP}{{\mathscr{P}}}
\newcommand{\sQ}{{\mathscr{Q}}}   \newcommand{\sR}{{\mathscr{R}}}
\newcommand{\sS}{{\mathscr{S}}}   \newcommand{\sT}{{\mathscr{T}}}
\newcommand{\sU}{{\mathscr{U}}}   \newcommand{\sV}{{\mathscr{V}}}
\newcommand{\sW}{{\mathscr{W}}}   \newcommand{\sX}{{\mathscr{X}}}
\newcommand{\sY}{{\mathscr{Y}}}   \newcommand{\sZ}{{\mathscr{Z}}}


%---------------------------------------
% Math Fraktur Font
%---------------------------------------

%Captital Letters
\newcommand{\mfA}{\mathfrak{A}}	\newcommand{\mfB}{\mathfrak{B}}
\newcommand{\mfC}{\mathfrak{C}}	\newcommand{\mfD}{\mathfrak{D}}
\newcommand{\mfE}{\mathfrak{E}}	\newcommand{\mfF}{\mathfrak{F}}
\newcommand{\mfG}{\mathfrak{G}}	\newcommand{\mfH}{\mathfrak{H}}
\newcommand{\mfI}{\mathfrak{I}}	\newcommand{\mfJ}{\mathfrak{J}}
\newcommand{\mfK}{\mathfrak{K}}	\newcommand{\mfL}{\mathfrak{L}}
\newcommand{\mfM}{\mathfrak{M}}	\newcommand{\mfN}{\mathfrak{N}}
\newcommand{\mfO}{\mathfrak{O}}	\newcommand{\mfP}{\mathfrak{P}}
\newcommand{\mfQ}{\mathfrak{Q}}	\newcommand{\mfR}{\mathfrak{R}}
\newcommand{\mfS}{\mathfrak{S}}	\newcommand{\mfT}{\mathfrak{T}}
\newcommand{\mfU}{\mathfrak{U}}	\newcommand{\mfV}{\mathfrak{V}}
\newcommand{\mfW}{\mathfrak{W}}	\newcommand{\mfX}{\mathfrak{X}}
\newcommand{\mfY}{\mathfrak{Y}}	\newcommand{\mfZ}{\mathfrak{Z}}
%Small Letters
\newcommand{\mfa}{\mathfrak{a}}	\newcommand{\mfb}{\mathfrak{b}}
\newcommand{\mfc}{\mathfrak{c}}	\newcommand{\mfd}{\mathfrak{d}}
\newcommand{\mfe}{\mathfrak{e}}	\newcommand{\mff}{\mathfrak{f}}
\newcommand{\mfg}{\mathfrak{g}}	\newcommand{\mfh}{\mathfrak{h}}
\newcommand{\mfi}{\mathfrak{i}}	\newcommand{\mfj}{\mathfrak{j}}
\newcommand{\mfk}{\mathfrak{k}}	\newcommand{\mfl}{\mathfrak{l}}
\newcommand{\mfm}{\mathfrak{m}}	\newcommand{\mfn}{\mathfrak{n}}
\newcommand{\mfo}{\mathfrak{o}}	\newcommand{\mfp}{\mathfrak{p}}
\newcommand{\mfq}{\mathfrak{q}}	\newcommand{\mfr}{\mathfrak{r}}
\newcommand{\mfs}{\mathfrak{s}}	\newcommand{\mft}{\mathfrak{t}}
\newcommand{\mfu}{\mathfrak{u}}	\newcommand{\mfv}{\mathfrak{v}}
\newcommand{\mfw}{\mathfrak{w}}	\newcommand{\mfx}{\mathfrak{x}}
\newcommand{\mfy}{\mathfrak{y}}	\newcommand{\mfz}{\mathfrak{z}}


\title{\Huge{Mechanics of Materials}\\Semester 4}
\author{}
\date{}

\begin{document}

\maketitle
\newpage% or \cleardoublepage
% \pdfbookmark[<level>]{<title>}{<dest>}
\pdfbookmark[section]{\contentsname}{toc}
\tableofcontents
\pagebreak

\chapter{Mathematical Concepts}
\section{Tensors}
\dfn{Einstein Notation}{
	Also known as summation notation, says that if we have a repeated index then we are summing over that index. For example
	\[
		y = c_i \vu{e}_i
		.\]
	implies that
	\[
		y = \sum_{i=1}^{3} c_i \vu{e}_i = c_1\vu{e}_1 + c_2\vu{e}_2 + c_3\vu{e}_3
		.\]
	same thing with
	\[
		a_i\cdot b_i = a_1\cdot b_1 + a_2\cdot b_2 + a_3\cdot b_3
		.\]
}



\dfn{}{
	Kronecker delta is defined to be
	\[
		\delta_{ij} = \begin{cases}
			0 & \text{if }i \neq j \\
			1 & \text{if }i = j
		\end{cases}
		.\]
	and the permutation symbol
	\[
		\varepsilon _{ijk}=\begin{cases}
			+1           & {\text{if }}(i,j,k){\text{ is }}(1,2,3),(2,3,1),{\text{ or }}(3,1,2), \\
			-1           & {\text{if }}(i,j,k){\text{ is }}(3,2,1),(1,3,2),{\text{ or }}(2,1,3), \\
			\phantom{+}0 & {\text{if }}i=j,{\text{ or }}j=k,{\text{ or }}k=i
		\end{cases}
		.\]
	And they appear in
	\begin{align*}
		\vu{e}_i \cdot \vu{e}_j  & = \delta_{ij}               \\
		\vu{e}_i \times \vu{e}_j & = \varepsilon_{ijk}\vu{e}_k
	\end{align*}
}

\dfn{Tensors}{
	In an $m$-dimensional space, a tensor of rank $n$ is a mathematical object that has $n$ indices, $m^n$ components, and obeys certain \emph{transformation rules}\\
}
\nt{
	Typically $m=3$ corresponding to the 3D space.
}
\ex{}{
	\begin{itemize}
		\item A rank 0 tensor is a scalar
		      \[
			      A
			      .\]
		\item A rank 1 tensor is a vector
		      \[
			      A \vu{x} = A_i x_i = A_1 x_1 + A_2 x_2 + A_3 x_3 = \begin{bmatrix}
				      A_1 \\A_2\\A_3
			      \end{bmatrix}
			      .\]
		\item A rank 2 tensor is a matrix
		      \[
			      A (\vu{x},\vu{y}) = A_{ij} x_i y_j = \begin{bmatrix}
				      A_{11} & A_{12} & A_{13} \\
				      A_{21} & A_{22} & A_{23} \\
				      A_{31} & A_{32} & A_{33}
			      \end{bmatrix}
			      .\]
	\end{itemize}
}
Some notable tensors are:
\begin{enumerate}
	\item Symmetric tensors
	      \[
		      A_{ij} = A_{ji}
		      .\]
	\item Anti-symmetric tensors
	      \[
		      A_{ij} = -A_{ji}
		      .\]
	\item General tensor. It can be represented using a symmetric and an anti symmetric tensor
	      \[
		      A = A^S + A^A
		      .\]
	      where
	      \begin{align*}
		      A^S & = \frac{1}{2}(A+A^T)      \\
		      A^A & = \frac{1}{2}(A\cdot A^T)
	      \end{align*}
\end{enumerate}

The identity tensor is the tensor whose components $I_{ij} = \delta_{ij}$
\[
	I = \begin{bmatrix}
		1 & 0 & 0 \\
		0 & 1 & 0 \\
		0 & 0 & 1
	\end{bmatrix}
	.\]

The scalar invariants of a tensor
\begin{enumerate}
	\ii $I_1 = \tr(A) = A_{ii} = A_{11} + A_{22} + A_{33}$
	\ii $I_2 = \frac{1}{2} \lt[\tr(A)^2 - tr(A^2)\rt] = \frac{1}{2} \lt(A_{ii}A_{jj} - A_{ij}A_{ji}\rt)$
	\ii $I_3 = \det(A) = \varepsilon_{iij}T_{i1}T_{j2}T_{k3}$
\end{enumerate}

The characteristic polynomial of a tensor $\det(A-\lambda I)$ can be expressed as
\[
	\det(A-\lambda I) = -\lambda^3 + I_1 \lambda^2 - I_2 \lambda + I_3
	.\]

\dfn{Tensor Product}{
	We define the tensor product between 2 tensors $X$ and $Y$ of order 3 to be
	\[
		(X\otimes Y)_{ij} = X_iY_j
		.\]
	and with a tensor of order 2 $T$
	\[
		(T\otimes X)_{ij} = T_{ij}X_k
		.\]

}

\ex{Tensor Product}{
	\begin{align*}
		\begin{bmatrix}
			1 & \alpha \\\alpha^* & 1
		\end{bmatrix}
		\otimes
		\begin{bmatrix}
			1 & \beta \\\beta^* & 1
		\end{bmatrix} & =
		\begin{bmatrix}
			1
			\begin{bmatrix}
				1 & \beta \\\beta^* & 1
			\end{bmatrix}
			 &
			\alpha
			\begin{bmatrix}
				1 & \beta \\\beta^* & 1
			\end{bmatrix}
			\\
			\alpha^*
			\begin{bmatrix}
				1 & \beta \\\beta^* & 1
			\end{bmatrix}
			 &
			1 \begin{bmatrix}
				1 & \beta \\\beta^* & 1
			\end{bmatrix}
		\end{bmatrix}
		\\
		                           & =
		\begin{bmatrix}
			1               & \beta         & \alpha        & \alpha\beta \\
			\beta^*         & 1             & \alpha\beta^* & \alpha      \\
			\alpha^*        & \alpha^*\beta & 1             & \beta       \\
			\alpha^*\beta^* & \alpha^*      & \beta^*       & 1           \\
		\end{bmatrix}
	\end{align*}
}

\url{https://www.math3ma.com/blog/the-tensor-product-demystified}

\nt{
	The order of a tensor product $X\otimes Y$ is the sum of the orders of $X$ and $Y$.
}
\dfn{Contraction}{
	We define the tensor product between 2 tensors $X$ and $Y$ to be
	\[
		X\cdot Y = X_iY_j
		.\]
}

\nt{
	From what I understand, a tensor product is the outer product and a contraction is an inner product
	\begin{align*}
		X\otimes Y & = X \times Y^T \\
		X\cdot Y   & = X^T \times Y
	\end{align*}
}

\section{Tensor Calculus}

\dfn{Gradient operator}{
	The gradient operator on a scalar tensor is defined to be
	\[
		\nabla f = \pdv{f}{x_i}\vu{e}_i = \pdv{f}{x_1}\vu{e}_1 + \pdv{f}{x_2}\vu{e}_2 + \pdv{f}{x_3}\vu{e}_3
		.\]
	in cylindrical coordinates
	\[
		\nabla f = \pdv{f}{r}\vu{e}_r + \frac{1}{r}\pdv{f}{\theta}\vu{e}_\theta + \pdv{f}{z}\vu{e}_z
		.\]
}

\dfn{Gradient of a vector}{
	The gradient of a vector tensor is
	\[
		\nabla \va{a} = \begin{bmatrix}
			\pdv{a_1}{x_1} & \pdv{a_1}{x_2} & \pdv{a_1}{x_3} \\
			\pdv{a_2}{x_1} & \pdv{a_2}{x_2} & \pdv{a_2}{x_3} \\
			\pdv{a_3}{x_1} & \pdv{a_3}{x_2} & \pdv{a_3}{x_3}
		\end{bmatrix}
		.\]
	in cylindrical coordinates
	\[
		\nabla \va{a} = \begin{bmatrix}
			\pdv{a_r}{r}      & \frac{1}{r}\lt(\pdv{a_r}{\theta}-a_\theta\rt) & \pdv{a_r}{z}      \\
			\pdv{a_\theta}{r} & \frac{1}{r}\lt(\pdv{a_\theta}{\theta}+a_r\rt) & \pdv{a_\theta}{z} \\
			\pdv{a_z}{r}      & \frac{1}{r}\pdv{a_z}{\theta}                  & \pdv{a_z}{z}
		\end{bmatrix}
		.\]
}
\nt{
	The order of a gradient tensor is 1 order higher than the tensor it operates on.
}

\dfn{Divergence}{
	The divergence is defined to be
	\[
		\nabla\cdot \va{a} = \tr(\nabla \va{a})
		.\]
	unlike the gradient, it reduces the order of the tensor.
}
\dfn{Laplacian}{
	The Laplacian is is the composition of a divergence and a gradient. It keeps the same order of the tensor
	\[
		\Delta f = \nabla\cdot(\nabla f) =\pdv[2]{f}{x_1} + \pdv[2]{f}{x_2} + \pdv[2]{f}{x_3}
		.\]
}

\dfn{Rotation}{
	Rotation mostly applies to vector tensors and retains the same order as it
	\[
		(\nabla\times \va{a})_i = \varepsilon_{ijk}\pdv{a_k}{x_j}
		.\]
	\[
		\nabla \times \va{a} =
		\lt(\pdv{a_3}{x_2} - \pdv{a_2}{x_3}\rt)\vu{e}_1 +
		\lt(\pdv{a_1}{x_3} - \pdv{a_3}{x_1}\rt)\vu{e}_2 +
		\lt(\pdv{a_2}{x_1} - \pdv{a_1}{x_2}\rt)\vu{e}_3
		.\]
	in cylindrical coordinates
	\[
		\nabla \times \va{a} =
		\lt(\frac{1}{r}\pdv{a_z}{\theta} - \pdv{a_\theta}{z}\rt)\vu{e}_r +
		\lt(\pdv{a_r}{r} - \pdv{a_z}{x_r}\rt)\vu{e}_\theta +
		\lt(\pdv{a_\theta}{r} - \frac{1}{r}\pdv{a_r}{\theta}+\frac{a_\theta}{r}\rt)\vu{e}_z
		.\]
}

\nt{
	\begin{align*}
		\nabla\times (\nabla \va{a})      & =0                          \\
		\nabla \cdot(\nabla\times \va{a}) & =0                          \\
		\nabla (ab)                       & = a(\nabla b) + b(\nabla a)
	\end{align*}
}

\thm{Ostrogradsky's theorem}{
	Denote $\iiint_D \dots\dd{V}$ as a volume integral and $\iint_S \dots\vu{n}\dd{S}$ as a surface integral.
	\begin{align*}
		\iiint_D \nabla f \dd{V}           & = \iint_S f\vu{n}\dd{S}      \\
		\iiint_D \nabla\cdot \va{U} \dd{V} & = \iint_S \va{U}\vu{n}\dd{S} \\
		\iiint_D \nabla \cdot T \dd{V}     & = \iint_S T\vu{n}\dd{S}      \\
	\end{align*} % TODO make all 2nd order tensors \vb{}
}

\chapter{Deformation}

We consider a body under some deformation, at time $t=0$, a point $P$ on that body can be described as
\[
	\va{X} = X_k\vu{e}_k
	.\]

after some time $t$ the object has deformed and the position of the point $P$ is now $\va{x}$. The relation between it's initial position and it's new position is
\[
	\va{x} = \va*{\Phi}(\va{X},t)
	.\]
where $\va*{\Phi}$ is a bijective transformation($\forall \va*{\Phi}, \exists \va*{\Phi}^{-1}$). The vector $\va{x}$ is a function of the initial position and time.

The displacement vector is
\[
	\va{u}\lt(\va{X},t\rt) = \va{x}-\va{X}
	.\]
velocity vector
\[
	\va{v}\lt(\va{X},t\rt) = \pdv{\va{x}}{t}
	.\]
and acceleration vector
\[
	\va{a} = \pdv{\va{v}}{t}
	.\]

We consider a point $P$ on a body and 2 points on the same body $Q_1$ and $Q_2$ described with respect to the point $P$. The differentials of $Q_1$ and $Q_2$ are
\begin{align*}
	\dd{\va{X}_1} & = \va{X}_{Q_1} - \va{X}_P \\
	\dd{\va{X}_2} & = \va{X}_{Q_2} - \va{X}_P
\end{align*}
and after the deformation
\begin{align*}
	\dd{\va{x}_1} & = \va*{\Phi}\lt(\va{X}_P+\dd{\va{X}_1},t\rt) - \va*{\Phi}\lt(\va{X}_P,t\rt) \\
	\dd{\va{x}_2} & = \va*{\Phi}\lt(\va{X}_P+\dd{\va{X}_2},t\rt) - \va*{\Phi}\lt(\va{X}_P,t\rt) \\
\end{align*}

we define a differential tensor of the transformation
\[
	\vb{F}\lt(\va{X},t\rt) = \pdv{\va*{\Phi}}{\va{X}}
	.\]

aka the Jacobian matrix
\[
	\vb{F} = \begin{bmatrix}
		\pdv{x_1}{X_1} & \pdv{x_1}{X_2} & \pdv{x_1}{X_3} \\
		\pdv{x_2}{X_1} & \pdv{x_2}{X_2} & \pdv{x_2}{X_3} \\
		\pdv{x_3}{X_1} & \pdv{x_3}{X_2} & \pdv{x_3}{X_3}
	\end{bmatrix}
	.\]

The differential can be written as
\begin{align*}
	\dd{\va{x}_1} & =\vb{F}\lt(\va{X}_P,t\rt)\cdot\dd{\va{X}_1} \\
	\dd{\va{x}_2} & =\vb{F}\lt(\va{X}_P,t\rt)\cdot\dd{\va{X}_2} \\
\end{align*}

The Jacobian is also useful for a change of reference when integrating
\[
	\int_v a(\va{x})\dd{v} = \int_V a\lt(\va{x}\lt(\va{X},t\rt)\rt)\det(\vb{F})\dd{V}
	.\]

The relation between vectors before and after deformation
\[
	\dd{\va{x}_1}\cdot\dd{\va{x}_2} = \dd{\va{X}_1}\cdot\vb{C}\cdot\dd{\va{X}_2}
	.\]
where $\vb{C}$ is the Cauchy–Green deformation tensor
\[
	\vb{C} = \vb{F}^T \cdot \vb{F}
	.\]

The elongation after deformation in a given direction
\[
	\delta(\dd{\va{X}})=\dv{\va{x}}{\va{X}}-1 = \dv{\sqrt{\dd{\va{X}}\cdot\vb{C}\cdot\dd{\va{X}}}}{\va{X}} - 1
	.\]

We define
\[
	\lambda = \dv{\sqrt{\dd{\va{X}}\cdot\vb{C}\cdot\dd{\va{X}}}}{\va{X}} = \delta + 1
	.\]

\[
	\delta \begin{cases}
		>0 & \text{elongation in the direction of }\dd{\va{x}}  \\
		<0 & \text{contraction in the direction of }\dd{\va{x}}
	\end{cases}
	.\]

Consider 2 orthogonal vectors, $X_1$ and $X_2$. The new angle formed $\alpha = \frac{\pi}{2}-\gamma$ is calculated using the formula
\[
	\sin(\gamma) = \frac{\dd{\va{X}}_1\cdot\vb{C}\cdot\dd{\va{X}_2}}{\sqrt{\dd{\va{X}}_1\cdot\vb{C}\cdot\dd{\va{X}_1}}\cdot\sqrt{\dd{\va{X}}_2\cdot\vb{C}\cdot\dd{\va{X}_2}}}
	.\]

We define the Green-Lagrangian strain tensor to be
\[
	\vb{E}=\frac{1}{2}(\vb{C}-\vb{I}) = \frac{1}{2}(\vb{F}^T\vb{F}-\vb{I})
	.\]

The diagonal elements of $\vb{E}$ represent scaling of basis vectors of the body, while non-diagonal elements represent the change in angle (rotation).\\

We define $\dd{L}$ to the magnitude of $\dd{\va{X}}$ and $\vu{N}$ to be its direction vector.
\[
	\dd{\va{X}} = \dd{L}\vu{N}
	.\]
similarly for $\dd{\va{x}}$
\[
	\dd{\va{x}} = \dd{l}\vu{n}
	.\]

it follows that
\[
	\frac{1}{2}\lt(\frac{\dd{l}^2-\dd{L}^2}{\dd{L}^2}\rt) = \vu{N}\cdot\vb{E}\cdot\vu{N}
	.\]

and the angle between the 2 transformed vectors becomes $\alpha = \frac{\pi}{2}-\gamma$
\[
	\frac{1}{2}\sin(\gamma)\dv{l_1}{L_1}\dv{l_2}{L_2}=\vu{N}_1\cdot \vb{C}\cdot\vu{N}_2
	.\]

We can decompose the gradient tensor $\vb{F}$ in to 2 other tensors where $\vb{R}$ is an orthogonal matrix ($\vb{R}^T = \vb{R}^{-1}$) and $\vb{U}$ is a symmetric matrix
\[
	\vb{F}=\vb{R}\cdot\vb{U}
	.\]

\[
	\vb{C} = \vb{U}\cdot\vb{U}
	.\]
and
\[
	\vb{E} = \frac{1}{2}(\vb{U}^2-\vb{I})
	.\]

in a small displacement
\[
	\vb{E} = \frac{1}{2}\lt(\pdv{\va{u}}{\va{X}} + \lt(\pdv{\va{u}}{\va{X}}\rt)^T + \lt(\pdv{\va{u}}{\va{X}}\rt)^T\cdot \pdv{\va{u}}{\va{X}}\rt)
	.\]

We can ignore the quadratic terms to obtain the strain tensor for small displacement $\vb*{\varepsilon}$
\[
	\vb*{\varepsilon} \approx \frac{1}{2}\lt(\pdv{\va{u}}{\va{X}} + \lt(\pdv{\va{u}}{\va{X}}\rt)^T\rt)
	.\]
\[
	\varepsilon_{ij} = \frac{1}{2}\lt(\pdv{u_i}{X_j} + \pdv{u_j}{X_i}\rt)
	.\]
Using the above definition we can explicitly define the matrix
\[
	\vb{\varepsilon} = \begin{bmatrix}
		\pdv{u_1}{X_1}                                     & \frac{1}{2}\lt(\pdv{u_1}{X_2} + \pdv{u_2}{X_1}\rt) & \frac{1}{2}\lt(\pdv{u_1}{X_3} + \pdv{u_3}{X_1}\rt) \\
		\frac{1}{2}\lt(\pdv{u_1}{X_2} + \pdv{u_2}{X_1}\rt) & \pdv{u_2}{X_2}                                     & \frac{1}{2}\lt(\pdv{u_2}{X_3} + \pdv{u_3}{X_2}\rt) \\
		\frac{1}{2}\lt(\pdv{u_1}{X_3} + \pdv{u_3}{X_1}\rt) & \frac{1}{2}\lt(\pdv{u_2}{X_3} + \pdv{u_3}{X_2}\rt) & \pdv{u_3}{X_3}
	\end{bmatrix}
	.\]

We can also prove that $\gamma$ between the 2 base vectors $\vu{e}_1$ and $\vu{e}_2$ is
\[
	\frac{\gamma}{2} = \varepsilon_{12}
	.\]
Plane strain in a displacement plane $\va{u}=(u,v)$ of a body that has a unit thickness $\dd{x}$ and $\dd{y}$. The Jacobian becomes
\[
	\vb{F} = \begin{bmatrix}
		\pdv{u}{x} & \pdv{u}{y} & 0 \\
		\pdv{v}{x} & \pdv{v}{y} & 0 \\
		0          & 0          & 0
	\end{bmatrix}
	+ \begin{bmatrix}
		1 & 0 & 0 \\
		0 & 1 & 0 \\
		0 & 0 & 1
	\end{bmatrix}
	.\]

\begin{align*}
	\dd{x'} & =\vb{F}\begin{bmatrix}
		\dd{x} \\0
	\end{bmatrix}=\begin{bmatrix}
		\pdv{u}{x}\dd{x}+\dd{x} \\
		\pdv{v}{x}\dd{x}
	\end{bmatrix} \\
	\dd{y'} & =\vb{F}\begin{bmatrix}
		0 \\\dd{y}
	\end{bmatrix}=\begin{bmatrix}
		\pdv{u}{y}\dd{y} \\
		\pdv{v}{y}\dd{y}+\dd{y}
	\end{bmatrix}
\end{align*}

The strain tensor becomes
\[
	\vb{\varepsilon} = \begin{bmatrix}
		\varepsilon_{xx}       & \frac{1}{2}\gamma_{xy} \\
		\frac{1}{2}\gamma_{xy} & \varepsilon_{yy}
	\end{bmatrix}
	.\]

where
\begin{align*}
	\varepsilon_{xx} & = \pdv{u}{x}              \\
	\varepsilon_{yy} & = \pdv{v}{y}              \\
	\gamma_{xy}      & = \pdv{u}{y} + \pdv{v}{x}
\end{align*}

We can scale it up to 3D ($\dd{x},\dd{y},\dd{z}$ and $\va{u}=(u,v,w)$)

\begin{align*}
	\varepsilon_{xx} & =\pdv{u}{x} & \gamma_{xy} & =\pdv{u}{y} + \pdv{v}{x} \\
	\varepsilon_{yy} & =\pdv{v}{y} & \gamma_{yz} & =\pdv{w}{y} + \pdv{v}{z} \\
	\varepsilon_{zz} & =\pdv{w}{z} & \gamma_{xz} & =\pdv{w}{x} + \pdv{u}{z} \\
\end{align*}

\[
	\vb{\varepsilon} = \begin{bmatrix}
		\varepsilon_{xx}       & \frac{1}{2}\gamma_{xy} & \frac{1}{2}\gamma_{xz} \\
		\frac{1}{2}\gamma_{xy} & \varepsilon_{yy}       & \frac{1}{2}\gamma_{yz} \\
		\frac{1}{2}\gamma_{xz} & \frac{1}{2}\gamma_{yz} & \varepsilon_{zz}
	\end{bmatrix}
	.\]

Consider a small displacement, to study the change in volume we need to look at the Jacobian. To do that we define the tensor
\[
	\vb{H} = \pdv{\va{u}}{\va{X}}
	.\]

such that
\[
	\vb{F} = \vb{H} + \vb{I}
	.\]

The Jacobian becomes
\[
	J = \det\vb{F} = 1 + \tr\vb{H} = 1 + \tr\vb{\varepsilon}
	.\]

For a tensor to be considered a stress tensor it has to satisfy the 6 compatibility equations

\begin{align*}
	\pdv[2]{\varepsilon_{11}}{X_2} + \pdv[2]{\varepsilon_{22}}{X_1} & = 2\pdv{\varepsilon_{12}}{X_1}{X_2}                                                                                                       & \pdv{\varepsilon_{11}}{X_2}{X_3} + \pdv{}{X_1}\lt(\pdv{\varepsilon_{23}}{X_1}-\pdv{\varepsilon_{31}}{X_2}-\pdv{\varepsilon_{12}}{X_3}\rt) & =0 \\
	\pdv[2]{\varepsilon_{22}}{X_2} + \pdv[2]{\varepsilon_{33}}{X_2} & = 2\pdv{\varepsilon_{23}}{X_2}{X_3}                                                                                                       & \pdv{\varepsilon_{22}}{X_3}{X_1} + \pdv{}{X_2}\lt(\pdv{\varepsilon_{31}}{X_2}-\pdv{\varepsilon_{12}}{X_3}-\pdv{\varepsilon_{23}}{X_1}\rt) & =0 \\
	\pdv[2]{\varepsilon_{33}}{X_1} + \pdv[2]{\varepsilon_{11}}{X_3} & = 2\pdv{\varepsilon_{31}}{X_3}{X_1}
	                                                                & \pdv{\varepsilon_{33}}{X_1}{X_2} + \pdv{}{X_3}\lt(\pdv{\varepsilon_{12}}{X_3}-\pdv{\varepsilon_{23}}{X_1}-\pdv{\varepsilon_{31}}{X_2}\rt) & =0
\end{align*}

\end{document}

\documentclass{report}

\input{../template/preamble}
\input{../template/macros}
\input{../template/letterfonts}

\title{\Huge{Some Class}\\Random Examples}
\author{\huge{Your Name}}
\date{}

\begin{document}

\maketitle
\newpage% or \cleardoublepage
% \pdfbookmark[<level>]{<title>}{<dest>}
\pdfbookmark[section]{\contentsname}{toc}
\tableofcontents
\pagebreak

\chapter{Mathematical Concepts}
\section{Tensors}
\dfn{Einstein Notation}{
	Also known as summation notation, says that if we have a repeated index then we are summing over that index. For example
	\[
		y = c_i \vu{e}_i
		.\]
	implies that
	\[
		y = \sum_{i=1}^{3} c_i \vu{e}_i = c_1\vu{e}_1 + c_2\vu{e}_2 + c_3\vu{e}_3
		.\]
	same thing with
	\[
		a_i\cdot b_i = a_1\cdot b_1 + a_2\cdot b_2 + a_3\cdot b_3
		.\]
}

\dfn{}{
	Kronecker delta is defined to be
	\[
		\delta_{ij} = \begin{cases}
			0 & \text{if }i \neq j \\
			1 & \text{if }i = j
		\end{cases}
		.\]
	and the permutation symbol
	\[
		\varepsilon _{ijk}=\begin{cases}
			+1           & {\text{if }}(i,j,k){\text{ is }}(1,2,3),(2,3,1),{\text{ or }}(3,1,2), \\
			-1           & {\text{if }}(i,j,k){\text{ is }}(3,2,1),(1,3,2),{\text{ or }}(2,1,3), \\
			\phantom{+}0 & {\text{if }}i=j,{\text{ or }}j=k,{\text{ or }}k=i
		\end{cases}
		.\]
	And they appear in
	\begin{align*}
		\vu{e}_i \cdot \vu{e}_j  & = \delta_{ij}               \\
		\vu{e}_i \times \vu{e}_j & = \varepsilon_{ijk}\vu{e}_k
	\end{align*}
}

\dfn{Tensors}{
	In an $m$-dimensional space, a tensor of rank $n$ is a mathematical object that has $n$ indices, $m^n$ components, and obeys certain \emph{transformation rules}\\
}
\nt{
	Typically $m=3$ corresponding to the 3D space.
}
\ex{}{
	\begin{itemize}
		\item A rank 0 tensor is a scalar
		      \[
			      A
			      .\]
		\item A rank 1 tensor is a vector
		      \[
			      A \vu{x} = A_i x_i = A_1 x_1 + A_2 x_2 + A_3 x_3 = \begin{bmatrix}
				      A_1 \\A_2\\A_3
			      \end{bmatrix}
			      .\]
		\item A rank 2 tensor is a matrix
		      \[
			      A (\vu{x},\vu{y}) = A_{ij} x_i y_j = \begin{bmatrix}
				      A_{11} & A_{12} & A_{13} \\
				      A_{21} & A_{22} & A_{23} \\
				      A_{31} & A_{32} & A_{33}
			      \end{bmatrix}
			      .\]
	\end{itemize}
}
Some notable tensors are:
\begin{enumerate}
	\item Symmetric tensors
	      \[
		      A_{ij} = A_{ji}
		      .\]
	\item Anti-symmetric tensors
	      \[
		      A_{ij} = -A_{ji}
		      .\]
	\item General tensor. It can be represented using a symmetric and an anti symmetric tensor
	      \[
		      A = A^S + A^A
		      .\]
	      where
	      \begin{align*}
		      A^S & = \frac{1}{2}(A+A^T)      \\
		      A^A & = \frac{1}{2}(A\cdot A^T)
	      \end{align*}
\end{enumerate}

\end{document}

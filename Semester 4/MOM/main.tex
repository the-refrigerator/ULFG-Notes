\documentclass{report}

\input{../template/preamble}
\input{../template/macros}
\input{../template/letterfonts}

\title{\Huge{Mechanics of Materials}\\Semester 4}
\author{}
\date{}

\begin{document}

\maketitle
\newpage% or \cleardoublepage
% \pdfbookmark[<level>]{<title>}{<dest>}
\pdfbookmark[section]{\contentsname}{toc}
\tableofcontents
\pagebreak

\chapter{Mathematical Concepts}
\section{Tensors}
\dfn{Einstein Notation}{
	Also known as summation notation, says that if we have a repeated index then we are summing over that index. For example
	\[
		y = c_i \vu{e}_i
		.\]
	implies that
	\[
		y = \sum_{i=1}^{3} c_i \vu{e}_i = c_1\vu{e}_1 + c_2\vu{e}_2 + c_3\vu{e}_3
		.\]
	same thing with
	\[
		a_i\cdot b_i = a_1\cdot b_1 + a_2\cdot b_2 + a_3\cdot b_3
		.\]
}



\dfn{}{
	Kronecker delta is defined to be
	\[
		\delta_{ij} = \begin{cases}
			0 & \text{if }i \neq j \\
			1 & \text{if }i = j
		\end{cases}
		.\]
	and the permutation symbol
	\[
		\varepsilon _{ijk}=\begin{cases}
			+1           & {\text{if }}(i,j,k){\text{ is }}(1,2,3),(2,3,1),{\text{ or }}(3,1,2), \\
			-1           & {\text{if }}(i,j,k){\text{ is }}(3,2,1),(1,3,2),{\text{ or }}(2,1,3), \\
			\phantom{+}0 & {\text{if }}i=j,{\text{ or }}j=k,{\text{ or }}k=i
		\end{cases}
		.\]
	And they appear in
	\begin{align*}
		\vu{e}_i \cdot \vu{e}_j  & = \delta_{ij}               \\
		\vu{e}_i \times \vu{e}_j & = \varepsilon_{ijk}\vu{e}_k
	\end{align*}
}

\dfn{Tensors}{
	In an $m$-dimensional space, a tensor of rank $n$ is a mathematical object that has $n$ indices, $m^n$ components, and obeys certain \emph{transformation rules}\\
}
\nt{
	Typically $m=3$ corresponding to the 3D space.
}
\ex{}{
	\begin{itemize}
		\item A rank 0 tensor is a scalar
		      \[
			      A
			      .\]
		\item A rank 1 tensor is a vector
		      \[
			      A \vu{x} = A_i x_i = A_1 x_1 + A_2 x_2 + A_3 x_3 = \begin{bmatrix}
				      A_1 \\A_2\\A_3
			      \end{bmatrix}
			      .\]
		\item A rank 2 tensor is a matrix
		      \[
			      A (\vu{x},\vu{y}) = A_{ij} x_i y_j = \begin{bmatrix}
				      A_{11} & A_{12} & A_{13} \\
				      A_{21} & A_{22} & A_{23} \\
				      A_{31} & A_{32} & A_{33}
			      \end{bmatrix}
			      .\]
	\end{itemize}
}
Some notable tensors are:
\begin{enumerate}
	\item Symmetric tensors
	      \[
		      A_{ij} = A_{ji}
		      .\]
	\item Anti-symmetric tensors
	      \[
		      A_{ij} = -A_{ji}
		      .\]
	\item General tensor. It can be represented using a symmetric and an anti symmetric tensor
	      \[
		      A = A^S + A^A
		      .\]
	      where
	      \begin{align*}
		      A^S & = \frac{1}{2}(A+A^T)      \\
		      A^A & = \frac{1}{2}(A\cdot A^T)
	      \end{align*}
\end{enumerate}

The identity tensor is the tensor whose components $I_{ij} = \delta_{ij}$
\[
	I = \begin{bmatrix}
		1 & 0 & 0 \\
		0 & 1 & 0 \\
		0 & 0 & 1
	\end{bmatrix}
	.\]

The scalar invariants of a tensor
\begin{enumerate}
	\ii $I_1 = \tr(A) = A_{ii} = A_{11} + A_{22} + A_{33}$
	\ii $I_2 = \frac{1}{2} \lt[\tr(A)^2 - tr(A^2)\rt] = \frac{1}{2} \lt(A_{ii}A_{jj} - A_{ij}A_{ji}\rt)$
	\ii $I_3 = \det(A) = \varepsilon_{iij}T_{i1}T_{j2}T_{k3}$
\end{enumerate}

The characteristic polynomial of a tensor $\det(A-\lambda I)$ can be expressed as
\[
	\det(A-\lambda I) = -\lambda^3 + I_1 \lambda^2 - I_2 \lambda + I_3
	.\]

\dfn{Tensor Product}{
	We define the tensor product between 2 tensors $X$ and $Y$ of order 3 to be
	\[
		(X\otimes Y)_{ij} = X_iY_j
		.\]
	and with a tensor of order 2 $T$
	\[
		(T\otimes X)_{ij} = T_{ij}X_k
		.\]

}

\ex{Tensor Product}{
	\begin{align*}
		\begin{bmatrix}
			1 & \alpha \\\alpha^* & 1
		\end{bmatrix}
		\otimes
		\begin{bmatrix}
			1 & \beta \\\beta^* & 1
		\end{bmatrix} & =
		\begin{bmatrix}
			1
			\begin{bmatrix}
				1 & \beta \\\beta^* & 1
			\end{bmatrix}
			 &
			\alpha
			\begin{bmatrix}
				1 & \beta \\\beta^* & 1
			\end{bmatrix}
			\\
			\alpha^*
			\begin{bmatrix}
				1 & \beta \\\beta^* & 1
			\end{bmatrix}
			 &
			1 \begin{bmatrix}
				  1 & \beta \\\beta^* & 1
			  \end{bmatrix}
		\end{bmatrix}
		\\
		                        & =
		\begin{bmatrix}
			1               & \beta         & \alpha        & \alpha\beta \\
			\beta^*         & 1             & \alpha\beta^* & \alpha      \\
			\alpha^*        & \alpha^*\beta & 1             & \beta       \\
			\alpha^*\beta^* & \alpha^*      & \beta^*       & 1           \\
		\end{bmatrix}
	\end{align*}
}

\url{https://www.math3ma.com/blog/the-tensor-product-demystified}

\nt{
	The order of a tensor product $X\otimes Y$ is the sum of the orders of $X$ and $Y$.
}
\dfn{Contraction}{
	We define the tensor product between 2 tensors $X$ and $Y$ to be
	\[
		X\cdot Y = X_iY_j
		.\]
}

\nt{
	From what I understand, a tensor product is the outer product and a contraction is an inner product
	\begin{align*}
		X\otimes Y & = X \times Y^T \\
		X\cdot Y   & = X^T \times Y
	\end{align*}
}

\section{Tensor Calculus}

\dfn{Gradient operator}{
	The gradient operator on a scalar tensor is defined to be
	\[
		\nabla f = \pdv{f}{x_i}\vu{e}_i = \pdv{f}{x_1}\vu{e}_1 + \pdv{f}{x_2}\vu{e}_2 + \pdv{f}{x_3}\vu{e}_3
		.\]
	in cylindrical coordinates
	\[
		\nabla f = \pdv{f}{r}\vu{e}_r + \frac{1}{r}\pdv{f}{\theta}\vu{e}_\theta + \pdv{f}{z}\vu{e}_z
		.\]
}

\dfn{Gradient of a vector}{
	The gradient of a vector tensor is
	\[
		\nabla \va{a} = \begin{bmatrix}
			\pdv{a_1}{x_1} & \pdv{a_1}{x_2} & \pdv{a_1}{x_3} \\
			\pdv{a_2}{x_1} & \pdv{a_2}{x_2} & \pdv{a_2}{x_3} \\
			\pdv{a_3}{x_1} & \pdv{a_3}{x_2} & \pdv{a_3}{x_3}
		\end{bmatrix}
		.\]
	in cylindrical coordinates
	\[
		\nabla \va{a} = \begin{bmatrix}
			\pdv{a_r}{r}      & \frac{1}{r}\lt(\pdv{a_r}{\theta}-a_\theta\rt) & \pdv{a_r}{z}      \\
			\pdv{a_\theta}{r} & \frac{1}{r}\lt(\pdv{a_\theta}{\theta}+a_r\rt) & \pdv{a_\theta}{z} \\
			\pdv{a_z}{r}      & \frac{1}{r}\pdv{a_z}{\theta}                  & \pdv{a_z}{z}
		\end{bmatrix}
		.\]
}
\nt{
	The order of a gradient tensor is 1 order higher than the tensor it operates on.
}

\dfn{Divergence}{
	The divergence is defined to be
	\[
		\nabla\cdot \va{a} = \tr(\nabla \va{a})
		.\]
	unlike the gradient, it reduces the order of the tensor.
}
\dfn{Laplacian}{
	The Laplacian is is the composition of a divergence and a gradient. It keeps the same order of the tensor
	\[
		\Delta f = \nabla\cdot(\nabla f) =\pdv[2]{f}{x_1} + \pdv[2]{f}{x_2} + \pdv[2]{f}{x_3}
		.\]
}

\dfn{Rotation}{
	Rotation mostly applies to vector tensors and retains the same order as it
	\[
		(\nabla\times \va{a})_i = \varepsilon_{ijk}\pdv{a_k}{x_j}
		.\]
	\[
		\nabla \times \va{a} =
		\lt(\pdv{a_3}{x_2} - \pdv{a_2}{x_3}\rt)\vu{e}_1 +
		\lt(\pdv{a_1}{x_3} - \pdv{a_3}{x_1}\rt)\vu{e}_2 +
		\lt(\pdv{a_2}{x_1} - \pdv{a_1}{x_2}\rt)\vu{e}_3
		.\]
	in cylindrical coordinates
	\[
		\nabla \times \va{a} =
		\lt(\frac{1}{r}\pdv{a_z}{\theta} - \pdv{a_\theta}{z}\rt)\vu{e}_r +
		\lt(\pdv{a_r}{r} - \pdv{a_z}{x_r}\rt)\vu{e}_\theta +
		\lt(\pdv{a_\theta}{r} - \frac{1}{r}\pdv{a_r}{\theta}+\frac{a_\theta}{r}\rt)\vu{e}_z
		.\]
}

\nt{
	\begin{align*}
		\nabla\times (\nabla \va{a})      & =0                          \\
		\nabla \cdot(\nabla\times \va{a}) & =0                          \\
		\nabla (ab)                       & = a(\nabla b) + b(\nabla a)
	\end{align*}
}

\thm{Ostrogradsky's theorem}{
	Denote $\iiint_D \dots\dd{V}$ as a volume integral and $\iint_S \dots\vu{n}\dd{S}$ as a surface integral.
	\begin{align*}
		\iiint_D \nabla f \dd{V}           & = \iint_S f\vu{n}\dd{S}      \\
		\iiint_D \nabla\cdot \va{U} \dd{V} & = \iint_S \va{U}\vu{n}\dd{S} \\
		\iiint_D \nabla \cdot T \dd{V}     & = \iint_S T\vu{n}\dd{S}      \\
	\end{align*} % TODO make all 2nd order tensors \vb{}
}

\chapter{Deformation}

We consider a body under some deformation, at time $t=0$, a point $P$ on that body can be described as
\[
	\va{X} = X_k\vu{e}_k
	.\]

after some time $t$ the object has deformed and the position of the point $P$ is now $\va{x}$. The relation between it's initial position and it's new position is
\[
	\va{x} = \va*{\Phi}(\va{X},t)
	.\]
where $\va*{\Phi}$ is a bijective transformation($\forall \va*{\Phi}, \exists \va*{\Phi}^{-1}$). The vector $\va{x}$ is a function of the initial position and time.

The displacement vector is
\[
	\va{u}\lt(\va{X},t\rt) = \va{x}-\va{X}
	.\]
velocity vector
\[
	\va{v}\lt(\va{X},t\rt) = \pdv{\va{x}}{t}
	.\]
and acceleration vector
\[
	\va{a} = \pdv{\va{v}}{t}
	.\]

We consider a point $P$ on a body and 2 points on the same body $Q_1$ and $Q_2$ described with respect to the point $P$. The differentials of $Q_1$ and $Q_2$ are
\begin{align*}
	\dd{\va{X}_1} & = \va{X}_{Q_1} - \va{X}_P \\
	\dd{\va{X}_2} & = \va{X}_{Q_2} - \va{X}_P
\end{align*}
and after the deformation
\begin{align*}
	\dd{\va{x}_1} & = \va*{\Phi}\lt(\va{X}_P+\dd{\va{X}_1},t\rt) - \va*{\Phi}\lt(\va{X}_P,t\rt) \\
	\dd{\va{x}_2} & = \va*{\Phi}\lt(\va{X}_P+\dd{\va{X}_2},t\rt) - \va*{\Phi}\lt(\va{X}_P,t\rt) \\
\end{align*}

we define a differential tensor of the transformation
\[
	\vb{F}\lt(\va{X},t\rt) = \pdv{\va*{\Phi}}{\va{X}}
	.\]

aka the Jacobian matrix
\[
	\vb{F} = \begin{bmatrix}
		\pdv{x_1}{X_1} & \pdv{x_1}{X_2} & \pdv{x_1}{X_3} \\
		\pdv{x_2}{X_1} & \pdv{x_2}{X_2} & \pdv{x_2}{X_3} \\
		\pdv{x_3}{X_1} & \pdv{x_3}{X_2} & \pdv{x_3}{X_3}
	\end{bmatrix}
	.\]

The differential can be written as
\begin{align*}
	\dd{\va{x}_1} & =\vb{F}\lt(\va{X}_P,t\rt)\cdot\dd{\va{X}_1} \\
	\dd{\va{x}_2} & =\vb{F}\lt(\va{X}_P,t\rt)\cdot\dd{\va{X}_2} \\
\end{align*}

The Jacobian is also useful for a change of reference when integrating
\[
	\int_v a(\va{x})\dd{v} = \int_V a\lt(\va{x}\lt(\va{X},t\rt)\rt)\det(\vb{F})\dd{V}
	.\]

The relation between vectors before and after deformation
\[
	\dd{\va{x}_1}\cdot\dd{\va{x}_2} = \dd{\va{X}_1}\cdot\vb{C}\cdot\dd{\va{X}_2}
	.\]
where $\vb{C}$ is the Cauchy–Green deformation tensor
\[
	\vb{C} = \vb{F}^T \cdot \vb{F}
	.\]

The elongation after deformation in a given direction
\[
	\delta(\dd{\va{X}})=\dv{\va{x}}{\va{X}}-1 = \dv{\sqrt{\dd{\va{X}}\cdot\vb{C}\cdot\dd{\va{X}}}}{\va{X}} - 1
	.\]

We define
\[
	\lambda = \dv{\sqrt{\dd{\va{X}}\cdot\vb{C}\cdot\dd{\va{X}}}}{\va{X}} = \delta + 1
	.\]

\[
	\delta \begin{cases}
		>0 & \text{elongation in the direction of }\dd{\va{x}}  \\
		<0 & \text{contraction in the direction of }\dd{\va{x}}
	\end{cases}
	.\]

Consider 2 orthogonal vectors, $X_1$ and $X_2$. The new angle formed $\alpha = \frac{\pi}{2}-\gamma$ is calculated using the formula
\[
	\sin(\gamma) = \frac{\dd{\va{X}}_1\cdot\vb{C}\cdot\dd{\va{X}_2}}{\sqrt{\dd{\va{X}}_1\cdot\vb{C}\cdot\dd{\va{X}_1}}\cdot\sqrt{\dd{\va{X}}_2\cdot\vb{C}\cdot\dd{\va{X}_2}}}
	.\]

We define the Green-Lagrangian strain tensor to be
\[
	\vb{E}=\frac{1}{2}(\vb{C}-\vb{I}) = \frac{1}{2}(\vb{F}^T\vb{F}-\vb{I})
	.\]

The diagonal elements of $\vb{E}$ represent scaling of basis vectors of the body, while non-diagonal elements represent the change in angle (rotation).\\

We define $\dd{L}$ to the magnitude of $\dd{\va{X}}$ and $\vu{N}$ to be its direction vector.
\[
	\dd{\va{X}} = \dd{L}\vu{N}
	.\]
similarly for $\dd{\va{x}}$
\[
	\dd{\va{x}} = \dd{l}\vu{n}
	.\]

it follows that
\[
	\frac{1}{2}\lt(\frac{\dd{l}^2-\dd{L}^2}{\dd{L}^2}\rt) = \vu{N}\cdot\vb{E}\cdot\vu{N}
	.\]

and the angle between the 2 transformed vectors becomes $\alpha = \frac{\pi}{2}-\gamma$
\[
	\frac{1}{2}\sin(\gamma)\dv{l_1}{L_1}\dv{l_2}{L_2}=\vu{N}_1\cdot \vb{C}\cdot\vu{N}_2
	.\]

We can decompose the gradient tensor $\vb{F}$ in to 2 other tensors where $\vb{R}$ is an orthogonal matrix ($\vb{R}^T = \vb{R}^{-1}$) and $\vb{U}$ is a symmetric matrix
\[
	\vb{F}=\vb{R}\cdot\vb{U}
	.\]

\[
	\vb{C} = \vb{U}\cdot\vb{U}
	.\]
and
\[
	\vb{E} = \frac{1}{2}(\vb{U}^2-\vb{I})
	.\]

in a small displacement
\[
	\vb{E} = \frac{1}{2}\lt(\pdv{\va{u}}{\va{X}} + \lt(\pdv{\va{u}}{\va{X}}\rt)^T + \lt(\pdv{\va{u}}{\va{X}}\rt)^T\cdot \pdv{\va{u}}{\va{X}}\rt)
	.\]

We can ignore the quadratic terms to obtain the strain tensor for small displacement $\vb*{\varepsilon}$
\[
	\vb*{\varepsilon} \approx \frac{1}{2}\lt(\pdv{\va{u}}{\va{X}} + \lt(\pdv{\va{u}}{\va{X}}\rt)^T\rt)
	.\]
\[
	\varepsilon_{ij} = \frac{1}{2}\lt(\pdv{u_i}{X_j} + \pdv{u_j}{X_i}\rt)
	.\]
Using the above definition we can explicitly define the matrix
\[
	\vb{\varepsilon} = \begin{bmatrix}
		\pdv{u_1}{X_1}                                     & \frac{1}{2}\lt(\pdv{u_1}{X_2} + \pdv{u_2}{X_1}\rt) & \frac{1}{2}\lt(\pdv{u_1}{X_3} + \pdv{u_3}{X_1}\rt) \\
		\frac{1}{2}\lt(\pdv{u_1}{X_2} + \pdv{u_2}{X_1}\rt) & \pdv{u_2}{X_2}                                     & \frac{1}{2}\lt(\pdv{u_2}{X_3} + \pdv{u_3}{X_2}\rt) \\
		\frac{1}{2}\lt(\pdv{u_1}{X_3} + \pdv{u_3}{X_1}\rt) & \frac{1}{2}\lt(\pdv{u_2}{X_3} + \pdv{u_3}{X_2}\rt) & \pdv{u_3}{X_3}
	\end{bmatrix}
	.\]

We can also prove that $\gamma$ between the 2 base vectors $\vu{e}_1$ and $\vu{e}_2$ is
\[
	\frac{\gamma}{2} = \varepsilon_{12}
	.\]
Plane strain in a displacement plane $\va{u}=(u,v)$ of a body that has a unit thickness $\dd{x}$ and $\dd{y}$. The Jacobian becomes
\[
	\vb{F} = \begin{bmatrix}
		\pdv{u}{x} & \pdv{u}{y} & 0 \\
		\pdv{v}{x} & \pdv{v}{y} & 0 \\
		0          & 0          & 0
	\end{bmatrix}
	+ \begin{bmatrix}
		1 & 0 & 0 \\
		0 & 1 & 0 \\
		0 & 0 & 1
	\end{bmatrix}
	.\]

\begin{align*}
	\dd{x'} & =\vb{F}\begin{bmatrix}
		                 \dd{x} \\0
	                 \end{bmatrix}=\begin{bmatrix}
		                               \pdv{u}{x}\dd{x}+\dd{x} \\
		                               \pdv{v}{x}\dd{x}
	                               \end{bmatrix} \\
	\dd{y'} & =\vb{F}\begin{bmatrix}
		                 0 \\\dd{y}
	                 \end{bmatrix}=\begin{bmatrix}
		                               \pdv{u}{y}\dd{y} \\
		                               \pdv{v}{y}\dd{y}+\dd{y}
	                               \end{bmatrix}
\end{align*}

The strain tensor becomes
\[
	\vb{\varepsilon} = \begin{bmatrix}
		\varepsilon_{xx}       & \frac{1}{2}\gamma_{xy} \\
		\frac{1}{2}\gamma_{xy} & \varepsilon_{yy}
	\end{bmatrix}
	.\]

where
\begin{align*}
	\varepsilon_{xx} & = \pdv{u}{x}              \\
	\varepsilon_{yy} & = \pdv{v}{y}              \\
	\gamma_{xy}      & = \pdv{u}{y} + \pdv{v}{x}
\end{align*}

We can scale it up to 3D ($\dd{x},\dd{y},\dd{z}$ and $\va{u}=(u,v,w)$)

\begin{align*}
	\varepsilon_{xx} & =\pdv{u}{x} & \gamma_{xy} & =\pdv{u}{y} + \pdv{v}{x} \\
	\varepsilon_{yy} & =\pdv{v}{y} & \gamma_{yz} & =\pdv{w}{y} + \pdv{v}{z} \\
	\varepsilon_{zz} & =\pdv{w}{z} & \gamma_{xz} & =\pdv{w}{x} + \pdv{u}{z} \\
\end{align*}

\[
	\vb{\varepsilon} = \begin{bmatrix}
		\varepsilon_{xx}       & \frac{1}{2}\gamma_{xy} & \frac{1}{2}\gamma_{xz} \\
		\frac{1}{2}\gamma_{xy} & \varepsilon_{yy}       & \frac{1}{2}\gamma_{yz} \\
		\frac{1}{2}\gamma_{xz} & \frac{1}{2}\gamma_{yz} & \varepsilon_{zz}
	\end{bmatrix}
	.\]

Consider a small displacement, to study the change in volume we need to look at the Jacobian. To do that we define the tensor
\[
	\vb{H} = \pdv{\va{u}}{\va{X}}
	.\]

such that
\[
	\vb{F} = \vb{H} + \vb{I}
	.\]

The Jacobian becomes
\[
	J = \det\vb{F} = 1 + \tr\vb{H} = 1 + \tr\vb{\varepsilon}
	.\]

For a tensor to be considered a stress tensor it has to satisfy the 6 compatibility equations

\begin{align*}
	\pdv[2]{\varepsilon_{11}}{X_2} + \pdv[2]{\varepsilon_{22}}{X_1} & = 2\pdv{\varepsilon_{12}}{X_1}{X_2}                                                                                                       & \pdv{\varepsilon_{11}}{X_2}{X_3} + \pdv{}{X_1}\lt(\pdv{\varepsilon_{23}}{X_1}-\pdv{\varepsilon_{31}}{X_2}-\pdv{\varepsilon_{12}}{X_3}\rt) & =0 \\
	\pdv[2]{\varepsilon_{22}}{X_2} + \pdv[2]{\varepsilon_{33}}{X_2} & = 2\pdv{\varepsilon_{23}}{X_2}{X_3}                                                                                                       & \pdv{\varepsilon_{22}}{X_3}{X_1} + \pdv{}{X_2}\lt(\pdv{\varepsilon_{31}}{X_2}-\pdv{\varepsilon_{12}}{X_3}-\pdv{\varepsilon_{23}}{X_1}\rt) & =0 \\
	\pdv[2]{\varepsilon_{33}}{X_1} + \pdv[2]{\varepsilon_{11}}{X_3} & = 2\pdv{\varepsilon_{31}}{X_3}{X_1}
	                                                                & \pdv{\varepsilon_{33}}{X_1}{X_2} + \pdv{}{X_3}\lt(\pdv{\varepsilon_{12}}{X_3}-\pdv{\varepsilon_{23}}{X_1}-\pdv{\varepsilon_{31}}{X_2}\rt) & =0
\end{align*}

\chapter{Stress}

External forces can be classified as body forces and surface forces. Body forces are forces that act on the entire body, such as gravity. Surface forces are forces that act on the surface of the body, such as pressure. \\

A body is at equilibrium if the external forces resisted by internal forces.

\section{Stress Vectors}

\[
	\va{T} = \lim_{\dd{s}\to 0} \dv{\va{F}}{s}
	.\]

The stress vector is a vector that acts on a surface. It is defined as the force per unit area acting on a surface $\dd{s}$. It can be split in to 2 vectors
\[
	\va{T} = \va{T}_n + \va{T}_t
	.\]

where $\va{T}_n$ is the normal stress vector and $\va{T}_t$ is the shear stress vector. \\

To know the stress vector of a plane we can look at $T$ in 3 perpendicular directions. The stress vector is then

\[
	\sigma_{ij} = \va{T}_i \cdot \vu{e}_j
	.\]

where the index $i$ is the direction of the normal vector and $j$ is the direction of the stress vector.

\[
	\va{T} = \begin{pmatrix}
		\sigma_{11} \\
		\sigma_{12} \\
		\sigma_{13}
	\end{pmatrix}
	.\]

To find the stress at every plane we can use the stress tensor

\[
	\vb*{\sigma} = \begin{bmatrix}
		\sigma_{11} & \sigma_{21} & \sigma_{31} \\
		\sigma_{12} & \sigma_{22} & \sigma_{32} \\
		\sigma_{13} & \sigma_{23} & \sigma_{33}
	\end{bmatrix}
	.\]

\nt{The stress tensor is symmetric.}

The stress at a plane $P$ with normal vector $\vu{n}$ is

\[
	\va{T} = \vb*{\sigma} \cdot \vu{n}
	.\]

\begin{align*}
	\va{T}_n & = \vu{n} \cdot \vb*{\sigma} \cdot \vu{n} \\
	\va{T}_t & = \va{T} - \va{T}_n
\end{align*}

\section{Priciple Stresses}

The principle stresses are the maximum and minimum stresses at a point. They are the eigenvalues of the stress tensor. The eigenvectors are the directions of the principle stresses. \\

\[
	\vb{\sigma}_p = \begin{bmatrix}
		\sigma_{11} & 0           & 0           \\
		0           & \sigma_{22} & 0           \\
		0           & 0           & \sigma_{33}
	\end{bmatrix}
	.\]

\section{Plane stress}

Plane stress is when the stress in the $z$ direction is 0. This means that $\sigma_{13} = \sigma_{23} = \sigma_{33} = 0$.

\[
	\vb{\sigma} = \begin{bmatrix}
		\sigma_{11} & \sigma_{12} & 0 \\
		\sigma_{21} & \sigma_{22} & 0 \\
		0           & 0           & 0
	\end{bmatrix}
	.\]

Let $\va{T}$ be the stress vector acting on a plane with normal vector $\vu{n}$, such that $\vu{n}$ is making an angle $\theta$ with the horizontal.

\[
	\va{T} = \begin{pmatrix}
		T_1 \\
		T_2 \\
		0
	\end{pmatrix} = \begin{pmatrix}
		\sigma_{11}\cos\theta + \sigma_{12}\sin\theta \\
		\sigma_{12}\cos\theta + \sigma_{22}\sin\theta \\
		0
	\end{pmatrix}
	.\]

\begin{align*}
	\sigma_n  & = \frac{\sigma_{11}+\sigma_{22}}{2} + \frac{\sigma_{11}-\sigma_{22}}{2}\cos2\theta + \sigma_{12}\sin2\theta \\
	\tau_{nt} & = -\frac{\sigma_{11}-\sigma_{22}}{2}\sin2\theta + \sigma_{12}\cos2\theta
\end{align*}

\subsection{Mohr's Circle in Plane Stress}

Mohr's circle is a graphical method of finding the principle stresses. It's a circle with centr $(\sigma_{11}+\sigma_{22})/2$ and radius $\sqrt{\lt(\frac{\sigma_{11}-\sigma_{22}}{2}\rt)^2 + \sigma_{12}^2}$. The sign convention in Mohr circle space is that $\tau_{nt}$ is in the negative $y$ direction and $\sigma_n$ is in the positive $x$ direction, and an angle of $\theta$ in real space is an angle of $2\theta$ in Mohr circle space.

The principle stresses are the $x$ intercepts of the circle.

\[
	\sigma_{1,2} = \frac{\sigma_{11}+\sigma_{22}}{2} \pm \sqrt{\lt(\frac{\sigma_{11}-\sigma_{22}}{2}\rt)^2 + \tau_{xy}^2}
	.\]

\[
	\tau_\text{max} = \sqrt{\lt(\frac{\sigma_{11}-\sigma_{22}}{2}\rt)^2 + \tau_{xy}^2}
	.\]

The principle angles $\theta_{1,2}$ are

\[
	\tan2\theta_{1,2} = \frac{2\tau_{xy}}{\sigma_{11}-\sigma_{22}}
	.\]

\section{Mohr's Circle in 3D}

Mohr's circle in case of 3D stress consists of 3 circles. We calculate the 3 principle stresses $\sigma_{1,2,3}$ and the 3 principle directions of those stresses $n_{1,2,3}$.

\begin{align*}
	{n_1}^2 & = \frac{{\tau_n}^2 + (\sigma_n - \sigma_2)(\sigma_n - \sigma_3)}{(\sigma_1 - \sigma_2)(\sigma_1-\sigma_3)} \\
	{n_2}^2 & = \frac{{\tau_n}^2 + (\sigma_n - \sigma_1)(\sigma_n - \sigma_3)}{(\sigma_2 - \sigma_1)(\sigma_2-\sigma_3)} \\
	{n_3}^2 & = \frac{{\tau_n}^2 + (\sigma_n - \sigma_1)(\sigma_n - \sigma_2)}{(\sigma_3 - \sigma_1)(\sigma_3-\sigma_2)}
\end{align*}

Thus we obtain the 3 equations of the 3 circles

\begin{align*}
	C_1 = \lt(\frac{\sigma_1 + \sigma_2}{2},0\rt) \qquad R_1 = \frac{\sigma_1 - \sigma_2}{2} \\
	C_2 = \lt(\frac{\sigma_2 + \sigma_3}{2},0\rt) \qquad R_2 = \frac{\sigma_2 - \sigma_3}{2} \\
	C_3 = \lt(\frac{\sigma_3 + \sigma_1}{2},0\rt) \qquad R_3 = \frac{\sigma_1 - \sigma_3}{2}
\end{align*}

\section{Equation of Equilibrium}

\[
	\pdv{\sigma_{ij}}{x_j} + f_i = 0
	.\]

\[
	\nabla \cdot \vb*{\sigma} + \vb{f} = 0
	.\]

\chapter{Elasticity}

\section{Tensile Test}

Nominal stress
\[
	\sigma = \frac{F}{S_0}
	.\]

Nominal strain
\[
	\varepsilon = \frac{\Delta L}{L_0}
	.\]

Hooke's Law (in case $\sigma<R_e$ where $R_e$ is the yield strength)
\[
	\sigma = E \varepsilon
	.\]

Where $E$ is the Young's Modulus.\\

We define the Poisson's ratio in case of an isotropic material as

\begin{align*}
	\vartheta_x & = -\frac{\varepsilon_x}{\varepsilon_z} \\
	\vartheta_y & = -\frac{\varepsilon_y}{\varepsilon_z}
\end{align*}

\[
	\sigma_z = E \varepsilon_z
	.\]

\[
	\vartheta_x = \vartheta_y = \vartheta
	.\]

\section{Pure Shear Test}

\[
	\tau = \frac{F}{A}
	.\]

\[
	2\varepsilon_{xy} = \tau_{xy} = \tan\theta
	.\]

If the material is still elastic, the
\[
	\tau_{xy} = G \varepsilon_{xy}
	.\]

Where $G=\mu$ is the shear modulus.

\[
	G = \mu = \frac{E}{2(1+\vartheta)}
	.\]

\section{Generalized Hooke's Law}

\begin{align*}
	\varepsilon_1 & = \frac{1}{E}\lt(\sigma_1 - \vartheta(\sigma_2 + \sigma_3)\rt) \\
	\varepsilon_2 & = \frac{1}{E}\lt(\sigma_2 - \vartheta(\sigma_1 + \sigma_3)\rt) \\
	\varepsilon_3 & = \frac{1}{E}\lt(\sigma_3 - \vartheta(\sigma_1 + \sigma_2)\rt)
\end{align*}

If a material experiences a deformation $\varepsilon_{ij}$, the new length of the material across the $x_i$ dimension is
\[
	x_i' = (1+\varepsilon_{ii})x_i
	.\]

\begin{align*}
	\sigma_{ij}      & = \lambda \delta_{ij} \varepsilon_{kk} + 2\mu \varepsilon_{ij}                     \\
	\varepsilon_{ij} & = \frac{1}{E} \lt((1+\vartheta)\sigma_{ij} - \vartheta \sigma_{kk}\delta_{ij} \rt)
\end{align*}

where $\lambda$ is the Lame's constant.

\[
	\lambda = \frac{E\vartheta}{(1+\vartheta)(1-2\vartheta)}
	.\]

\section{Deformation Approach}

\begin{enumerate}
	\ii Postulate the strain field.
	\ii Verify the boundary conditions.
	\ii Verify the Lam\'e-Navier equations.
	\[
		(\lambda + 2\mu) \nabla(\nabla \cdot \vb{u}) + + \nabla \cdot \va{f} =  0
		.\]
	\ii Find the strain components first and then the stress components.
	\ii Check boundary conditions in stress.
\end{enumerate}

\section{Stress Approach}

\begin{enumerate}
	\ii Postulate the stress filed.
	\ii Verify the conditions of equilibrium.
	\ii Verify the boundary conditions in terms of stress.
	\ii Verify the equations of Beltrami-Michell.

	\[
		\nabla^2 \sigma_ij + \frac{1}{1+\vartheta}\pdv{\sigma_{kk}}{x_i}{x_j} + \frac{\vartheta}{1+\vartheta}\dv{f_l}{x_l} + \pdv{f_j}{x_i} + \pdv{f_i}{x_j} = 0
		.\]
	\ii Verify the boundary conditions in displacement.
\end{enumerate}

\section{Stress functions}

We call $\Phi$ a stress function if it verfies

\[
	\pdv[4]{\Phi}{x_1} + 2\frac{\partial^4\Phi}{\partial{x_1}^2 \partial{x_2}^2} + \pdv[4]{\Phi}{x_2} = 0
	.\]

\begin{align*}
	\sigma_{11} & = \pdv[2]{\Phi}{x_2}       \\
	\sigma_{22} & = \pdv[2]{\Phi}{x_1}       \\
	\sigma_{12} & = -\pdv[2]{\Phi}{x_1}{x_2}
\end{align*}

\end{document}

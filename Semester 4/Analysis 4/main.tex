\documentclass{report}

\input{../template/preamble}
\input{../template/macros}
\input{../template/letterfonts}

\title{\Huge{Complex Analysis}\\Semester 4}
\author{}
\date{}

\begin{document}

\maketitle
\newpage% or \cleardoublepage
% \pdfbookmark[<level>]{<title>}{<dest>}
\pdfbookmark[section]{\contentsname}{toc}
\tableofcontents
\pagebreak

\chapter{The Complex Plane}
\section{Algebra of the complex plane}

\begin{figure}[h]
	\centering
	\begin{tikzpicture}
		\node[anchor=south west,inner sep=0] at (0,0) {\includegraphics[width=0.5\textwidth]{wojak.jpg}};
		\draw (4,3.5) node {$i = \sqrt{-1}$};
	\end{tikzpicture}
\end{figure}

Euler’s formulas for $\sin$ and $\cos$
\begin{align*}
	\cos(\theta) & = \frac{e^{i\theta} + e^{-i\theta}}{2}                                 \\
	\sin(\theta) & = \frac{e^{i\theta} - e^{-i\theta}}{2i}                                \\
	\tan(\theta) & = \frac{e^{i\theta} - e^{-i\theta}}{i\lt(e^{i\theta}+e^{-i\theta}\rt)}
\end{align*}

The $n$-th roots of unity are the set of complex numbers $(\zeta_1,\zeta_2,\dots,\zeta_n)$ are the complex numbers that satisfy the equation

\[
	z^n = w
	.\]
where $w=Re^{i\alpha}$. The solutions equation are
\[
	\zeta_k = \sqrt[n]{R} e^{i\left( \frac{\alpha + 2k\pi}{n} \right)}
	.\]

\section{Topology of the complex plane}

\thm{}{
	The mapping
	\begin{align*}
		|z|: \mathbb{C} & \longrightarrow \mathbb{R}^+          \\
		z=x+yi          & \longmapsto |x+yi| = \sqrt{x^2 + y^2}
		.\end{align*}
	defines a norm on $\mathbb{C}$, so the complex plane is a normed space.
}

\thm{}{
	The mapping
	\begin{align*}
		d(.,.): \mathbb{C}\times\mathbb{C} & \longrightarrow \mathbb{R}^+ \\
		(z,w)                              & \longmapsto d(z,w) = |z-w|
		.\end{align*}
	defined a distance on $\mathbb{C}$, so the complex plane is a metric space.
}

\dfn{Neighborhood}{
	We call $\delta$-neighborhood of $z_0$ an open disk centered at $z_0$ of radius $\delta$
	\[
		N_\delta(z_0) = \{z\in\mathbb{C}: \; |z-z_0|<\delta\}
		.\]
}

We call $N_\delta(z_0)-\{z_0\}$ a deleted $\delta$-neighborhood. ($\{z\in\mathbb{C}: \; 0 < |z-z_0|<\delta\}$)\\

\dfn{}{
	Let $z_0\in\mathbb{C}$ and $\Omega\subset\mathbb{C}$.
	\begin{enumerate}
		\ii $z_0$ is called an \emph{interior point} of $\Omega$ if
		\[
			\exists\delta>0, N_\delta(z_0)\subset\Omega
			.\]
		\ii $z_0$ is an \emph{exterior point} of $\Omega$ if
		\[
			\exists\delta>0, N_\delta(z_0)\cap\Omega=\emptyset
			.\]
		\ii $z_0$ is a \emph{boundary point} of $\Omega$ if
		\[
			\forall \delta>0, \; N_\delta(z_0)\cap\Omega\neq\emptyset \quad \text{and} \quad N_\delta(z_0)\cap \underbrace{C^{\Omega}_{\mathbb{C}}}_{\mathbb{C} - \Omega}\neq\emptyset
			.\]
	\end{enumerate}
}

\dfn{}{
	The set of all:
	\begin{enumerate}
		\ii interior points: $\dot{\Omega}$
		\ii boundary points: $\partial\Omega$
		\ii the set $\Omega\cup\partial\Omega$ is called a closure of $\Omega$ denoted $\bar{\Omega}$
	\end{enumerate}
}

\dfn{}{
	We call a set $\Omega$
	\begin{enumerate}
		\ii an \emph{open set} if it only contains it's interior points
		\[
			\Omega \cup \partial\Omega =\emptyset \quad\text{and}\quad\Omega =\dot{\Omega}
			.\]
		\ii a \emph{closed set} if it contains all it's boundary points
		\[
			\partial\Omega \subset \Omega \quad\text{and}\quad\Omega =\bar{\Omega}
			.\]
	\end{enumerate}
}

\nt{
	$\Omega$ is said to be \emph{compact} if it is both \emph{bounded and closed}.
}

\dfn{Limit (accumulation point)}{
	Given a point $z_0\in\Omega$. $z_0$ is a limit point if for all $\delta>0$, $\exists$ infinitely many points $\in N_\delta(z_0)$.
}

\nt{
	If a set is finite then it doesn't have any limit points.\\
	If $z_0$ is a boundary point of $\Omega$ and $\not\in\Omega\Rightarrow z_0$ is a limit point\\
	$\Omega$ is a closed set $\Leftrightarrow\Omega\subset\{\text{All limit points}\}$\\
	The set of all limit points of $\Omega$ is called the derivative set $\Omega'$
}

\nt{
	A set $\Omega$ is bounded if
	\[
		\exists M \in \RR_+ \Big/ \forall z\in \Omega\;|z|\leq M
		.\]
}

\thm{Bolzano-Weirstrass theorem}{
	Every \emph{bounded infinte} set admits at least one limit point
}

\subsubsection{Paths}
A path is a set of complex points $\Gamma$ where
\[
	\Gamma = \lt\{z(t) = x(t) + i\,y(t)\; t\in[a,b[\rt\}
	.\]
A simple path/Jordan arc if it does not cross itself
\[
	\forall t_1,t_2 \in [a,b[ \;t_1\neq t_2 \Rightarrow z(t_1)\neq z(t_2)
	.\]

A closed path is a path such that
\[
	z(a) = z(b)
	.\]
A differentiable path (aka. a contour) is a path of equation $\lt(x(t),y(t)\rt)$ such that $x$ and $y$ are of class $C^1$ on the domain of $t$.\\

A piecewise differentiable path is union of several differentiable paths.

\nt{
	A set is connected if we can connect 2 points $z_1,z_2\in\Omega$ using a broken line\\
	A connected set simply connected if we can connect any 2 points within using a straight line (no holes in the set), otherwise it is multiply connected.
}

\chapter{Complex Functions}

\section{Limits and Differentiability}

\nt{
	When taking limits we can do the 2D limit where $x = \Re(z)$ and $y=\Im(z)$
	\[
		\lim_{(x,y)\to(x_0,y_0)} f(x+iy)
		.\]

	then we can take multiple paths to find the limit. However we can't take sufficient paths to prove a limit exists as there could exist one path that causes the limit to not exist, however we can use polar limits to prove that the limit exists. We take $x = r\cos(\theta)-x_0$ and $y = r\sin(\theta)-y_0$
	\[
		\lim_{r\to0} f\lt(r\cos(\theta)-x_0+ i\lt(r\sin(\theta)-y_0\rt)\rt)
		.\]

}

\thm{Cauchy-Riemann equations}{
	We define a complex function
	\[
		f(x+iy) = u(x,y) + iv(x,y)
		.\]

	If $f$ is differentiable on a point $z_0 = x_0 + iy_0$ then $u$ and $v$ satisfy the Cauchy-Riemann equations:
	\begin{align*}
		\pdv{u}{x}(x_0,y_0) & = \pdv{v}{y}(x_0,y_0)  \\
		\pdv{u}{y}(x_0,y_0) & = -\pdv{v}{x}(x_0,y_0)
		.\end{align*}

	Note that the converse is not true
}

To prove that a function $f$ is differentiable at $z_0$ then we have to prove that $u_x$, $u_y$, $v_x$, and $v_y$

\[
	\begin{cases}
		\text{exist in } \Omega            \\
		\text{are continous at } (x_0,y_0) \\
		\text{satisfy the Cauchy-Riemann equations at }(x_0,y_0)
	\end{cases}
\]

\subsection{Hyperbolic functions}

\begin{align*}
	\cosh z & = \frac{e^z + e^{-z}}{2}  \\
	\sinh z & = \frac{e^z - e^{-z}}{2}  \\
	\tanh z & = \frac{\sinh z}{\cosh z} \\
\end{align*}

Properties
\begin{tasks}(2)
	\task $\cosh^2z - \sinh^2z = 1$
	\task $\cosh^2z + \sinh^2z = \cosh 2z$
	\task $\cosh z_1+z_2 = \cosh z_1 \cdot \cosh z_2 + \sinh z_1 \cdot \sinh z_2$
	\task $\sinh z_1+z_2 = \sinh z_1\cdot \cosh z_2 +\sinh z_2\cdot \cosh z_1$
	\task $\cos iz = \cosh z$
	\task $\sin iz = i\sinh z$
	\task $\cosh iz = \cos z$
	\task $\sinh iz = i\sin z$
\end{tasks}

\thm{}{
	Consider 2 functions $u$ and $v$, and the 2 curves $u=\alpha$ and $v=\beta$ such that $\alpha,\beta\in\RR$. The 2 curves are orthogonal at their intersection points if and only if they satisfy the Cauchy-Riemann conditions.
}

\section{Harmonic functions}

\dfn{Harmonic function}{
	A function $u(x,y)$, of class $C^2$ and defined on $\Omega$, is said to be harmonic if
	\[
		\pdv[2]{u}{x} + \pdv[2]{u}{y} = 0
		.\]
	or in other words the Laplacian is equal to 0
	\[
		\Delta u = \nabla^2 u = 0
		.\]
}

\thm{}{
	Let a function $f = u + iv$ defined on $\Omega$
	\[
		f \text{ is holomorphic } \Leftrightarrow \begin{cases}
			u,v \text{ are of class } C^\infty \text{ in } \Omega       \\
			u,v \text{ satisfy the Cauchy–Riemann equations in } \Omega \\
			u,v \text{ are harmonic in }\Omega
		\end{cases}
		.\]
}

\chapter{Integrals}

\dfn{Complex Integral}{
	Let $\Omega$ be an open subset of $\CC$ and $\Gamma$ a piecewise differentiable path from $z_1$ to $z_2$. We define the integral of $f$ along the path to be 2 different line integrals:
	\[
		\int f(z)\dd{z} = \int_\Gamma (u+iv)(\dd{x}+i\dd{y}) = \int_\Gamma(u\dd{x}-v\dd{y}) + i\int_\Gamma(v\dd{x}+u\dd{y})
		.\]
}

\thm{Parametrization of the path}{
	If the path $\Gamma$ is parametrized by $\gamma(t)=x(t) + iy(t)$ where $x,y$ are of class $c^1$ on $[a,b]$ then
	\[
		\int_\Gamma f(z)\dd{z}=\int_a^b f\lt(\gamma(t)\rt)\cdot\gamma'(t)\dd{t}
		.\]
}

\thm{$ML$-rule}{
In a path of $\Gamma$ of length $L$, we can approximate the value of an integral along that path
\[
	\lt|\int_\Gamma f(z)\dd{z}\rt|\leq M\cdot L
	.\]
where
\[
	M=\sup_{z\in\Gamma}|f(z)|\quad\text{and}\quad L=\text{Length of the path } \Gamma = \int_a^b\sqrt{{x'}(t)^2 + {y'}(t)^2}\dd{t}
	.\]
}

\thm{Cauchy's theorem}{
	Let $\Gamma$ be a simple closed curve. Let $f$ be a holomorphic function on $\Gamma$ and inside $\Gamma$, then
	\[
		\oint_\Gamma f(z)\dd{z} = 0
		.\]
}

\nt{
	Green-Riemann theorem states that
	\[
		\oint_{\partial\Omega}\lt(P(x,y)\dd{x}+Q(x,y)\dd{y}\rt) = \iint_\Omega \lt(\pdv{Q}{x} - \pdv{P}{y}\rt)\dd{x}\dd{y}
		.\]
}

\nt{
	\[
		\int_{\Gamma^-}f(z)\dd{z} = -\int_\Gamma f(z)\dd{z}
		.\]
}

A consequence of Cauchy's theorem is that if a closed path $C$ contains a discontinuity then the path of integration doesn't matter as long as the new path also contains the exact same discontinuity.

\thm{}{
	Let $\Omega$ be a simply closed region. Let $f$ be a holomorphic function on $\Omega$, $z_1$ and $z_2$ be 2 point $\in\Omega$. Then the integral of $f(z)$ is independent of the path taken from $z_1$ to $z_2$
	\[
		\int_{\gamma_1} f(z)\dd{z} = \int_{\gamma_2}f(z)\dd{z}
		.\]
}

\thm{Liouville's theorem}{
	\begin{itemize}
		\ii $f$ is holomorphic in $\CC$
		\ii $f$ is bounded in $\CC$
		\[
			\exists M\in\RR_+,\forall z\in\CC,|f(z)\leq M|
			.\]
	\end{itemize}
	then $f$ is constant in $\CC$
}

\thm{Mean value theorem}{
	Let $\gamma_r$ be a circle of center $a$ and radius $r>0$. If $f$ is a holomorphic on and in $\gamma_r$ then
	\[
		f(a) = \frac{1}{2\pi}\int_0^{2\pi} f\lt(a+re^{i\theta}\rt)\dd{\theta}
		.\]
}

\thm{Cauchy's integral formula}{
	Let $\Gamma$ is a simple closed curve and the function $f(z)$ is holomorphic on $\Gamma$ and its interior. Then:
	\[
		f(a) = \frac{1}{2\pi i}\oint_{\Gamma^+}\frac{f(z)}{z-a}\dd{z}
		.\]

	and the general form of the formula is
	\[
		f^{(n)}(a) = \frac{n!}{2\pi i}\oint_{\Gamma^+}\frac{f(z)}{(z-a)^{n+1}}\dd{z}
		.\]
}

\thm{Tangent half-angle substitution}{
	We can transform the integral of the form
	\[
		\int f(\sin x, \cos x)\dd{x}=\int f\lt(\frac{2t}{1+t^2},\frac{1-t^2}{1+t^2}\rt)\dd{x}
		.\]
	by letterfonts
	\[
		t = \tan \frac{x}{2}
		.\]
	\[
		\sin x = \frac{2t}{1+t^2}\qquad\cos x=\frac{1-t^2}{1+t^2}\qquad \dd{x}=\frac{2}{1+t^2}\dd{t}
	\]
}

\section{Primitives}
\dfn{Primitives}{
	Let $f$ be a complex function, defined in an open set $\Omega\subset\CC$.\\
	We call a primitive of $f$ on $\Omega$, any function $F$ such that $F$ is holomorphic in $\Omega$ and $\forall z\in\Omega\;F'(z) = f(z)$
	\[
		F(z) = \int f(z)\dd{z}
		.\]
}

\nt{
	If $f$ admits a primitive on the open set $\Omega$ then $f$ is holomorphic in $\Omega$
}

Let the path $\gamma$ goes from $z_1$ to $z_2$ in $\Omega$ then
\[
	\int_\gamma f(z)\dd{z} = F(z_2)-F(z_2)
	.\]

\nt{
	\[
		\oint_\gamma f(z)\dd{z}=0\;\Rightarrow f\text{ is holomorphic in }\Omega
		.\]

}

\end{document}

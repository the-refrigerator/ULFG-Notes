\documentclass{report}

\input{../template/preamble}
\input{../template/macros}
\input{../template/letterfonts}

\title{\Huge{Complex Analysis}\\Semester 4}
\author{}
\date{}

\begin{document}

\maketitle
\newpage% or \cleardoublepage
% \pdfbookmark[<level>]{<title>}{<dest>}
\pdfbookmark[section]{\contentsname}{toc}
\tableofcontents
\pagebreak

\chapter{The Complex Plane}
\section{Algebra of the complex plane}

\begin{figure}[h]
	\centering
	\begin{tikzpicture}
		\node[anchor=south west,inner sep=0] at (0,0) {\includegraphics[width=0.5\textwidth]{wojak.jpg}};
		\draw (4,3.5) node {$i = \sqrt{-1}$};
	\end{tikzpicture}
\end{figure}

Euler’s formulas for $\sin$ and $\cos$
\begin{align*}
	\cos(\theta) & = \frac{e^{i\theta} + e^{-i\theta}}{2}                                 \\
	\sin(\theta) & = \frac{e^{i\theta} - e^{-i\theta}}{2i}                                \\
	\tan(\theta) & = \frac{e^{i\theta} - e^{-i\theta}}{i\lt(e^{i\theta}+e^{-i\theta}\rt)}
\end{align*}

The $n$-th roots of unity are the set of complex numbers $(\zeta_1,\zeta_2,\dots,\zeta_n)$ are the complex numbers that satisfy the equation

\[
	z^n = w
	.\]
where $w=Re^{i\alpha}$. The solutions equation are
\[
	\zeta_k = \sqrt[n]{R} e^{i\left( \frac{\alpha + 2k\pi}{n} \right)}
	.\]

\section{Topology of the complex plane}

\thm{}{
	The mapping
	\begin{align*}
		|z|: \mathbb{C} & \longrightarrow \mathbb{R}^+          \\
		z=x+yi          & \longmapsto |x+yi| = \sqrt{x^2 + y^2}
		.\end{align*}
	defines a norm on $\mathbb{C}$, so the complex plane is a normed space.
}

\thm{}{
	The mapping
	\begin{align*}
		d(.,.): \mathbb{C}\times\mathbb{C} & \longrightarrow \mathbb{R}^+ \\
		(z,w)                              & \longmapsto d(z,w) = |z-w|
		.\end{align*}
	defined a distance on $\mathbb{C}$, so the complex plane is a metric space.
}

\dfn{Neighborhood}{
	We call $\delta$-neighborhood of $z_0$ an open disk centered at $z_0$ of radius $\delta$
	\[
		N_\delta(z_0) = \{z\in\mathbb{C}: \; |z-z_0|<\delta\}
		.\]
}

We call $N_\delta(z_0)-\{z_0\}$ a deleted $\delta$-neighborhood. ($\{z\in\mathbb{C}: \; 0 < |z-z_0|<\delta\}$)\\

\dfn{}{
	Let $z_0\in\mathbb{C}$ and $\Omega\subset\mathbb{C}$.
	\begin{enumerate}
		\ii $z_0$ is called an \emph{interior point} of $\Omega$ if
		\[
			\exists\delta>0, N_\delta(z_0)\subset\Omega
			.\]
		\ii $z_0$ is an \emph{exterior point} of $\Omega$ if
		\[
			\exists\delta>0, N_\delta(z_0)\cap\Omega=\emptyset
			.\]
		\ii $z_0$ is a \emph{boundary point} of $\Omega$ if
		\[
			\forall \delta>0, \; N_\delta(z_0)\cap\Omega\neq\emptyset \quad \text{and} \quad N_\delta(z_0)\cap \underbrace{C^{\Omega}_{\mathbb{C}}}_{\mathbb{C} - \Omega}\neq\emptyset
			.\]
	\end{enumerate}
}

\dfn{}{
	The set of all:
	\begin{enumerate}
		\ii interior points: $\dot{\Omega}$
		\ii boundary points: $\partial\Omega$
		\ii the set $\Omega\cup\partial\Omega$ is called a closure of $\Omega$ denoted $\bar{\Omega}$
	\end{enumerate}
}

\nt{
	\[
		\dot{\Omega}\subset\Omega\subset\bar{\Omega}
		.\]
	and
	\[
		\Omega \text{ is open} \Leftrightarrow \begin{cases}
			\Omega\cap\partial\Omega=\emptyset \\
			\Omega = \dot{\Omega}
		\end{cases}
		.\]
	If $\Omega$ is not open from all sides then it is not open. Same thing with closed.\\
	If $\Omega$ is not open then it is not connected.
}

$\Omega$ is said to be \emph{compact} if it is both \emph{bounded and closed}.

\thm{Bolzano-Weirstrass theorem}{
	Every \emph{bounded infinte} set admits at least one limit point
}

\subsubsection{Paths}
A path is a set of complex points $\Gamma$ where
\[
	\Gamma = \lt\{z(t) = x(t) + i\,y(t)\; t\in[a,b[\rt\}
	.\]
A simple path/Jordan arc if it does not cross itself
\[
	\forall t_1,t_2 \in [a,b[ \;t_1\neq t_2 \Rightarrow z(t_1)\neq z(t_2)
	.\]

A closed path is a path such that
\[
	z(a) = z(b)
	.\]


\chapter{Complex Functions}

\section{Limits and Differentiability}

\nt{
	When taking limits we can do the 2D limit where $x = \Re(z)$ and $y=\Im(z)$
	\[
		\lim_{(x,y)\to(x_0,y_0)} f(x+iy)
		.\]

	then we can take multiple paths to find the limit. However we can't take suffecient paths to prove a limit exists as there could exist one path that causes the limit to not exist, however we can use polar limits to prove that the limit exists. We take $x = r\cos(\theta)-x_0$ and $y = r\sin(\theta)-y_0$
	\[
		\lim_{(r,\theta)\to(0,0)} f\lt(r\cos(\theta)-x_0+ i\lt(r\sin(\theta)-y_0\rt)\rt)
		.\]

}

\thm{Cauchy-Riemann equations}{
	We define a complex function
	\[
		f(x+iy) = u(x,y) + iv(x,y)
		.\]

	If $f$ is differentiable on a point $z_0 = x_0 + iy_0$ then $u$ and $v$ satisfy the Cauchy-Riemann equations:
	\begin{align*}
		\pdv{u}{x}(x_0,y_0) & = \pdv{v}{y}(x_0,y_0)  \\
		\pdv{u}{y}(x_0,y_0) & = -\pdv{v}{x}(x_0,y_0)
		.\end{align*}

	Note that the converse is not true
}

To prove that a function $f$ is differentiable at $z_0$ then we have to prove that $u$ and $v$

\[
	\begin{cases}
		\text{exist in } \Omega            \\
		\text{are continous at } (x_0,y_0) \\
		\text{satisfy the Cauchy-Riemann equations at }(x_0,y_0)
	\end{cases}
\]

\subsection{Hyperbolic functions}

\begin{align*}
	\cosh z & = \frac{e^z + e^{-z}}{2}  \\
	\sinh z & = \frac{e^z - e^{-z}}{2}  \\
	\tanh z & = \frac{\sinh z}{\cosh z} \\
\end{align*}

Properties
\begin{tasks}(2)
	\task $\cosh^2z - \sinh^2z = 1$
	\task $\cosh^2z + \sinh^2z = \cosh 2z$
	\task $\cosh z_1+z_2 = \cosh z_1 \cdot \cosh z_2 + \sinh z_1 \cdot \sinh z_2$
	\task $\sinh z_1+z_2 = \sinh z_1\cdot \cosh z_2 +\sinh z_2\cdot \cosh z_1$
	\task $\cos iz = \cosh z$
	\task $\sin iz = i\sinh z$
	\task $\cosh iz = \cos z$
	\task $\sinh iz = i\sin z$
\end{tasks}

\section{Harmonic functions}

\dfn{Harmonic function}{
	A function $u(x,y)$, of class $C^2$ and defined on $\Omega$, is said to be harmonic if
	\[
		\pdv[2]{u}{x} + \pdv[2]{u}{y} = 0
		.\]
	or in other words the Laplacian is equal to 0
	\[
		\Delta u = \nabla^2 u = 0
		.\]
}

\thm{}{
	Let a function $f = u + iv$ defined on $\Omega$
	\[
		f \text{ is holomorphic } \Leftrightarrow \begin{cases}
			u,v \text{ are of class } C^\infty \text{ in } \Omega       \\
			u,v \text{ satisfy the Cauchy–Riemann equations in } \Omega \\
			u,v \text{ are harmonic in }\Omega
		\end{cases}
		.\]
}

\end{document}

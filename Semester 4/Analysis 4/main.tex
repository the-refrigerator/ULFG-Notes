\documentclass{report}

\input{../template/preamble}
\input{../template/macros}
\input{../template/letterfonts}

\title{\Huge{Complex Analysis}\\Semester 4}
\author{}
\date{}

\begin{document}

\maketitle
\newpage% or \cleardoublepage
% \pdfbookmark[<level>]{<title>}{<dest>}
\pdfbookmark[section]{\contentsname}{toc}
\tableofcontents
\pagebreak

\chapter{The Complex Plane}
\section{Algebra of the complex plane}

\begin{figure}[h]
	\centering
	\begin{tikzpicture}
		\node[anchor=south west,inner sep=0] at (0,0) {\includegraphics[width=0.5\textwidth]{wojak.jpg}};
		\draw (4,3.5) node {$i = \sqrt{-1}$};
	\end{tikzpicture}
\end{figure}

Euler’s formulas for $\sin$ and $\cos$
\begin{align*}
	\cos(\theta) & = \frac{e^{i\theta} + e^{-i\theta}}{2}                                 \\
	\sin(\theta) & = \frac{e^{i\theta} - e^{-i\theta}}{2i}                                \\
	\tan(\theta) & = \frac{e^{i\theta} - e^{-i\theta}}{i\lt(e^{i\theta}+e^{-i\theta}\rt)}
\end{align*}

The $n$-th roots of unity are the set of complex numbers $(\zeta_1,\zeta_2,\dots,\zeta_n)$ are the complex numbers that satisfy the equation

\[
	z^n = w
	.\]
where $w=Re^{i\alpha}$. The solutions equation are
\[
	\zeta_k = \sqrt[n]{R} e^{i\lt( \frac{\alpha + 2k\pi}{n} \rt)}
	.\]

\section{Topology of the complex plane}

\thm{}{
	The mapping
	\begin{align*}
		|z|: \mathbb{C} & \longrightarrow \mathbb{R}^+          \\
		z=x+yi          & \longmapsto |x+yi| = \sqrt{x^2 + y^2}
		.\end{align*}
	defines a norm on $\mathbb{C}$, so the complex plane is a normed space.
}

\thm{}{
	The mapping
	\begin{align*}
		d(.,.): \mathbb{C}\times\mathbb{C} & \longrightarrow \mathbb{R}^+ \\
		(z,w)                              & \longmapsto d(z,w) = |z-w|
		.\end{align*}
	defined a distance on $\mathbb{C}$, so the complex plane is a metric space.
}

\dfn{Neighborhood}{
	We call $\delta$-neighborhood of $z_0$ an open disk centered at $z_0$ of radius $\delta$
	\[
		N_\delta(z_0) = \{z\in\mathbb{C}: \; |z-z_0|<\delta\}
		.\]
}

We call $N_\delta(z_0)-\{z_0\}$ a deleted $\delta$-neighborhood. ($\{z\in\mathbb{C}: \; 0 < |z-z_0|<\delta\}$)\\

\dfn{}{
	Let $z_0\in\mathbb{C}$ and $\Omega\subset\mathbb{C}$.
	\begin{enumerate}
		\ii $z_0$ is called an \emph{interior point} of $\Omega$ if
		\[
			\exists\delta>0, N_\delta(z_0)\subset\Omega
			.\]
		\ii $z_0$ is an \emph{exterior point} of $\Omega$ if
		\[
			\exists\delta>0, N_\delta(z_0)\cap\Omega=\emptyset
			.\]
		\ii $z_0$ is a \emph{boundary point} of $\Omega$ if
		\[
			\forall \delta>0, \; N_\delta(z_0)\cap\Omega\neq\emptyset \quad \text{and} \quad N_\delta(z_0)\cap \underbrace{C^{\Omega}_{\mathbb{C}}}_{\mathbb{C} - \Omega}\neq\emptyset
			.\]
	\end{enumerate}
}

\dfn{}{
	The set of all:
	\begin{enumerate}
		\ii interior points: $\dot{\Omega}$
		\ii boundary points: $\partial\Omega$
		\ii the set $\Omega\cup\partial\Omega$ is called a closure of $\Omega$ denoted $\bar{\Omega}$
	\end{enumerate}
}

\dfn{}{
	We call a set $\Omega$
	\begin{enumerate}
		\ii an \emph{open set} if it only contains it's interior points
		\[
			\Omega \cup \partial\Omega =\emptyset \quad\text{and}\quad\Omega =\dot{\Omega}
			.\]
		\ii a \emph{closed set} if it contains all it's boundary points
		\[
			\partial\Omega \subset \Omega \quad\text{and}\quad\Omega =\bar{\Omega}
			.\]
	\end{enumerate}
}

\nt{
	$\Omega$ is said to be \emph{compact} if it is both \emph{bounded and closed}.
}

\dfn{Limit (accumulation point)}{
	Given a point $z_0\in\Omega$. $z_0$ is a limit point if for all $\delta>0$, $\exists$ infinitely many points $\in N_\delta(z_0)$.
}

\nt{
	If a set is finite then it doesn't have any limit points.\\
	If $z_0$ is a boundary point of $\Omega$ and $\not\in\Omega\Rightarrow z_0$ is a limit point\\
	$\Omega$ is a closed set $\Leftrightarrow\Omega\subset\{\text{All limit points}\}$\\
	The set of all limit points of $\Omega$ is called the derivative set $\Omega'$
}

\nt{
	A set $\Omega$ is bounded if
	\[
		\exists M \in \RR_+ \Big/ \forall z\in \Omega\;|z|\leq M
		.\]
}

\thm{Bolzano-Weirstrass theorem}{
	Every \emph{bounded infinte} set admits at least one limit point
}

\subsubsection{Paths}
A path is a set of complex points $\Gamma$ where
\[
	\Gamma = \lt\{z(t) = x(t) + i\,y(t)\; t\in[a,b[\rt\}
	.\]
A simple path/Jordan arc if it does not cross itself
\[
	\forall t_1,t_2 \in [a,b[ \;t_1\neq t_2 \Rightarrow z(t_1)\neq z(t_2)
	.\]

A closed path is a path such that
\[
	z(a) = z(b)
	.\]
A differentiable path (aka. a contour) is a path of equation $\lt(x(t),y(t)\rt)$ such that $x$ and $y$ are of class $C^1$ on the domain of $t$.\\

A piecewise differentiable path is union of several differentiable paths.

\nt{
	A set is connected if we can connect 2 points $z_1,z_2\in\Omega$ using a broken line\\
	A connected set simply connected if we can connect any 2 points within using a straight line (no holes in the set), otherwise it is multiply connected.
}

\chapter{Complex Functions}

\section{Limits and Differentiability}

\nt{
	When taking limits we can do the 2D limit where $x = \Re(z)$ and $y=\Im(z)$
	\[
		\lim_{(x,y)\to(x_0,y_0)} f(x+iy)
		.\]

	then we can take multiple paths to find the limit. However we can't take sufficient paths to prove a limit exists as there could exist one path that causes the limit to not exist, however we can use polar limits to prove that the limit exists. We take $x = r\cos(\theta)-x_0$ and $y = r\sin(\theta)-y_0$
	\[
		\lim_{r\to0} f\lt(r\cos(\theta)-x_0+ i\lt(r\sin(\theta)-y_0\rt)\rt)
		.\]

}

\thm{Cauchy-Riemann equations}{
	We define a complex function
	\[
		f(x+iy) = u(x,y) + iv(x,y)
		.\]

	If $f$ is differentiable on a point $z_0 = x_0 + iy_0$ then $u$ and $v$ satisfy the Cauchy-Riemann equations:
	\begin{align*}
		\pdv{u}{x}(x_0,y_0) & = \pdv{v}{y}(x_0,y_0)  \\
		\pdv{u}{y}(x_0,y_0) & = -\pdv{v}{x}(x_0,y_0)
		.\end{align*}

	Note that the converse is not true
}

To prove that a function $f$ is differentiable at $z_0$ then we have to prove that $u_x$, $u_y$, $v_x$, and $v_y$

\[
	\begin{cases}
		\text{exist in } \Omega            \\
		\text{are continous at } (x_0,y_0) \\
		\text{satisfy the Cauchy-Riemann equations at }(x_0,y_0)
	\end{cases}
\]

\subsection{Hyperbolic functions}

\begin{align*}
	\cosh z & = \frac{e^z + e^{-z}}{2}  \\
	\sinh z & = \frac{e^z - e^{-z}}{2}  \\
	\tanh z & = \frac{\sinh z}{\cosh z} \\
\end{align*}

Properties
\begin{tasks}(2)
	\task $\cosh^2z - \sinh^2z = 1$
	\task $\cosh^2z + \sinh^2z = \cosh 2z$
	\task $\cosh z_1+z_2 = \cosh z_1 \cdot \cosh z_2 + \sinh z_1 \cdot \sinh z_2$
	\task $\sinh z_1+z_2 = \sinh z_1\cdot \cosh z_2 +\sinh z_2\cdot \cosh z_1$
	\task $\cos iz = \cosh z$
	\task $\sin iz = i\sinh z$
	\task $\cosh iz = \cos z$
	\task $\sinh iz = i\sin z$
\end{tasks}

\thm{}{
	Consider 2 functions $u$ and $v$, and the 2 curves $u=\alpha$ and $v=\beta$ such that $\alpha,\beta\in\RR$. The 2 curves are orthogonal at their intersection points if and only if they satisfy the Cauchy-Riemann conditions.
}

\section{Harmonic functions}

\dfn{Harmonic function}{
	A function $u(x,y)$, of class $C^2$ and defined on $\Omega$, is said to be harmonic if
	\[
		\pdv[2]{u}{x} + \pdv[2]{u}{y} = 0
		.\]
	or in other words the Laplacian is equal to 0
	\[
		\Delta u = \nabla^2 u = 0
		.\]
}

\thm{}{
	Let a function $f = u + iv$ defined on $\Omega$
	\[
		f \text{ is holomorphic } \Leftrightarrow \begin{cases}
			u,v \text{ are of class } C^\infty \text{ in } \Omega       \\
			u,v \text{ satisfy the Cauchy–Riemann equations in } \Omega \\
			u,v \text{ are harmonic in }\Omega
		\end{cases}
		.\]
}

\chapter{Integrals}

\dfn{Complex Integral}{
	Let $\Omega$ be an open subset of $\CC$ and $\Gamma$ a piecewise differentiable path from $z_1$ to $z_2$. We define the integral of $f$ along the path to be 2 different line integrals:
	\[
		\int f(z)\dd{z} = \int_\Gamma (u+iv)(\dd{x}+i\dd{y}) = \int_\Gamma(u\dd{x}-v\dd{y}) + i\int_\Gamma(v\dd{x}+u\dd{y})
		.\]
}

\thm{Parametrization of the path}{
	If the path $\Gamma$ is parametrized by $\gamma(t)=x(t) + iy(t)$ where $x,y$ are of class $c^1$ on $[a,b]$ then
	\[
		\int_\Gamma f(z)\dd{z}=\int_a^b f\lt(\gamma(t)\rt)\cdot\gamma'(t)\dd{t}
		.\]
}

\thm{$ML$-rule}{
In a path of $\Gamma$ of length $L$, we can approximate the value of an integral along that path
\[
	\lt|\int_\Gamma f(z)\dd{z}\rt|\leq M\cdot L
	.\]
where
\[
	M=\sup_{z\in\Gamma}|f(z)|\quad\text{and}\quad L=\text{Length of the path } \Gamma = \int_a^b\sqrt{{x'}(t)^2 + {y'}(t)^2}\dd{t}
	.\]
}

\thm{Cauchy's theorem}{
	Let $\Gamma$ be a simple closed curve. Let $f$ be a holomorphic function on $\Gamma$ and inside $\Gamma$, then
	\[
		\oint_\Gamma f(z)\dd{z} = 0
		.\]
}

\nt{
	Green-Riemann theorem states that
	\[
		\oint_{\partial\Omega}\lt(P(x,y)\dd{x}+Q(x,y)\dd{y}\rt) = \iint_\Omega \lt(\pdv{Q}{x} - \pdv{P}{y}\rt)\dd{x}\dd{y}
		.\]
}

\nt{
	\[
		\int_{\Gamma^-}f(z)\dd{z} = -\int_\Gamma f(z)\dd{z}
		.\]
}

A consequence of Cauchy's theorem is that if a closed path $C$ contains a discontinuity then the path of integration doesn't matter as long as the new path also contains the exact same discontinuity.

\thm{}{
	Let $\Omega$ be a simply closed region. Let $f$ be a holomorphic function on $\Omega$, $z_1$ and $z_2$ be 2 point $\in\Omega$. Then the integral of $f(z)$ is independent of the path taken from $z_1$ to $z_2$
	\[
		\int_{\gamma_1} f(z)\dd{z} = \int_{\gamma_2}f(z)\dd{z}
		.\]
}

\thm{Liouville's theorem}{
	\begin{itemize}
		\ii $f$ is holomorphic in $\CC$
		\ii $f$ is bounded in $\CC$
		\[
			\exists M\in\RR_+,\forall z\in\CC,|f(z)\leq M|
			.\]
	\end{itemize}
	then $f$ is constant in $\CC$
}

\thm{Mean value theorem}{
	Let $\gamma_r$ be a circle of center $a$ and radius $r>0$. If $f$ is a holomorphic on and in $\gamma_r$ then
	\[
		f(a) = \frac{1}{2\pi}\int_0^{2\pi} f\lt(a+re^{i\theta}\rt)\dd{\theta}
		.\]
}

\thm{Cauchy's integral formula}{
	Let $\Gamma$ is a simple closed curve and the function $f(z)$ is holomorphic on $\Gamma$ and its interior. Then:
	\[
		f(a) = \frac{1}{2\pi i}\oint_{\Gamma^+}\frac{f(z)}{z-a}\dd{z}
		.\]

	and the general form of the formula is
	\[
		f^{(n)}(a) = \frac{n!}{2\pi i}\oint_{\Gamma^+}\frac{f(z)}{(z-a)^{n+1}}\dd{z}
		.\]
}

\thm{Tangent half-angle substitution}{
	We can transform the integral of the form
	\[
		\int f(\sin x, \cos x)\dd{x}=\int f\lt(\frac{2t}{1+t^2},\frac{1-t^2}{1+t^2}\rt)\dd{x}
		.\]
	by letterfonts
	\[
		t = \tan \frac{x}{2}
		.\]
	\[
		\sin x = \frac{2t}{1+t^2}\qquad\cos x=\frac{1-t^2}{1+t^2}\qquad \dd{x}=\frac{2}{1+t^2}\dd{t}
	\]
}

\section{Primitives}
\dfn{Primitives}{
	Let $f$ be a complex function, defined in an open set $\Omega\subset\CC$.\\
	We call a primitive of $f$ on $\Omega$, any function $F$ such that $F$ is holomorphic in $\Omega$ and $\forall z\in\Omega\;F'(z) = f(z)$
	\[
		F(z) = \int f(z)\dd{z}
		.\]
}

\nt{
	If $f$ admits a primitive on the open set $\Omega$ then $f$ is holomorphic in $\Omega$
}

Let the path $\gamma$ goes from $z_1$ to $z_2$ in $\Omega$ then
\[
	\int_\gamma f(z)\dd{z} = F(z_2)-F(z_2)
	.\]

\nt{
	\[
		\oint_\gamma f(z)\dd{z}=0\;\Rightarrow f\text{ is holomorphic in }\Omega
		.\]

}

\chapter{Multivalued Functions}

\section{Complex Logarithm}

We define the complex logarithm to be
\[
	\log(z) = \ln|z| + i\arg(z)
	.\]

where
\[
	\arg\lt(re^{i\theta}\rt) = \theta + 2k\pi \quad k \in \ZZ
	.\]

We define the principle valued logarithm to be
\[
	\Log(z) = \ln|z| + i \Arg(z)
	.\]
where

\[
	\Arg\lt(re^{i\theta}\rt) = \text{principle value of } \arg\lt(re^{i\theta}\rt)\quad \theta\in]-\pi,\pi]
	.\]

$\Log$ is defined on $\CC^*$ however it is only continous on $\CC - \RR_-$

\begin{align*}
	\Omega_1 & = \{z\in\CC: \Re(z)>0\} \\
	\Omega_2 & = \{z\in\CC: \Im(z)>0\} \\
	\Omega_3 & = \{z\in\CC: \Im(z)<0\}
\end{align*}

\[
	\CC-\RR_- = \Omega_1 \cup \Omega_2 \cup \Omega_3
	.\]

\[
	\Arg(x+iy) = \begin{cases}
		\arctan\lt(\dfrac{y}{x}\rt)               & \text{if }z\in\Omega_1 \\
		\arccos\lt(\dfrac{x}{\sqrt{x^2+y^2}}\rt)  & \text{if }z\in\Omega_2 \\
		\arccos\lt(-\dfrac{x}{\sqrt{x^2+y^2}}\rt) & \text{if }z\in\Omega_3
	\end{cases}
	.\]

\begin{align*}
	\Log(z_1\cdot z_2)           & = \Log(z_1) + \Log(z_2) + 2k\pi i \\
	\Log\lt(\dfrac{z_1}{z_2}\rt) & = \Log(z_1) - \Log(z_2) + 2k\pi i \\
	\overline{\Log(z)}           & = \Log(\bar{z})
\end{align*}

\subsection{Generalisation of the Complex Logarithm}

We define
\[
	\Log_\alpha(z) = \ln|z| + i\Arg(z) \quad \Arg(z)\in ]\alpha,\alpha+2\pi]
	.\]

It is continuous on $\CC-\Delta$ where

\[
	\Delta = \lt\{re^{i\alpha}: r\geq 0\rt\}
	.\]

\section{$n$-th root}

We call the square root of $z=re^{i\theta}$ any complex number $w\in\CC$ satisfying
\[
	w^2 = z
	.\]

\[
	w = \sqrt{r} e^{i\lt(\frac{\theta}{2}+k\pi\rt)}
	.\]

And the $n$-th root becomes

\[
	w = \sqrt[n]{r} e^{i\lt(\frac{\theta}{n}+\frac{2k\pi}{n}\rt)}
	.\]

\section{Power Functions}

A power function is any function $f(z) = z^\alpha$. For $\alpha\in\ZZ$ the function is single valued, while for values $\alpha\not\in\ZZ$ the function is multivalued. The principle determination of $f$
\[
	z^\alpha = e^{\alpha\Log(z)}
	.\]

\section{Inverse Trig Functions}

It can be shown that
\[
	\Arcsin(z) = \frac{1}{i}\log\lt(iz+\sqrt{1-z^2}\rt)
	.\]
is holomorphic on $\CC - ]-\infty,-1]\cup[1,+\infty[$ and
\[
	\dv{}{z}\Arcsin(z) = \frac{1}{\sqrt{1-z^2}}
	.\]

\begin{align*}
	 & \Arccos(z)=\frac{1}{i}\Log\lt(z+\sqrt{z^2-1}\rt)   & \dv{}{z}\Arccos(z)  & = -\frac{1}{\sqrt{1-z^2}} \\
	 & \Arctan(z)=\frac{i}{2}\Log\lt(\frac{i+z}{i-z}\rt)  & \dv{}{z}\Arctan(z)  & = \frac{1}{1+z^2}         \\
	 & \Arcsinh(z)=\Log\lt(z+\sqrt{z^2+1}\rt)             & \dv{}{z}\Arcsinh(z) & = \frac{1}{\sqrt{1+z^2}}  \\
	 & \Arccosh(z)=\Log\lt(z+\sqrt{z^2-1}\rt)             & \dv{}{z}\Arccosh(z) & = \frac{1}{\sqrt{z^2-1}}  \\
	 & \Arctanh(z)=\frac{1}{2}\Log\lt(\frac{1+z}{1-z}\rt) & \dv{}{z}\Arctanh(z) & = \frac{1}{1-z^2}         \\
\end{align*}

\chapter{Series}

\dfn{Power Series}{
	We define a power series to be a function of the from
	\[
		f(z) = \sum_{n=0}^\infty a_n(z-z_0)^n
		.\]
}

\nt{
	A power series is always convergent at $z=z_0$ and it's value is $a_0$.
}

\thm{}{
	Consider a power series $ \sum_{n=0}^\infty a_n(z-z_0)^n$.

	\begin{enumerate}
		\ii If this power series converges at $z_1\neq z_0$ then it converges absolutely for all $z$ such that
		\[
			|z-z_0|<|z_1-z_0|
			.\]
		\ii If this power series diverges at $z_2\neq z_0$ then it diverges for all $z$ such that
		\[
			|z-z_0|>|z_2-z_0|
			.\]
	\end{enumerate}
}

\subsection{Radius of Convergence}
Consider a power series $\sum_{n=0}^\infty a_n(z-z_0)^n$. We conider the cases for convergence.

\begin{enumerate}
	\ii \textbf{Case 1}: The power series cvonverges at $z_0$ only. Then the radius of convergence is $0$.
	\ii \textbf{Case 2}: The power series converges for all $z$ in a disc $D$ of radius $R$ centered at $z_0$. Then the radius of convergence is $R$. The power series may or many no converge in $\partial D$.
	\ii \textbf{Case 3}: The power series converges for all $z$. Then the radius of convergence is $\infty$.
\end{enumerate}

\thm{}{
	Consider a power series $\sum_{n=0}^\infty a_n(z-z_0)^n$ with radius of convergence $R$. Then the series converges uniformly on any closed disc $\gamma_r$ of radius $r<R$ centered at $z_0$.
	\[
		\gamma_r = \lt\{z\in\CC: |z-z_0|=r\rt\}
		.\]
}

\section{Taylor Series}

\dfn{Taylor Series}{
	Consider a function $f$ holomorphic on a disc $D$ of radius $R$ centered at $z_0$. Then the Taylor series of $f$ centered at $z_0$ is
	\[
		f(z) = \sum_{n=0}^\infty a_n(z-z_0)^n
		.\]
	where
	\[
		a_n = \frac{f^{(n)}(z_0)}{n!} = \frac{1}{2\pi i} \oint_{\gamma_r} \frac{f(z)}{(z-z_0)^{n+1}}\,\dd{z}\quad \forall n\in\NN
		.\]
}
\nt{
	The Taylor series of $f(z)$ at $z=0$ is called the Maclaurin series of $f(z)$.
}

\thm{}{
	The radius of convergence $R$ of the Taylor series of $f$ at $z_0$ is given by $R=|z_T-z_0|$ where $z_T$ is the closest singularity of $f$ to $z_0$.
}
\nt{
	\begin{enumerate}
		\ii If $f$ is holomorphic on $\CC$ then $z_T$ is located at infinity hence $R=\infty$.
		\ii $f$ is holomorphic on an open set $\Omega$ if an only if the Taylor series of $f$ converges to $f$ on $\Omega$.
	\end{enumerate}
}

\nt{
	The Taylor expension of some common functions are
	\begin{align*}
		\frac{1}{1-z} & =\sum_{n=0}^\infty z^n                                          & R=1                        \\
		\frac{1}{1+z} & =\sum_{n=0}^\infty (-1)^nz^n                                    & R=1                        \\
		\Log(1+z)     & =\sum_{n=1}^\infty \frac{(-1)^{n+1}}{n}z^n                      & R=1                        \\
		e^z           & =\sum_{n=0}^\infty \frac{z^n}{n!}                               & R=\infty                   \\
		\sin(z)       & =\sum_{n=0}^\infty \frac{(-1)^n}{(2n+1)!}z^{2n+1}               & R=\infty                   \\
		\cos(z)       & =\sum_{n=0}^\infty \frac{(-1)^n}{(2n)!}z^{2n}                   & R=\infty                   \\
		\sinh(z)      & =\sum_{n=0}^\infty \frac{z^{2n+1}}{(2n+1)!}                     & R=\infty                   \\
		\cosh(z)      & =\sum_{n=0}^\infty \frac{z^{2n}}{(2n)!}                         & R=\infty                   \\
		(1+z)^\alpha  & =\sum_{n=0}^\infty  \prod_{k=0}^{n-1}(\alpha-k)  \frac{z^n}{n!} & R\text{ depends on }\alpha
	\end{align*}
}

\nt{
$z_0$ is a zero of order $n$ of $f$ if $f(z_0)=f'(z_0)=\cdots=f^{(n-1)}(z_0)=0$ and $f^{(n)}(z_0)\neq 0$.
}

\cor{}{
	Consider a function $h = f \cdot g$ where $f$ and $g$ are holomorphic around $z_0$. If $z_0$ is a zero of order $n$ of $f$ and $z_0$ is a zero of order $m$ of $g$ then $z_0$ is a zero of order $n+m$ of $h$.
}

\cor{}{
	Consider a function $f$ holomorphic in an open $\Omega$ containing $z_0$, such that $f\neq0$ in $\Omega$. If $z_0$ is a zero of $f$ then $z_0$ is an isolated zero.
}

\cor{}{
	Consider a function $f$ holomorphic in an open set $\Omega$. Let $K$ be compact set in $\Omega$ such that $f\neq0$ on $K$. Then $f$ admits a finite number
}

\dfn{Poles}{
	Consider 2 functions $f$ and $g$ holomorphic in a neighborhood of $z_0$. Let $h=\frac{f}{g}$. If $g(z_0)\neq z_0$ then $z_0$ is a regular point. However if $g(z_0)=0$ then we have several cases of study. Let $l$ be the order of the root of $f$ at $z_0$ and $u$ be the order of the root of $g$ at $z_0$.
	\begin{itemize}
		\ii If $l<u$ then $z_0$ is a pole of order $u-l$ of $h$.
		\ii If $l\geq u$ then $z_0$ is a removable singularity of $h$.
	\end{itemize}
}

\dfn{Memomorphic functions}{
	A function $f$ is meromorphic in an open set $\Omega$ if $\exists A \subset \Omega$
	\begin{enumerate}
		\ii $A$ admits no accumulation points
		\ii $f$ is holomorphic in $\Omega - A$.
		\ii Each point of $A$ is a pole of $f$.
	\end{enumerate}
}

\section{Laurent Series}

\dfn{Laurent Series}{
	Consider a function $f$ holomorphic in a disk $\gamma_r$ of center $z_0$ not including $z_0$ and radii $r<R$. Then the Laurent series of $f$ centered at $z_0$ is
	\[
		f(z) = \sum_{n=-\infty}^\infty a_n(z-z_0)^n
		.\]
	where
	\[
		a_n = \frac{1}{2\pi i} \oint_{\gamma_r} \frac{f(z)}{(z-z_0)^{n+1}}\,\dd{z}\quad \forall n\in\ZZ
		.\]
}

The negative terms of the Laurent series are called the principal part of the Laurent series, while the positive terms are called the regular part of the Laurent series.

The domain of convergence of the Laurent series is the punctured disk of center $z_0$ and radius $R=|z_T-z_0|$ where $z_T$ is the closest singularity of $f$ to $z_0$.

\dfn{Analytic Extensions}{
	Let $\gamma_r$ be a punctured disk of center $z_0$ and radius $r$. Let $f$ be a holomorphic function in $\gamma_r$(not necessarily holomorphic at $z_0$). Then the following statements are equivalent
	\begin{enumerate}
		\ii $f$ is bounded in the deleted neighborhood of $z_0$.
		\ii The Laurent expansion of $f$ at $z-0$ is a Taylor expansion.
		\ii $f$ admits an analytic extension at $z_0$.
	\end{enumerate}
}

\nt{
	If $f$ is holomorphic at $z_0$ then it is bounded in a neighborhood of $z_0$. So the Laurent expansion of $f$ at $z_0$ coincides with its Taylor expansion at $z_0$.
}

Let $f$ be a function expandable into a Laurent series at $z_0$, and non-holomorphic at $z_0$. This is equivalent to say $z_0$ is an isolated singular point of $f$. We distinguish then three situations for this singular point:

\begin{enumerate}
	\ii If the Laurent series of $f$ at $z_0$ has only positive terms, then $z_0$ is a removable singularity of $f$.
	\ii If the Laurent series of $f$ at $z_0$ has only a finite number $k$ of negative terms, then $z_0$ is a pole of $f$ of order $k$.
	\ii If the Laurent series of $f$ at $z_0$ has an infinite number of negative terms, then $z_0$ is either an essential singularity of $f$ or a pole of $f$.
\end{enumerate}

\chapter{Residues}

\dfn{Residue}{
	Let $f$ be a function analytic in a punctured open disk $V$ centred at $z_0$ ($f$ may not be analytic at $z_0$). Then $f$ is expandable into a Laurent series at $z_0$ and this Laurent series is given by:
	\[
		f(z) = \sum_{n=-\infty}^\infty a_n(z-z_0)^n = \cdots + \frac{a_{-1}}{z-z_0} + a_0 + a_1(z-z_0) + \cdots
		.\]
	where
	\[
		a_n = \frac{1}{2\pi i} \oint_{\gamma_r} \frac{f(z)}{(z-z_0)^{n+1}}\,\dd{z}\quad \forall n\in\ZZ
		.\]

	The coefficient $a_{-1}$ is called the residue of $f$ at $z_0$ and is denoted by $\Res(f,z_0)$.
	\[
		\oint_{\gamma_r} f(z)\,\dd{z} = 2\pi i a_{-1}
		.\]
}

\thm{Residue Theorem}{
	Let $\Gamma$ be a simple closed curve and $f$ be a function analytic in an open set containing $\Gamma$ except for a finite number of singularities $z_1,\ldots,z_n$ inside $\Gamma$. Then
	\[
		\oint_\Gamma f(z)\,\dd{z} = 2\pi i \sum_{k=1}^n \Res(f,z_k)
		.\]
}

\section{Calculating Residues}

\begin{enumerate}
	\ii \textbf{Case of a simple pole}: Suppose that $z_0$ is a simple pole of the function $f$. The Laurent series of $f$ at $z_0$ is
	\[
		f(z) = \frac{a_{-1}}{z-z_0} + a_0 + a_1(z-z_0) + \cdots
		.\]
	and
	\[
		a_{-1} = \Res(f,z_0) = \lim_{z\to z_0} (z-z_0)f(z)
		.\]
	\ii \textbf{Case of a pole of order $k$}: Suppose that $z_0$ is a pole of order $k$ of the function $f$. The Laurent series of $f$ at $z_0$ is
	\[
		f(z) = \frac{a_{-k}}{(z-z_0)^k} + \frac{a_{-k+1}}{(z-z_0)^{k-1}} + \cdots + \frac{a_{-1}}{z-z_0} + a_0 + a_1(z-z_0) + \cdots
		.\]

	Thus
	\[
		\Res(f,z_0) = \lim_{z\to z_0} \frac{1}{(k-1)!} \dv[k-1]{z} \lt[ (z-z_0)^k f(z) \rt]
		.\]
	\ii \textbf{Case of an essential singularity or a pole of high order}: In this case, to calculate the residue of $f$ at $z_0$, it is sufficient to expand $f$ into a Laurent series at $z_0$ and examine the coefficient of the term $(z-z_0)^{-1}$
\end{enumerate}

\mprop{}{
	In the particular case where $f(z)=\frac{P(z)}{Q(z)}$ with $P(z_0)\neq 0$ and $z_0$ being a simple zero of $Q(z)$, then
	\[
		\Res(f,z_0) = \frac{P(z_0)}{Q'(z_0)}
		.\]
}

\section{Applications of the Residue Theorem}

\subsection{Jordan’s lemmas}

\mlenma{}{
	Let $f$ be a continuous function in $\Omega=\{z\in\CC: 0<|z|<R, 0\leq \theta_1\leq \Arg(z)\leq \theta_2\leq 2\pi\}$, where $R>0$ and $\theta_1$ and $\theta_2$ are both fixed.

	Consider an arc of circle $\gamma_r$ of center $0$ and radius $r$ contained in $\Omega$. Then if
	\[
		\lim_{z\to0} zf(z) = 0 \quad\text{then}\quad  \lim_{r\to0} \int_{\gamma_r} f(z)\,\dd{z} = 0
		.\]
}
\mlenma{}{
	Let $f$ be a continuous function in $\Omega=\{z\in\CC: |z|>R, 0\leq \theta_1\leq \Arg(z)\leq \theta_2\leq 2\pi\}$, where $R>0$ and $\theta_1$ and $\theta_2$ are both fixed.
	Consider an arc of circle $\gamma_r$ of center $0$ and radius $r$ contained in $\Omega$. Then if
	\[
		\lim_{|z|\to\infty} zf(z) = 0 \quad\text{then}\quad  \lim_{r\to\infty} \int_{\gamma_r} f(z)\,\dd{z} = 0
		.\]
}

\mlenma{}{
	Let $f$ be a continuous function in $\Omega=\{z\in\CC: |z|>R, 0\leq \theta_1\leq \Arg(z)\leq \theta_2\leq \pi\}$, where $R>0$ and $\theta_1$ and $\theta_2$ are both fixed.
	Consider an arc of circle $\gamma_r$ of center $0$ and radius $r$ contained in $\Omega$, and let $m$ be a positive constant. Then if
	\[
		\lim_{|z|\to\infty} f(z) = 0 \quad\text{then}\quad \lim_{r\to\infty} \int_{\gamma_r} f(z)e^{imz}\,\dd{z} = 0
		.\]
}

\mlenma{}{
Consider an open disk $D$ centered at $z_0$ and a function $f$ holomorphic in $D - {z_0}$ such that $z_0$ is a simple pole of $f$. Let $\gamma_r$ be an \emph{arc} of a circle of center $z_0$ and radius $r$ contained in $D$ and of angle $\alpha$. Then
\[
	\lim_{r\to0} \int_{\gamma_r} f(z)\,\dd{z} = \alpha i \Res(f,z_0)
	.\]
}

\subsection{Evaluation of real integrals}

\subsubsection{Integrals of the form $\int_0^{2\pi} f(\cos\theta,\sin\theta)\,\dd{\theta}$}
This type of integral can be evaluated by the substitution $z=e^{i\theta}$, $\dd{z}=ie^{i\theta}\dd{\theta}$, $\cos\theta=\frac{1}{2}(z+\frac{1}{z})$, and $\sin\theta=\frac{1}{2i}(z-\frac{1}{z})$. Hence
\[
	\int_0^{2\pi} f(\cos\theta,\sin\theta)\,\dd{\theta} = \int_\gamma \frac{1}{iz} f\lt( \frac{1}{2}\lt( z+\frac{1}{z} \rt), \frac{1}{2i}\lt( z-\frac{1}{z} \rt) \rt)\,\dd{z}
	.\]

Then we can simply evaluate the integral using the residue theorem, Cauchy's integral formula, or the Cauchy's theorem.

\subsubsection{Integrals of the form $\int_{-\infty}^\infty f(x)\,\dd{x}$}

Where $f(x) = \frac{P(x)}{Q(x)}$ with $\deg Q(x) \geq \deg P(x) + 2$ and $Q(x)$ has no real roots, then we can evaluate the integral by considering the path $\Gamma$ consisting of the real axis and a semicircle of radius $R$ in the positive imaginary plane, where $R$ is chosen to be big enough such that $\Gamma$ encloses all the singularities of $f$

\subsubsection{Integrals of the form $\int_{-\infty}^\infty f(x)e^{imx}\,\dd{x}$, $\int_{-\infty}^\infty f(x)\cos(mx)\,\dd{x}$, and $\int_{-\infty}^\infty f(x)\sin(mx)\,\dd{x}$}

Where $f(x) = \frac{P(x)}{Q(x)}$ with $\deg Q(x) \geq \deg P(x) + 1$, $Q(x)$ has no real roots, and $m$ is a real positive number, then we can evaluate the integral by considering the same semicircle path as before.

\subsubsection{Integrals of the form $\int_{-\infty}^\infty \frac{f(x)}{x^\alpha}\,\dd{x}$}

Where $f(x) = \frac{P(x)}{Q(x)}$ with $\deg Q(x) \geq \deg P(x) + 1$, $Q(x)$ has no real roots, and $\alpha\in]0,1[$, then we integrate the function $g(z)=\frac{f(z)}{z^\alpha}$ where $z^\alpha=e^{\alpha\Log(z)}$ along the closed path $\Gamma = ABCDEFGHA$ shown below

\begin{tikzpicture}
	\draw[->] (-3,0) -- (3,0) node[right] {$\Re(z)$};

	\draw[->] (0,-3) -- (0,3) node[above] {$\Im(z)$};
	\draw[->] (0,0) -- (2,2) node[above] {Imma do it later};
	\draw (0,0) node[below left] {$0$};
\end{tikzpicture}

\section{Special Functions}

\dfn{Gamma funcction}{
The \emph{Gamma function} is defined as
\[
	\Gamma(z) = \int_0^\infty t^{z-1}e^{-t}\,\dd{t}
	.\]
}

Some properties of the Gamma function are
\begin{enumerate}
	\ii $\Gamma(z+1) = z\Gamma(z)$
	\ii $\Gamma(n+1) = n!$
	\ii $\Gamma(z)\Gamma(1-z) = \frac{\pi}{\sin(\pi z)}$
	\ii $\Gamma(\frac{1}{2}) = \sqrt{\pi}$
\end{enumerate}


\dfn{Beta function}{
The \emph{Beta function} is defined as
\[
	B(x,y) = \int_0^1 (1-t)^{x-1}t^{y-1}\,\dd{y}
	.\]
}
Some properties of the Beta function are
\begin{enumerate}
	\ii $B(x,y) = B(y,x)$
	\ii $B(x,y) = \frac{\Gamma(x)\Gamma(y)}{\Gamma(x+y)}$
	\ii $B(x,y) = 2\int_0^{\pi/2} \sin^{2x-1}\theta\cos^{2y-1}\theta\,\dd{\theta}$
	\ii $B(p,1-p) = \frac{\pi}{\sin(\pi p)}$
\end{enumerate}


\dfn{Gauss error functions}{
	The \emph{Gauss error functions} are defined as
	\[
		\erf(z) = \frac{2}{\sqrt{\pi}} \int_0^z e^{-t^2}\,\dd{t}
		\quad\text{and}\quad
		\erfc(z) = 1 - \erf(z)
		.\]
}

Some properties of the Gauss error functions are
\begin{enumerate}
	\ii $\erf(-z) = -\erf(z)$
	\ii $\lim_{z\to\infty} \erf(z) = 1$
	\ii $\lim_{z\to-\infty} \erf(z) = -1$
	\ii $\erf(z) + \erfc(z) = 1$
\end{enumerate}

\thm{Stirling's approximation}{
	\[
		n! \sim \sqrt{2\pi n}\lt( \frac{n}{e} \rt)^n
		.\]
}

\end{document}

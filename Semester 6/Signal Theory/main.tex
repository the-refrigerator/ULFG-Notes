\documentclass{report}

%%%%%%%%%%%%%%%%%%%%%%%%%%%%%%%%%
% PACKAGE IMPORTS
%%%%%%%%%%%%%%%%%%%%%%%%%%%%%%%%%


\usepackage[tmargin=2cm,rmargin=1in,lmargin=1in,margin=0.85in,bmargin=2cm,footskip=.2in]{geometry}
\usepackage{amsmath,amsfonts,amsthm,amssymb,mathtools}
\usepackage[varbb]{newpxmath}
\usepackage{xfrac}
\usepackage[makeroom]{cancel}
\usepackage{mathtools}
\usepackage{bookmark}
\usepackage{enumitem}
\usepackage{hyperref,theoremref}
\usepackage{xparse}
\hypersetup{
	pdftitle={Hamboola my beloved},
	colorlinks=true, linkcolor=doc!90,
	bookmarksnumbered=true,
	bookmarksopen=true
}
\usepackage[most,many,breakable]{tcolorbox}
\usepackage{xcolor}
\usepackage{varwidth}
\usepackage{varwidth}
\usepackage{etoolbox}
\usepackage{bm}
%\usepackage{authblk}
\usepackage{pgfplots}
\usepackage{nameref}
\usepackage{multicol,array}
\usepackage{tikz-cd}
\usepackage[ruled,vlined,linesnumbered]{algorithm2e}
\usepackage{comment} % enables the use of multi-line comments (\ifx \fi) 
\usepackage{import}
\usepackage{xifthen}
\usepackage{pdfpages}
\usepackage{transparent}
\usepackage{minted}
\usepackage{fontspec}
\usepackage{tasks}
\usepackage{chemfig}
\usepackage[version=4]{mhchem}
\usepackage{suffix}
\usepackage{tabularx}
\usepackage{subcaption}
\usepackage{physics}


% \setmonofont{SpaceMono Nerd Font}
\setminted{fontsize=\footnotesize}

\newcommand\mycommfont[1]{\footnotesize\ttfamily\textcolor{blue}{#1}}
\SetCommentSty{mycommfont}
\newcommand{\incfig}[1]{%
	\def\svgwidth{\columnwidth}
	\import{./figures/}{#1.pdf_tex}
}

\usepackage{tikzsymbols}
% \renewcommand\qedsymbol{$\Laughey$}
\pgfplotsset{compat=1.18}

%\usepackage{import}
%\usepackage{xifthen}
%\usepackage{pdfpages}
%\usepackage{transparent}


%%%%%%%%%%%%%%%%%%%%%%%%%%%%%%
% SELF MADE COLORS
%%%%%%%%%%%%%%%%%%%%%%%%%%%%%%



\definecolor{myg}{RGB}{56, 140, 70}
\definecolor{myb}{RGB}{45, 111, 177}
\definecolor{myr}{RGB}{199, 68, 64}
\definecolor{mytheorembg}{HTML}{F2F2F9}
\definecolor{mytheoremfr}{HTML}{00007B}
\definecolor{mylenmabg}{HTML}{FFFAF8}
\definecolor{mylenmafr}{HTML}{983b0f}
\definecolor{mypropbg}{HTML}{f2fbfc}
\definecolor{mypropfr}{HTML}{191971}
\definecolor{myexamplebg}{HTML}{F2FBF8}
\definecolor{myexamplefr}{HTML}{88D6D1}
\definecolor{myexampleti}{HTML}{2A7F7F}
\definecolor{mydefinitbg}{HTML}{E5E5FF}
\definecolor{mydefinitfr}{HTML}{3F3FA3}
\definecolor{notesgreen}{RGB}{0,162,0}
\definecolor{myp}{RGB}{197, 92, 212}
\definecolor{mygr}{HTML}{2C3338}
\definecolor{myred}{RGB}{127,0,0}
\definecolor{myyellow}{RGB}{169,121,69}
\definecolor{myexercisebg}{HTML}{F2FBF8}
\definecolor{myexercisefg}{HTML}{88D6D1}
\definecolor{codebg}{HTML}{0D1117}

%%%%%%%%%%%%%%%%%%%%%%%%%%%%
% TCOLORBOX SETUPS
%%%%%%%%%%%%%%%%%%%%%%%%%%%%

\setlength{\parindent}{0cm}
%================================
% THEOREM BOX
%================================

\tcbuselibrary{theorems,skins,hooks}
\newtcbtheorem[number within=section]{Theorem}{Theorem}
{%
	enhanced,
	breakable,
	colback = mytheorembg,
	frame hidden,
	boxrule = 0sp,
	borderline west = {2pt}{0pt}{mytheoremfr},
	sharp corners,
	detach title,
	before upper = \tcbtitle\par\smallskip,
	coltitle = mytheoremfr,
	fonttitle = \bfseries\sffamily,
	description font = \mdseries,
	separator sign none,
	segmentation style={solid, mytheoremfr},
}
{th}

\tcbuselibrary{theorems,skins,hooks}
\newtcbtheorem[number within=chapter]{theorem}{Theorem}
{%
	enhanced,
	breakable,
	colback = mytheorembg,
	frame hidden,
	boxrule = 0sp,
	borderline west = {2pt}{0pt}{mytheoremfr},
	sharp corners,
	detach title,
	before upper = \tcbtitle\par\smallskip,
	coltitle = mytheoremfr,
	fonttitle = \bfseries\sffamily,
	description font = \mdseries,
	separator sign none,
	segmentation style={solid, mytheoremfr},
}
{th}


\tcbuselibrary{theorems,skins,hooks}
\newtcolorbox{Theoremcon}
{%
	enhanced
	,breakable
	,colback = mytheorembg
	,frame hidden
	,boxrule = 0sp
	,borderline west = {2pt}{0pt}{mytheoremfr}
	,sharp corners
	,description font = \mdseries
	,separator sign none
}

%================================
% Corollery
%================================
\tcbuselibrary{theorems,skins,hooks}
\newtcbtheorem[number within=section]{Corollary}{Corollary}
{%
	enhanced
	,breakable
	,colback = myp!10
	,frame hidden
	,boxrule = 0sp
	,borderline west = {2pt}{0pt}{myp!85!black}
	,sharp corners
	,detach title
	,before upper = \tcbtitle\par\smallskip
	,coltitle = myp!85!black
	,fonttitle = \bfseries\sffamily
	,description font = \mdseries
	,separator sign none
	,segmentation style={solid, myp!85!black}
}
{th}
\tcbuselibrary{theorems,skins,hooks}
\newtcbtheorem[number within=chapter]{corollary}{Corollary}
{%
	enhanced
	,breakable
	,colback = myp!10
	,frame hidden
	,boxrule = 0sp
	,borderline west = {2pt}{0pt}{myp!85!black}
	,sharp corners
	,detach title
	,before upper = \tcbtitle\par\smallskip
	,coltitle = myp!85!black
	,fonttitle = \bfseries\sffamily
	,description font = \mdseries
	,separator sign none
	,segmentation style={solid, myp!85!black}
}
{th}


%================================
% LENMA
%================================

\tcbuselibrary{theorems,skins,hooks}
\newtcbtheorem[number within=section]{Lenma}{Lenma}
{%
	enhanced,
	breakable,
	colback = mylenmabg,
	frame hidden,
	boxrule = 0sp,
	borderline west = {2pt}{0pt}{mylenmafr},
	sharp corners,
	detach title,
	before upper = \tcbtitle\par\smallskip,
	coltitle = mylenmafr,
	fonttitle = \bfseries\sffamily,
	description font = \mdseries,
	separator sign none,
	segmentation style={solid, mylenmafr},
}
{th}

\tcbuselibrary{theorems,skins,hooks}
\newtcbtheorem[number within=chapter]{lenma}{Lenma}
{%
	enhanced,
	breakable,
	colback = mylenmabg,
	frame hidden,
	boxrule = 0sp,
	borderline west = {2pt}{0pt}{mylenmafr},
	sharp corners,
	detach title,
	before upper = \tcbtitle\par\smallskip,
	coltitle = mylenmafr,
	fonttitle = \bfseries\sffamily,
	description font = \mdseries,
	separator sign none,
	segmentation style={solid, mylenmafr},
}
{th}


%================================
% PROPOSITION
%================================

\tcbuselibrary{theorems,skins,hooks}
\newtcbtheorem[number within=section]{Prop}{Proposition}
{%
	enhanced,
	breakable,
	colback = mypropbg,
	frame hidden,
	boxrule = 0sp,
	borderline west = {2pt}{0pt}{mypropfr},
	sharp corners,
	detach title,
	before upper = \tcbtitle\par\smallskip,
	coltitle = mypropfr,
	fonttitle = \bfseries\sffamily,
	description font = \mdseries,
	separator sign none,
	segmentation style={solid, mypropfr},
}
{th}

\tcbuselibrary{theorems,skins,hooks}
\newtcbtheorem[number within=chapter]{prop}{Proposition}
{%
	enhanced,
	breakable,
	colback = mypropbg,
	frame hidden,
	boxrule = 0sp,
	borderline west = {2pt}{0pt}{mypropfr},
	sharp corners,
	detach title,
	before upper = \tcbtitle\par\smallskip,
	coltitle = mypropfr,
	fonttitle = \bfseries\sffamily,
	description font = \mdseries,
	separator sign none,
	segmentation style={solid, mypropfr},
}
{th}


%================================
% CLAIM
%================================

\tcbuselibrary{theorems,skins,hooks}
\newtcbtheorem[number within=section]{claim}{Claim}
{%
	enhanced
	,breakable
	,colback = myg!10
	,frame hidden
	,boxrule = 0sp
	,borderline west = {2pt}{0pt}{myg}
	,sharp corners
	,detach title
	,before upper = \tcbtitle\par\smallskip
	,coltitle = myg!85!black
	,fonttitle = \bfseries\sffamily
	,description font = \mdseries
	,separator sign none
	,segmentation style={solid, myg!85!black}
}
{th}



%================================
% Exercise
%================================

\tcbuselibrary{theorems,skins,hooks}
\newtcbtheorem[number within=section]{Exercise}{Exercise}
{%
	enhanced,
	breakable,
	colback = myexercisebg,
	frame hidden,
	boxrule = 0sp,
	borderline west = {2pt}{0pt}{myexercisefg},
	sharp corners,
	detach title,
	before upper = \tcbtitle\par\smallskip,
	coltitle = myexercisefg,
	fonttitle = \bfseries\sffamily,
	description font = \mdseries,
	separator sign none,
	segmentation style={solid, myexercisefg},
}
{th}

\tcbuselibrary{theorems,skins,hooks}
\newtcbtheorem[number within=chapter]{exercise}{Exercise}
{%
	enhanced,
	breakable,
	colback = myexercisebg,
	frame hidden,
	boxrule = 0sp,
	borderline west = {2pt}{0pt}{myexercisefg},
	sharp corners,
	detach title,
	before upper = \tcbtitle\par\smallskip,
	coltitle = myexercisefg,
	fonttitle = \bfseries\sffamily,
	description font = \mdseries,
	separator sign none,
	segmentation style={solid, myexercisefg},
}
{th}

%================================
% EXAMPLE BOX
%================================

\newtcbtheorem[number within=section]{Example}{Example}
{%
	colback = myexamplebg
	,breakable
	,colframe = myexamplefr
	,coltitle = myexampleti
	,boxrule = 1pt
	,sharp corners
	,detach title
	,before upper=\tcbtitle\par\smallskip
	,fonttitle = \bfseries
	,description font = \mdseries
	,separator sign none
	,description delimiters parenthesis
}
{ex}

\newtcbtheorem[number within=chapter]{example}{Example}
{%
	colback = myexamplebg
	,breakable
	,colframe = myexamplefr
	,coltitle = myexampleti
	,boxrule = 1pt
	,sharp corners
	,detach title
	,before upper=\tcbtitle\par\smallskip
	,fonttitle = \bfseries
	,description font = \mdseries
	,separator sign none
	,description delimiters parenthesis
}
{ex}

%================================
% DEFINITION BOX
%================================

\newtcbtheorem[number within=section]{Definition}{Definition}{enhanced,
	before skip=2mm,after skip=2mm, colback=red!5,colframe=red!80!black,boxrule=0.5mm,
	attach boxed title to top left={xshift=1cm,yshift*=1mm-\tcboxedtitleheight}, varwidth boxed title*=-3cm,
	boxed title style={frame code={
					\path[fill=tcbcolback]
					([yshift=-1mm,xshift=-1mm]frame.north west)
					arc[start angle=0,end angle=180,radius=1mm]
					([yshift=-1mm,xshift=1mm]frame.north east)
					arc[start angle=180,end angle=0,radius=1mm];
					\path[left color=tcbcolback!60!black,right color=tcbcolback!60!black,
						middle color=tcbcolback!80!black]
					([xshift=-2mm]frame.north west) -- ([xshift=2mm]frame.north east)
					[rounded corners=1mm]-- ([xshift=1mm,yshift=-1mm]frame.north east)
					-- (frame.south east) -- (frame.south west)
					-- ([xshift=-1mm,yshift=-1mm]frame.north west)
					[sharp corners]-- cycle;
				},interior engine=empty,
		},
	fonttitle=\bfseries,
	title={#2},#1}{def}
\newtcbtheorem[number within=chapter]{definition}{Definition}{enhanced,
	before skip=2mm,after skip=2mm, colback=red!5,colframe=red!80!black,boxrule=0.5mm,
	attach boxed title to top left={xshift=1cm,yshift*=1mm-\tcboxedtitleheight}, varwidth boxed title*=-3cm,
	boxed title style={frame code={
					\path[fill=tcbcolback]
					([yshift=-1mm,xshift=-1mm]frame.north west)
					arc[start angle=0,end angle=180,radius=1mm]
					([yshift=-1mm,xshift=1mm]frame.north east)
					arc[start angle=180,end angle=0,radius=1mm];
					\path[left color=tcbcolback!60!black,right color=tcbcolback!60!black,
						middle color=tcbcolback!80!black]
					([xshift=-2mm]frame.north west) -- ([xshift=2mm]frame.north east)
					[rounded corners=1mm]-- ([xshift=1mm,yshift=-1mm]frame.north east)
					-- (frame.south east) -- (frame.south west)
					-- ([xshift=-1mm,yshift=-1mm]frame.north west)
					[sharp corners]-- cycle;
				},interior engine=empty,
		},
	fonttitle=\bfseries,
	title={#2},#1}{def}



%================================
% Solution BOX
%================================

\makeatletter
\newtcbtheorem{question}{Question}{enhanced,
	breakable,
	colback=white,
	colframe=myb!80!black,
	attach boxed title to top left={yshift*=-\tcboxedtitleheight},
	fonttitle=\bfseries,
	title={#2},
	boxed title size=title,
	boxed title style={%
			sharp corners,
			rounded corners=northwest,
			colback=tcbcolframe,
			boxrule=0pt,
		},
	underlay boxed title={%
			\path[fill=tcbcolframe] (title.south west)--(title.south east)
			to[out=0, in=180] ([xshift=5mm]title.east)--
			(title.center-|frame.east)
			[rounded corners=\kvtcb@arc] |-
			(frame.north) -| cycle;
		},
	#1
}{def}
\makeatother

%================================
% SOLUTION BOX
%================================

\makeatletter
\newtcolorbox{solution}{enhanced,
	breakable,
	colback=white,
	colframe=myg!80!black,
	attach boxed title to top left={yshift*=-\tcboxedtitleheight},
	title=Solution,
	boxed title size=title,
	boxed title style={%
			sharp corners,
			rounded corners=northwest,
			colback=tcbcolframe,
			boxrule=0pt,
		},
	underlay boxed title={%
			\path[fill=tcbcolframe] (title.south west)--(title.south east)
			to[out=0, in=180] ([xshift=5mm]title.east)--
			(title.center-|frame.east)
			[rounded corners=\kvtcb@arc] |-
			(frame.north) -| cycle;
		},
}
\makeatother

%================================
% Question BOX
%================================

\makeatletter
\newtcbtheorem{qstion}{Question}{enhanced,
	breakable,
	colback=white,
	colframe=mygr,
	attach boxed title to top left={yshift*=-\tcboxedtitleheight},
	fonttitle=\bfseries,
	title={#2},
	boxed title size=title,
	boxed title style={%
			sharp corners,
			rounded corners=northwest,
			colback=tcbcolframe,
			boxrule=0pt,
		},
	underlay boxed title={%
			\path[fill=tcbcolframe] (title.south west)--(title.south east)
			to[out=0, in=180] ([xshift=5mm]title.east)--
			(title.center-|frame.east)
			[rounded corners=\kvtcb@arc] |-
			(frame.north) -| cycle;
		},
	#1
}{def}
\makeatother

\newtcbtheorem[number within=chapter]{wconc}{Wrong Concept}{
	breakable,
	enhanced,
	colback=white,
	colframe=myr,
	arc=0pt,
	outer arc=0pt,
	fonttitle=\bfseries\sffamily\large,
	colbacktitle=myr,
	attach boxed title to top left={},
	boxed title style={
			enhanced,
			skin=enhancedfirst jigsaw,
			arc=3pt,
			bottom=0pt,
			interior style={fill=myr}
		},
	#1
}{def}



%================================
% NOTE BOX
%================================

\usetikzlibrary{hobby}
\usetikzlibrary{arrows,calc,shadows.blur}
\tcbuselibrary{skins}
\newtcolorbox{note}[1][]{%
	enhanced jigsaw,
	colback=gray!20!white,%
	colframe=gray!80!black,
	size=small,
	boxrule=1pt,
	title=\textbf{Note:-},
	halign title=flush center,
	coltitle=black,
	breakable,
	drop shadow=black!50!white,
	attach boxed title to top left={xshift=1cm,yshift=-\tcboxedtitleheight/2,yshifttext=-\tcboxedtitleheight/2},
	minipage boxed title=1.5cm,
	boxed title style={%
			colback=white,
			size=fbox,
			boxrule=1pt,
			boxsep=2pt,
			underlay={%
					\coordinate (dotA) at ($(interior.west) + (-0.5pt,0)$);
					\coordinate (dotB) at ($(interior.east) + (0.5pt,0)$);
					\begin{scope}
						\clip (interior.north west) rectangle ([xshift=3ex]interior.east);
						\filldraw [white, blur shadow={shadow opacity=60, shadow yshift=-.75ex}, rounded corners=2pt] (interior.north west) rectangle (interior.south east);
					\end{scope}
					\begin{scope}[gray!80!black]
						\fill (dotA) circle (2pt);
						\fill (dotB) circle (2pt);
					\end{scope}
				},
		},
	#1,
}

%%%%%%%%%%%%%%%%%%%%%%%%%%%%%%
% SELF MADE COMMANDS
%%%%%%%%%%%%%%%%%%%%%%%%%%%%%%


\newcommand{\thm}[2]{\begin{Theorem}{#1}{}#2\end{Theorem}}
\newcommand{\cor}[2]{\begin{Corollary}{#1}{}#2\end{Corollary}}
\newcommand{\mlenma}[2]{\begin{Lenma}{#1}{}#2\end{Lenma}}
\newcommand{\mprop}[2]{\begin{Prop}{#1}{}#2\end{Prop}}
\newcommand{\clm}[3]{\begin{claim}{#1}{#2}#3\end{claim}}
\newcommand{\wc}[2]{\begin{wconc}{#1}{}\setlength{\parindent}{1cm}#2\end{wconc}}
\newcommand{\thmcon}[1]{\begin{Theoremcon}{#1}\end{Theoremcon}}
\newcommand{\ex}[2]{\begin{Example}{#1}{}#2\end{Example}}
\newcommand{\dfn}[2]{\begin{Definition}[colbacktitle=red!75!black]{#1}{}#2\end{Definition}}
\newcommand{\dfnc}[2]{\begin{definition}[colbacktitle=red!75!black]{#1}{}#2\end{definition}}
\newcommand{\qs}[2]{\begin{question}{#1}{}#2\end{question}}
\newcommand{\pf}[2]{\begin{myproof}[#1]#2\end{myproof}}
\newcommand{\nt}[1]{\begin{note}#1\end{note}}

\newcommand*\circled[1]{\tikz[baseline=(char.base)]{
		\node[shape=circle,draw,inner sep=1pt] (char) {#1};}}
\newcommand\getcurrentref[1]{%
	\ifnumequal{\value{#1}}{0}
	{??}
	{\the\value{#1}}%
}
\newcommand{\getCurrentSectionNumber}{\getcurrentref{section}}
\newenvironment{myproof}[1][\proofname]{%
	\proof[\bfseries #1: ]%
}{\endproof}

\newcommand{\mclm}[2]{\begin{myclaim}[#1]#2\end{myclaim}}
\newenvironment{myclaim}[1][\claimname]{\proof[\bfseries #1: ]}{}

\newcounter{mylabelcounter}

\makeatletter
\newcommand{\setword}[2]{%
	\phantomsection
	#1\def\@currentlabel{\unexpanded{#1}}\label{#2}%
}
\makeatother




\tikzset{
	symbol/.style={
			draw=none,
			every to/.append style={
					edge node={node [sloped, allow upside down, auto=false]{$#1$}}}
		}
}


% deliminators
% \DeclarePairedDelimiter{\abs}{\lvert}{\rvert}
% \DeclarePairedDelimiter{\norm}{\lVert}{\rVert}

\DeclarePairedDelimiter{\ceil}{\lceil}{\rceil}
\DeclarePairedDelimiter{\floor}{\lfloor}{\rfloor}
\DeclarePairedDelimiter{\round}{\lfloor}{\rceil}

\newsavebox\diffdbox
\newcommand{\slantedromand}{{\mathpalette\makesl{d}}}
\newcommand{\makesl}[2]{%
	\begingroup
	\sbox{\diffdbox}{$\mathsurround=0pt#1\mathrm{#2}$}%
	\pdfsave
	\pdfsetmatrix{1 0 0.2 1}%
	\rlap{\usebox{\diffdbox}}%
	\pdfrestore
	\hskip\wd\diffdbox
	\endgroup
}
% \newcommand{\dd}[1][]{\ensuremath{\mathop{}\!\ifstrempty{#1}{%
% 			\slantedromand\@ifnextchar^{\hspace{0.2ex}}{\hspace{0.1ex}}}%
% 		{\slantedromand\hspace{0.2ex}^{#1}}}}
\ProvideDocumentCommand\dv{o m g}{%
	\ensuremath{%
		\IfValueTF{#3}{%
			\IfNoValueTF{#1}{%
				\frac{\dd #2}{\dd #3}%
			}{%
				\frac{\dd^{#1} #2}{\dd #3^{#1}}%
			}%
		}{%
			\IfNoValueTF{#1}{%
				\frac{\dd}{\dd #2}%
			}{%
				\frac{\dd^{#1}}{\dd #2^{#1}}%
			}%
		}%
	}%
}
\DeclareDocumentCommand\pdv{ s o m g g d() }
{ % Partial derivative
	% s: star for \flatfrac flat derivative
	% o: optional n for nth derivative
	% m: mandatory (x in df/dx)
	% g: optional (f in df/dx)
	% g: optional (y in d^2f/dxdy)
	% d: long-form d/dx(...)
	\IfBooleanTF{#1}
	{\let\fractype\flatfrac}
	{\let\fractype\frac}
	\IfNoValueTF{#4}
	{
		\IfNoValueTF{#6}
		{\fractype{\partial \IfNoValueTF{#2}{}{^{#2}}}{\partial #3\IfNoValueTF{#2}{}{^{#2}}}}
		{\fractype{\partial \IfNoValueTF{#2}{}{^{#2}}}{\partial #3\IfNoValueTF{#2}{}{^{#2}}} \argopen(#6\argclose)}
	}
	{
		\IfNoValueTF{#5}
		{\fractype{\partial \IfNoValueTF{#2}{}{^{#2}} #3}{\partial #4\IfNoValueTF{#2}{}{^{#2}}}}
		{\fractype{\partial^2 #3}{\partial #4 \partial #5}}
	}
}
% \providecommand*{\pdv}[3][]{\frac{\partial^{#1}#2}{\partial#3^{#1}}}
%  - others
\DeclareMathOperator{\Lap}{\mathcal{L}}
\DeclareMathOperator{\Var}{Var} % varience
\DeclareMathOperator{\Cov}{Cov} % covarience
\DeclareMathOperator{\E}{E} % expected

% Since the amsthm package isn't loaded

% I prefer the slanted \leq
\let\oldleq\leq % save them in case they're every wanted
\let\oldgeq\geq
\renewcommand{\leq}{\leqslant}
\renewcommand{\geq}{\geqslant}

% % redefine matrix env to allow for alignment, use r as default
% \renewcommand*\env@matrix[1][r]{\hskip -\arraycolsep
%     \let\@ifnextchar\new@ifnextchar
%     \array{*\c@MaxMatrixCols #1}}


%\usepackage{framed}
%\usepackage{titletoc}
%\usepackage{etoolbox}
%\usepackage{lmodern}


%\patchcmd{\tableofcontents}{\contentsname}{\sffamily\contentsname}{}{}

%\renewenvironment{leftbar}
%{\def\FrameCommand{\hspace{6em}%
%		{\color{myyellow}\vrule width 2pt depth 6pt}\hspace{1em}}%
%	\MakeFramed{\parshape 1 0cm \dimexpr\textwidth-6em\relax\FrameRestore}\vskip2pt%
%}
%{\endMakeFramed}

%\titlecontents{chapter}
%[0em]{\vspace*{2\baselineskip}}
%{\parbox{4.5em}{%
%		\hfill\Huge\sffamily\bfseries\color{myred}\thecontentspage}%
%	\vspace*{-2.3\baselineskip}\leftbar\textsc{\small\chaptername~\thecontentslabel}\\\sffamily}
%{}{\endleftbar}
%\titlecontents{section}
%[8.4em]
%{\sffamily\contentslabel{3em}}{}{}
%{\hspace{0.5em}\nobreak\itshape\color{myred}\contentspage}
%\titlecontents{subsection}
%[8.4em]
%{\sffamily\contentslabel{3em}}{}{}  
%{\hspace{0.5em}\nobreak\itshape\color{myred}\contentspage}



%%%%%%%%%%%%%%%%%%%%%%%%%%%%%%%%%%%%%%%%%%%
% TABLE OF CONTENTS
%%%%%%%%%%%%%%%%%%%%%%%%%%%%%%%%%%%%%%%%%%%

\usepackage{tikz}
\definecolor{doc}{RGB}{0,60,110}
\usepackage{titletoc}
\contentsmargin{0cm}
\titlecontents{chapter}[3.7pc]
{\addvspace{30pt}%
	\begin{tikzpicture}[remember picture, overlay]%
		\draw[fill=doc!60,draw=doc!60] (-7,-.1) rectangle (-0.9,.5);%
		\pgftext[left,x=-3.5cm,y=0.2cm]{\color{white}\Large\sc\bfseries Chapter\ \thecontentslabel};%
	\end{tikzpicture}\color{doc!60}\large\sc\bfseries}%
{}
{}
{\;\titlerule\;\large\sc\bfseries Page \thecontentspage
	\begin{tikzpicture}[remember picture, overlay]
		\draw[fill=doc!60,draw=doc!60] (2pt,0) rectangle (4,0.1pt);
	\end{tikzpicture}}%
\titlecontents{section}[3.7pc]
{\addvspace{2pt}}
{\contentslabel[\thecontentslabel]{2pc}}
{}
{\hfill\small \thecontentspage}
[]
\titlecontents*{subsection}[3.7pc]
{\addvspace{-1pt}\small}
{}
{}
{\ --- \small\thecontentspage}
[ \textbullet\ ][]

\makeatletter
\renewcommand{\tableofcontents}{%
	\chapter*{%
	  \vspace*{-20\p@}%
	  \begin{tikzpicture}[remember picture, overlay]%
		  \pgftext[right,x=15cm,y=0.2cm]{\color{doc!60}\Huge\sc\bfseries \contentsname};%
		  \draw[fill=doc!60,draw=doc!60] (13,-.75) rectangle (20,1);%
		  \clip (13,-.75) rectangle (20,1);
		  \pgftext[right,x=15cm,y=0.2cm]{\color{white}\Huge\sc\bfseries \contentsname};%
	  \end{tikzpicture}}%
	\@starttoc{toc}}
\makeatother

%From M275 "Topology" at SJSU
\newcommand{\id}{\mathrm{id}}
\newcommand{\taking}[1]{\xrightarrow{#1}}
\newcommand{\inv}{^{-1}}

%From M170 "Introduction to Graph Theory" at SJSU
\DeclareMathOperator{\diam}{diam}
\DeclareMathOperator{\ord}{ord}
\newcommand{\defeq}{\overset{\mathrm{def}}{=}}

%From the USAMO .tex files
\newcommand{\ts}{\textsuperscript}
\newcommand{\dg}{^\circ}
\newcommand{\ii}{\item}

% % From Math 55 and Math 145 at Harvard
% \newenvironment{subproof}[1][Proof]{%
% \begin{proof}[#1] \renewcommand{\qedsymbol}{$\blacksquare$}}%
% {\end{proof}}

\newcommand{\liff}{\leftrightarrow}
\newcommand{\lthen}{\rightarrow}
\newcommand{\opname}{\operatorname}
\newcommand{\surjto}{\twoheadrightarrow}
\newcommand{\injto}{\hookrightarrow}
\newcommand{\On}{\mathrm{On}} % ordinals
% \newcommand{\EE}{\mathbb{E}} % Expectance
\DeclareMathOperator{\img}{im} % Image
\DeclareMathOperator{\Img}{Im} % Image
\DeclareMathOperator{\coker}{coker} % Cokernel
\DeclareMathOperator{\Coker}{Coker} % Cokernel
\DeclareMathOperator{\Ker}{Ker} % Kernel
\DeclareMathOperator{\rank}{rank}
\DeclareMathOperator{\Spec}{Spec} % spectrum
\DeclareMathOperator{\Tr}{Tr} % trace
\DeclareMathOperator{\pr}{pr} % projection
\DeclareMathOperator{\ext}{ext} % extension
\DeclareMathOperator{\pred}{pred} % predecessor
\DeclareMathOperator{\dom}{dom} % domain
\DeclareMathOperator{\ran}{ran} % range
\DeclareMathOperator{\Hom}{Hom} % homomorphism
\DeclareMathOperator{\Mor}{Mor} % morphisms
\DeclareMathOperator{\End}{End} % endomorphism
% \DeclareMathOperator{\Pr}{Pr} % probability
% \DeclareMathOperator{\Var}{Var} % variance

\newcommand{\eps}{\epsilon}
\newcommand{\veps}{\varepsilon}
\newcommand{\ol}{\overline}
\newcommand{\ul}{\underline}
\newcommand{\wt}{\widetilde}
\newcommand{\wh}{\widehat}
\newcommand{\vocab}[1]{\textbf{\color{blue} #1}}
\providecommand{\half}{\frac{1}{2}}
\newcommand{\dang}{\measuredangle} %% Directed angle
\newcommand{\ray}[1]{\overrightarrow{#1}}
\newcommand{\seg}[1]{\overline{#1}}
\newcommand{\arc}[1]{\wideparen{#1}}
\DeclareMathOperator{\cis}{cis}
\DeclareMathOperator*{\lcm}{lcm}
\DeclareMathOperator*{\argmin}{arg min}
\DeclareMathOperator*{\argmax}{arg max}
\newcommand{\cycsum}{\sum_{\mathrm{cyc}}}
\newcommand{\symsum}{\sum_{\mathrm{sym}}}
\newcommand{\cycprod}{\prod_{\mathrm{cyc}}}
\newcommand{\symprod}{\prod_{\mathrm{sym}}}
\newcommand{\Qed}{\begin{flushright}\qed\end{flushright}}
\newcommand{\parinn}{\setlength{\parindent}{1cm}}
\newcommand{\parinf}{\setlength{\parindent}{0cm}}
% \newcommand{\norm}{\|\cdot\|}
\newcommand{\inorm}{\norm_{\infty}}
\newcommand{\opensets}{\{V_{\alpha}\}_{\alpha\in I}}
\newcommand{\oset}{V_{\alpha}}
\newcommand{\opset}[1]{V_{\alpha_{#1}}}
\newcommand{\lub}{\text{lub}}
\newcommand{\del}[2]{\frac{\partial #1}{\partial #2}}
\newcommand{\Del}[3]{\frac{\partial^{#1} #2}{\partial^{#1} #3}}
\newcommand{\deld}[2]{\dfrac{\partial #1}{\partial #2}}
\newcommand{\Deld}[3]{\dfrac{\partial^{#1} #2}{\partial^{#1} #3}}
\newcommand{\lm}{\lambda}
\newcommand{\uin}{\mathbin{\rotatebox[origin=c]{90}{$\in$}}}
\newcommand{\usubset}{\mathbin{\rotatebox[origin=c]{90}{$\subset$}}}
\newcommand{\lt}{\left}
\newcommand{\rt}{\right}
\newcommand{\bs}[1]{\boldsymbol{#1}}
\newcommand{\exs}{\exists}
\newcommand{\st}{\strut}
\newcommand{\dps}[1]{\displaystyle{#1}}
\newcommand{\va}[1]{\vec{\bm{\mathrm{#1}}}}
\WithSuffix\newcommand\va*[1]{\vec{\bm{#1}}}
\newcommand{\vb}[1]{\bm{\mathrm{#1}}}
\WithSuffix\newcommand\vb*[1]{\bm{#1}}
\newcommand{\vu}[1]{\hat{\bm{\mathrm{#1}}}}
\WithSuffix\newcommand\vu*[1]{\hat{\bm{#1}}}
\renewcommand{\dd}[1]{\mathrm{d}#1}
\renewcommand{\Re}{\mathrm{Re}}
\renewcommand{\Im}{\mathrm{Im}}
\DeclareMathOperator{\tr}{tr}
\newcommand{\csin}[1]{\mintinline{csharp}|#1|}
\renewcommand{\Pr}[1]{\mathrm{Pr}\lt( #1 \rt)}
\renewcommand{\Var}[1]{\mathrm{Var}\lt( #1 \rt)}
\newcommand{\cov}[1]{\mathrm{cov}\lt( #1 \rt)}

\newcommand{\sol}{\setlength{\parindent}{0cm}\textbf{\textit{Solution:}}\setlength{\parindent}{1cm} }
\newcommand{\solve}[1]{\setlength{\parindent}{0cm}\textbf{\textit{Solution: }}\setlength{\parindent}{1cm}#1 \Qed}

% Things Lie
\newcommand{\kb}{\mathfrak b}
\newcommand{\kg}{\mathfrak g}
\newcommand{\kh}{\mathfrak h}
\newcommand{\kn}{\mathfrak n}
\newcommand{\ku}{\mathfrak u}
\newcommand{\kz}{\mathfrak z}
\DeclareMathOperator{\Ext}{Ext} % Ext functor
\DeclareMathOperator{\Tor}{Tor} % Tor functor
\newcommand{\gl}{\opname{\mathfrak{gl}}} % frak gl group
\renewcommand{\sl}{\opname{\mathfrak{sl}}} % frak sl group chktex 6

% More script letters etc.
\newcommand{\SA}{\mathcal A}
\newcommand{\SB}{\mathcal B}
\newcommand{\SC}{\mathcal C}
\newcommand{\SF}{\mathcal F}
\newcommand{\SG}{\mathcal G}
\newcommand{\SH}{\mathcal H}
\newcommand{\OO}{\mathcal O}

\newcommand{\SCA}{\mathscr A}
\newcommand{\SCB}{\mathscr B}
\newcommand{\SCC}{\mathscr C}
\newcommand{\SCD}{\mathscr D}
\newcommand{\SCE}{\mathscr E}
\newcommand{\SCF}{\mathscr F}
\newcommand{\SCG}{\mathscr G}
\newcommand{\SCH}{\mathscr H}

% Mathfrak primes
\newcommand{\km}{\mathfrak m}
\newcommand{\kp}{\mathfrak p}
\newcommand{\kq}{\mathfrak q}

% number sets
\newcommand{\RR}[1][]{\ensuremath{\ifstrempty{#1}{\mathbb{R}}{\mathbb{R}^{#1}}}}
\newcommand{\NN}[1][]{\ensuremath{\ifstrempty{#1}{\mathbb{N}}{\mathbb{N}^{#1}}}}
\newcommand{\ZZ}[1][]{\ensuremath{\ifstrempty{#1}{\mathbb{Z}}{\mathbb{Z}^{#1}}}}
\newcommand{\QQ}[1][]{\ensuremath{\ifstrempty{#1}{\mathbb{Q}}{\mathbb{Q}^{#1}}}}
\newcommand{\CC}[1][]{\ensuremath{\ifstrempty{#1}{\mathbb{C}}{\mathbb{C}^{#1}}}}
\newcommand{\PP}[1][]{\ensuremath{\ifstrempty{#1}{\mathbb{P}}{\mathbb{P}^{#1}}}}
\newcommand{\HH}[1][]{\ensuremath{\ifstrempty{#1}{\mathbb{H}}{\mathbb{H}^{#1}}}}
\newcommand{\FF}[1][]{\ensuremath{\ifstrempty{#1}{\mathbb{F}}{\mathbb{F}^{#1}}}}
% expected value
\newcommand{\EE}{\ensuremath{\mathbb{E}}}
\newcommand{\charin}{\text{ char }}
\DeclareMathOperator{\sign}{sign}
\DeclareMathOperator{\Aut}{Aut}
\DeclareMathOperator{\Inn}{Inn}
\DeclareMathOperator{\Syl}{Syl}
\DeclareMathOperator{\Gal}{Gal}
\DeclareMathOperator{\GL}{GL} % General linear group
\DeclareMathOperator{\SL}{SL} % Special linear group

%---------------------------------------
% BlackBoard Math Fonts :-
%---------------------------------------

%Captital Letters
\newcommand{\bbA}{\mathbb{A}}	\newcommand{\bbB}{\mathbb{B}}
\newcommand{\bbC}{\mathbb{C}}	\newcommand{\bbD}{\mathbb{D}}
\newcommand{\bbE}{\mathbb{E}}	\newcommand{\bbF}{\mathbb{F}}
\newcommand{\bbG}{\mathbb{G}}	\newcommand{\bbH}{\mathbb{H}}
\newcommand{\bbI}{\mathbb{I}}	\newcommand{\bbJ}{\mathbb{J}}
\newcommand{\bbK}{\mathbb{K}}	\newcommand{\bbL}{\mathbb{L}}
\newcommand{\bbM}{\mathbb{M}}	\newcommand{\bbN}{\mathbb{N}}
\newcommand{\bbO}{\mathbb{O}}	\newcommand{\bbP}{\mathbb{P}}
\newcommand{\bbQ}{\mathbb{Q}}	\newcommand{\bbR}{\mathbb{R}}
\newcommand{\bbS}{\mathbb{S}}	\newcommand{\bbT}{\mathbb{T}}
\newcommand{\bbU}{\mathbb{U}}	\newcommand{\bbV}{\mathbb{V}}
\newcommand{\bbW}{\mathbb{W}}	\newcommand{\bbX}{\mathbb{X}}
\newcommand{\bbY}{\mathbb{Y}}	\newcommand{\bbZ}{\mathbb{Z}}

%---------------------------------------
% MathCal Fonts :-
%---------------------------------------

%Captital Letters
\newcommand{\mcA}{\mathcal{A}}	\newcommand{\mcB}{\mathcal{B}}
\newcommand{\mcC}{\mathcal{C}}	\newcommand{\mcD}{\mathcal{D}}
\newcommand{\mcE}{\mathcal{E}}	\newcommand{\mcF}{\mathcal{F}}
\newcommand{\mcG}{\mathcal{G}}	\newcommand{\mcH}{\mathcal{H}}
\newcommand{\mcI}{\mathcal{I}}	\newcommand{\mcJ}{\mathcal{J}}
\newcommand{\mcK}{\mathcal{K}}	\newcommand{\mcL}{\mathcal{L}}
\newcommand{\mcM}{\mathcal{M}}	\newcommand{\mcN}{\mathcal{N}}
\newcommand{\mcO}{\mathcal{O}}	\newcommand{\mcP}{\mathcal{P}}
\newcommand{\mcQ}{\mathcal{Q}}	\newcommand{\mcR}{\mathcal{R}}
\newcommand{\mcS}{\mathcal{S}}	\newcommand{\mcT}{\mathcal{T}}
\newcommand{\mcU}{\mathcal{U}}	\newcommand{\mcV}{\mathcal{V}}
\newcommand{\mcW}{\mathcal{W}}	\newcommand{\mcX}{\mathcal{X}}
\newcommand{\mcY}{\mathcal{Y}}	\newcommand{\mcZ}{\mathcal{Z}}


%---------------------------------------
% Bold Math Fonts :-
%---------------------------------------

%Captital Letters
\newcommand{\bmA}{\boldsymbol{A}}	\newcommand{\bmB}{\boldsymbol{B}}
\newcommand{\bmC}{\boldsymbol{C}}	\newcommand{\bmD}{\boldsymbol{D}}
\newcommand{\bmE}{\boldsymbol{E}}	\newcommand{\bmF}{\boldsymbol{F}}
\newcommand{\bmG}{\boldsymbol{G}}	\newcommand{\bmH}{\boldsymbol{H}}
\newcommand{\bmI}{\boldsymbol{I}}	\newcommand{\bmJ}{\boldsymbol{J}}
\newcommand{\bmK}{\boldsymbol{K}}	\newcommand{\bmL}{\boldsymbol{L}}
\newcommand{\bmM}{\boldsymbol{M}}	\newcommand{\bmN}{\boldsymbol{N}}
\newcommand{\bmO}{\boldsymbol{O}}	\newcommand{\bmP}{\boldsymbol{P}}
\newcommand{\bmQ}{\boldsymbol{Q}}	\newcommand{\bmR}{\boldsymbol{R}}
\newcommand{\bmS}{\boldsymbol{S}}	\newcommand{\bmT}{\boldsymbol{T}}
\newcommand{\bmU}{\boldsymbol{U}}	\newcommand{\bmV}{\boldsymbol{V}}
\newcommand{\bmW}{\boldsymbol{W}}	\newcommand{\bmX}{\boldsymbol{X}}
\newcommand{\bmY}{\boldsymbol{Y}}	\newcommand{\bmZ}{\boldsymbol{Z}}
%Small Letters
\newcommand{\bma}{\boldsymbol{a}}	\newcommand{\bmb}{\boldsymbol{b}}
\newcommand{\bmc}{\boldsymbol{c}}	\newcommand{\bmd}{\boldsymbol{d}}
\newcommand{\bme}{\boldsymbol{e}}	\newcommand{\bmf}{\boldsymbol{f}}
\newcommand{\bmg}{\boldsymbol{g}}	\newcommand{\bmh}{\boldsymbol{h}}
\newcommand{\bmi}{\boldsymbol{i}}	\newcommand{\bmj}{\boldsymbol{j}}
\newcommand{\bmk}{\boldsymbol{k}}	\newcommand{\bml}{\boldsymbol{l}}
\newcommand{\bmm}{\boldsymbol{m}}	\newcommand{\bmn}{\boldsymbol{n}}
\newcommand{\bmo}{\boldsymbol{o}}	\newcommand{\bmp}{\boldsymbol{p}}
\newcommand{\bmq}{\boldsymbol{q}}	\newcommand{\bmr}{\boldsymbol{r}}
\newcommand{\bms}{\boldsymbol{s}}	\newcommand{\bmt}{\boldsymbol{t}}
\newcommand{\bmu}{\boldsymbol{u}}	\newcommand{\bmv}{\boldsymbol{v}}
\newcommand{\bmw}{\boldsymbol{w}}	\newcommand{\bmx}{\boldsymbol{x}}
\newcommand{\bmy}{\boldsymbol{y}}	\newcommand{\bmz}{\boldsymbol{z}}

%---------------------------------------
% Scr Math Fonts :-
%---------------------------------------

\newcommand{\sA}{{\mathscr{A}}}   \newcommand{\sB}{{\mathscr{B}}}
\newcommand{\sC}{{\mathscr{C}}}   \newcommand{\sD}{{\mathscr{D}}}
\newcommand{\sE}{{\mathscr{E}}}   \newcommand{\sF}{{\mathscr{F}}}
\newcommand{\sG}{{\mathscr{G}}}   \newcommand{\sH}{{\mathscr{H}}}
\newcommand{\sI}{{\mathscr{I}}}   \newcommand{\sJ}{{\mathscr{J}}}
\newcommand{\sK}{{\mathscr{K}}}   \newcommand{\sL}{{\mathscr{L}}}
\newcommand{\sM}{{\mathscr{M}}}   \newcommand{\sN}{{\mathscr{N}}}
\newcommand{\sO}{{\mathscr{O}}}   \newcommand{\sP}{{\mathscr{P}}}
\newcommand{\sQ}{{\mathscr{Q}}}   \newcommand{\sR}{{\mathscr{R}}}
\newcommand{\sS}{{\mathscr{S}}}   \newcommand{\sT}{{\mathscr{T}}}
\newcommand{\sU}{{\mathscr{U}}}   \newcommand{\sV}{{\mathscr{V}}}
\newcommand{\sW}{{\mathscr{W}}}   \newcommand{\sX}{{\mathscr{X}}}
\newcommand{\sY}{{\mathscr{Y}}}   \newcommand{\sZ}{{\mathscr{Z}}}


%---------------------------------------
% Math Fraktur Font
%---------------------------------------

%Captital Letters
\newcommand{\mfA}{\mathfrak{A}}	\newcommand{\mfB}{\mathfrak{B}}
\newcommand{\mfC}{\mathfrak{C}}	\newcommand{\mfD}{\mathfrak{D}}
\newcommand{\mfE}{\mathfrak{E}}	\newcommand{\mfF}{\mathfrak{F}}
\newcommand{\mfG}{\mathfrak{G}}	\newcommand{\mfH}{\mathfrak{H}}
\newcommand{\mfI}{\mathfrak{I}}	\newcommand{\mfJ}{\mathfrak{J}}
\newcommand{\mfK}{\mathfrak{K}}	\newcommand{\mfL}{\mathfrak{L}}
\newcommand{\mfM}{\mathfrak{M}}	\newcommand{\mfN}{\mathfrak{N}}
\newcommand{\mfO}{\mathfrak{O}}	\newcommand{\mfP}{\mathfrak{P}}
\newcommand{\mfQ}{\mathfrak{Q}}	\newcommand{\mfR}{\mathfrak{R}}
\newcommand{\mfS}{\mathfrak{S}}	\newcommand{\mfT}{\mathfrak{T}}
\newcommand{\mfU}{\mathfrak{U}}	\newcommand{\mfV}{\mathfrak{V}}
\newcommand{\mfW}{\mathfrak{W}}	\newcommand{\mfX}{\mathfrak{X}}
\newcommand{\mfY}{\mathfrak{Y}}	\newcommand{\mfZ}{\mathfrak{Z}}
%Small Letters
\newcommand{\mfa}{\mathfrak{a}}	\newcommand{\mfb}{\mathfrak{b}}
\newcommand{\mfc}{\mathfrak{c}}	\newcommand{\mfd}{\mathfrak{d}}
\newcommand{\mfe}{\mathfrak{e}}	\newcommand{\mff}{\mathfrak{f}}
\newcommand{\mfg}{\mathfrak{g}}	\newcommand{\mfh}{\mathfrak{h}}
\newcommand{\mfi}{\mathfrak{i}}	\newcommand{\mfj}{\mathfrak{j}}
\newcommand{\mfk}{\mathfrak{k}}	\newcommand{\mfl}{\mathfrak{l}}
\newcommand{\mfm}{\mathfrak{m}}	\newcommand{\mfn}{\mathfrak{n}}
\newcommand{\mfo}{\mathfrak{o}}	\newcommand{\mfp}{\mathfrak{p}}
\newcommand{\mfq}{\mathfrak{q}}	\newcommand{\mfr}{\mathfrak{r}}
\newcommand{\mfs}{\mathfrak{s}}	\newcommand{\mft}{\mathfrak{t}}
\newcommand{\mfu}{\mathfrak{u}}	\newcommand{\mfv}{\mathfrak{v}}
\newcommand{\mfw}{\mathfrak{w}}	\newcommand{\mfx}{\mathfrak{x}}
\newcommand{\mfy}{\mathfrak{y}}	\newcommand{\mfz}{\mathfrak{z}}


\title{\Huge{Signal Theory}\\Semester 6}
\author{\huge{Ahmad Abu Zainab}}
\date{}

\newcommand{\scalar}[1]{\lt\langle #1 \rt\rangle}
\newcommand{\Ft}[1]{\SF_f\!\lt\{ #1 \rt\}}
\newcommand{\Fti}[1]{\SF_f^{-1}\!\lt\{ #1 \rt\}}

\DeclareMathOperator{\rect}{rect}
\DeclareMathOperator{\tri}{tri}
\DeclareMathOperator{\sinc}{sinc}
\DeclareMathOperator{\sgn}{sgn}

\pgfmathdeclarefunction{u}{1}{
\pgfmathparse{
((#1<0)  * 0 )  +
((#1>=0) * 1 )
}
}

\pgfmathdeclarefunction{rect}{2}{
\pgfmathparse{
((#1>= -#2/2) * (#1<=#2/2) * 1 )
}
}

\pgfmathdeclarefunction{tri}{2}{
\pgfmathparse{
((#1>= -#2) * (#1<=#2) * (1-abs(#1)/#2) ) 
}
}

\pgfmathdeclarefunction{sinc}{1}{
\pgfmathparse{
    sin(deg(pi*#1))/(pi*#1)
}
}

\pgfmathdeclarefunction{sgn}{1}{
\pgfmathparse{
    (#1>0)  -  (#1<0)
}
}


\begin{document}

\maketitle
\newpage% or \cleardoublepage
% \pdfbookmark[<level>]{<title>}{<dest>}
\pdfbookmark[section]{\contentsname}{toc}
\tableofcontents
\pagebreak

\chapter{Definition of a Signal}

A signal is a function that conveys information about a phenomenon. A signal can be a function of time, space, or any other variable. A signal can be continuous or discrete.
Examples of signals include:
\begin{itemize}
	\ii Speech
	\ii Images
	\ii Audio
	\ii Video
	\ii Temperature
	\ii Pressure
	\ii Voltage
	\ii Current
	\ii etc.
\end{itemize}

\section{Classification of Signals}

\begin{description}
	\ii[Continuous-Time Signals] A signal that is defined for all values of time.
	\ii[Discrete-Time Signals] A signal that is defined only at discrete values of time.
	\ii[Analog Signals] A signal that can take any value in a given range.
	\ii[Digital Signals] A signal that can take only a finite number of values.
	\ii[Periodic Signals] A signal that repeats itself after a certain period of time.
	\ii[Energy Signals] A signal that has finite energy.
	\[
		E \coloneq \int_{-\infty}^{\infty} \abs{x(t)}^2 \dd{t}
		.\]
	\ii[Power Signals] A signal that has finite power.
	\[
		P_{(t_1,t_2)} \coloneq \lim_{T\to\infty} \frac{1}{2T} \int_{-T}^{T} \abs{x(t)}^2 \dd{t}
		.\]
	\ii [Even/Odd Signals] A signal is even if $x(t) = x(-t)$ and odd if $x(t) = -x(-t)$. Any signal can be represented as the sum of an even and odd signal.
	\[
		x(t) = \underbrace{\frac{x(t) + x(-t)}{2}}_{\text{Even}} + \underbrace{\frac{x(t) - x(-t)}{2}}_{\text{Odd}}
		.\]
\end{description}

\section{Operations on Signals}

\subsection{Unary Operations}

\begin{description}
	\ii[Inverse] The inverse of a signal $x(t)$ is $x(-t)$.
	\nt{
		The inverse of $x(t-1)$ is $x(-t-1)$.
	}
	\ii[Time Shift] The time shift of a signal $x(t)$ by $t_0$ is $x(t+t_0)$. If $t_0$ is negative then the signal is shifted to the left.
	\nt{
		The time shift of $x(-t-1)$ by $1$ is $x(-(t+1)-1)$.
	}
	\ii[Time Scaling] The time scaling of a signal $x(t)$ by $a$ is $x(at)$. If $a$ is greater than $1$ then the signal is compressed and if $a$ is less than $1$ then the signal is expanded.
\end{description}

\ex{}{
	Given a signal $x(t)$, to get the signal $x(-2t+4)$ we have 2 methods:
	\begin{enumerate}
		\ii $\displaystyle x(t) \xrightarrow{\text{Shift to the left}} x(t+4) \xrightarrow{\text{Inverse}} x(-t+4) \xrightarrow{\text{Time Scaling}} x(-2t+4)$
		\ii $\displaystyle x(t) \xrightarrow{\text{Time Scaling}} x(2t) \xrightarrow{\text{Inverse}} x(-2t) \xrightarrow{\text{Shift to the right}} x(-2(t-2)) = x(-2t+4)$
	\end{enumerate}
}

\subsection{Binary Operations}

\begin{description}
	\ii[Convolution]
	\[
		z(t) = x(t) * y(t) \coloneq \int_{-\infty}^{\infty} x(u)\,y(t-u)\dd{u}
		.\]

	The convolution operation is

	\begin{enumerate}
		\ii Commutative

		\[
			x(t) * y(t) = y(t) * x(t)
			.\]

		\ii Associative

		\[
			(x(t) * y(t)) * z(t) = x(t) * (y(t) * z(t)) = x(t) * y(t) * z(t)
			.\]

		\ii Distributive over addition

		\[
			x(t) * (y(t) + z(t)) = x(t) * y(t) + x(t) * z(t)
			.\]

		\nt{
			Generally speaking,
			\[
				x(at) * y(at) \neq a(x(t) * y(t)) \quad \text{for } a \neq 1
				.\]
		}
	\end{enumerate}

	\nt{
		When convolving two signals aim to have the signal $y(t-u)$ be the signal with the simplest $t$ input.
	}

	\ii[Scalar Product]

	\[
		\scalar{x(t), y^*(t)} \coloneq \int_{t_1}^{t_2} x(t)\, y^*(t) \dd{t}
		.\]

	If the scalar product is zero, then the two signals are orthogonal in the interval $[t_1, t_2]$.
	The scalar product is not commutative.
	\[
		\scalar{x(t), y^*(t)} \neq \scalar{y(t), x^*(t)} = \int_{t_1}^{t_2} y(t)\, x^*(t) \dd{t} = \lt( \int_{t_1}^{t_2} x(t)\, y^*(t) \dd{t} \rt)^* = \scalar{x(t), y^*(t)}^*
		.\]

	\nt{
		\begin{align*}
			(A+B)^* & = A^* + B^* \\
			(AB)^*  & = A^*B^*
		\end{align*}
	}

	\ii[Correlation]

	\[
		\varphi_{xy}(\tau) \coloneq \int_{-\infty}^{\infty} x^*(u)\, y(u+\tau) \dd{u}
		.\]

	The correlation operation is not commutative.
	\[
		\varphi_{xy}(\tau) \neq \varphi_{yx}(\tau)
		.\]

	\begin{align*}
		x(t) & \neq y(t) \quad \text{intercorrelation} \\
		x(t) & = y(-t) \quad \text{autocorrelation}
	\end{align*}
	\nt{
		\[
			\varphi_{xy}(\tau) = x^*(-\tau) * y(\tau)
			.\]
	}
\end{description}

\section{Particular Signals}

\begin{description}
	\ii[Unit Step Signal]

	\[
		u(t) \coloneq \begin{cases}
			1 & t \geq 0 \\
			0 & t < 0
		\end{cases}
	\]

	\begin{figure}[H]
		\centering
		\begin{tikzpicture}
			\begin{axis}[
					axis lines = center,
					xlabel = $t$,
					ylabel = $u(t)$,
					legend pos = north west,
					xmin = -2,
					xmax = 2,
					ymin = -0.5,
					ymax = 1.25,
					xtick = {0},
					ytick = {0,1},
					ymajorgrids=true,
					grid,
					grid style=dashed,
				]

				\addplot[domain=-2:2, samples=500, cyan, very thick] {u(x)};
			\end{axis}
		\end{tikzpicture}
	\end{figure}

	\ii[Rectangular Signal]

	\[
		\rect\lt(\frac{t}{T}\rt) \coloneq \begin{cases}
			1 & -\frac{T}{2} \leq t \leq \frac{T}{2} \\
			0 & \text{otherwise}
		\end{cases}
	\]

	\begin{figure}[H]
		\centering
		\begin{tikzpicture}
			\begin{axis}[
					axis lines = center,
					xlabel = $t$,
					ylabel = $\text{rect}\lt(\frac{t}{T}\rt)$,
					legend pos = north west,
					xmin = -2,
					xmax = 2,
					ymin = -0.5,
					ymax = 1.25,
					xtick = {-1,1},
					xticklabels = {$-\frac{T}{2}$,$\frac{T}{2}$},
					ytick = {0,1},
				]

				\addplot[domain=-2:2, samples=500, cyan, very thick] {rect(x,2)};

				\draw[dashed, thin, <->] (axis cs:-1,-0.25) -- (axis cs:1,-0.25) node[below right, midway] {$T$};
			\end{axis}
		\end{tikzpicture}
	\end{figure}

	\nt{
		The rect function is typically used to sample a signal over a period of time. The signal is sampled at the points where the rect function is equal to 1. For example
		\[
			x(t) \, \text{rect}\lt(\frac{t-2}{4}\rt)
			.\]

		This signal is sampled in the interval $[2,6]$.
	}

	\ii[Triangular Signal]

	\[
		\tri\lt( \frac{t}{T} \rt) \coloneq \begin{cases}
			\frac{t}{T} + 1  & -T \leq t \leq 0 \\
			-\frac{t}{T} + 1 & 0 \leq t \leq T  \\
			0                & \text{otherwise}
		\end{cases} =
		\begin{cases}
			1 - \frac{\abs{t}}{T} & -T \leq t \leq T \\
			0                     & \text{otherwise}
		\end{cases}
	\]

	\begin{figure}[H]
		\centering
		\begin{tikzpicture}
			\begin{axis}[
					axis lines = center,
					xlabel = $t$,
					ylabel = $\text{tri}\lt(\frac{t}{T}\rt)$,
					legend pos = north west,
					xmin = -2,
					xmax = 2,
					ymin = -0.5,
					ymax = 1.25,
					xtick = {-1,1},
					xticklabels = {$-T$,$T$},
					ytick = {0,1},
				]

				\addplot[domain=-2:2, samples=500, cyan, very thick] {tri(x,1)};

				\draw[dashed, thin, <->] (axis cs:-1,-0.25) -- (axis cs:1,-0.25) node[below right, midway] {$2T$};
			\end{axis}
		\end{tikzpicture}
	\end{figure}

	\ii[Sinc]

	\[
		\sinc(t) \coloneq \begin{cases}
			\frac{\sin(\pi t)}{\pi t} & t \neq 0 \\
			1                         & t = 0
		\end{cases}
	\]

	\nt{
		\[
			\sinc(x) = \sinc(-x)
			.\]
	}

	\begin{figure}[H]
		\centering
		\begin{tikzpicture}
			\begin{axis}[
					axis lines = center,
					xlabel = $t$,
					ylabel = $\sinc(tT)$,
					legend pos = north west,
					xmin = -5,
					xmax = 5,
					ymin = -0.5,
					ymax = 1.25,
					xtick = {-4,-3,-2,-1,1,2,3,4},
					xticklabels = {$-4T$,$-3T$,$-2T$,$-T$,$T$,$2T$,$3T$,$4T$},
					ytick = {0,1},
					width = 0.6\textwidth,
				]

				\addplot[domain=-5:5, samples=300, cyan, very thick] {sinc(x)};
			\end{axis}
		\end{tikzpicture}
	\end{figure}

	\ii[Sign Function]

	\[
		\sgn(t) \coloneq \begin{cases}
			1  & t > 0 \\
			0  & t = 0 \\
			-1 & t < 0
		\end{cases}
		.\]

	\begin{figure}[H]
		\centering
		\begin{tikzpicture}
			\begin{axis}[
					axis lines = center,
					xlabel = $t$,
					ylabel = $\sgn(t)$,
					legend pos = north west,
					xmin = -2,
					xmax = 2,
					ymin = -1.5,
					ymax = 1.25,
					xtick = {0},
					ytick = {-1,0,1},
				]

				\addplot[domain=-2:2, samples=500, cyan, very thick] {sgn(x)};
			\end{axis}
		\end{tikzpicture}
	\end{figure}

	\nt{
		\begin{align*}
			\sgn(t) & = \frac{u(t) - u(-t)}{2} \\
			u(t)    & = \frac{\sgn(t) + 1}{2}  \\
		\end{align*}
	}

\end{description}

\section{Other Quantities}

\begin{description}
	\ii[Mean]
	\[
		\bar{x} \coloneq \frac{1}{t_2-t_1}\int_{t_1}^{t_2} x(t) \dd{t}
		.\]

	Over the entire domain
	\[
		\bar{x} \coloneq \lim_{T\to\infty}\int_{-T/2}^{T/2}x(t) \dd{t}
		.\]

	\ii[Energy]
	\[
		E_{(t_1,t_2)} \coloneq \int_{t_1}^{t_2} \abs{x(t)}^2 \dd{t}
		.\]

	\ii[Power]
	\[
		P_{(t_1,t_2)} \coloneq \frac{1}{t_2-t_1} \int_{t_1}^{t_2} \abs{x(t)}^2 \dd{t}
		.\]
\end{description}

\ex{}{
	\begin{enumerate}
		\ii For $u(t)$
		\begin{align*}
			\bar{u} & = \lim_{T\to\infty} \frac{1}{T} \int_{-T/2}^{T/2} u(t) \dd{t} \\
			        & = \lim_{T\to\infty} \frac{1}{T} \int_{0}^{T/2} 1 \dd{t}       \\
			        & = \lim_{T\to\infty} \frac{1}{T} \frac{T}{2}                   \\
			        & = \frac{1}{2}
		\end{align*}

		\begin{align*}
			\int_{-T/2}^{T/2} \abs{u(t)}^2 \dd{t} & = \int_{0}^{T/2} 1 \dd{t} = \frac{T}{2} \\
		\end{align*}

		\begin{align*}
			E_{(-\infty,\infty)} & = \lim_{T\to\infty} \int_{-T/2}^{T/2} \abs{u(t)}^2 \dd{t} = \lim_{T\to\infty} \frac{T}{2} = \infty                              \\
			P_{(-\infty,\infty)} & = \lim_{T\to\infty} \frac{1}{T} \int_{-T/2}^{T/2} \abs{u(t)}^2 \dd{t} = \lim_{T\to\infty} \frac{1}{T} \frac{T}{2} = \frac{1}{2}
		\end{align*}

		\ii For $\sgn(t)$
		Since the signal is odd, the mean is zero.
		\begin{align*}
			E_{(-\infty,\infty)} & = \lim_{T\to\infty} \int_{-T/2}^{T/2} \abs{\sgn(t)}^2 \dd{t}             \\
			                     & = \lim_{T\to\infty} \int_{-T/2}^{T/2} 1 \dd{t}                           \\
			                     & = \lim_{T\to\infty} \frac{T}{2}                                          \\
			                     & = \infty                                                                 \\
			P_{(-\infty,\infty)} & = \lim_{T\to\infty} \frac{1}{T} \int_{-T/2}^{T/2} \abs{\sgn(t)}^2 \dd{t} \\
			                     & = \lim_{T\to\infty} \frac{1}{T} \int_{-T/2}^{T/2} 1 \dd{t}               \\
			                     & = \lim_{T\to\infty} \frac{1}{T} \frac{T}{2}                              \\
			                     & = \frac{1}{2}
		\end{align*}
	\end{enumerate}
}

\ex{Convolution}{
	Consider the 2 signals
	\[
		x(t) = \rect\lt(t\rt) \quad \text{and} \quad y(t) = \rect\lt(t\rt)
		.\]

	\[
		z(t) = x(t) * y(t) = \int_{-\infty}^{\infty} \rect\lt(u\rt)\, \rect\lt(t-u\rt) \dd{u}
		.\]

	Since the rect function is even
	\[
		\int_{-\infty}^{\infty} \rect\lt(u\rt)\, \rect\lt(u-t\rt) \dd{u}
		.\]

	\begin{figure}[H]
		\centering
		\begin{tikzpicture}
			\begin{axis}[
					axis lines = center,
					xlabel = $u$,
					legend pos = north west,
					xmin = -3,
					xmax = 3,
					ymin = -0.5,
					ymax = 1.25,
					xtick = {0},
					ytick = {0,1},
				]

				\addplot[domain=-3:3, samples=500, cyan, very thick] {rect(x,2)};
				\addplot[domain=-3:3, samples=500, blue, thick, densely dashed] {rect(x-1.25,2)};

				\node[anchor=north] at (axis cs:0.25,0) {$t-\sfrac{1}{2}$};
			\end{axis}
		\end{tikzpicture}
	\end{figure}
	The integral only really takes a value when the two rect functions overlap.\\

	The boundaries of the integral are $t-\sfrac{1}{2}$ and $\sfrac{1}{2}$ when $0\leq t\leq 1$.

	\[
		z(t)  = \int_{t-1/2}^{1/2} 1 \dd{u} = 1 - t
		.\]

	\begin{figure}[H]
		\centering
		\begin{tikzpicture}
			\begin{axis}[
					axis lines = center,
					xlabel = $u$,
					legend pos = north west,
					xmin = -3,
					xmax = 3,
					ymin = -0.5,
					ymax = 1.25,
					xtick = {0},
					ytick = {0,1},
				]

				\addplot[domain=-3:3, samples=500, cyan, very thick] {rect(x,2)};
				\addplot[domain=-3:3, samples=500, blue, thick, densely dashed] {rect(x+1.25,2)};

				\node[anchor=north] at (axis cs:-0.25,0) {$t+\sfrac{1}{2}$};
			\end{axis}
		\end{tikzpicture}
	\end{figure}

	Similarly, the boundaries of the integral are $t+\sfrac{1}{2}$ and $\sfrac{1}{2}$ when $-1\leq t\leq 0$.

	\[
		z(t) = \int_{1/2}^{t+1/2} 1 \dd{u} = 1 + t
		.\]

	Thus
	\[
		z(t) = \begin{cases}
			1-t & 0\leq t\leq 1    \\
			1+t & -1\leq t\leq 0   \\
			0   & \text{otherwise}
		\end{cases}
		= \tri\lt(t\rt)
		.\]
}

\chapter{Deterministic Signals}

\section{Fourier Transform}

Recall that a Fourier series is a representation of a periodic signal as a sum of sines and cosines.

\[
	x(t) = \frac{a_0}{2} + \sum_{n=1}^{\infty} \underbrace{\lt( a_n\cos\lt(\frac{2\pi nt}{T}\rt) + b_n\sin\lt(\frac{2\pi nt}{T}\rt) \rt)}_{
		A_n\cos\lt(n\omega_0t + \phi_n\rt)
	}
	.\]

\begin{align*}
	a_n & = \frac{2}{T}\int_{T} x(t)\cos\lt(\frac{2\pi nt}{T}\rt) \dd{t} \\
	b_n & = \frac{2}{T}\int_{T} x(t)\sin\lt(\frac{2\pi nt}{T}\rt) \dd{t}
\end{align*}

where $\omega_0 = \frac{2\pi}{T}$.\\

And in complex form
\[
	x(t) = \sum_{n=-\infty}^{\infty} \lambda_n e^{jn\omega_0t}
	.\]

Where
\[
	\lambda_n = \frac{1}{T}\int_{T} x(t)e^{-jn\omega_0t} \dd{t}
	.\]

Consider an interval $[-T,T]$ for the signal $x(t)$. We then assume that the function is periodic with period $2T$ over the rest of the domain (the section $[-T,T]$ will repeat infinitely). Since the signal is periodic we can represent it using a Fourier series $x_\text{rep}(t,T)$.
As $T$ approaches infinity then we obtain the original signal periodic over it's entire domain

\[
	\lim_{T\to\infty} x_\text{rep}(t,T) = x(t)
	.\]
\[
	x_\text{rep}(t,T) = \sum_{n=-\infty}^{\infty} \frac{1}{2T} \int_{-T}^{T} x_\text{rep}(t,T)e^{-j\frac{n\pi t}{T}} \dd{t} e^{j\frac{n\pi t}{T}}
	.\]

Since we have the amplitude if the signal at each frequency, we can represent the signal in the frequency domain (i.e. the sum of frequencies that make up the signal). This is the Fourier transform.

\[
	\Ft{x(t)} = \hat{x}(f) = X(f) \coloneq \int_{-\infty}^{\infty} x(t)e^{-j2\pi ft} \dd{t}
	.\]

\[
	\Fti{X(f)} = x(t) \coloneq \int_{-\infty}^{\infty} X(f)e^{j2\pi ft} \dd{f}
	.\]

A signal has a Fourier transform if the signal belongs to $\bbL^2$ (i.e. the signal has finite energy).
\[
	\bbL^2 = \lt\{ x(t)  \Big/ E \text{ is finite} \rt\}
	.\]

\ex{}{
	Consider the signal $x(t) = \rect\lt(t\rt)$.

	\begin{align*}
		\Ft{x(t)}= \text{Rect}\lt(f\rt) & = \int_{-\infty}^{\infty} \rect\lt(t\rt)e^{-j2\pi ft} \dd{t} \\
		                                & = \int_{-\sfrac{1}{2}}^{\sfrac{1}{2}} e^{-j2\pi ft} \dd{t}   \\
		                                & = \frac{e^{-j\pi f} - e^{j\pi f}}{j2\pi f}                   \\
		                                & = \sinc(f)
	\end{align*}

	Since we have a division by $f$, we have to take a special case for $f=0$.
	\begin{align*}
		\Ft{x(t)} & = \int_{-\infty}^{\infty} \dd{t} \\
		          & = 1
	\end{align*}

	The signal is mostly constant so the Fourier transform is highest closest to zero and the farther away from zero the lower the amplitude of the associated frequency.
}

\subsection{Properties of the Fourier Transform}

Here we assume that the signals $x(t),y(t),\dots\,\in\bbL^2$

\begin{description}
	\ii[Linearity]
	\[
		\Ft{ax(t) + by(t)} = a\hat{x}(f) + b\hat{y}(f) \quad \forall a,b \in \RR
		.\]

	\ii[Inverse]
	\[
		\Ft{x(-t)} = \hat{x}(-f)
		.\]

	\ii[Time Shift]
	\[
		\Ft{x(t-t_0)} = e^{-j2\pi ft_0}\hat{x}(f)
		.\]

	\ii[Modulation]
	\[
		\Ft{e^{j2\pi f_0t}x(t)} = \hat{x}(f-f_0)
		.\]
	\ii[Differentiation]
	\begin{align*}
		\Ft{\dv{x(t)}{t}}    & = j2\pi f\hat{x}(f)     \\
		\Ft{\dv[n]{x(t)}{t}} & = (j2\pi f)^n\hat{x}(f)
	\end{align*}
	\ii[Symmetry]
	\[
		\hat{x}(f) = \abs{\hat{x}(f)} e^{j \varphi(f)}
		.\]
	\[
		\begin{rcases}
			\text{even: } & \abs{\hat{x}(f)} = \abs{\hat{x}(-f)} \\
			\text{odd: }  & \varphi(f) = -\varphi(-f)
		\end{rcases} \implies \underset{\text{Hermitian Symmetry}}{\hat{x}(f) = \hat{x}^*(-f)}
	\]
	or
	\begin{align*}
		\text{even: } & \Re(f) = \Re(-f)  \\
		\text{odd: }  & \Im(f) = -\Im(-f)
	\end{align*}
	\ii[Convolution]
	\[
		\Ft{x(t) * y(t)} = \hat{x}(f)\cdot\hat{y}(f)
		.\]
	\ii[Product]
	\[
		\Ft{x(t)y(t)} = \hat{x}(f) * \hat{y}(f)
		.\]
\end{description}

\thm{Parseval's Theorem}{
	Given a signal $x(t)$ with Fourier transform $\hat{x}(f)$. The energy of the signal in the time domain is equal to the energy of the signal in the frequency domain.
	\[
		E = \int_{-\infty}^{\infty} \abs{x(t)}^2 \dd{t} = \int_{-\infty}^{\infty} \abs{\hat{x}(f)}^2 \dd{f}
		.\]

	\[
		\underset{\text{Energy Density}}{\dv{E}{f} = \abs{\hat{x}(f)}^2}
		.\]
}

\end{document}

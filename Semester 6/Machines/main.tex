\documentclass{report}

\input{../template/preamble}
\input{../template/macros}
\input{../template/letterfonts}

\usetikzlibrary{angles}

\title{\Huge{Electric Machines}\\Semester 6}
\author{Ahmad Abu Zainab}
\date{}

\begin{document}

\maketitle
\newpage% or \cleardoublepage
% \pdfbookmark[<level>]{<title>}{<dest>}
\pdfbookmark[section]{\contentsname}{toc}
\tableofcontents
\pagebreak

\chapter{Transformers}

\section{Introduction}

\begin{itemize}
	\ii A transformer is a static device that transfers electrical energy from one circuit to another through inductively coupled conductors.
	\ii It is used to increase or decrease the voltage level of an AC signal.
	\ii The transformer is based on the principle of electromagnetic induction.
	\ii The transformer is a passive device, meaning it does not require any external power source to operate.
	\ii The transformer is used in power distribution systems to step up or step down the voltage level.
	\ii The transformer is used in electronic circuits to isolate the input and output signals.
\end{itemize}

\section{Ideal Transformer}

\begin{itemize}
	\ii The primary and secondary windings have zero resistance.
	\[
		R_1 = R_2 = 0
		.\]
	\ii The core has infinite permeability.
	\[
		\mu = \infty
		.\]
	\ii The transformer has no leakage flux.
	\ii The transformer has no hysteresis loss.
	\ii The transformer has no eddy current loss.
	\begin{align*}
		\mcR_{P_{\text{core}}} & = 0 \\
		\Delta P               & = 0 \\
		\Delta P_j             & = 0
	\end{align*}
\end{itemize}

\begin{figure}[H]
	\centering
	\begin{circuitikz}
		\draw (0,0) node[transformer core] (T) {};
		\draw (T.A1) -- ++(-0.5,0) node[anchor=east] {1};
		\draw (T.A2) -- ++(-0.5,0) node[anchor=east] {2};
		\draw (T.B1) -- ++(0.5,0) node[anchor=west] {3};
		\draw (T.B2) -- ++(0.5,0) node[anchor=west] {4};
		\draw (T.base) node[above]{T};
		\draw (T.outer dot A1) node[circ] {};
		\draw (T.outer dot B1) node[circ] {};
		\draw (T.A1) to[american, open, v=$V_1$] (T.A2);
		\draw (T.B1) to[american, open, v=$V_2$] (T.B2);
	\end{circuitikz}
	\caption{Ideal Transformer}
\end{figure}

\begin{align*}
	\text{Turns Ratio:} \quad m & = \frac{N_1}{N_2} = \frac{V_1}{V_2} = \frac{I_2}{I_1}
\end{align*}

\section{Real Transformer}

We define the voltage equations of the transformer as follows:

\begin{align}
	\dot{V}_1 + \dot{E}_1 & = \dot{I}_1 Z_1 \\
	\dot{E}_2 - \dot{V}_2 & = \dot{I}_2 Z_2
\end{align}

\begin{align*}
	E_1 & = 4.44fN_1\Phi_m \\
	E_2 & = 4.44fN_2\Phi_m
\end{align*}

If we consider $\dot{I}_1Z_1\ll\dot{V}_1$ and $\dot{I}_2Z_2\ll\dot{V}_2$, we can simplify the equations as follows:

\begin{align*}
	\dot{V}_1 & = -\dot{E}_1 \\
	\dot{V}_2 & = \dot{E}_2
\end{align*}

We define the magnetization current as the current required to produce the magnetic flux in the core.

\[
	I_m = \dot{I}_{10} = I_1 + \underbrace{mI_2}_{I_2'}
	.\]

We define the equivalent circuit of the transformer as follows:

\begin{figure}[H]
	\centering
	\begin{circuitikz}
		\draw (0,0) to[R=$R_1$, o-]
		++(2,0) to[L=$jX_1$]
		++(2,0) to[short, i=$I_1$]
		++(0.5,0) coordinate (A) to[short, i=$I_2$]
		++(0.5,0) to[R=$R_2$]
		++(2,0) to[L=$jX_2$, -o]
		++(2,0) to[american, open, invert, v=$\dot{V}_2$]
		++(0,-3.5) to[short, o-o]
		++(-9,0);
		\draw (0,0) to[american, open, v=$\dot{V}_1$] ++(0,-3.5);
		\draw (A) to[short, *-*,,i=$I_m$] ++(0,-0.75) coordinate(B) to[short] ++(-0.5,0) to[R, l_=$R_f$] ++(0,-2) to[short,-*] ++(0.5,0) to[short,-*] ++(0,-0.75);
		\draw (B) to[short] ++(0.5,0) to[L=$jX_m$] ++(0,-2) to[short] ++(-0.5,0);
	\end{circuitikz}
	\caption{Transformer Equivalent Circuit}
\end{figure}

\subsection{$\Gamma$ Equivalent Circuit}

\begin{figure}[H]
	\centering
	\begin{circuitikz}
		\draw (0,0) to[short, o-] ++(1.5,0) coordinate (A) to[R=$R_p$] ++(2,0) to[L=$jX_p$] ++(2,0) to[short, -o] ++(1,0);
		\draw (A) to[short, *-*] ++(0,-0.75) coordinate(B) to[short] ++(-0.5,0) to[R, l_=$R_f$] ++(0,-2) to[short,-*] ++(0.5,0) to[short,-*] ++(0,-0.75);
		\draw (B) to[short] ++(0.5,0) to[L=$jX_m$] ++(0,-2) to[short] ++(-0.5,0);
		\draw (0,-3.5) to[short, o-o] ++(6.5,0);
	\end{circuitikz}
	\caption{$\Gamma$ Equivalent Circuit with the Primary Side as the Reference}
\end{figure}

\begin{align*}
	R_p & = R_1 + R_2' = R_1 + \frac{R_2}{m^2} \\
	X_p & = X_1 + X_2' = X_1 + \frac{X_2}{m^2} \\
	Z_p & = R_p + jX_p
\end{align*}

\begin{figure}[H]
	\centering
	\begin{circuitikz}
		\draw (0,0) to[short, o-] ++(1.5,0) coordinate (A) to[R=$R_s$] ++(2,0) to[L=$jX_s$] ++(2,0) to[short, -o] ++(1,0);
		\draw (A) to[short, *-*] ++(0,-0.75) coordinate(B) to[short] ++(-0.5,0) to[R, l_=$R_f$] ++(0,-2) to[short,-*] ++(0.5,0) to[short,-*] ++(0,-0.75);
		\draw (B) to[short] ++(0.5,0) to[L=$jX_m$] ++(0,-2) to[short] ++(-0.5,0);
		\draw (0,-3.5) to[short, o-o] ++(6.5,0);
	\end{circuitikz}
	\caption{$\Gamma$ Equivalent Circuit with the Secondary Side as the Reference}
\end{figure}

\begin{align*}
	R_s & = m^2R_1 + R_2 = m^2R_1 + R_2 \\
	X_s & = m^2X_1 + X_2 = m^2X_1 + X_2 \\
	Z_s & = R_s + jX_s
\end{align*}

\subsection{$\kappa$ Equivalent Circuit}

\begin{figure}[H]
	\centering
	\begin{circuitikz}
		\draw (0,0) to[short,o-] ++(1,0) to[R=$R_p$] ++(2,0) to[L=$jX_p$] ++(2,0) to[short,-o] ++(1,0);
		\draw (0,-2) to[short,o-o] ++(6,0);
	\end{circuitikz}
	\caption{$\kappa$ Equivalent Circuit with the Primary Side as the Reference}
\end{figure}

\section{Transformer Tests}

\subsection{Open Circuit Test}

\begin{figure}[H]
	\centering
	\begin{circuitikz}[american]
		\draw (0,0) node[transformer core] (T) {};
		\draw (T.base) node{$m$};
		\draw (T.outer dot A1) node[circ] {};
		\draw (T.outer dot B1) node[circ] {};

		\draw let \p1=(T.A1), \p2=(T.A2) in (T.A1) to[short] ++(-1,0) to[sV, l_=$V_1$] ++($ (0,\y2) - (0,\y1) $) to[short] (T.A2);
		\draw let \p1=(T.B1), \p2=(T.B2) in (T.B1) to[short] ++(1,0) to[qvprobe, l={$I_2{=}0$}] ++($ (0,\y2) - (0,\y1) $) to[short] (T.B2);
	\end{circuitikz}
	\caption{Open Circuit Test}
\end{figure}

We measure $V_{10}, I_{10}, P_{10}$ and $V_{20}$.\\

We can calculate the following:

\begin{align*}
	Q_{10}          & = V_{10}I_{10}\sin\varphi_{o} \\
	P_{10}          & = V_{10}I_{10}\cos\varphi_{o} \\
	\cos\varphi_{o} & = \frac{P_{10}}{V_{10}I_{10}} \\
	S_{10}          & = V_{10}I_{10}                \\
	R_f             & = \frac{{V_{10}}^2}{P_{10}}   \\
	X_m             & = \frac{{V_{10}}^2}{Q_{10}}   \\
\end{align*}

\subsection{Short Circuit Test}

\begin{figure}[H]
	\centering
	\begin{circuitikz}
		\draw (0,0) node[transformer core] (T) {};
		\draw (T.base) node{$m$};
		\draw (T.outer dot A1) node[circ] {};
		\draw (T.outer dot B1) node[circ] {};

		\draw let \p1=(T.A1), \p2=(T.A2) in (T.A1) to[short] ++(-1,0) to[sI, l=$I_{1c}$] ++($ (0,\y2) - (0,\y1) $) to[short] (T.A2);
		\draw let \p1=(T.B1), \p2=(T.B2) in (T.B1) to[short] ++(1,0) to[qiprobe, l={$V_2{=}0$}] ++($ (0,\y2) - (0,\y1) $) to[short] (T.B2);
	\end{circuitikz}
	\caption{Short Circuit Test}
\end{figure}

Typically we run the short circuit test at $5-15\%$ of the nominal current.\\

\begin{align*}
	I_{2c} & \leq I_{2n}      \\
	V_{1c} & = (5-15)\%V_{1n} \\
	I_{2c} & \ll I_{2n}       \\
	I_{1c} & \ll I_{1n}
\end{align*}

We measure $V_{1c}, I_{1c}, P_{1c}$ and $I_{2c}$.\\

We can calculate the following:

\begin{align*}
	R_p & = \frac{P_{1c}}{I_{1c}^2} \\
	X_p & = \frac{Q_{1c}}{I_{1c}^2} \\
\end{align*}

\section{Variation of the Voltage of Conduction}

\begin{enumerate}
	\ii \textbf{Exact Calculation:}
	\begin{align*}
		V_2                   & < V_{20} \\
		m_c = \frac{V_2}{V_1} & < m      \\
	\end{align*}

	\[
		V_1 = V_2' + \Delta V_p = V_2' + I_2'\lt( R_p + j X_p \rt)
		.\]

	\ii \textbf{Approximate Calculation (Variation of the Secondary Voltage):}
	\[
		\Delta V_s = V_20 - V-2
		.\]

	\[
		\Delta V_p = \frac{\Delta V_s}{m}
		.\]

	\begin{align*}
		\Delta V_s \% & = \frac{V_{20} - V_2}{V_{20}} \times 100 \\
		              & = \frac{mV_1 - V_2}{mV_1} \times 100     \\
		              & = \frac{V_1' - V_2}{V_1'} \times 100
	\end{align*}

	The KAPP formula states

	\[
		\Delta V_s = I_2 \lt( R_s \cos \varphi_2 \pm X_s \sin \varphi_2 \rt)
		.\]

	The $\pm$ sign depends on the nature of the load (inductive or capacitive).

	From the KAPP formula, we define the voltage regulation as

	\[
		\Sigma = \frac{\Delta V_s}{V_2} \times 100
		.\]

	\[
		\Delta V_p = I_1 \lt( R_p \cos \varphi_2 \pm X_p \sin \varphi_2 \rt)
		.\]
\end{enumerate}

\section{Vector Diagram of the Transformer}

\begin{figure}[H]
	\centering
	\begin{tikzpicture}
		% draw grid
		\draw[step=1cm,gray,thin] (-7,-0.25) grid (6,7);
		\begin{scope}[xshift=-5cm]
			\draw[-Stealth, thick] (0,0) coordinate (0) -- (0,4) node[above left]{$V_2$} coordinate(V2);
			\draw[-Stealth, thick] (V2) -- ++(60:1.5) node[right]{$I_2R_s$};
			\draw[-Stealth, thick] (V2) ++(60:1.5) -- ++(150:2) node[above]{$I_2X_s$} coordinate(V20);
			\draw[-Stealth, cyan, thick] (V2) -- (V20) node[midway, right]{$I_2Z_s$};
			\draw[-Stealth, cyan, thick] (0,0) -- (V20) node[midway, left]{$V_{20}$};
			\draw[-Stealth, thick] (0,0) -- ++(60:2) node[right]{$I_2$} coordinate(I2);
			% angle between V2 and I2 without using \pic
			\draw pic["$\varphi_2$", draw=orange, <->, angle eccentricity=1.5, angle radius=1cm] {angle=I2--0--V2};

			\node at (0,-0.5) [below] {Partly Inductive};
		\end{scope}
		\begin{scope}
			\draw[-Stealth, thick] (0,0) coordinate (0) -- (0,4) node[above right]{$V_2$} coordinate(V2);
			\draw[-Stealth, thick] (V2) -- ++(90:1.5) node[right]{$I_2R_s$};
			\draw[-Stealth, thick] (V2) ++(90:1.5) -- ++(180:2) node[above]{$I_2X_s$} coordinate(V20);
			\draw[-Stealth, cyan, thick] (V2) -- (V20) node[midway, right]{$I_2Z_s$};
			\draw[-Stealth, cyan, thick] (0,0) -- (V20) node[midway, left]{$V_{20}$};
			\draw[-Stealth, thick] (0,0) -- ++(90:2) node[right]{$I_2$} coordinate(I2) node[midway, right]{$\varphi_2{=}0$};

			\node at (0,-0.5) [below] {Purely Resistive};
		\end{scope}
		\begin{scope}[xshift=5cm]
			\draw[-Stealth, thick] (0,0) coordinate (0) -- (0,4) node[above right]{$V_2$} coordinate(V2);
			\draw[-Stealth, thick] (V2) -- ++(120:1.5) node[right]{$I_2R_s$};
			\draw[-Stealth, thick] (V2) ++(120:1.5) -- ++(210:2) node[above]{$I_2X_s$} coordinate(V20);
			\draw[-Stealth, cyan, thick] (V2) -- (V20) node[midway, above right]{$I_2Z_s$};
			\draw[-Stealth, cyan, thick] (0,0) -- (V20) node[midway, left]{$V_{20}$};
			\draw[-Stealth, thick] (0,0) -- ++(120:2) node[right]{$I_2$} coordinate(I2);

			\draw pic["$\varphi_2$", draw=orange, <->, angle eccentricity=1.5, angle radius=1cm] {angle=V2--0--I2};

			\node at (0,-0.5) [below] {Partly Capacitive};
		\end{scope}
	\end{tikzpicture}
	\caption{Vector Diagram of a Transformer}
\end{figure}

\[
	\eta = \frac{P_2}{P_1} = \frac{P_2}{P_2 + \Delta P_{Jt} + \Delta P_{f}}
	.\]

Where

\begin{align*}
	\Delta P_{Jt} & = \Delta P_{J1} + \Delta P_{J2} = {I_{11}}^2R_p = {I_{21}}^2R_s \\
	\Delta P_f    & = P_{10} = P_{20}
\end{align*}

\ex{}{
	Given a single phase transformer $200/400 \unit{\volt}$ - $\SI{50}{\hertz}$. By tests, we measure the following:
	\begin{itemize}
		\ii Open Circuit Test: $V_{10} = \SI{200}{\volt}$, $I_{10} = \SI{0.7}{\ampere}$, $P_{10} = \SI{70}{\watt}$.
		\ii Short Circuit Test: $V_{2c} = \SI{15}{\volt}$, $I_{1c} = \SI{10}{\ampere}$, $P_{1c} = \SI{85}{\watt}$.
	\end{itemize}

	\begin{enumerate}
		\ii Calculate the equivalent circuit parameters.
		\ii Calculate $V_2$ if the charge is $\SI{5}{\kilo\watt}$, $\cos\varphi = 0.8$, and $V_1=\SI{200}{\volt}$. Deduce the voltage regulation ($\Sigma$).
		\ii Deduce the efficiency of the transformer.
	\end{enumerate}

	\textbf{Solution:}
	$m=2$
	\begin{enumerate}
		\ii
		\begin{align*}
			R_f    & = \frac{{V_{10}}^2}{P_{10}} = \frac{{200}^2}{70} = \SI{571.43}{\ohm}         \\
			S_{10} & = V_{10}I_{10} = 200 \times 0.7 = \SI{140}{\volt\ampere}                     \\
			Q_{10} & = \sqrt{{S_{10}}^2 - {P_{10}}^2} = \sqrt{{140}^2 - {70}^2} = \SI{121.6}{VAR} \\
			X_m    & = \frac{{V_{10}}^2}{Q_{10}} = \frac{{200}^2}{121.6} = \SI{328.95}{\ohm}
		\end{align*}
		To find $R_p$ and $X_p$, we need to find $R_s$ and $X_s$ first.
		\begin{align*}
			R_s    & = \frac{P_{2c}}{I_{1c}^2} = \frac{85}{100} = \SI{0.85}{\ohm}        \\
			S_{2c} & = V_{2c}I_{1c} = 15 \times 10 = \SI{150}{\volt\ampere}              \\
			Q_{2c} & = \sqrt{S_{2c}^2 - P_{1c}^2} = \sqrt{150^2 - 85^2} = \SI{12.3}{VAR} \\
			X_s    & = \frac{Q_{2c}}{I_{1c}^2} = \frac{12.3}{100} = \SI{1.23}{\ohm}      \\
			R_p    & = \frac{R_s}{m^2} = \frac{0.85}{4} = \SI{0.2125}{\ohm}              \\
			X_p    & = \frac{X_s}{m^2} = \frac{1.23}{4} = \SI{0.3075}{\ohm}
		\end{align*}

		\ii
		\begin{align*}
			I_2        & = \frac{P_2}{V_2\cos\varphi} = \frac{5000}{400 \times 0.8} = \SI{15.625}{\ampere}                                           \\
			\Delta V_s & = I_2 \lt( R_s \cos\varphi \pm X_s \sin\varphi \rt) = 15.625 \lt( 0.85 \times 0.8 + 1.23 \times 0.6 \rt) = \SI{22.2}{\volt} \\
		\end{align*}

		\[
			\Sigma = \frac{\Delta V_s}{V_2} \times 100 = \frac{22.2}{400} \times 100 = 5.55\% \quad \text{If } V_{20} = \SI{400}{\volt}
			.\]

		To obtain $V_2=\SI{400}{\volt}$, we need to increase the primary voltage to
		\begin{align*}
			\Delta V_p & = \frac{\Delta V_s}{m} = \frac{22.2}{2} = \SI{11.1}{\volt}     \\
			V_1        & = V_2' + \Delta V_p = \frac{400}{2} + 11.1 = \SI{211.1}{\volt}
		\end{align*}

		\ii
		\begin{align*}
			\Delta P_{Jt} & = I_{11}^2R_p = 15.6^2 \times 0.85 = \SI{206}{\watt} \\
			\Delta P_f    & = P_{10} = P_{20} = \SI{70}{\watt}                   \\
			\eta          & = \frac{5000}{5000 + 206 + 70} = 0.947
		\end{align*}
	\end{enumerate}
}

\end{document}

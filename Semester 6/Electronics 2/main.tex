\documentclass{report}

%%%%%%%%%%%%%%%%%%%%%%%%%%%%%%%%%
% PACKAGE IMPORTS
%%%%%%%%%%%%%%%%%%%%%%%%%%%%%%%%%


\usepackage[tmargin=2cm,rmargin=1in,lmargin=1in,margin=0.85in,bmargin=2cm,footskip=.2in]{geometry}
\usepackage{amsmath,amsfonts,amsthm,amssymb,mathtools}
\usepackage[varbb]{newpxmath}
\usepackage{xfrac}
\usepackage[makeroom]{cancel}
\usepackage{mathtools}
\usepackage{bookmark}
\usepackage{enumitem}
\usepackage{hyperref,theoremref}
\usepackage{xparse}
\hypersetup{
	pdftitle={Hamboola my beloved},
	colorlinks=true, linkcolor=doc!90,
	bookmarksnumbered=true,
	bookmarksopen=true
}
\usepackage[most,many,breakable]{tcolorbox}
\usepackage{xcolor}
\usepackage{varwidth}
\usepackage{varwidth}
\usepackage{etoolbox}
\usepackage{bm}
%\usepackage{authblk}
\usepackage{pgfplots}
\usepackage{nameref}
\usepackage{multicol,array}
\usepackage{tikz-cd}
\usepackage[ruled,vlined,linesnumbered]{algorithm2e}
\usepackage{comment} % enables the use of multi-line comments (\ifx \fi) 
\usepackage{import}
\usepackage{xifthen}
\usepackage{pdfpages}
\usepackage{transparent}
\usepackage{minted}
\usepackage{fontspec}
\usepackage{tasks}
\usepackage{chemfig}
\usepackage[version=4]{mhchem}
\usepackage{suffix}
\usepackage{tabularx}
\usepackage{subcaption}
\usepackage{physics}


% \setmonofont{SpaceMono Nerd Font}
\setminted{fontsize=\footnotesize}

\newcommand\mycommfont[1]{\footnotesize\ttfamily\textcolor{blue}{#1}}
\SetCommentSty{mycommfont}
\newcommand{\incfig}[1]{%
	\def\svgwidth{\columnwidth}
	\import{./figures/}{#1.pdf_tex}
}

\usepackage{tikzsymbols}
% \renewcommand\qedsymbol{$\Laughey$}
\pgfplotsset{compat=1.18}

%\usepackage{import}
%\usepackage{xifthen}
%\usepackage{pdfpages}
%\usepackage{transparent}


%%%%%%%%%%%%%%%%%%%%%%%%%%%%%%
% SELF MADE COLORS
%%%%%%%%%%%%%%%%%%%%%%%%%%%%%%



\definecolor{myg}{RGB}{56, 140, 70}
\definecolor{myb}{RGB}{45, 111, 177}
\definecolor{myr}{RGB}{199, 68, 64}
\definecolor{mytheorembg}{HTML}{F2F2F9}
\definecolor{mytheoremfr}{HTML}{00007B}
\definecolor{mylenmabg}{HTML}{FFFAF8}
\definecolor{mylenmafr}{HTML}{983b0f}
\definecolor{mypropbg}{HTML}{f2fbfc}
\definecolor{mypropfr}{HTML}{191971}
\definecolor{myexamplebg}{HTML}{F2FBF8}
\definecolor{myexamplefr}{HTML}{88D6D1}
\definecolor{myexampleti}{HTML}{2A7F7F}
\definecolor{mydefinitbg}{HTML}{E5E5FF}
\definecolor{mydefinitfr}{HTML}{3F3FA3}
\definecolor{notesgreen}{RGB}{0,162,0}
\definecolor{myp}{RGB}{197, 92, 212}
\definecolor{mygr}{HTML}{2C3338}
\definecolor{myred}{RGB}{127,0,0}
\definecolor{myyellow}{RGB}{169,121,69}
\definecolor{myexercisebg}{HTML}{F2FBF8}
\definecolor{myexercisefg}{HTML}{88D6D1}
\definecolor{codebg}{HTML}{0D1117}

%%%%%%%%%%%%%%%%%%%%%%%%%%%%
% TCOLORBOX SETUPS
%%%%%%%%%%%%%%%%%%%%%%%%%%%%

\setlength{\parindent}{0cm}
%================================
% THEOREM BOX
%================================

\tcbuselibrary{theorems,skins,hooks}
\newtcbtheorem[number within=section]{Theorem}{Theorem}
{%
	enhanced,
	breakable,
	colback = mytheorembg,
	frame hidden,
	boxrule = 0sp,
	borderline west = {2pt}{0pt}{mytheoremfr},
	sharp corners,
	detach title,
	before upper = \tcbtitle\par\smallskip,
	coltitle = mytheoremfr,
	fonttitle = \bfseries\sffamily,
	description font = \mdseries,
	separator sign none,
	segmentation style={solid, mytheoremfr},
}
{th}

\tcbuselibrary{theorems,skins,hooks}
\newtcbtheorem[number within=chapter]{theorem}{Theorem}
{%
	enhanced,
	breakable,
	colback = mytheorembg,
	frame hidden,
	boxrule = 0sp,
	borderline west = {2pt}{0pt}{mytheoremfr},
	sharp corners,
	detach title,
	before upper = \tcbtitle\par\smallskip,
	coltitle = mytheoremfr,
	fonttitle = \bfseries\sffamily,
	description font = \mdseries,
	separator sign none,
	segmentation style={solid, mytheoremfr},
}
{th}


\tcbuselibrary{theorems,skins,hooks}
\newtcolorbox{Theoremcon}
{%
	enhanced
	,breakable
	,colback = mytheorembg
	,frame hidden
	,boxrule = 0sp
	,borderline west = {2pt}{0pt}{mytheoremfr}
	,sharp corners
	,description font = \mdseries
	,separator sign none
}

%================================
% Corollery
%================================
\tcbuselibrary{theorems,skins,hooks}
\newtcbtheorem[number within=section]{Corollary}{Corollary}
{%
	enhanced
	,breakable
	,colback = myp!10
	,frame hidden
	,boxrule = 0sp
	,borderline west = {2pt}{0pt}{myp!85!black}
	,sharp corners
	,detach title
	,before upper = \tcbtitle\par\smallskip
	,coltitle = myp!85!black
	,fonttitle = \bfseries\sffamily
	,description font = \mdseries
	,separator sign none
	,segmentation style={solid, myp!85!black}
}
{th}
\tcbuselibrary{theorems,skins,hooks}
\newtcbtheorem[number within=chapter]{corollary}{Corollary}
{%
	enhanced
	,breakable
	,colback = myp!10
	,frame hidden
	,boxrule = 0sp
	,borderline west = {2pt}{0pt}{myp!85!black}
	,sharp corners
	,detach title
	,before upper = \tcbtitle\par\smallskip
	,coltitle = myp!85!black
	,fonttitle = \bfseries\sffamily
	,description font = \mdseries
	,separator sign none
	,segmentation style={solid, myp!85!black}
}
{th}


%================================
% LENMA
%================================

\tcbuselibrary{theorems,skins,hooks}
\newtcbtheorem[number within=section]{Lenma}{Lenma}
{%
	enhanced,
	breakable,
	colback = mylenmabg,
	frame hidden,
	boxrule = 0sp,
	borderline west = {2pt}{0pt}{mylenmafr},
	sharp corners,
	detach title,
	before upper = \tcbtitle\par\smallskip,
	coltitle = mylenmafr,
	fonttitle = \bfseries\sffamily,
	description font = \mdseries,
	separator sign none,
	segmentation style={solid, mylenmafr},
}
{th}

\tcbuselibrary{theorems,skins,hooks}
\newtcbtheorem[number within=chapter]{lenma}{Lenma}
{%
	enhanced,
	breakable,
	colback = mylenmabg,
	frame hidden,
	boxrule = 0sp,
	borderline west = {2pt}{0pt}{mylenmafr},
	sharp corners,
	detach title,
	before upper = \tcbtitle\par\smallskip,
	coltitle = mylenmafr,
	fonttitle = \bfseries\sffamily,
	description font = \mdseries,
	separator sign none,
	segmentation style={solid, mylenmafr},
}
{th}


%================================
% PROPOSITION
%================================

\tcbuselibrary{theorems,skins,hooks}
\newtcbtheorem[number within=section]{Prop}{Proposition}
{%
	enhanced,
	breakable,
	colback = mypropbg,
	frame hidden,
	boxrule = 0sp,
	borderline west = {2pt}{0pt}{mypropfr},
	sharp corners,
	detach title,
	before upper = \tcbtitle\par\smallskip,
	coltitle = mypropfr,
	fonttitle = \bfseries\sffamily,
	description font = \mdseries,
	separator sign none,
	segmentation style={solid, mypropfr},
}
{th}

\tcbuselibrary{theorems,skins,hooks}
\newtcbtheorem[number within=chapter]{prop}{Proposition}
{%
	enhanced,
	breakable,
	colback = mypropbg,
	frame hidden,
	boxrule = 0sp,
	borderline west = {2pt}{0pt}{mypropfr},
	sharp corners,
	detach title,
	before upper = \tcbtitle\par\smallskip,
	coltitle = mypropfr,
	fonttitle = \bfseries\sffamily,
	description font = \mdseries,
	separator sign none,
	segmentation style={solid, mypropfr},
}
{th}


%================================
% CLAIM
%================================

\tcbuselibrary{theorems,skins,hooks}
\newtcbtheorem[number within=section]{claim}{Claim}
{%
	enhanced
	,breakable
	,colback = myg!10
	,frame hidden
	,boxrule = 0sp
	,borderline west = {2pt}{0pt}{myg}
	,sharp corners
	,detach title
	,before upper = \tcbtitle\par\smallskip
	,coltitle = myg!85!black
	,fonttitle = \bfseries\sffamily
	,description font = \mdseries
	,separator sign none
	,segmentation style={solid, myg!85!black}
}
{th}



%================================
% Exercise
%================================

\tcbuselibrary{theorems,skins,hooks}
\newtcbtheorem[number within=section]{Exercise}{Exercise}
{%
	enhanced,
	breakable,
	colback = myexercisebg,
	frame hidden,
	boxrule = 0sp,
	borderline west = {2pt}{0pt}{myexercisefg},
	sharp corners,
	detach title,
	before upper = \tcbtitle\par\smallskip,
	coltitle = myexercisefg,
	fonttitle = \bfseries\sffamily,
	description font = \mdseries,
	separator sign none,
	segmentation style={solid, myexercisefg},
}
{th}

\tcbuselibrary{theorems,skins,hooks}
\newtcbtheorem[number within=chapter]{exercise}{Exercise}
{%
	enhanced,
	breakable,
	colback = myexercisebg,
	frame hidden,
	boxrule = 0sp,
	borderline west = {2pt}{0pt}{myexercisefg},
	sharp corners,
	detach title,
	before upper = \tcbtitle\par\smallskip,
	coltitle = myexercisefg,
	fonttitle = \bfseries\sffamily,
	description font = \mdseries,
	separator sign none,
	segmentation style={solid, myexercisefg},
}
{th}

%================================
% EXAMPLE BOX
%================================

\newtcbtheorem[number within=section]{Example}{Example}
{%
	colback = myexamplebg
	,breakable
	,colframe = myexamplefr
	,coltitle = myexampleti
	,boxrule = 1pt
	,sharp corners
	,detach title
	,before upper=\tcbtitle\par\smallskip
	,fonttitle = \bfseries
	,description font = \mdseries
	,separator sign none
	,description delimiters parenthesis
}
{ex}

\newtcbtheorem[number within=chapter]{example}{Example}
{%
	colback = myexamplebg
	,breakable
	,colframe = myexamplefr
	,coltitle = myexampleti
	,boxrule = 1pt
	,sharp corners
	,detach title
	,before upper=\tcbtitle\par\smallskip
	,fonttitle = \bfseries
	,description font = \mdseries
	,separator sign none
	,description delimiters parenthesis
}
{ex}

%================================
% DEFINITION BOX
%================================

\newtcbtheorem[number within=section]{Definition}{Definition}{enhanced,
	before skip=2mm,after skip=2mm, colback=red!5,colframe=red!80!black,boxrule=0.5mm,
	attach boxed title to top left={xshift=1cm,yshift*=1mm-\tcboxedtitleheight}, varwidth boxed title*=-3cm,
	boxed title style={frame code={
					\path[fill=tcbcolback]
					([yshift=-1mm,xshift=-1mm]frame.north west)
					arc[start angle=0,end angle=180,radius=1mm]
					([yshift=-1mm,xshift=1mm]frame.north east)
					arc[start angle=180,end angle=0,radius=1mm];
					\path[left color=tcbcolback!60!black,right color=tcbcolback!60!black,
						middle color=tcbcolback!80!black]
					([xshift=-2mm]frame.north west) -- ([xshift=2mm]frame.north east)
					[rounded corners=1mm]-- ([xshift=1mm,yshift=-1mm]frame.north east)
					-- (frame.south east) -- (frame.south west)
					-- ([xshift=-1mm,yshift=-1mm]frame.north west)
					[sharp corners]-- cycle;
				},interior engine=empty,
		},
	fonttitle=\bfseries,
	title={#2},#1}{def}
\newtcbtheorem[number within=chapter]{definition}{Definition}{enhanced,
	before skip=2mm,after skip=2mm, colback=red!5,colframe=red!80!black,boxrule=0.5mm,
	attach boxed title to top left={xshift=1cm,yshift*=1mm-\tcboxedtitleheight}, varwidth boxed title*=-3cm,
	boxed title style={frame code={
					\path[fill=tcbcolback]
					([yshift=-1mm,xshift=-1mm]frame.north west)
					arc[start angle=0,end angle=180,radius=1mm]
					([yshift=-1mm,xshift=1mm]frame.north east)
					arc[start angle=180,end angle=0,radius=1mm];
					\path[left color=tcbcolback!60!black,right color=tcbcolback!60!black,
						middle color=tcbcolback!80!black]
					([xshift=-2mm]frame.north west) -- ([xshift=2mm]frame.north east)
					[rounded corners=1mm]-- ([xshift=1mm,yshift=-1mm]frame.north east)
					-- (frame.south east) -- (frame.south west)
					-- ([xshift=-1mm,yshift=-1mm]frame.north west)
					[sharp corners]-- cycle;
				},interior engine=empty,
		},
	fonttitle=\bfseries,
	title={#2},#1}{def}



%================================
% Solution BOX
%================================

\makeatletter
\newtcbtheorem{question}{Question}{enhanced,
	breakable,
	colback=white,
	colframe=myb!80!black,
	attach boxed title to top left={yshift*=-\tcboxedtitleheight},
	fonttitle=\bfseries,
	title={#2},
	boxed title size=title,
	boxed title style={%
			sharp corners,
			rounded corners=northwest,
			colback=tcbcolframe,
			boxrule=0pt,
		},
	underlay boxed title={%
			\path[fill=tcbcolframe] (title.south west)--(title.south east)
			to[out=0, in=180] ([xshift=5mm]title.east)--
			(title.center-|frame.east)
			[rounded corners=\kvtcb@arc] |-
			(frame.north) -| cycle;
		},
	#1
}{def}
\makeatother

%================================
% SOLUTION BOX
%================================

\makeatletter
\newtcolorbox{solution}{enhanced,
	breakable,
	colback=white,
	colframe=myg!80!black,
	attach boxed title to top left={yshift*=-\tcboxedtitleheight},
	title=Solution,
	boxed title size=title,
	boxed title style={%
			sharp corners,
			rounded corners=northwest,
			colback=tcbcolframe,
			boxrule=0pt,
		},
	underlay boxed title={%
			\path[fill=tcbcolframe] (title.south west)--(title.south east)
			to[out=0, in=180] ([xshift=5mm]title.east)--
			(title.center-|frame.east)
			[rounded corners=\kvtcb@arc] |-
			(frame.north) -| cycle;
		},
}
\makeatother

%================================
% Question BOX
%================================

\makeatletter
\newtcbtheorem{qstion}{Question}{enhanced,
	breakable,
	colback=white,
	colframe=mygr,
	attach boxed title to top left={yshift*=-\tcboxedtitleheight},
	fonttitle=\bfseries,
	title={#2},
	boxed title size=title,
	boxed title style={%
			sharp corners,
			rounded corners=northwest,
			colback=tcbcolframe,
			boxrule=0pt,
		},
	underlay boxed title={%
			\path[fill=tcbcolframe] (title.south west)--(title.south east)
			to[out=0, in=180] ([xshift=5mm]title.east)--
			(title.center-|frame.east)
			[rounded corners=\kvtcb@arc] |-
			(frame.north) -| cycle;
		},
	#1
}{def}
\makeatother

\newtcbtheorem[number within=chapter]{wconc}{Wrong Concept}{
	breakable,
	enhanced,
	colback=white,
	colframe=myr,
	arc=0pt,
	outer arc=0pt,
	fonttitle=\bfseries\sffamily\large,
	colbacktitle=myr,
	attach boxed title to top left={},
	boxed title style={
			enhanced,
			skin=enhancedfirst jigsaw,
			arc=3pt,
			bottom=0pt,
			interior style={fill=myr}
		},
	#1
}{def}



%================================
% NOTE BOX
%================================

\usetikzlibrary{hobby}
\usetikzlibrary{arrows,calc,shadows.blur}
\tcbuselibrary{skins}
\newtcolorbox{note}[1][]{%
	enhanced jigsaw,
	colback=gray!20!white,%
	colframe=gray!80!black,
	size=small,
	boxrule=1pt,
	title=\textbf{Note:-},
	halign title=flush center,
	coltitle=black,
	breakable,
	drop shadow=black!50!white,
	attach boxed title to top left={xshift=1cm,yshift=-\tcboxedtitleheight/2,yshifttext=-\tcboxedtitleheight/2},
	minipage boxed title=1.5cm,
	boxed title style={%
			colback=white,
			size=fbox,
			boxrule=1pt,
			boxsep=2pt,
			underlay={%
					\coordinate (dotA) at ($(interior.west) + (-0.5pt,0)$);
					\coordinate (dotB) at ($(interior.east) + (0.5pt,0)$);
					\begin{scope}
						\clip (interior.north west) rectangle ([xshift=3ex]interior.east);
						\filldraw [white, blur shadow={shadow opacity=60, shadow yshift=-.75ex}, rounded corners=2pt] (interior.north west) rectangle (interior.south east);
					\end{scope}
					\begin{scope}[gray!80!black]
						\fill (dotA) circle (2pt);
						\fill (dotB) circle (2pt);
					\end{scope}
				},
		},
	#1,
}

%%%%%%%%%%%%%%%%%%%%%%%%%%%%%%
% SELF MADE COMMANDS
%%%%%%%%%%%%%%%%%%%%%%%%%%%%%%


\newcommand{\thm}[2]{\begin{Theorem}{#1}{}#2\end{Theorem}}
\newcommand{\cor}[2]{\begin{Corollary}{#1}{}#2\end{Corollary}}
\newcommand{\mlenma}[2]{\begin{Lenma}{#1}{}#2\end{Lenma}}
\newcommand{\mprop}[2]{\begin{Prop}{#1}{}#2\end{Prop}}
\newcommand{\clm}[3]{\begin{claim}{#1}{#2}#3\end{claim}}
\newcommand{\wc}[2]{\begin{wconc}{#1}{}\setlength{\parindent}{1cm}#2\end{wconc}}
\newcommand{\thmcon}[1]{\begin{Theoremcon}{#1}\end{Theoremcon}}
\newcommand{\ex}[2]{\begin{Example}{#1}{}#2\end{Example}}
\newcommand{\dfn}[2]{\begin{Definition}[colbacktitle=red!75!black]{#1}{}#2\end{Definition}}
\newcommand{\dfnc}[2]{\begin{definition}[colbacktitle=red!75!black]{#1}{}#2\end{definition}}
\newcommand{\qs}[2]{\begin{question}{#1}{}#2\end{question}}
\newcommand{\pf}[2]{\begin{myproof}[#1]#2\end{myproof}}
\newcommand{\nt}[1]{\begin{note}#1\end{note}}

\newcommand*\circled[1]{\tikz[baseline=(char.base)]{
		\node[shape=circle,draw,inner sep=1pt] (char) {#1};}}
\newcommand\getcurrentref[1]{%
	\ifnumequal{\value{#1}}{0}
	{??}
	{\the\value{#1}}%
}
\newcommand{\getCurrentSectionNumber}{\getcurrentref{section}}
\newenvironment{myproof}[1][\proofname]{%
	\proof[\bfseries #1: ]%
}{\endproof}

\newcommand{\mclm}[2]{\begin{myclaim}[#1]#2\end{myclaim}}
\newenvironment{myclaim}[1][\claimname]{\proof[\bfseries #1: ]}{}

\newcounter{mylabelcounter}

\makeatletter
\newcommand{\setword}[2]{%
	\phantomsection
	#1\def\@currentlabel{\unexpanded{#1}}\label{#2}%
}
\makeatother




\tikzset{
	symbol/.style={
			draw=none,
			every to/.append style={
					edge node={node [sloped, allow upside down, auto=false]{$#1$}}}
		}
}


% deliminators
% \DeclarePairedDelimiter{\abs}{\lvert}{\rvert}
% \DeclarePairedDelimiter{\norm}{\lVert}{\rVert}

\DeclarePairedDelimiter{\ceil}{\lceil}{\rceil}
\DeclarePairedDelimiter{\floor}{\lfloor}{\rfloor}
\DeclarePairedDelimiter{\round}{\lfloor}{\rceil}

\newsavebox\diffdbox
\newcommand{\slantedromand}{{\mathpalette\makesl{d}}}
\newcommand{\makesl}[2]{%
	\begingroup
	\sbox{\diffdbox}{$\mathsurround=0pt#1\mathrm{#2}$}%
	\pdfsave
	\pdfsetmatrix{1 0 0.2 1}%
	\rlap{\usebox{\diffdbox}}%
	\pdfrestore
	\hskip\wd\diffdbox
	\endgroup
}
% \newcommand{\dd}[1][]{\ensuremath{\mathop{}\!\ifstrempty{#1}{%
% 			\slantedromand\@ifnextchar^{\hspace{0.2ex}}{\hspace{0.1ex}}}%
% 		{\slantedromand\hspace{0.2ex}^{#1}}}}
\ProvideDocumentCommand\dv{o m g}{%
	\ensuremath{%
		\IfValueTF{#3}{%
			\IfNoValueTF{#1}{%
				\frac{\dd #2}{\dd #3}%
			}{%
				\frac{\dd^{#1} #2}{\dd #3^{#1}}%
			}%
		}{%
			\IfNoValueTF{#1}{%
				\frac{\dd}{\dd #2}%
			}{%
				\frac{\dd^{#1}}{\dd #2^{#1}}%
			}%
		}%
	}%
}
\DeclareDocumentCommand\pdv{ s o m g g d() }
{ % Partial derivative
	% s: star for \flatfrac flat derivative
	% o: optional n for nth derivative
	% m: mandatory (x in df/dx)
	% g: optional (f in df/dx)
	% g: optional (y in d^2f/dxdy)
	% d: long-form d/dx(...)
	\IfBooleanTF{#1}
	{\let\fractype\flatfrac}
	{\let\fractype\frac}
	\IfNoValueTF{#4}
	{
		\IfNoValueTF{#6}
		{\fractype{\partial \IfNoValueTF{#2}{}{^{#2}}}{\partial #3\IfNoValueTF{#2}{}{^{#2}}}}
		{\fractype{\partial \IfNoValueTF{#2}{}{^{#2}}}{\partial #3\IfNoValueTF{#2}{}{^{#2}}} \argopen(#6\argclose)}
	}
	{
		\IfNoValueTF{#5}
		{\fractype{\partial \IfNoValueTF{#2}{}{^{#2}} #3}{\partial #4\IfNoValueTF{#2}{}{^{#2}}}}
		{\fractype{\partial^2 #3}{\partial #4 \partial #5}}
	}
}
% \providecommand*{\pdv}[3][]{\frac{\partial^{#1}#2}{\partial#3^{#1}}}
%  - others
\DeclareMathOperator{\Lap}{\mathcal{L}}
\DeclareMathOperator{\Var}{Var} % varience
\DeclareMathOperator{\Cov}{Cov} % covarience
\DeclareMathOperator{\E}{E} % expected

% Since the amsthm package isn't loaded

% I prefer the slanted \leq
\let\oldleq\leq % save them in case they're every wanted
\let\oldgeq\geq
\renewcommand{\leq}{\leqslant}
\renewcommand{\geq}{\geqslant}

% % redefine matrix env to allow for alignment, use r as default
% \renewcommand*\env@matrix[1][r]{\hskip -\arraycolsep
%     \let\@ifnextchar\new@ifnextchar
%     \array{*\c@MaxMatrixCols #1}}


%\usepackage{framed}
%\usepackage{titletoc}
%\usepackage{etoolbox}
%\usepackage{lmodern}


%\patchcmd{\tableofcontents}{\contentsname}{\sffamily\contentsname}{}{}

%\renewenvironment{leftbar}
%{\def\FrameCommand{\hspace{6em}%
%		{\color{myyellow}\vrule width 2pt depth 6pt}\hspace{1em}}%
%	\MakeFramed{\parshape 1 0cm \dimexpr\textwidth-6em\relax\FrameRestore}\vskip2pt%
%}
%{\endMakeFramed}

%\titlecontents{chapter}
%[0em]{\vspace*{2\baselineskip}}
%{\parbox{4.5em}{%
%		\hfill\Huge\sffamily\bfseries\color{myred}\thecontentspage}%
%	\vspace*{-2.3\baselineskip}\leftbar\textsc{\small\chaptername~\thecontentslabel}\\\sffamily}
%{}{\endleftbar}
%\titlecontents{section}
%[8.4em]
%{\sffamily\contentslabel{3em}}{}{}
%{\hspace{0.5em}\nobreak\itshape\color{myred}\contentspage}
%\titlecontents{subsection}
%[8.4em]
%{\sffamily\contentslabel{3em}}{}{}  
%{\hspace{0.5em}\nobreak\itshape\color{myred}\contentspage}



%%%%%%%%%%%%%%%%%%%%%%%%%%%%%%%%%%%%%%%%%%%
% TABLE OF CONTENTS
%%%%%%%%%%%%%%%%%%%%%%%%%%%%%%%%%%%%%%%%%%%

\usepackage{tikz}
\definecolor{doc}{RGB}{0,60,110}
\usepackage{titletoc}
\contentsmargin{0cm}
\titlecontents{chapter}[3.7pc]
{\addvspace{30pt}%
	\begin{tikzpicture}[remember picture, overlay]%
		\draw[fill=doc!60,draw=doc!60] (-7,-.1) rectangle (-0.9,.5);%
		\pgftext[left,x=-3.5cm,y=0.2cm]{\color{white}\Large\sc\bfseries Chapter\ \thecontentslabel};%
	\end{tikzpicture}\color{doc!60}\large\sc\bfseries}%
{}
{}
{\;\titlerule\;\large\sc\bfseries Page \thecontentspage
	\begin{tikzpicture}[remember picture, overlay]
		\draw[fill=doc!60,draw=doc!60] (2pt,0) rectangle (4,0.1pt);
	\end{tikzpicture}}%
\titlecontents{section}[3.7pc]
{\addvspace{2pt}}
{\contentslabel[\thecontentslabel]{2pc}}
{}
{\hfill\small \thecontentspage}
[]
\titlecontents*{subsection}[3.7pc]
{\addvspace{-1pt}\small}
{}
{}
{\ --- \small\thecontentspage}
[ \textbullet\ ][]

\makeatletter
\renewcommand{\tableofcontents}{%
	\chapter*{%
	  \vspace*{-20\p@}%
	  \begin{tikzpicture}[remember picture, overlay]%
		  \pgftext[right,x=15cm,y=0.2cm]{\color{doc!60}\Huge\sc\bfseries \contentsname};%
		  \draw[fill=doc!60,draw=doc!60] (13,-.75) rectangle (20,1);%
		  \clip (13,-.75) rectangle (20,1);
		  \pgftext[right,x=15cm,y=0.2cm]{\color{white}\Huge\sc\bfseries \contentsname};%
	  \end{tikzpicture}}%
	\@starttoc{toc}}
\makeatother

%From M275 "Topology" at SJSU
\newcommand{\id}{\mathrm{id}}
\newcommand{\taking}[1]{\xrightarrow{#1}}
\newcommand{\inv}{^{-1}}

%From M170 "Introduction to Graph Theory" at SJSU
\DeclareMathOperator{\diam}{diam}
\DeclareMathOperator{\ord}{ord}
\newcommand{\defeq}{\overset{\mathrm{def}}{=}}

%From the USAMO .tex files
\newcommand{\ts}{\textsuperscript}
\newcommand{\dg}{^\circ}
\newcommand{\ii}{\item}

% % From Math 55 and Math 145 at Harvard
% \newenvironment{subproof}[1][Proof]{%
% \begin{proof}[#1] \renewcommand{\qedsymbol}{$\blacksquare$}}%
% {\end{proof}}

\newcommand{\liff}{\leftrightarrow}
\newcommand{\lthen}{\rightarrow}
\newcommand{\opname}{\operatorname}
\newcommand{\surjto}{\twoheadrightarrow}
\newcommand{\injto}{\hookrightarrow}
\newcommand{\On}{\mathrm{On}} % ordinals
% \newcommand{\EE}{\mathbb{E}} % Expectance
\DeclareMathOperator{\img}{im} % Image
\DeclareMathOperator{\Img}{Im} % Image
\DeclareMathOperator{\coker}{coker} % Cokernel
\DeclareMathOperator{\Coker}{Coker} % Cokernel
\DeclareMathOperator{\Ker}{Ker} % Kernel
\DeclareMathOperator{\rank}{rank}
\DeclareMathOperator{\Spec}{Spec} % spectrum
\DeclareMathOperator{\Tr}{Tr} % trace
\DeclareMathOperator{\pr}{pr} % projection
\DeclareMathOperator{\ext}{ext} % extension
\DeclareMathOperator{\pred}{pred} % predecessor
\DeclareMathOperator{\dom}{dom} % domain
\DeclareMathOperator{\ran}{ran} % range
\DeclareMathOperator{\Hom}{Hom} % homomorphism
\DeclareMathOperator{\Mor}{Mor} % morphisms
\DeclareMathOperator{\End}{End} % endomorphism
% \DeclareMathOperator{\Pr}{Pr} % probability
% \DeclareMathOperator{\Var}{Var} % variance

\newcommand{\eps}{\epsilon}
\newcommand{\veps}{\varepsilon}
\newcommand{\ol}{\overline}
\newcommand{\ul}{\underline}
\newcommand{\wt}{\widetilde}
\newcommand{\wh}{\widehat}
\newcommand{\vocab}[1]{\textbf{\color{blue} #1}}
\providecommand{\half}{\frac{1}{2}}
\newcommand{\dang}{\measuredangle} %% Directed angle
\newcommand{\ray}[1]{\overrightarrow{#1}}
\newcommand{\seg}[1]{\overline{#1}}
\newcommand{\arc}[1]{\wideparen{#1}}
\DeclareMathOperator{\cis}{cis}
\DeclareMathOperator*{\lcm}{lcm}
\DeclareMathOperator*{\argmin}{arg min}
\DeclareMathOperator*{\argmax}{arg max}
\newcommand{\cycsum}{\sum_{\mathrm{cyc}}}
\newcommand{\symsum}{\sum_{\mathrm{sym}}}
\newcommand{\cycprod}{\prod_{\mathrm{cyc}}}
\newcommand{\symprod}{\prod_{\mathrm{sym}}}
\newcommand{\Qed}{\begin{flushright}\qed\end{flushright}}
\newcommand{\parinn}{\setlength{\parindent}{1cm}}
\newcommand{\parinf}{\setlength{\parindent}{0cm}}
% \newcommand{\norm}{\|\cdot\|}
\newcommand{\inorm}{\norm_{\infty}}
\newcommand{\opensets}{\{V_{\alpha}\}_{\alpha\in I}}
\newcommand{\oset}{V_{\alpha}}
\newcommand{\opset}[1]{V_{\alpha_{#1}}}
\newcommand{\lub}{\text{lub}}
\newcommand{\del}[2]{\frac{\partial #1}{\partial #2}}
\newcommand{\Del}[3]{\frac{\partial^{#1} #2}{\partial^{#1} #3}}
\newcommand{\deld}[2]{\dfrac{\partial #1}{\partial #2}}
\newcommand{\Deld}[3]{\dfrac{\partial^{#1} #2}{\partial^{#1} #3}}
\newcommand{\lm}{\lambda}
\newcommand{\uin}{\mathbin{\rotatebox[origin=c]{90}{$\in$}}}
\newcommand{\usubset}{\mathbin{\rotatebox[origin=c]{90}{$\subset$}}}
\newcommand{\lt}{\left}
\newcommand{\rt}{\right}
\newcommand{\bs}[1]{\boldsymbol{#1}}
\newcommand{\exs}{\exists}
\newcommand{\st}{\strut}
\newcommand{\dps}[1]{\displaystyle{#1}}
\newcommand{\va}[1]{\vec{\bm{\mathrm{#1}}}}
\WithSuffix\newcommand\va*[1]{\vec{\bm{#1}}}
\newcommand{\vb}[1]{\bm{\mathrm{#1}}}
\WithSuffix\newcommand\vb*[1]{\bm{#1}}
\newcommand{\vu}[1]{\hat{\bm{\mathrm{#1}}}}
\WithSuffix\newcommand\vu*[1]{\hat{\bm{#1}}}
\renewcommand{\dd}[1]{\mathrm{d}#1}
\renewcommand{\Re}{\mathrm{Re}}
\renewcommand{\Im}{\mathrm{Im}}
\DeclareMathOperator{\tr}{tr}
\newcommand{\csin}[1]{\mintinline{csharp}|#1|}
\renewcommand{\Pr}[1]{\mathrm{Pr}\lt( #1 \rt)}
\renewcommand{\Var}[1]{\mathrm{Var}\lt( #1 \rt)}
\newcommand{\cov}[1]{\mathrm{cov}\lt( #1 \rt)}

\newcommand{\sol}{\setlength{\parindent}{0cm}\textbf{\textit{Solution:}}\setlength{\parindent}{1cm} }
\newcommand{\solve}[1]{\setlength{\parindent}{0cm}\textbf{\textit{Solution: }}\setlength{\parindent}{1cm}#1 \Qed}

% Things Lie
\newcommand{\kb}{\mathfrak b}
\newcommand{\kg}{\mathfrak g}
\newcommand{\kh}{\mathfrak h}
\newcommand{\kn}{\mathfrak n}
\newcommand{\ku}{\mathfrak u}
\newcommand{\kz}{\mathfrak z}
\DeclareMathOperator{\Ext}{Ext} % Ext functor
\DeclareMathOperator{\Tor}{Tor} % Tor functor
\newcommand{\gl}{\opname{\mathfrak{gl}}} % frak gl group
\renewcommand{\sl}{\opname{\mathfrak{sl}}} % frak sl group chktex 6

% More script letters etc.
\newcommand{\SA}{\mathcal A}
\newcommand{\SB}{\mathcal B}
\newcommand{\SC}{\mathcal C}
\newcommand{\SF}{\mathcal F}
\newcommand{\SG}{\mathcal G}
\newcommand{\SH}{\mathcal H}
\newcommand{\OO}{\mathcal O}

\newcommand{\SCA}{\mathscr A}
\newcommand{\SCB}{\mathscr B}
\newcommand{\SCC}{\mathscr C}
\newcommand{\SCD}{\mathscr D}
\newcommand{\SCE}{\mathscr E}
\newcommand{\SCF}{\mathscr F}
\newcommand{\SCG}{\mathscr G}
\newcommand{\SCH}{\mathscr H}

% Mathfrak primes
\newcommand{\km}{\mathfrak m}
\newcommand{\kp}{\mathfrak p}
\newcommand{\kq}{\mathfrak q}

% number sets
\newcommand{\RR}[1][]{\ensuremath{\ifstrempty{#1}{\mathbb{R}}{\mathbb{R}^{#1}}}}
\newcommand{\NN}[1][]{\ensuremath{\ifstrempty{#1}{\mathbb{N}}{\mathbb{N}^{#1}}}}
\newcommand{\ZZ}[1][]{\ensuremath{\ifstrempty{#1}{\mathbb{Z}}{\mathbb{Z}^{#1}}}}
\newcommand{\QQ}[1][]{\ensuremath{\ifstrempty{#1}{\mathbb{Q}}{\mathbb{Q}^{#1}}}}
\newcommand{\CC}[1][]{\ensuremath{\ifstrempty{#1}{\mathbb{C}}{\mathbb{C}^{#1}}}}
\newcommand{\PP}[1][]{\ensuremath{\ifstrempty{#1}{\mathbb{P}}{\mathbb{P}^{#1}}}}
\newcommand{\HH}[1][]{\ensuremath{\ifstrempty{#1}{\mathbb{H}}{\mathbb{H}^{#1}}}}
\newcommand{\FF}[1][]{\ensuremath{\ifstrempty{#1}{\mathbb{F}}{\mathbb{F}^{#1}}}}
% expected value
\newcommand{\EE}{\ensuremath{\mathbb{E}}}
\newcommand{\charin}{\text{ char }}
\DeclareMathOperator{\sign}{sign}
\DeclareMathOperator{\Aut}{Aut}
\DeclareMathOperator{\Inn}{Inn}
\DeclareMathOperator{\Syl}{Syl}
\DeclareMathOperator{\Gal}{Gal}
\DeclareMathOperator{\GL}{GL} % General linear group
\DeclareMathOperator{\SL}{SL} % Special linear group

%---------------------------------------
% BlackBoard Math Fonts :-
%---------------------------------------

%Captital Letters
\newcommand{\bbA}{\mathbb{A}}	\newcommand{\bbB}{\mathbb{B}}
\newcommand{\bbC}{\mathbb{C}}	\newcommand{\bbD}{\mathbb{D}}
\newcommand{\bbE}{\mathbb{E}}	\newcommand{\bbF}{\mathbb{F}}
\newcommand{\bbG}{\mathbb{G}}	\newcommand{\bbH}{\mathbb{H}}
\newcommand{\bbI}{\mathbb{I}}	\newcommand{\bbJ}{\mathbb{J}}
\newcommand{\bbK}{\mathbb{K}}	\newcommand{\bbL}{\mathbb{L}}
\newcommand{\bbM}{\mathbb{M}}	\newcommand{\bbN}{\mathbb{N}}
\newcommand{\bbO}{\mathbb{O}}	\newcommand{\bbP}{\mathbb{P}}
\newcommand{\bbQ}{\mathbb{Q}}	\newcommand{\bbR}{\mathbb{R}}
\newcommand{\bbS}{\mathbb{S}}	\newcommand{\bbT}{\mathbb{T}}
\newcommand{\bbU}{\mathbb{U}}	\newcommand{\bbV}{\mathbb{V}}
\newcommand{\bbW}{\mathbb{W}}	\newcommand{\bbX}{\mathbb{X}}
\newcommand{\bbY}{\mathbb{Y}}	\newcommand{\bbZ}{\mathbb{Z}}

%---------------------------------------
% MathCal Fonts :-
%---------------------------------------

%Captital Letters
\newcommand{\mcA}{\mathcal{A}}	\newcommand{\mcB}{\mathcal{B}}
\newcommand{\mcC}{\mathcal{C}}	\newcommand{\mcD}{\mathcal{D}}
\newcommand{\mcE}{\mathcal{E}}	\newcommand{\mcF}{\mathcal{F}}
\newcommand{\mcG}{\mathcal{G}}	\newcommand{\mcH}{\mathcal{H}}
\newcommand{\mcI}{\mathcal{I}}	\newcommand{\mcJ}{\mathcal{J}}
\newcommand{\mcK}{\mathcal{K}}	\newcommand{\mcL}{\mathcal{L}}
\newcommand{\mcM}{\mathcal{M}}	\newcommand{\mcN}{\mathcal{N}}
\newcommand{\mcO}{\mathcal{O}}	\newcommand{\mcP}{\mathcal{P}}
\newcommand{\mcQ}{\mathcal{Q}}	\newcommand{\mcR}{\mathcal{R}}
\newcommand{\mcS}{\mathcal{S}}	\newcommand{\mcT}{\mathcal{T}}
\newcommand{\mcU}{\mathcal{U}}	\newcommand{\mcV}{\mathcal{V}}
\newcommand{\mcW}{\mathcal{W}}	\newcommand{\mcX}{\mathcal{X}}
\newcommand{\mcY}{\mathcal{Y}}	\newcommand{\mcZ}{\mathcal{Z}}


%---------------------------------------
% Bold Math Fonts :-
%---------------------------------------

%Captital Letters
\newcommand{\bmA}{\boldsymbol{A}}	\newcommand{\bmB}{\boldsymbol{B}}
\newcommand{\bmC}{\boldsymbol{C}}	\newcommand{\bmD}{\boldsymbol{D}}
\newcommand{\bmE}{\boldsymbol{E}}	\newcommand{\bmF}{\boldsymbol{F}}
\newcommand{\bmG}{\boldsymbol{G}}	\newcommand{\bmH}{\boldsymbol{H}}
\newcommand{\bmI}{\boldsymbol{I}}	\newcommand{\bmJ}{\boldsymbol{J}}
\newcommand{\bmK}{\boldsymbol{K}}	\newcommand{\bmL}{\boldsymbol{L}}
\newcommand{\bmM}{\boldsymbol{M}}	\newcommand{\bmN}{\boldsymbol{N}}
\newcommand{\bmO}{\boldsymbol{O}}	\newcommand{\bmP}{\boldsymbol{P}}
\newcommand{\bmQ}{\boldsymbol{Q}}	\newcommand{\bmR}{\boldsymbol{R}}
\newcommand{\bmS}{\boldsymbol{S}}	\newcommand{\bmT}{\boldsymbol{T}}
\newcommand{\bmU}{\boldsymbol{U}}	\newcommand{\bmV}{\boldsymbol{V}}
\newcommand{\bmW}{\boldsymbol{W}}	\newcommand{\bmX}{\boldsymbol{X}}
\newcommand{\bmY}{\boldsymbol{Y}}	\newcommand{\bmZ}{\boldsymbol{Z}}
%Small Letters
\newcommand{\bma}{\boldsymbol{a}}	\newcommand{\bmb}{\boldsymbol{b}}
\newcommand{\bmc}{\boldsymbol{c}}	\newcommand{\bmd}{\boldsymbol{d}}
\newcommand{\bme}{\boldsymbol{e}}	\newcommand{\bmf}{\boldsymbol{f}}
\newcommand{\bmg}{\boldsymbol{g}}	\newcommand{\bmh}{\boldsymbol{h}}
\newcommand{\bmi}{\boldsymbol{i}}	\newcommand{\bmj}{\boldsymbol{j}}
\newcommand{\bmk}{\boldsymbol{k}}	\newcommand{\bml}{\boldsymbol{l}}
\newcommand{\bmm}{\boldsymbol{m}}	\newcommand{\bmn}{\boldsymbol{n}}
\newcommand{\bmo}{\boldsymbol{o}}	\newcommand{\bmp}{\boldsymbol{p}}
\newcommand{\bmq}{\boldsymbol{q}}	\newcommand{\bmr}{\boldsymbol{r}}
\newcommand{\bms}{\boldsymbol{s}}	\newcommand{\bmt}{\boldsymbol{t}}
\newcommand{\bmu}{\boldsymbol{u}}	\newcommand{\bmv}{\boldsymbol{v}}
\newcommand{\bmw}{\boldsymbol{w}}	\newcommand{\bmx}{\boldsymbol{x}}
\newcommand{\bmy}{\boldsymbol{y}}	\newcommand{\bmz}{\boldsymbol{z}}

%---------------------------------------
% Scr Math Fonts :-
%---------------------------------------

\newcommand{\sA}{{\mathscr{A}}}   \newcommand{\sB}{{\mathscr{B}}}
\newcommand{\sC}{{\mathscr{C}}}   \newcommand{\sD}{{\mathscr{D}}}
\newcommand{\sE}{{\mathscr{E}}}   \newcommand{\sF}{{\mathscr{F}}}
\newcommand{\sG}{{\mathscr{G}}}   \newcommand{\sH}{{\mathscr{H}}}
\newcommand{\sI}{{\mathscr{I}}}   \newcommand{\sJ}{{\mathscr{J}}}
\newcommand{\sK}{{\mathscr{K}}}   \newcommand{\sL}{{\mathscr{L}}}
\newcommand{\sM}{{\mathscr{M}}}   \newcommand{\sN}{{\mathscr{N}}}
\newcommand{\sO}{{\mathscr{O}}}   \newcommand{\sP}{{\mathscr{P}}}
\newcommand{\sQ}{{\mathscr{Q}}}   \newcommand{\sR}{{\mathscr{R}}}
\newcommand{\sS}{{\mathscr{S}}}   \newcommand{\sT}{{\mathscr{T}}}
\newcommand{\sU}{{\mathscr{U}}}   \newcommand{\sV}{{\mathscr{V}}}
\newcommand{\sW}{{\mathscr{W}}}   \newcommand{\sX}{{\mathscr{X}}}
\newcommand{\sY}{{\mathscr{Y}}}   \newcommand{\sZ}{{\mathscr{Z}}}


%---------------------------------------
% Math Fraktur Font
%---------------------------------------

%Captital Letters
\newcommand{\mfA}{\mathfrak{A}}	\newcommand{\mfB}{\mathfrak{B}}
\newcommand{\mfC}{\mathfrak{C}}	\newcommand{\mfD}{\mathfrak{D}}
\newcommand{\mfE}{\mathfrak{E}}	\newcommand{\mfF}{\mathfrak{F}}
\newcommand{\mfG}{\mathfrak{G}}	\newcommand{\mfH}{\mathfrak{H}}
\newcommand{\mfI}{\mathfrak{I}}	\newcommand{\mfJ}{\mathfrak{J}}
\newcommand{\mfK}{\mathfrak{K}}	\newcommand{\mfL}{\mathfrak{L}}
\newcommand{\mfM}{\mathfrak{M}}	\newcommand{\mfN}{\mathfrak{N}}
\newcommand{\mfO}{\mathfrak{O}}	\newcommand{\mfP}{\mathfrak{P}}
\newcommand{\mfQ}{\mathfrak{Q}}	\newcommand{\mfR}{\mathfrak{R}}
\newcommand{\mfS}{\mathfrak{S}}	\newcommand{\mfT}{\mathfrak{T}}
\newcommand{\mfU}{\mathfrak{U}}	\newcommand{\mfV}{\mathfrak{V}}
\newcommand{\mfW}{\mathfrak{W}}	\newcommand{\mfX}{\mathfrak{X}}
\newcommand{\mfY}{\mathfrak{Y}}	\newcommand{\mfZ}{\mathfrak{Z}}
%Small Letters
\newcommand{\mfa}{\mathfrak{a}}	\newcommand{\mfb}{\mathfrak{b}}
\newcommand{\mfc}{\mathfrak{c}}	\newcommand{\mfd}{\mathfrak{d}}
\newcommand{\mfe}{\mathfrak{e}}	\newcommand{\mff}{\mathfrak{f}}
\newcommand{\mfg}{\mathfrak{g}}	\newcommand{\mfh}{\mathfrak{h}}
\newcommand{\mfi}{\mathfrak{i}}	\newcommand{\mfj}{\mathfrak{j}}
\newcommand{\mfk}{\mathfrak{k}}	\newcommand{\mfl}{\mathfrak{l}}
\newcommand{\mfm}{\mathfrak{m}}	\newcommand{\mfn}{\mathfrak{n}}
\newcommand{\mfo}{\mathfrak{o}}	\newcommand{\mfp}{\mathfrak{p}}
\newcommand{\mfq}{\mathfrak{q}}	\newcommand{\mfr}{\mathfrak{r}}
\newcommand{\mfs}{\mathfrak{s}}	\newcommand{\mft}{\mathfrak{t}}
\newcommand{\mfu}{\mathfrak{u}}	\newcommand{\mfv}{\mathfrak{v}}
\newcommand{\mfw}{\mathfrak{w}}	\newcommand{\mfx}{\mathfrak{x}}
\newcommand{\mfy}{\mathfrak{y}}	\newcommand{\mfz}{\mathfrak{z}}


\title{\Huge{Electronics 2}\\Semester 6}
\author{Ahmad Abu Zainab}
\date{}

\ctikzset{tripoles/mos style/arrows}
\ctikzset{transistors/arrow pos=end}
\ctikzset{tripoles/.style={scale=2}}
\ctikzset{tr circle=true}

\begin{document}

\maketitle
\newpage% or \cleardoublepage
% \pdfbookmark[<level>]{<title>}{<dest>}
\pdfbookmark[section]{\contentsname}{toc}
\tableofcontents
\pagebreak

\chapter{Differential Amplifiers}

\section{BJT Differential Amplifier}

\begin{figure}[H]
	\centering
	\begin{subfigure}[b]{0.45\textwidth}
		\centering
		\begin{circuitikz}
			% Transistors
			\draw (0,0) node[npn] (Q1) {};
			\draw (2,0) node[npn, xscale=-1] (Q2) {};

			% Resistors
			\draw (Q1.E) to[short, i_=$I_{EE}/2$] ++(0,-0.75) to[short, -*] ++(1,0) coordinate (Vp);
			\draw (Q2.E) to[short, i=$I_{EE}/2$] ++(0,-0.75) to[short, -*] (Vp);

			\draw (Q1.C) to[R, l=$R_C$] ++(0,1.75) node[vcc] {$V_{CC}$};
			\draw (Q2.C) to[R, l_=$R_C$] ++(0,1.75) node[vcc] {$V_{CC}$};

			% Input
			\draw (Q1.B) to[short, -o] ++(-0.5,0) node[left] {$v_{in1}$};
			\draw (Q2.B) to[short, -o] ++(0.5,0) node[right] {$v_{in2}$};

			% Output
			\draw (Q1.C) to[short, -o] ++(-0.5,0) node[left] {$v_{out1}$};
			\draw (Q2.C) to[short, -o] ++(0.5,0) node[right] {$v_{out2}$};

			% Biasing
			\draw (Vp) to[american, isource, l=$I_{EE}$] ++(0,-1.5) node[vcc, rotate=180] {};
		\end{circuitikz}
		\caption{BJT Differential Amplifier}
		\label{fig:diffamp}
	\end{subfigure}
	\hfill
	\begin{subfigure}[b]{0.45\textwidth}
		\centering
		\begin{circuitikz}
			% Transistors
			\draw (0,0) node[npn] (Q1) {};
			\draw (2,0) node[npn, xscale=-1] (Q2) {};

			% Resistors
			\draw (Q1.E) to[R, l=$R_E$] ++(0,-1.75) to[short, -*] ++(1,0) coordinate (Vp);
			\draw (Q2.E) to[R, l=$R_E$] ++(0,-1.75) to[short, -*] (Vp);

			\draw (Q1.C) to[R, l=$R_C$] ++(0,1.75) node[vcc] {$V_{CC}$};
			\draw (Q2.C) to[R, l_=$R_C$] ++(0,1.75) node[vcc] {$V_{CC}$};

			% Input
			\draw (Q1.B) to[short, -o] ++(-0.5,0) node[left] {$v_{in1}$};
			\draw (Q2.B) to[short, -o] ++(0.5,0) node[right] {$v_{in2}$};

			% Output
			\draw (Q1.C) to[short, -o] ++(-0.5,0) node[left] {$v_{out1}$};
			\draw (Q2.C) to[short, -o] ++(0.5,0) node[right] {$v_{out2}$};

			% Biasing
			\draw (Vp) to[american, isource, l=$I_{EE}$] ++(0,-1.5) node[vcc, rotate=180] {};
		\end{circuitikz}
		\caption{BJT Differential Amplifier with Emitter Resistance}
	\end{subfigure}
\end{figure}

The gain of the differential amplifier is given by

\[
	|A_d| = \begin{cases}
		\displaystyle \frac{\text{Total resistance in the collectors}}{\text{Total resistance in the emitters}} & \text{with emitter resistance}    \\
		g_m \cdot R_C                                                                                           & \text{without emitter resistance}
	\end{cases}
	.\]

The common mode gain is given by

\[
	|A_\text{cm}| = \frac{\Delta R_C}{2R_EE} = \frac{\Delta R_C}{2R_E + 2r_e}
	.\]

The common mode rejection ratio is given by

\[
	\text{CMRR} = 20 \log \frac{|A_d|}{|A_\text{cm}|}
	.\]

The minimum and maximum common mode input voltage (to operate the amplifier) is given by

\[
	V_{CM \max} = V_{CC} - \alpha \frac{1}{2} R_C + \SI{0.4}{\volt} \quad ; \quad V_{CM \min} = -V_{EE} + V_{CS} + V_{EE}
	.\]

\section{MOSFET Differential Amplifier}

\begin{figure}[H]
	\centering
	\begin{circuitikz}
		% Transistors
		\draw (0,0) node[nmos] (Q1) {};
		\draw (2,0) node[nmos, xscale=-1] (Q2) {};

		% Resistors
		\draw (Q1.S) to[short] ++(0,-0.75) to[short, -*] ++(1,0) coordinate (Vp);
		\draw (Q2.S) to[short] ++(0,-0.75) to[short, -*] (Vp);

		\draw (Q1.D) to[R, l=$R_D$] ++(0,1.75) node[vcc] {$V_{DD}$};
		\draw (Q2.D) to[R, l_=$R_D$] ++(0,1.75) node[vcc] {$V_{DD}$};

		% Input
		\draw (Q1.G) to[short, -o] ++(-0.5,0) node[left] {$v_{in1}$};
		\draw (Q2.G) to[short, -o] ++(0.5,0) node[right] {$v_{in2}$};

		% Output
		\draw (Q1.D) to[short, -o] ++(-0.5,0) node[left] {$v_{out1}$};
		\draw (Q2.D) to[short, -o] ++(0.5,0) node[right] {$v_{out2}$};

		% Biasing
		\draw (Vp) to[american, isource, l=$I_{EE}$] ++(0,-1.5) node[vcc, rotate=180] {};
	\end{circuitikz}
	\caption{MOSFET Differential Amplifier}
	\label{fig:diffampmos}
\end{figure}

Similarly, the gain of the differential amplifier is given by

\[
	|A_d| = g_m \cdot R_D
	.\]

The minimum and maximum common mode input voltage (to operate the amplifier) is given by

\[
	V_{CM\max} = V_t + V_{DD} - \frac{I}{2}R_{D} \quad ; \quad V_{CM\min} = -V_{SS} + V_{CS} +V_{t}+V_{OV}
	.\]

\chapter{Current Mirrors}

\section{Current Mirror}

\begin{figure}[H]
	\centering
	\begin{circuitikz}
		\draw (0,0) node[npn, xscale=-1] (Q1) {};
		\draw (2,0) node[npn] (Q2) {};

		\draw (Q1.C) -- ++(0,0.1) to[R, l=$R$, f<=$I_\text{Ref}$] ++(0,1.5) node[vcc] {$V_{CC}$};
		\draw[dotted] (Q2.C) to[short, f_<=$I_o \approx I_\text{Ref}$] ++(0,1);

		\draw (Q1.C) node[circ]{} -| (Q1.B) node[circ]{}  -- (Q2.B);

		\draw (Q1.E) node[ground] {};
		\draw (Q2.E) node[ground] {};
	\end{circuitikz}
	\caption{BJT Current Mirror}
	\label{fig:bjtcm}
\end{figure}

\section{Wilson Current Mirror}

A Wilson current mirror is a current mirror that uses two transistors to provide a more accurate current mirror.

\begin{figure}[H]
	\centering
	\begin{circuitikz}
		\draw (0,0) node[npn, xscale=-1] (Q1) {};
		\draw (2,0) node[npn] (Q2) {};
		\draw (2,2) node[npn] (Q3) {};

		\draw (Q2.C) -- (Q3.E);

		\draw[dotted] (Q3.C) to[short, f_<=$I_o \approx I_\text{Ref}$] ++(0,1);
		\draw (Q1.C) |- (Q3.B);
		\draw (Q1.C) -- ++(0,1) to[american, isource, l=$I_\text{Ref}$] ++(0,2) node[vcc] {};
		\draw (Q2.C) node[circ]{} -| (Q2.B) node[circ]{}  -- (Q1.B);

		\draw (Q1.E) node[vcc, rotate=180] {\raisebox{0.5em}{\rotatebox{-180}{$-V_{EE}$}}};
		\draw (Q2.E) node[vcc, rotate=180] {\raisebox{0.5em}{\rotatebox{-180}{$-V_{EE}$}}};
	\end{circuitikz}
	\caption{Wilson Current Mirror}
	\label{fig:wilsoncm}
\end{figure}

\section{Widlar Current Source}

\begin{figure}[H]
	\centering
	\begin{circuitikz}
		\draw (0,0) node[npn, xscale=-1] (Q1) {};
		\draw (2,0) node[npn] (Q2) {};

		\draw (Q1.C) -- ++(0,0.1) to[R, l=$R$, f<=$I_\text{Ref}$] ++(0,1.5) node[vcc] {$V_{CC}$};
		\draw[dotted] (Q2.C) to[short, f_<=$I_o \approx I_\text{Ref}$] ++(0,1);

		\draw (Q1.C) node[circ]{} -| (Q1.B) node[circ]{}  -- (Q2.B);

		\draw (Q1.E) -- ++(0,-1.5) node[ground] {};
		\draw (Q2.E) to[R, l=$R_E \approx \frac{V_T}{I_o} \ln \frac{I_\text{Ref}}{I_o} $] ++(0,-1.5) node[ground] {};
	\end{circuitikz}
	\caption{Widlar Current Source}
	\label{fig:widlarcs}
\end{figure}

\chapter{Frequency Response}

The amplifier's gain varies with frequency. The response is studied at 3 points:
\begin{enumerate}
	\ii Low frequency: Coupling and bypass capacitors are not shorted.
	\ii Mid frequency: Typical gain, all capacitors are shorted.
	\ii High frequency: Coupling and bypass capacitors are shorted, but internal capacitances of the transistors are considered.
\end{enumerate}

\begin{figure}[H]
	\centering
	% example bode plot
	\begin{tikzpicture}
		\begin{semilogxaxis}[
				width=0.8\textwidth,
				height=0.4\textwidth,
				grid=both,
				grid style={line width=.1pt, draw=gray!10},
				major grid style={line width=.2pt,draw=gray!50},
				xlabel={Frequency (\si{\hertz})},
				ylabel={Gain (\si{\decibel})},
				xmin=1, xmax=1e10,
				ymin=0, ymax=250,
				xtick={1, 10, 100, 1000, 10000, 100000, 1000000, 10000000, 100000000, 1000000000, 10000000000},
				ytick={0, 141.42, 200},
				yticklabels={0, $0.707A_{v\text{mid}}$, $A_{v\text{mid}}$},
				yticklabel style={/pgf/number format/fixed},
				xticklabel style={/pgf/number format/fixed},
				legend pos=north west,
				legend style={at={(0.5,-0.2)},anchor=north},
				legend cell align={left},
			]

			\addplot[domain=1:1e10, samples=200, red, thick] {200 * log10(x/5 * 1/((1+x/50)* (1+x/100000000)))};
			\addplot[domain=1:1e10, samples=200, blue, dashed] {141.42};

			\legend{Gain, -3dB}

			\draw[dashed] (axis cs: 50, 0) -- (axis cs: 50, 250);
			\draw[dashed] (axis cs: 100000000, 0) -- (axis cs: 100000000, 250);
			\draw (axis cs: 50, 141.42) node[circ]{} node[above left]{$f_{L}$};
			\draw (axis cs: 100000000, 141.42) node[circ]{} node[above right]{$f_{H}$};
		\end{semilogxaxis}
	\end{tikzpicture}
\end{figure}

The cutoff frequencies are the frequencies at which the gain is reduced by \SI{3}{\decibel}, and the bandwidth is the difference between the two cutoff frequencies.

\[
	\text{BW} = f_H - f_L = f_2 - f_1 \quad ; \quad \begin{cases}
		f_L \text{ determined using the critical frequency of } C_i, C_o, C_E \\
		f_H \text{ determined using the critical frequency of } C_{\pi}, C_{\mu}, C_{\text{cs}}
	\end{cases}
	.\]

\section{Low Frequency Response of Amplifiers}

\subsection{Low Frequency Response of CE Amplifiers}

\begin{figure}[H]
	\centering
	\begin{circuitikz}
		\draw (0,0) node[npn] (Q) {};

		% Input 
		\draw (Q.B) ++(-0.5,0) to[C, l=$C_i$] ++(-2,0) to[R, l=$R_s$] ++(0,-2) to[sV, l=$V_{s}$] ++(0,-1.5) coordinate (GND) node[ground] {};

		% Base
		\draw (Q.B) -- ++(-0.5,0) to[R, l=$R_1$] ++(0,2.5) coordinate (VCC) node[vcc] {$V_{CC}$};
		\draw let \p1=(GND), \p2=(Q.B) in (Q.B) ++(-0.5,0) to[R, l=$R_2$, *-] ++(0,\y1-\y2) node[ground] {};

		% Emitter
		\draw let \p1=(GND), \p2=(Q.E) in (Q.E) to[R, l=$R_E$, *-] ++(0,\y1-\y2) node[ground] {};
		\draw let \p1=(GND), \p2=(Q.E) in (Q.E) -- ++(1.5,0) to[C, l=$C_E$] ++(0,\y1-\y2) node[ground] {};

		% Collector 
		\draw let \p1=(VCC), \p2=(Q.C) in (Q.C) to[R, l=$R_C$, *-] ++(0,\y1-\y2) node[vcc] {$V_{CC}$};
		\draw let \p1=(GND), \p2=(Q.C) in (Q.C) to[C, l=$C_o$, *-*] ++(3,0) coordinate (Vo) to[R, l=$R_L$] ++(0,\y1-\y2) node[ground] {};
		\draw (Vo) to[short, -o] ++(0.5,0) node[right] {$v_{out}$};
	\end{circuitikz}
	\caption{CE Amplifier}
	\label{fig:ceamp}
\end{figure}

To determine the cutoff frequencies for each capacitor, we short the other capacitors and determine the critical frequency for each capacitor.

\begin{align*}
	f_{Ci} & = \frac{1}{2 \pi (R_{s} + R_{i}) C_{i}} \quad ; \quad R_{i} = R_{1}\parallel R_{2} \parallel (\beta + 1) r_{e}                                               \\
	f_{Co} & = \frac{1}{2 \pi (R_{o} + R_{L}) C_{o}} \quad ; \quad R_{o} = R_{C}\parallel r_{o}                                                                           \\
	f_{CE} & = \frac{1}{2 \pi R_\text{eq} C_{E}} \quad ; \quad R_\text{eq} = R_{E}\parallel \left[ \frac{R_{s}\parallel R_{1} \parallel R_{2}}{\beta + 1} + r_{e} \right]
\end{align*}

\subsection{Low Frequency Response of CS Amplifiers}

\begin{figure}[H]
	\centering
	\begin{circuitikz}
		\draw (0,0) node[nmos] (Q) {};

		% Input 
		\draw (Q.G) -- ++(-0.5,0) to[C, l=$C_i$] ++(-2,0) to[R, l=$R_s$] ++(0,-2) to[sV, l=$V_{s}$] ++(0,-1.5) coordinate (GND) node[ground] {};

		% Gate
		\draw let \p1=(GND), \p2=(Q.G) in (Q.G) ++(-0.5,0) to[R, l=$R_G$, *-] ++(0,\y1-\y2) node[ground] {};

		% Source
		\draw let \p1=(GND), \p2=(Q.S) in (Q.S) to[R, l=$R_S$, *-] ++(0,\y1-\y2) node[ground] {};
		\draw let \p1=(GND), \p2=(Q.S) in (Q.S) -- ++(1.5,0) to[C, l=$C_S$] ++(0,\y1-\y2) node[ground] {};

		% Drain
		\draw (Q.D) to[R, l=$R_D$, *-] ++(0,2.5) node[vcc] {$V_{CC}$};
		\draw let \p1=(GND), \p2=(Q.D) in (Q.D) to[C, l=$C_o$, *-*] ++(3,0) coordinate (Vo) to[R, l=$R_L$] ++(0,\y1-\y2) node[ground] {};
		\draw (Vo) to[short, -o] ++(0.5,0) node[right] {$v_{out}$};
	\end{circuitikz}
	\caption{CS Amplifier}
	\label{fig:csamp}
\end{figure}

\begin{align*}
	f_{Ci} & = \frac{1}{2 \pi (R_{s} + R_{G}) C_{i}}                                                       \\
	f_{Co} & = \frac{1}{2 \pi (R_{L} + R_{o}) C_{L}} \quad ; \quad R_{o} = R_{D}\parallel r_{o}            \\
	f_{CS} & = \frac{1}{2 \pi R_\text{eq} C_{S}} \quad ; \quad R_\text{eq} = R_{S} \parallel \frac{1}{g_m}
\end{align*}

\section{High Frequency Response of Amplifiers}

\subsection{High Frequency Response of CE Amplifiers}

We consider the internal capacitances of the transistor, $C_{\pi}$($C_\text{be}$), $C_{\mu}$($C_\text{bc}$), and $C_{\text{ce}}$ as well as the wiring capacitances $C_{wi}$ and $C_{wo}$.
We say that $C_i$, $C_o$, and $C_E$ are shorted.

\begin{figure}[H]
	\centering
	\scalebox{1.5}{
		\begin{circuitikz}
			\ctikzset{capacitors/scale=0.45}
			\draw (0,0) node[npn] (Q) {};
			\draw let \p1=(Q.B), \p2=(Q.C) in (Q.C) to[C, l_=\scalebox{0.75}{$C_{\mu}$}, *-] (\x1,\y2) -- (Q.B);
			\draw let \p1=(Q.C), \p2=(Q.E) in (Q.C) -- ++(0.5,0) to[C, l=\scalebox{0.75}{$C_\text{ce}$}] ++(0,\y2-\y1) -- (Q.E) to[short, *-*] (Q.E);
			\draw let \p1=(Q.B), \p2=(Q.E) in (Q.B) to[C, l_=\scalebox{0.75}{$C_{\pi}$}, *-] (\x1,\y2) -- (Q.E);
		\end{circuitikz}
	}
\end{figure}

\thm{Miller Effect}{
	The capacitor connected between the input and output is called the feedback capacitor. Miller's theorem states that the feedback capacitor is equivalent to two capacitors at the inout and output.
	\begin{align*}
		C_{Mi} & = C_{f} (1 - A_v)                 \\
		C_{Mo} & = C_{f} \lt(1 - \frac{1}{A_v}\rt)
	\end{align*}
}

Using Miller's theorem, we say that
\begin{align*}
	C_{Mi} & = (1 - A_v)C_\text{bc} \quad                 & \quad C_{i} & = C_{wi} + C_\text{be} + C_{Mi}   \\
	C_{Mo} & = \lt(1 - \frac{1}{A_v}\rt)C_\text{bc} \quad & \quad C_{o} & = C_{wo} + C_{\text{ce}} + C_{Mo}
\end{align*}

The cutoff frequency is given by

\begin{align*}
	f_{Hi} & = \frac{1}{2 \pi R_{Thi} C_i} \quad & \quad R_{Thi} & = R_s \parallel R_1 \parallel R_2 \parallel r_{\pi} \\
	f_{Ho} & = \frac{1}{2 \pi R_{Tho} C_o} \quad & \quad R_{Tho} & = R_L \parallel R_C \parallel r_{o}
\end{align*}

\subsection{High Frequency Response of CS Amplifiers}

\begin{align*}
	R_{Thi} & = R_s \parallel R_G \quad               & \quad R_{Tho} & = R_D \parallel R_L \parallel r_d      \\
	C_{Mi}  & = (1 - A_v)C_\text{gd} \quad            & \quad C_{Mo}  & = \lt(1 - \frac{1}{A_v}\rt)C_\text{gd} \\
	C_i     & = C_{wi} + C_{\text{gs}} + C_{Mi} \quad & \quad C_o     & = C_{wo} + C_{\text{ds}} + C_{Mo}      \\
	f_{Hi}  & = \frac{1}{2 \pi R_{Thi} C_i} \quad     & \quad f_{Ho}  & = \frac{1}{2 \pi R_{Tho} C_o}
\end{align*}

\section{Open-Circuit Time Constants}

Open-circuit time constants is a method used to approximate the cutoff frequencies of the amplifier.

\begin{align*}
	b_1 & = \sum_{i=1}^n C_i R_i^0                  \\
	b_2 & = \sum_{i=1}^n \sum_j C_i R_i^0 R_j^i C_j
\end{align*}

Where $R_i^0$ is the resistance seen by the capacitor when all other capacitors are shorted.

\[
	\omega_H \approx p_1 \approx \frac{1}{b_1} = \frac{1}{\sum_{i=1}^n C_i R_i^0}
	.\]

\chapter{Negative Feedback}

There are 4 types of negative feedback topologies:

\begin{enumerate}
	\ii Feedback Voltage Amplifier (Series—Shunt)
	\ii Feedback Transconductance Amplifier (Series—Series)
	\ii Feedback Transresistance Amplifier (Shunt—Shunt)
	\ii Feedback Current Amplifier (Shunt—Series)
\end{enumerate}

We define

\[
	\begin{cases} \displaystyle
		\beta = \frac{x_f}{x_o}    & \text{Feedback factor}  \\
		A                          & \text{Open-loop gain}   \\
		A \beta                    & \text{Loop gain}        \\
		\displaystyle
		A_f = \frac{A}{1 + A\beta} & \text{Closed-loop gain}
	\end{cases}
	.\]

\begin{table}[H]
	\centering
	\def\arraystretch{2.5}
	\begin{tabular}[c]{|l|l|l|l|l|}
		\hline
		         & Series-Shunt V-V        & Shunt-Shunt I-V         & Series-Series V-I  & Shunt-Series I-I        \\
		\hline
		$R_{if}$ & $R_i (1+A\beta)$        & $\dfrac{R_i}{1+A\beta}$ & $R_i (1+A\beta)$   & $\dfrac{R_i}{1+A\beta}$ \\
		$R_{of}$ & $\dfrac{R_o}{1+A\beta}$ & $\dfrac{R_o}{1+A\beta}$ & $R_o (1+A\beta)$   & $R_o (1+A\beta)$        \\
		$\beta$  & $\dfrac{V_f}{V_o}$      & $\dfrac{I_f}{V_o}$      & $\dfrac{V_f}{I_o}$ & $\dfrac{I_f}{I_o}$      \\
		\hline
	\end{tabular}
\end{table}


\end{document}

\documentclass{report}

\input{../template/preamble}
\input{../template/macros}
\input{../template/letterfonts}

\title{\Huge{Electronics 2}\\Semester 6}
\author{Ahmad Abu Zainab}
\date{}

\ctikzset{tripoles/mos style/arrows}
\ctikzset{transistors/arrow pos=end}
\ctikzset{tripoles/.style={scale=2}}
\ctikzset{tr circle=true}

\begin{document}

\maketitle
\newpage% or \cleardoublepage
% \pdfbookmark[<level>]{<title>}{<dest>}
\pdfbookmark[section]{\contentsname}{toc}
\tableofcontents
\pagebreak

\chapter{Differential Amplifiers}

\section{BJT Differential Amplifier}

\begin{figure}[H]
	\centering
	\begin{subfigure}[b]{0.45\textwidth}
		\centering
		\begin{circuitikz}
			% Transistors
			\draw (0,0) node[npn] (Q1) {};
			\draw (2,0) node[npn, xscale=-1] (Q2) {};

			% Resistors
			\draw (Q1.E) to[short, i_=$I_{EE}/2$] ++(0,-0.75) to[short, -*] ++(1,0) coordinate (Vp);
			\draw (Q2.E) to[short, i=$I_{EE}/2$] ++(0,-0.75) to[short, -*] (Vp);

			\draw (Q1.C) to[R, l=$R_C$] ++(0,1.75) node[vcc] {$V_{CC}$};
			\draw (Q2.C) to[R, l_=$R_C$] ++(0,1.75) node[vcc] {$V_{CC}$};

			% Input
			\draw (Q1.B) to[short, -o] ++(-0.5,0) node[left] {$v_{in1}$};
			\draw (Q2.B) to[short, -o] ++(0.5,0) node[right] {$v_{in2}$};

			% Output
			\draw (Q1.C) to[short, -o] ++(-0.5,0) node[left] {$v_{out1}$};
			\draw (Q2.C) to[short, -o] ++(0.5,0) node[right] {$v_{out2}$};

			% Biasing
			\draw (Vp) to[american, isource, l=$I_{EE}$] ++(0,-1.5) node[vcc, rotate=180] {};
		\end{circuitikz}
		\caption{BJT Differential Amplifier}
		\label{fig:diffamp}
	\end{subfigure}
	\hfill
	\begin{subfigure}[b]{0.45\textwidth}
		\centering
		\begin{circuitikz}
			% Transistors
			\draw (0,0) node[npn] (Q1) {};
			\draw (2,0) node[npn, xscale=-1] (Q2) {};

			% Resistors
			\draw (Q1.E) to[R, l=$R_E$] ++(0,-1.75) to[short, -*] ++(1,0) coordinate (Vp);
			\draw (Q2.E) to[R, l=$R_E$] ++(0,-1.75) to[short, -*] (Vp);

			\draw (Q1.C) to[R, l=$R_C$] ++(0,1.75) node[vcc] {$V_{CC}$};
			\draw (Q2.C) to[R, l_=$R_C$] ++(0,1.75) node[vcc] {$V_{CC}$};

			% Input
			\draw (Q1.B) to[short, -o] ++(-0.5,0) node[left] {$v_{in1}$};
			\draw (Q2.B) to[short, -o] ++(0.5,0) node[right] {$v_{in2}$};

			% Output
			\draw (Q1.C) to[short, -o] ++(-0.5,0) node[left] {$v_{out1}$};
			\draw (Q2.C) to[short, -o] ++(0.5,0) node[right] {$v_{out2}$};

			% Biasing
			\draw (Vp) to[american, isource, l=$I_{EE}$] ++(0,-1.5) node[vcc, rotate=180] {};
		\end{circuitikz}
		\caption{BJT Differential Amplifier with Emitter Resistance}
	\end{subfigure}
\end{figure}

The gain of the differential amplifier is given by

\[
	|A_d| = \begin{cases}
		\displaystyle \frac{\text{Total resistance in the collectors}}{\text{Total resistance in the emitters}} & \text{with emitter resistance}    \\
		g_m \cdot R_C                                                                                           & \text{without emitter resistance}
	\end{cases}
	.\]

The common mode gain is given by

\[
	|A_\text{cm}| = \frac{\Delta R_C}{2R_EE} = \frac{\Delta R_C}{2R_E + 2r_e}
	.\]

The common mode rejection ratio is given by

\[
	\text{CMRR} = 20 \log \frac{|A_d|}{|A_\text{cm}|}
	.\]

The minimum and maximum common mode input voltage (to operate the amplifier) is given by

\[
	V_{CM \max} = V_{CC} - \alpha \frac{1}{2} R_C + \SI{0.4}{\volt} \quad ; \quad V_{CM \min} = -V_{EE} + V_{CS} + V_{EE}
	.\]

\section{MOSFET Differential Amplifier}

\begin{figure}[H]
	\centering
	\begin{circuitikz}
		% Transistors
		\draw (0,0) node[nmos] (Q1) {};
		\draw (2,0) node[nmos, xscale=-1] (Q2) {};

		% Resistors
		\draw (Q1.S) to[short] ++(0,-0.75) to[short, -*] ++(1,0) coordinate (Vp);
		\draw (Q2.S) to[short] ++(0,-0.75) to[short, -*] (Vp);

		\draw (Q1.D) to[R, l=$R_D$] ++(0,1.75) node[vcc] {$V_{DD}$};
		\draw (Q2.D) to[R, l_=$R_D$] ++(0,1.75) node[vcc] {$V_{DD}$};

		% Input
		\draw (Q1.G) to[short, -o] ++(-0.5,0) node[left] {$v_{in1}$};
		\draw (Q2.G) to[short, -o] ++(0.5,0) node[right] {$v_{in2}$};

		% Output
		\draw (Q1.D) to[short, -o] ++(-0.5,0) node[left] {$v_{out1}$};
		\draw (Q2.D) to[short, -o] ++(0.5,0) node[right] {$v_{out2}$};

		% Biasing
		\draw (Vp) to[american, isource, l=$I_{EE}$] ++(0,-1.5) node[vcc, rotate=180] {};
	\end{circuitikz}
	\caption{MOSFET Differential Amplifier}
	\label{fig:diffampmos}
\end{figure}

Similarly, the gain of the differential amplifier is given by

\[
	|A_d| = g_m \cdot R_D
	.\]

The minimum and maximum common mode input voltage (to operate the amplifier) is given by

\[
	V_{CM\max} = V_t + V_{DD} - \frac{I}{2}R_{D} \quad ; \quad V_{CM\min} = -V_{SS} + V_{CS} +V_{t}+V_{OV}
	.\]

\chapter{Current Mirrors}

\section{Current Mirror}

\begin{figure}[H]
	\centering
	\begin{circuitikz}
		\draw (0,0) node[npn, xscale=-1] (Q1) {};
		\draw (2,0) node[npn] (Q2) {};

		\draw (Q1.C) -- ++(0,0.1) to[R, l=$R$, f<=$I_\text{Ref}$] ++(0,1.5) node[vcc] {$V_{CC}$};
		\draw[dotted] (Q2.C) to[short, f_<=$I_o \approx I_\text{Ref}$] ++(0,1);

		\draw (Q1.C) node[circ]{} -| (Q1.B) node[circ]{}  -- (Q2.B);

		\draw (Q1.E) node[ground] {};
		\draw (Q2.E) node[ground] {};
	\end{circuitikz}
	\caption{BJT Current Mirror}
	\label{fig:bjtcm}
\end{figure}

\section{Wilson Current Mirror}

A Wilson current mirror is a current mirror that uses two transistors to provide a more accurate current mirror.

\begin{figure}[H]
	\centering
	\begin{circuitikz}
		\draw (0,0) node[npn, xscale=-1] (Q1) {};
		\draw (2,0) node[npn] (Q2) {};
		\draw (2,2) node[npn] (Q3) {};

		\draw (Q2.C) -- (Q3.E);

		\draw[dotted] (Q3.C) to[short, f_<=$I_o \approx I_\text{Ref}$] ++(0,1);
		\draw (Q1.C) |- (Q3.B);
		\draw (Q1.C) -- ++(0,1) to[american, isource, l=$I_\text{Ref}$] ++(0,2) node[vcc] {};
		\draw (Q2.C) node[circ]{} -| (Q2.B) node[circ]{}  -- (Q1.B);

		\draw (Q1.E) node[vcc, rotate=180] {\raisebox{0.5em}{\rotatebox{-180}{$-V_{EE}$}}};
		\draw (Q2.E) node[vcc, rotate=180] {\raisebox{0.5em}{\rotatebox{-180}{$-V_{EE}$}}};
	\end{circuitikz}
	\caption{Wilson Current Mirror}
	\label{fig:wilsoncm}
\end{figure}

\section{Widlar Current Source}

\begin{figure}[H]
	\centering
	\begin{circuitikz}
		\draw (0,0) node[npn, xscale=-1] (Q1) {};
		\draw (2,0) node[npn] (Q2) {};

		\draw (Q1.C) -- ++(0,0.1) to[R, l=$R$, f<=$I_\text{Ref}$] ++(0,1.5) node[vcc] {$V_{CC}$};
		\draw[dotted] (Q2.C) to[short, f_<=$I_o \approx I_\text{Ref}$] ++(0,1);

		\draw (Q1.C) node[circ]{} -| (Q1.B) node[circ]{}  -- (Q2.B);

		\draw (Q1.E) -- ++(0,-1.5) node[ground] {};
		\draw (Q2.E) to[R, l=$R_E \approx \frac{V_T}{I_o} \ln \frac{I_\text{Ref}}{I_o} $] ++(0,-1.5) node[ground] {};
	\end{circuitikz}
	\caption{Widlar Current Source}
	\label{fig:widlarcs}
\end{figure}

\chapter{Frequency Response}

The amplifier's gain varies with frequency. The response is studied at 3 points:
\begin{enumerate}
	\ii Low frequency: Coupling and bypass capacitors are not shorted.
	\ii Mid frequency: Typical gain, all capacitors are shorted.
	\ii High frequency: Coupling and bypass capacitors are shorted, but internal capacitances of the transistors are considered.
\end{enumerate}

\begin{figure}[H]
	\centering
	% example bode plot
	\begin{tikzpicture}
		\begin{semilogxaxis}[
				width=0.8\textwidth,
				height=0.4\textwidth,
				grid=both,
				grid style={line width=.1pt, draw=gray!10},
				major grid style={line width=.2pt,draw=gray!50},
				xlabel={Frequency (\si{\hertz})},
				ylabel={Gain (\si{\decibel})},
				xmin=1, xmax=1e10,
				ymin=0, ymax=250,
				xtick={1, 10, 100, 1000, 10000, 100000, 1000000, 10000000, 100000000, 1000000000, 10000000000},
				ytick={0, 141.42, 200},
				yticklabels={0, $0.707A_{v\text{mid}}$, $A_{v\text{mid}}$},
				yticklabel style={/pgf/number format/fixed},
				xticklabel style={/pgf/number format/fixed},
				legend pos=north west,
				legend style={at={(0.5,-0.2)},anchor=north},
				legend cell align={left},
			]

			\addplot[domain=1:1e10, samples=200, red, thick] {200 * log10(x/5 * 1/((1+x/50)* (1+x/100000000)))};
			\addplot[domain=1:1e10, samples=200, blue, dashed] {141.42};

			\legend{Gain, -3dB}

			\draw[dashed] (axis cs: 50, 0) -- (axis cs: 50, 250);
			\draw[dashed] (axis cs: 100000000, 0) -- (axis cs: 100000000, 250);
			\draw (axis cs: 50, 141.42) node[circ]{} node[above left]{$f_{L}$};
			\draw (axis cs: 100000000, 141.42) node[circ]{} node[above right]{$f_{H}$};
		\end{semilogxaxis}
	\end{tikzpicture}
\end{figure}

The cutoff frequencies are the frequencies at which the gain is reduced by \SI{3}{\decibel}, and the bandwidth is the difference between the two cutoff frequencies.

\[
	\text{BW} = f_H - f_L = f_2 - f_1 \quad ; \quad \begin{cases}
		f_L \text{ determined using the critical frequency of } C_i, C_o, C_E \\
		f_H \text{ determined using the critical frequency of } C_{\pi}, C_{\mu}, C_{\text{cs}}
	\end{cases}
	.\]

\section{Low Frequency Response of Amplifiers}

\subsection{Low Frequency Response of CE Amplifiers}

\begin{figure}[H]
	\centering
	\begin{circuitikz}
		\draw (0,0) node[npn] (Q) {};

		% Input 
		\draw (Q.B) ++(-0.5,0) to[C, l=$C_i$] ++(-2,0) to[R, l=$R_s$] ++(0,-2) to[sV, l=$V_{s}$] ++(0,-1.5) coordinate (GND) node[ground] {};

		% Base
		\draw (Q.B) -- ++(-0.5,0) to[R, l=$R_1$] ++(0,2.5) coordinate (VCC) node[vcc] {$V_{CC}$};
		\draw let \p1=(GND), \p2=(Q.B) in (Q.B) ++(-0.5,0) to[R, l=$R_2$, *-] ++(0,\y1-\y2) node[ground] {};

		% Emitter
		\draw let \p1=(GND), \p2=(Q.E) in (Q.E) to[R, l=$R_E$, *-] ++(0,\y1-\y2) node[ground] {};
		\draw let \p1=(GND), \p2=(Q.E) in (Q.E) -- ++(1.5,0) to[C, l=$C_E$] ++(0,\y1-\y2) node[ground] {};

		% Collector 
		\draw let \p1=(VCC), \p2=(Q.C) in (Q.C) to[R, l=$R_C$, *-] ++(0,\y1-\y2) node[vcc] {$V_{CC}$};
		\draw let \p1=(GND), \p2=(Q.C) in (Q.C) to[C, l=$C_o$, *-*] ++(3,0) coordinate (Vo) to[R, l=$R_L$] ++(0,\y1-\y2) node[ground] {};
		\draw (Vo) to[short, -o] ++(0.5,0) node[right] {$v_{out}$};
	\end{circuitikz}
	\caption{CE Amplifier}
	\label{fig:ceamp}
\end{figure}

To determine the cutoff frequencies for each capacitor, we short the other capacitors and determine the critical frequency for each capacitor.

\begin{align*}
	f_{Ci} & = \frac{1}{2 \pi (R_{s} + R_{i}) C_{i}} \quad ; \quad R_{i} = R_{1}\parallel R_{2} \parallel (\beta + 1) r_{e}                                               \\
	f_{Co} & = \frac{1}{2 \pi (R_{o} + R_{L}) C_{o}} \quad ; \quad R_{o} = R_{C}\parallel r_{o}                                                                           \\
	f_{CE} & = \frac{1}{2 \pi R_\text{eq} C_{E}} \quad ; \quad R_\text{eq} = R_{E}\parallel \left[ \frac{R_{s}\parallel R_{1} \parallel R_{2}}{\beta + 1} + r_{e} \right]
\end{align*}

\subsection{Low Frequency Response of CS Amplifiers}

\begin{figure}[H]
	\centering
	\begin{circuitikz}
		\draw (0,0) node[nmos] (Q) {};

		% Input 
		\draw (Q.G) -- ++(-0.5,0) to[C, l=$C_i$] ++(-2,0) to[R, l=$R_s$] ++(0,-2) to[sV, l=$V_{s}$] ++(0,-1.5) coordinate (GND) node[ground] {};

		% Gate
		\draw let \p1=(GND), \p2=(Q.G) in (Q.G) ++(-0.5,0) to[R, l=$R_G$, *-] ++(0,\y1-\y2) node[ground] {};

		% Source
		\draw let \p1=(GND), \p2=(Q.S) in (Q.S) to[R, l=$R_S$, *-] ++(0,\y1-\y2) node[ground] {};
		\draw let \p1=(GND), \p2=(Q.S) in (Q.S) -- ++(1.5,0) to[C, l=$C_S$] ++(0,\y1-\y2) node[ground] {};

		% Drain
		\draw (Q.D) to[R, l=$R_D$, *-] ++(0,2.5) node[vcc] {$V_{CC}$};
		\draw let \p1=(GND), \p2=(Q.D) in (Q.D) to[C, l=$C_o$, *-*] ++(3,0) coordinate (Vo) to[R, l=$R_L$] ++(0,\y1-\y2) node[ground] {};
		\draw (Vo) to[short, -o] ++(0.5,0) node[right] {$v_{out}$};
	\end{circuitikz}
	\caption{CS Amplifier}
	\label{fig:csamp}
\end{figure}

\begin{align*}
	f_{Ci} & = \frac{1}{2 \pi (R_{s} + R_{G}) C_{i}}                                                       \\
	f_{Co} & = \frac{1}{2 \pi (R_{L} + R_{o}) C_{L}} \quad ; \quad R_{o} = R_{D}\parallel r_{o}            \\
	f_{CS} & = \frac{1}{2 \pi R_\text{eq} C_{S}} \quad ; \quad R_\text{eq} = R_{S} \parallel \frac{1}{g_m}
\end{align*}

\section{High Frequency Response of Amplifiers}

\subsection{High Frequency Response of CE Amplifiers}

We consider the internal capacitances of the transistor, $C_{\pi}$($C_\text{be}$), $C_{\mu}$($C_\text{bc}$), and $C_{\text{ce}}$ as well as the wiring capacitances $C_{wi}$ and $C_{wo}$.
We say that $C_i$, $C_o$, and $C_E$ are shorted.

\begin{figure}[H]
	\centering
	\scalebox{1.5}{
		\begin{circuitikz}
			\ctikzset{capacitors/scale=0.45}
			\draw (0,0) node[npn] (Q) {};
			\draw let \p1=(Q.B), \p2=(Q.C) in (Q.C) to[C, l_=\scalebox{0.75}{$C_{\mu}$}, *-] (\x1,\y2) -- (Q.B);
			\draw let \p1=(Q.C), \p2=(Q.E) in (Q.C) -- ++(0.5,0) to[C, l=\scalebox{0.75}{$C_\text{ce}$}] ++(0,\y2-\y1) -- (Q.E) to[short, *-*] (Q.E);
			\draw let \p1=(Q.B), \p2=(Q.E) in (Q.B) to[C, l_=\scalebox{0.75}{$C_{\pi}$}, *-] (\x1,\y2) -- (Q.E);
		\end{circuitikz}
	}
\end{figure}

\thm{Miller Effect}{
	The capacitor connected between the input and output is called the feedback capacitor. Miller's theorem states that the feedback capacitor is equivalent to two capacitors at the inout and output.
	\begin{align*}
		C_{Mi} & = C_{f} (1 - A_v)                 \\
		C_{Mo} & = C_{f} \lt(1 - \frac{1}{A_v}\rt)
	\end{align*}
}

Using Miller's theorem, we say that
\begin{align*}
	C_{Mi} & = (1 - A_v)C_\text{bc} \quad                 & \quad C_{i} & = C_{wi} + C_\text{be} + C_{Mi}   \\
	C_{Mo} & = \lt(1 - \frac{1}{A_v}\rt)C_\text{bc} \quad & \quad C_{o} & = C_{wo} + C_{\text{ce}} + C_{Mo}
\end{align*}

The cutoff frequency is given by

\begin{align*}
	f_{Hi} & = \frac{1}{2 \pi R_{Thi} C_i} \quad & \quad R_{Thi} & = R_s \parallel R_1 \parallel R_2 \parallel r_{\pi} \\
	f_{Ho} & = \frac{1}{2 \pi R_{Tho} C_o} \quad & \quad R_{Tho} & = R_L \parallel R_C \parallel r_{o}
\end{align*}

\subsection{High Frequency Response of CS Amplifiers}

\begin{align*}
	R_{Thi} & = R_s \parallel R_G \quad               & \quad R_{Tho} & = R_D \parallel R_L \parallel r_d      \\
	C_{Mi}  & = (1 - A_v)C_\text{gd} \quad            & \quad C_{Mo}  & = \lt(1 - \frac{1}{A_v}\rt)C_\text{gd} \\
	C_i     & = C_{wi} + C_{\text{gs}} + C_{Mi} \quad & \quad C_o     & = C_{wo} + C_{\text{ds}} + C_{Mo}      \\
	f_{Hi}  & = \frac{1}{2 \pi R_{Thi} C_i} \quad     & \quad f_{Ho}  & = \frac{1}{2 \pi R_{Tho} C_o}
\end{align*}

\section{Open-Circuit Time Constants}

Open-circuit time constants is a method used to approximate the cutoff frequencies of the amplifier.

\begin{align*}
	b_1 & = \sum_{i=1}^n C_i R_i^0                  \\
	b_2 & = \sum_{i=1}^n \sum_j C_i R_i^0 R_j^i C_j
\end{align*}

Where $R_i^0$ is the resistance seen by the capacitor when all other capacitors are shorted.

\[
	\omega_H \approx p_1 \approx \frac{1}{b_1} = \frac{1}{\sum_{i=1}^n C_i R_i^0}
	.\]

\chapter{Negative Feedback}

There are 4 types of negative feedback topologies:

\begin{enumerate}
	\ii Feedback Voltage Amplifier (Series—Shunt)
	\ii Feedback Transconductance Amplifier (Series—Series)
	\ii Feedback Transresistance Amplifier (Shunt—Shunt)
	\ii Feedback Current Amplifier (Shunt—Series)
\end{enumerate}

We define

\[
	\begin{cases} \displaystyle
		\beta = \frac{x_f}{x_o}    & \text{Feedback factor}  \\
		A                          & \text{Open-loop gain}   \\
		A \beta                    & \text{Loop gain}        \\
		\displaystyle
		A_f = \frac{A}{1 + A\beta} & \text{Closed-loop gain}
	\end{cases}
	.\]

\begin{table}[H]
	\centering
	\def\arraystretch{2.5}
	\begin{tabular}[c]{|l|l|l|l|l|}
		\hline
		         & Series-Shunt V-V        & Shunt-Shunt I-V         & Series-Series V-I  & Shunt-Series I-I        \\
		\hline
		$R_{if}$ & $R_i (1+A\beta)$        & $\dfrac{R_i}{1+A\beta}$ & $R_i (1+A\beta)$   & $\dfrac{R_i}{1+A\beta}$ \\
		$R_{of}$ & $\dfrac{R_o}{1+A\beta}$ & $\dfrac{R_o}{1+A\beta}$ & $R_o (1+A\beta)$   & $R_o (1+A\beta)$        \\
		$\beta$  & $\dfrac{V_f}{V_o}$      & $\dfrac{I_f}{V_o}$      & $\dfrac{V_f}{I_o}$ & $\dfrac{I_f}{I_o}$      \\
		\hline
	\end{tabular}
\end{table}


\end{document}
